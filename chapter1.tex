%!TEX root = dissertation.tex
%\pagestyle{plain}
\chapter*{Глава 1. Метод численного решения трехмерных задач динамики несжимаемой жидкости 
на подвижных сетках}
\label{s:1}
\setcounter{chapter}{1}
\addcontentsline{toc}{chapter}{Глава~\thechapter~ Метод численного решения трехмерных задач 
динамики несжимаемой жидкости на подвижных сетках}
\setcounter{section}{0}

В~\cite{Cher} для моделирования установившихся (стационарных или периодически нестационарных) течений в 
проточном тракте (ПТ) гидротурбины при неподвижных лопатках направляющего аппарата предложен 
численный метод решения трехмерных уравнений несжимаемой жидкости, хорошо зарекомендовавший себя в работе на 
фиксированных сетках. В случае переходных течений, возникающих вследствие изменения положения
лопаток направляющего аппарата (НА), требуется обобщение метода на подвижные сетки, 
подстраивающиеся со временем под положения жестких 
границ расчетных областей. Выписать основные уравнения движения несжимаемой жидкости в криволинейных 
координатах, зависящих от времени, не представляет трудности. Однако проведение правильной
дискретизации этих уравнений для построения эффективного численного метода требует выполнения определенных 
условий. Главным из них является условие геометрической консервативности (УГК), которое заключается в 
сохранении без какого-либо возмущения постоянного однородного потока при расчете его на двигающейся сетке.

В диссертации метод~\cite{Cher} обобщается на задачи с подвижными сетками. Параллельно с проводимой
дискретизацией уравнений рассматриваются известные подходы построения численных методов на подвижных сетках, 
предложенные в \cite{trulio,thomas,demper,shyy,lesoin,koobus,forst,engel,zaits,volkov}. Осуществляется их 
сравнительный анализ и сопоставление с предлагаемым автором методом.

Отметим преимущества представленного метода при решении задач на подвижных сетках. 
Во-первых, в методе сохранены все важные свойства 
исходного численного алгоритма моделирования течений на фиксированных сетках. Для высокой 
экономичности численного решения трехмерных нестационарных уравнений Рейнольдса несжимаемой жидкости в 
реальных аэрогидродинамических установках, имеющих сложные многосвязные проточные части, в алгоритме 
использован максимально устойчивый и в то же время достаточно просто реализуемый неявный метод конечных 
объемов. Экономичность алгоритма повышается тем, что сегменты, в которых находится решение, строятся 
топологически эквивалентными параллелепипедам с регулярными сетками в них. Благодаря этому, а также введению 
фиктивных слоев сетки алгоритмы остаются однородными во всем сегменте, в том числе и у его границ. Обращение 
неявных операторов посредством попеременно-треугольного метода становится в данном случае крайне
простым. Сложная область проточного тракта с помощью созданного аппарата геометрического моделирования 
покрывается указанными сегментами. При этом используется экономичный метод задания и хранения в 
памяти компьютера типов граничных условий для каждого сегмента. Метод позволяет задавать произвольные 
граничные условия на различных участках границы и организовывать оперативный обмен данными между сегментами 
при минимальных затратах памяти и времени выполнения операций. Проводимая естественным образом сегментация 
проточного тракта аэрогидродинамической установки позволяет, с одной стороны, эффективно находить решения 
уравнений в каждом сегменте, с другой~--- строить в каждом сегменте достаточно качественные сетки, 
независимые от сеток соседних сегментов.

Помимо сохранения свойств исходного алгоритма, новый метод дополнен 
важным для работы на подвижных сетках свойством точного выполнения УГК.

\section{Основные уравнения в случае подвижного конечного объема}
\label{s:11}
\subsection{Дифференциальная форма записи в декартовой системе координат}
\label{s:111}
Течение вязкой несжимаемой жидкости в проточном тракте ГТ, как и в~\cite{Cher}, описывается нестационарными 
трехмерными уравнениями Рейнольдса, имеющими в декартовой системе координат
$\left( {x,\,y,\,z} \right) = \left( {x_1 ,\,x_2 ,\,x_3 } \right)$ следующий вид (по повторяющимся индексам 
предполагается суммирование $\sum\limits_{j=1}^3$):
\begin{equation}
  \label{1} 
  \frac{{\partial w_j }}{{\partial x_j }} = 0,
\end{equation}
\begin{equation}
  \label{2} 
  \frac{{\partial w_i }}{{\partial {\kern 1pt} t}} +
  \frac{{\partial {\kern 1pt} w_i w_j }}{{\partial {\kern 1pt} x_j}} + 
  \frac{{\partial {\kern 1pt} p}}{{\partial {\kern 1pt} x_i }} = 
  \frac{{\partial {\kern 1pt} }}{{\partial {\kern 1pt} x_j}}
  \left[ {\nu _{\text{eff}} \left( {\frac{{\partial {\kern 1pt} w_i }}{{\partial {\kern 1pt} x_j }} + 
  \frac{{\partial {\kern 1pt} w_j }}{{\partial {\kern 1pt} x_i }}} \right)} \right] + f_i,
  \ \ \ i = 1,2,3,
\end{equation}
где $ w_1 ,\ w_2 ,\ w_3$~--- компоненты вектора скорости; $p = p_c + \cfrac{2}{3}k$;  $p_c$~--- 
гидростатическое давление, деленное на плотность жидкости; $k$~--- кинетическая энергия турбулентности;
$f_1=x_1\omega^2+2w_2\omega$; $f_2=x_2\omega^2-2w_1\omega$; $f_3=g$~--- ускорение свободного падения;
$\omega$~--- угловая скорость вращения рабочего колеса вокруг оси $x_3$;  $\omega=0$ в остальных элементах 
проточной части.

Величина $\nu _{\text{eff}}$\, есть сумма молекулярной $\nu$ и турбулентной $\nu_t$ вязкостей
\begin{equation}
  \label{3} 
  \nu _{\text{eff}} = \nu  + \nu _t.
\end{equation}
Турбулентная вязкость $\nu_t$ определяется по стандартной $k - \varepsilon $ модели 
турбулентности~\cite{sharma}.

Для удобства дальнейшей работы с уравнениями запишем их в виде векторного уравнения
\begin{equation}
  \label{4} 
  {\bf{R}}^t {\bf{Q}}_t  + {\bf{E}}_x  + {\bf{G}}_y  + {\bf{H}}_z  = {\bf{F}},
\end{equation}
где ${\bf{Q}} = \left( {p,w_1 ,w_2 ,w_3 } \right)^T$, ${\bf{R}}^t =
\text{diag}\left( {0,1,1,1} \right), {\bf{F}} = \left( {0,\, f_1 ,\, f_2 ,\, f_3 } \right)^T$,
\begin{gather}
  \label{5} 
  {\bf{E}} = \left( \begin{array}{l}
   w_1  \\
   w_1^2  + p - \tau _{11}  \\
   w_1 w_2  - \tau _{21}  \\
   w_1 w_3  - \tau _{31}  \\
  \end{array} \right),\ {\bf{G}} = \left( \begin{array}{l}
   w_2  \\
   w_1 w_2  - \tau _{12}  \\
   w_2^2  + p - \tau _{22}  \\
   w_2 w_3  - \tau _{32}  \\
  \end{array} \right),\ {\bf{H}} = \left( \begin{array}{l}
   w_3  \\
   w_1 w_3  - \tau _{13}  \\
   w_2 w_3  - \tau _{23}  \\
   w_3^2  + p - \tau _{33}  \\
  \end{array} \right),\  \\
  \tau _{ij}  = \nu _{\text{eff}} \left( {\frac{{\partial w_i }}{{\partial x_j }} + \frac{{\partial w_j }}{{\partial x_i }}}
  \right).
\end{gather}

Перейдем от дифференциального векторного уравнения \eqref{4} к интегральному, выполняющемуся на произвольном 
подвижном объеме.

\subsection{Интегральные законы сохранения для подвижного объема}
\label{s:112}
Интегрируя уравнение \eqref{4} по произвольному подвижному объему $V(t)$  с границей $\partial V(t)$,
ориентация которой задается внешней единичной нормалью ${\bf{n}}$, и применяя теорему 
Гаусса"=Остроградского, получим
\begin{equation}
  \label{6} 
  {\bf{R}}^t \int\limits_{V(t)}
  {\frac{{\partial {\bf{Q}}}}{{\partial t}}dV}  +
  \displaystyle\oint\limits_{\partial V(t)} {\bf{K}} d{\bf{S}} =
  \int\limits_{V(t)} {{\bf{F}}dV},
\end{equation}
где $ d{\bf{S}} = {\bf{n}}dS$;  $dS$~-- элемент поверхности $\partial V (t )$; матрица ${\bf{K}}$ имеет 
структуру ${\bf{K}} = ( {{\bf{E}},\, {\bf{G}},\, {\bf{H}}} ).$ Интеграл в первом слагаемом левой части 
уравнения \eqref{6} преобразуется по правилу Лейбница для производной от интеграла с переменными 
пределами~\cite{fiht}
\begin{equation}
  \label{7} 
  \int\limits_{V(t)} {\frac{{\partial
  {\bf{Q}}}}{{\partial t}}dV} = \frac{\partial }{{\partial t}}\int\limits_{V(t)} {{\bf{Q}}dV} -
  \displaystyle\oint\limits_{\partial V(t)} {{\bf{Q}}
  {\frac{{\partial {\bf{x}}_{} }}{{\partial t}} \cdot d{\bf{S}}} },
\end{equation}
где ${\bf{x}}$~--- радиус-вектор элемента поверхности $dS$;
\begin{equation*}
  \cfrac{{\partial {\bf{x}}}}{{\partial t}} \cdot d{\bf{S}} = {\bf{x}}_t  \cdot d{\bf{S}} =
  ( {x_t ,\,y_t ,\, z_t })  \left(
  \begin{array}{l}
   dS_x  \\
   dS_y  \\
   dS_z  \\
  \end{array} \right)~\text{---}
\end{equation*}
нормальная скорость перемещения этого элемента. Подставляя \eqref{7} в \eqref{6}, получим
\begin{equation}
  \label{8} 
  {\bf{R}}^t \frac{\partial }{{\partial t}}\int\limits_{V(t)} {{\bf{Q}}dV} +
  \displaystyle\oint\limits_{\partial V(t)} {\left(
  {{\bf{K}}d{\bf{S}} - {\bf{R}}^t {\bf{Qx}}_t  \cdot d{\bf{S}}}
  \right)}  = \int\limits_{V(t)} {{\bf{F}}dV}.
\end{equation}
Преобразуем второй член в круглых скобках \eqref{8} под поверхностным интегралом
\begin{equation*}
  \label{9}
  \begin{array}{l}
   {\bf{R}}^t {\bf{Qx}}_t  \cdot d{\bf{S}} = {\bf{R}}^t \left( {\begin{array}{*{20}c}
     {p{\bf{x}}_t  \cdot d{\bf{S}}}  \\
     {w_1 {\bf{x}}_t  \cdot d{\bf{S}}}  \\
     {w_2 {\bf{x}}_t  \cdot d{\bf{S}}}  \\
     {w_3 {\bf{x}}_t  \cdot d{\bf{S}}}  \\
  \end{array}} \right) = {\bf{R}}^t \left( {\begin{array}{*{20}c}
     {p\left( {x_t dS_x  + y_t dS_y  + z_t dS_z } \right)}  \\
     {w_1 \left( {x_t dS_x  + y_t dS_y  + z_t dS_z } \right)}  \\
     {w_2 \left( {x_t dS_x  + y_t dS_y  + z_t dS_z } \right)}  \\
     {w_3 \left( {x_t dS_x  + y_t dS_y  + z_t dS_z } \right)}  \\
  \end{array}} \right) =  \\
    = {\bf{R}}^t \left( {\begin{array}{*{20}c}
     {px_t } & {py_t } & {pz_t }  \\
     {w_1 x_t } & {w_1 y_t } & {w_1 z_t }  \\
     {w_2 x_t } & {w_2 y_t } & {w_2 z_t }  \\
     {w_3 x_t } & {w_3 y_t } & {w_3 z_t }  \\
  \end{array}} \right)\left( {\begin{array}{*{20}c}
     {dS_x }  \\
     {dS_y }  \\
     {dS_z }  \\
  \end{array}} \right) = \left( {\begin{array}{*{20}c}
     0 & 0 & 0  \\
     {w_1 x_t } & {w_1 y_t } & {w_1 z_t }  \\
     {w_2 x_t } & {w_2 y_t } & {w_2 z_t }  \\
     {w_3 x_t } & {w_3 y_t } & {w_3 z_t }  \\
  \end{array}} \right)\left( {\begin{array}{*{20}c}
     {dS_x }  \\
     {dS_y }  \\
     {dS_z }  \\
  \end{array}} \right), \\
 \end{array}
\end{equation*}
сгруппируем его с матрицей потоков ${\bf{K}}$ и обозначим объединенную матрицу через ${\bf{K}}^t $
\begin{equation}
  \label{10} 
  {\bf{K}}^t  = \left( {\begin{array}{*{20}c}
   \begin{array}{l}
   w_1  \\
   w_1^2  + p - \tau _{11}  - w_1 x_t  \\
   w_1 w_2  - \tau _{21}  - w_2 x_t  \\
   w_1 w_3  - \tau _{31}  - w_3 x_t  \\
   \end{array} & \begin{array}{l}
   w_2  \\
   w_1 w_2  - \tau _{12}  - w_1 y_t  \\
   w_2^2  + p - \tau _{22}  - w_2 y_t  \\
   w_2 w_3  - \tau _{32}  - w_3 y_t  \\
   \end{array} & \begin{array}{l}
   w_3  \\
   w_1 w_3  - \tau _{13}  - w_1 z_t  \\
   w_2 w_3  - \tau _{23}  - w_2 z_t  \\
   w_3^2  + p - \tau _{33}  - w_3 z_t  \\
   \end{array}  \\
  \end{array}} \right).
\end{equation}
Окончательно интегральные законы сохранения массы и количества движения для подвижного объема $V(t)$ примут 
вид
\begin{equation}
  \label{11} 
  {\bf{R}}^t \frac{\partial }{{\partial t}}\int\limits_{V(t)} {{\bf{Q}}dV +
  \displaystyle\oint\limits_{\partial V(t)} {{\bf{K}}^t d{\bf{S}}} =
  \int\limits_{V(t)} {{\bf{F}}dV} }.
\end{equation}
\subsection{Условие геометрической консервативности в интегральной форме}
\label{s:113}
Уравнения, описывающие течение жидкости, должны сохранять со временем постоянный однородный поток в 
отсутствие внешних сил. Поэтому равенства ${\bf{F}} = 0,\ {\bf{Q}} =$~const и условие
замкнутости $\displaystyle\oint\limits_{\partial V(t)} {d{\bf{S}}} = 0$ выделенного объема $V(t)$, 
подставленные в уравнение \eqref{11}, дают тождественное скалярное уравнение
\begin{equation}
  \label{12} 
  \frac{\partial }{{\partial t}}V(t) =
  \displaystyle\oint\limits_{\partial V(t)} {{\bf{x}}_t  \cdot d{\bf{S}}},
\end{equation}
которое также является следствием правила Лейбница \eqref{7} и называется условием геометрической 
консервативности. Оно имеет простую геометрическую интерпретацию~--- изменение
выделенного объема во времени равно сумме объемов, образованных геометрическими элементами поверхности объема 
при их движении. Впервые понятие УГК было введено в \cite{trulio} для конечно-разностного метода, а позднее 
распространено на метод конечных объемов \cite{thomas,demper}. В работе~\cite{demper} было предложено 
вычислять скорости движения граней ячеек исходя из известных перемещений узлов так, чтобы условие 
геометрической консервативности автоматически выполнялось. При этом в~\cite{demper} рассмотрен только случай 
двумерных течений. В настоящей работе указанный подход получения метода, удовлетворяющего УГК, распространен 
на случай трехмерных течений.

В~\cite{demper,shyy,lesoin} показано, что нарушение УГК приводит к возникновению осцилляций решения, 
вызываемых только движением сетки. В работах~\cite{lesoin,koobus} предложены удовлетворяющие УГК схемы
соответственно первого и второго порядка по времени для двумерного и трехмерного случаев движения сетки.

Если в дискретном аналоге уравнения \eqref{11} не удовлетворить условие геометрической консервативности, то 
на задаче с решением в виде постоянного однородного потока численный метод на подвижной
сетке даст возмущенное, отличное от начальных данных решение. Ниже в разделе~\ref{s:12} проводится обобщение 
метода~\cite{Cher} на случай подвижных сеток. Для полноты изложения приведен также геометрический закон 
сохранения в дифференциальной форме.

\subsection{Условие геометрической консервативности в дифференциальной форме}
\label{s:114}
Наряду с физической областью $(x,y,z)$ с подвижной сеткой и подвижным элементом объема физического 
пространства $dV$ рассмотрим вычислительную область $(\xi,\eta,\zeta)$ с~неподвижным элементом объема 
вычислительного пространства $dV_0$. Между декартовыми координатами физической области и криволинейными
координатами вычислительной области устанавливается взаимно-однозначное отображение
\begin{gather}
  \label{xyzt}
  \xi=\xi(x,y,z,t),\notag \\
  \eta=\eta(x,y,z,t),\notag \\
  \zeta=\zeta(x,y,z,t),\notag \\
  t'=t.
\end{gather}
Рассмотрим матрицу
\begin{equation}
  \label{Amatr}
   {\bf{A}} = \left( {\begin{array}{*{20}c}
   {\xi _x } & {\xi _y } & {\xi _z } & {\xi _t } \\
   {\eta _x } & {\eta _y } & {\eta _z } & {\eta _t } \\
   {\zeta _x } & {\zeta _y } & {\zeta _y } & {\zeta _t } \\
   0 & 0 & 0 & 1 \\
   \end{array}} \right)
\end{equation}
и определим величину
\begin{equation}
  \label{JdetA}
  J=\text{det}\,{\bf{A}}.
\end{equation}
Тогда связь между элементами объемов в физическом и вычислительном пространствах задается равенством
\begin{equation}
  \label{JdV}
  J\,dV=dV_0.
\end{equation}
Следует отметить, что сетка и объемы в вычислительном пространстве фиксированы и не зависят от времени. 
Из-за подвижности сетки и объемов в физическом пространстве преобразование координат (13) и соответствующая 
ему матрица Якоби \eqref{Amatr} зависят от времени. В силу теоремы Гаусса"=Остроградского УГК
\eqref{12} переписывается в виде
\begin{equation}
  \label{difUGK} 
  \frac{\partial}{{\partial t}}\int\limits_V dV = \int\limits_V \text{div}\left({\bf{x}}_t\right) dV,
\end{equation}
с учетом \eqref{JdV} и независимости $dV_0$ от времени получаем
\begin{equation}
  \label{difUGK_V0} 
  \int\limits_{V_0}\frac{\partial }{{\partial t}}\left(\frac{1}{J}\right)dV_0 = \int\limits_{V_0}\frac{1}{J}
  \text{div}\left({\bf{x}}_t\right)dV_0.
\end{equation}
Из \eqref{difUGK_V0} следует дифференциальная форма геометрического закона сохранения
\begin{equation}
  \label{difUGK_1}
  J\frac{\partial }{{\partial t}}\frac{1}{J} - \text{div}\,{\bf{x}}_t  = 0.
\end{equation}

\section{Численный метод}
\label{s:12}
\subsection{Дискретизация уравнений для подвижного объема}
\label{s:121}
Расчетная физическая область разбивается на элементарные ячейки в виде криволинейных 
шестигранников. Центрам ячеек приписываются осредненные по их объемам $V^n_{ijk}$ значения переменных и
массовых сил
\begin{equation}
  \label{13} 
  \left( {{\bf{Q}}_{ijk} V_{ijk} } \right)^n  = \int\limits_{V_{ijk}^n } {{\bf{Q}}^n dV} ,\quad   \left(
  {{\bf{F}}_{ijk} V_{ijk} } \right)^n  = \int\limits_{V_{ijk}^n }
  {{\bf{F}}^n dV}.
\end{equation}

Неявная конечно"=объемная аппроксимация уравнения \eqref{11} на ячейке ${ijk}$ дает
\begin{equation}
  \label{14} 
  {\bf{R}}^t \frac{{3\left( {{\bf{Q}}V} \right)^{n + 1} - 4\left( {{\bf{Q}}V} \right)^n  + \left( {{\bf{Q}}V}
  \right)^{n - 1} }}{{2\Delta t}} = \left( {{\bf{RHS}}^{\,t} } \right)^{n + 1},
\end{equation}
где $\Delta t$~--- шаг по времени, $n$~--- номер слоя по времени, $V^n$~--- объем ячейки на $n$-м слое по 
времени. Правая часть \eqref{14} имеет структуру
\begin{equation*}
  \begin{array}{c}
  {\bf{RHS}}^t  =  - \left( {\left( {{\bf{K}}^t {\bf{S}}} \right)_{i + 1/2}  - \left( {{\bf{K}}^t {\bf{S}}} 
  \right)_{i - 1/2}  + \left( {{\bf{K}}^t {\bf{S}}} \right)_{j + 1/2}  - \left( {{\bf{K}}^t {\bf{S}}} 
  \right)_{j - 1/2}  + } \right. \\
  \left. { + \left( {{\bf{K}}^t {\bf{S}}} \right)_{k + 1/2}  - \left( {{\bf{K}}^t {\bf{S}}} \right)_{k - 1/2}}
  \right) + {\bf{F}}V_{ijk}, \\
  \end{array}
\end{equation*}
где выражения $ \left( {{\bf{K}}^t {\bf{S}}} \right)_{i + 1/2},\ \left( {{\bf{K}}^t {\bf{S}}} 
\right)_{j + 1/2} ,\ \left({{\bf{K}}^t {\bf{S}}} \right)_{k + 1/2}$ представляют собой
разностные потоки через грани $S_{i + 1/2},\, S_{j + 1/2},\, S_{k + 1/2}$ ячейки $ijk$ и объемом 
$V_{ijk} $. Определим векторы
\begin{gather}
  {\bf{S}}_{i + 1/2}  = \left( {{{S}}_x ,\ {{S}}_y ,\ {{S}}_z } \right)_{i + 1/2} ,\notag \\
  {\bf{S}}_{j + 1/2}  = \left( {{{S}}_x ,\ {{S}}_y ,\ {{S}}_z } \right)_{j + 1/2} ,\notag \\
  {\bf{S}}_{k + 1/2}  = \left( {{{S}}_x ,\ {{S}}_y ,\ {{S}}_z } \right)_{k + 1/2}
  \label{15}
\end{gather}
как нормали к граням $S_{i + 1/2},\, S_{j + 1/2},\, S_{k + 1/2}$ ячейки $ijk$. Длины этих векторов 
равны площадям соответствующих граней данной ячейки. Направления нормалей выбираются таким образом, чтобы к 
граням $S_{i + 1/2},\, S_{j + 1/2},\, S_{k + 1/2}$ они были внешними, а к граням $S_{i - 1/2},\,
S_{j - 1/2},\, S_{k - 1/2}$~--- внутренними по отношению к ячейке сетки с~индексом $ijk$. Такой выбор 
нормалей упрощает их построение, поскольку при этом выделяются одни глобальные для всех ячеек сетки, 
положительные по каждой из координат направления. Нормали к граням ячейки определяются таким образом, чтобы
выполнялось соотношение
\begin{equation}
  \label{16} 
  {\bf{S}}_{i + 1/2}  - {\bf{S}}_{i - 1/2}  + {\bf{S}}_{j + 1/2}  - {\bf{S}}_{j - 1/2}  + {\bf{S}}_{k + 1/2}
  - {\bf{S}}_{k - 1/2}  = 0,
\end{equation}
являющееся следствием интегрального условия замкнутости $\displaystyle\oint\limits_{\partial V(t)} 
{d{\bf{S}}} = 0$.

\subsection{Дискретное условие геометрической консервативности}
\label{s:122}
Подставив в \eqref{14} значения ${\bf{F}} = 0,\ {\bf{Q}} =$~const и учитывая дискретное условие замкнутости 
ячейки \eqref{16}, получим дискретное условие геометрической консервативности
\begin{gather}
  \frac{{3 V^{n + 1}  - 4V^n  + V^{n - 1} }}{{2\Delta t}} = \left( {{\bf{x}}_t  \cdot {\bf{S}}} 
  \right)_{i + 1/2}  - \left( {{\bf{x}}_t  \cdot {\bf{S}}} \right)_{i - 1/2} +
  \left( {{\bf{x}}_t  \cdot {\bf{S}}} \right)_{j + 1/2}  - \notag \\ -
  \left( {{\bf{x}}_t  \cdot {\bf{S}}} \right)_{j - 1/2} + 
  \left( {{\bf{x}}_t  \cdot {\bf{S}}} \right)_{k + 1/2}  - \left( {{\bf{x}}_t  \cdot {\bf{S}}} 
  \right)_{k - 1/2}.   
  \label{eq1:17}
 \end{gather}
Соотношение (\ref{eq1:17}) связывает способы вычисления объема ячейки и 
нормальной составляющей скорости движения грани ${\bf{x}}_t \cdot {\bf{S}}$.

\subsection{Скорость движения грани ячейки}
\label{s:123}
Выведем из \eqref{eq1:17} конкретные выражения для скорости движения грани ячейки ${\bf{x}}_t \cdot {\bf{S}}$.

\subsubsection{Одномерный случай}
\label{s:1231}
В одномерном случае скорости движения граней и нормали вырождаются в скалярные величины и УГК \eqref{eq1:17} 
принимает вид
\begin{equation}
  \label{18} 
  \frac{{3V_i^{n + 1}  - 4V_i^n + V_i^{n - 1} }}{{2\Delta t}} = \left( {x_t } \right)_{i + 1/2} - 
  \left( {x_t } \right)_{i - 1/2}.
\end{equation}
Объемы ячеек вычисляются как $V_i = x_{i + 1/2} - x_{i - 1/2}$, где $x_{i \pm 1/2}$~--- координаты 
краев $i$-й ячейки. Левая часть уравнения \eqref{18} может быть преобразована как
\begin{gather}
  \frac{{3 {V_i}^{n + 1}  - 4{V_i}^n  + {V_i}^{n - 1} }}{{2\Delta t}} =\notag  \\ = 
  \frac{1}{{2\Delta t}}\left[ {3\left( {x_{i + 1/2}  - x_{i - 1/2} } \right)^{n + 1}  - 4
  \left( {x_{i + 1/2}  - x_{i - 1/2} } \right)^n  + \left( {x_{i + 1/2}  - x_{i - 1/2} } 
  \right)^{n - 1} } \right] = \notag  \\ = 
  \frac{{3x_{i + 1/2}^{n + 1}  - 4x_{i + 1/2}^n  + x_{i + 1/2}^{n - 1} }}{{2\Delta t}} - 
  \frac{{3x_{i - 1/2}^{n + 1}  - 4x_{i - 1/2}^n  + x_{i - 1/2}^{n - 1} }}{{2\Delta t}} = \notag \\ =
  \left( {\frac{{3x_{}^{n + 1}  - 4x_{}^n  + x_{}^{n - 1} }}{{2\Delta t}}} \right)_{i + 1/2}  - 
  \left( {\frac{{3x_{}^{n + 1}  - 4x_{}^n  + x_{}^{n - 1} }}{{2\Delta t}}} \right)_{i - 1/2}.
  \label{19}
\end{gather}
Из \eqref{19} следует, что для тождественного выполнения УГК \eqref{18} скорости движения граней ячеек в 
одномерном случае должны рассчитываться по направленной назад разности второго порядка
\begin{equation}
  \label{20} 
  x_t  = \frac{{3x^{n + 1}  - 4x^n  + x^{n - 1} }}{{2\Delta t}}.
\end{equation}

\subsubsection{Двумерный случай}
\label{s:1232}
В двумерном случае ячейками сетки являются многоугольники, грани ячеек~--- отрезки прямых, объемы 
ячеек~--- это площади многоугольников. За основу построения 
выражения для скорости движения грани ячейки в этом случае взята методика, предложенная для двуслойной схемы 
в работе \cite{demper}. Для двуслойной схемы УГК имеет вид
\begin{equation}
  \label{21} 
  \frac{V_{ij}^{n + 1}  - V_{ij}^n }{\Delta t} = \left(
  {{\bf{x}}_t  \cdot {\bf{S}}} \right)_{i + 1/2}  - \left(
  {{\bf{x}}_t  \cdot {\bf{S}}} \right)_{i - 1/2}  + \left(
  {{\bf{x}}_t  \cdot {\bf{S}}} \right)_{j + 1/2}  - \left(
  {{\bf{x}}_t  \cdot {\bf{S}}} \right)_{j - 1/2},
\end{equation}
что можно переписать как
\begin{equation}
  \label{22} 
  \frac{\Delta V_{ij}^n}{\Delta t}= \left( {{\bf{x}}_t \cdot {\bf{S}}} \right)_{i + 1/2} - 
  \left( {{\bf{x}}_t  \cdot {\bf{S}}} \right)_{i - 1/2}  + \left( {{\bf{x}}_t  \cdot {\bf{S}}}
  \right)_{j + 1/2}  - \left( {{\bf{x}}_t  \cdot {\bf{S}}}
  \right)_{j - 1/2} ,
\end{equation}
где
\begin{equation}
  \label{23} 
  \Delta V_{ij}^n  \equiv V_{ij}^{n + 1} - V_{ij}^n
\end{equation}
есть изменение площади ячейки при переходе с $n$-го слоя на $n+1$-й слой.

В работе \cite{lesoin} показано, что
\begin{equation}
  \label{24} 
  \Delta V_{}^n  = V_{i + 1/2}^n  - V_{i - 1/2}^n  + V_{j + 1/2}^n - V_{j - 1/2}^n,
\end{equation}
где
\begin{equation*}
  V_{m \pm 1/2}^n  = \left( \left( {{\bf{x}}^{n + 1}  - {\bf{x}}^n } \right) {\frac{{{\bf{S}}^{n + 1}
  + {\bf{S}}^n }}{2}} \right)_{m \pm 1/2},\quad m=i,j,
\end{equation*}
представляют собой площади фигур, образованных гранями-отрезками ${{S}}_{m \pm 1/2}$ при их движении за время 
$\Delta t$. Плюсы при $V_{i+1/2},\  V_{j+1/2}$ и минусы при $V_{i-1/2},\  V_{j-1/2}$ в правой части \eqref{24}
обусловлены тем, что направления нормалей к отрезкам ${{S}}_{i+1/2},\  {{S}}_{j+1/2}$ являются внешними, а к
${{S}}_{i-1/2},\  {{S}}_{j-1/2}$~--- внутренними по отношению к ячейке сетки с~индексом $ij$.

Подставляя \eqref{24} в \eqref{22}, получим 
\begin{gather}
  \label{26}
  \frac{V^n_{i + 1/2}}{\Delta t} -  \frac{V^n_{i - 1/2}}{\Delta t} +  \frac{V^n_{j + 1/2}}{\Delta t}- 
  \frac{V^n_{j - 1/2}}{\Delta t} =\notag\\ = 
  \left( {{\bf{x}}_t  \cdot {\bf{S}}} \right)_{i + 1/2}  - \left( {{\bf{x}}_t  \cdot {\bf{S}}} 
  \right)_{i - 1/2}  + \left( {{\bf{x}}_t  \cdot {\bf{S}}} \right)_{j + 1/2}  - \left( {{\bf{x}}_t  
  \cdot {\bf{S}}} \right)_{j - 1/2}.
\end{gather}
Следовательно, если положить
\begin{equation}
  \label{27}
  \left( {{\bf{x}}_t  \cdot {\bf{S}}} \right)_{m \pm 1/2} = 
  \frac{{V^n_{m \pm 1/2}}}{{\Delta t}},\quad m = i,j,
\end{equation}
то УГК будет выполняться точно.

В случае трехслойной схемы УГК имеет вид
\begin{equation}
  \label{121} 
  \frac{{3V_{ij}^{n + 1}  - 4V_{ij}^n  + V_{ij}^{n - 1} }}{{2\Delta t}} = \left( {{\bf{x}}_t  \cdot {\bf{S}}}
  \right)_{i + 1/2}  - \left( {{\bf{x}}_t  \cdot {\bf{S}}} \right)_{i - 1/2} +
  \left( {{\bf{x}}_t  \cdot {\bf{S}}} \right)_{j + 1/2} - 
  \left( {{\bf{x}}_t  \cdot {\bf{S}}} \right)_{j - 1/2},
\end{equation}
что можно переписать как
\begin{equation}
  \label{122} 
  \frac{{3\Delta V_{}^n - \Delta V_{}^{n - 1} }}{{2\Delta t}}= \left( {{\bf{x}}_t  \cdot {\bf{S}}} 
  \right)_{i + 1/2}  - \left( {{\bf{x}}_t  \cdot {\bf{S}}} \right)_{i - 1/2} +
  \left( {{\bf{x}}_t  \cdot {\bf{S}}} \right)_{j + 1/2} - 
  \left( {{\bf{x}}_t  \cdot {\bf{S}}} \right)_{j - 1/2} ,
\end{equation}
где величина $\Delta V_{}^n$ определена в \eqref{23} и вычисляется также по формуле \eqref{24}.
Подставляя \eqref{24} в \eqref{122}, получим
\begin{gather}
  \left(\frac{3V^n  - V^{n - 1}}{2\Delta t}\right)_{i + 1/2}\! - \! 
  \left(\frac{3V^n  - V^{n - 1}}{2\Delta t}\right)_{i - 1/2}\! + \notag\\ +\! 
  \left(\frac{3V^n  - V^{n - 1}}{2\Delta t}\right)_{j + 1/2}\! -\! 
  \left(\frac{3V^n  - V^{n - 1}}{2\Delta t}\right)_{j - 1/2}\! =\notag\\ = 
  \left( {{\bf{x}}_t  \cdot {\bf{S}}} \right)_{i + 1/2}  - 
  \left( {{\bf{x}}_t  \cdot {\bf{S}}} \right)_{i - 1/2}  + 
  \left( {{\bf{x}}_t  \cdot {\bf{S}}} \right)_{j + 1/2}  - 
  \left( {{\bf{x}}_t  \cdot {\bf{S}}} \right)_{j - 1/2}.
  \label{126}
\end{gather}
Следовательно, если положить
\begin{equation}
 \label{127}
 \left( {{\bf{x}}_t  \cdot {\bf{S}}} \right)_{m \pm 1/2} = 
 \frac{{{\left(3V^n - V^{n - 1}\right)}_{m \pm 1/2}}}{{2\Delta t}},\quad m = i,j,
\end{equation}
то УГК будет выполняться точно.

\subsubsection{Трехмерный случай}
\label{s:1233}
В трехмерном случае автором развита идея работы \cite{demper} для двумерного случая. Дискретное условие
геометрической консервативности~\eqref{eq1:17} перепишем как
\begin{gather}
  \frac{{3\Delta V^n  - \Delta V^{n - 1} }}{{2\Delta t}} = 
  \left({{\bf{x}}_t  \cdot {\bf{S}}} \right)_{i + 1/2} - 
  \left({{\bf{x}}_t  \cdot {\bf{S}}} \right)_{i - 1/2} + 
  \left({{\bf{x}}_t  \cdot {\bf{S}}} \right)_{j + 1/2} - 
  \left({{\bf{x}}_t  \cdot {\bf{S}}} \right)_{j - 1/2} + \notag\\ + 
  \left({{\bf{x}}_t  \cdot {\bf{S}}} \right)_{k + 1/2} - 
  \left({{\bf{x}}_t  \cdot {\bf{S}}} \right)_{k - 1/2} ,
  \label{28}
\end{gather}
где $\Delta V_{}^n \equiv V^{n + 1} - V^n$. По аналогии с двумерным случаем изменение объема ячейки 
представим в виде
\begin{equation}
  \label{29} 
  \Delta V_{}^n = V_{i + 1/2}^n - V_{i - 1/2}^n + V_{j + 1/2}^n - V_{j - 1/2}^n 
  + V_{k + 1/2}^n - V_{k - 1/2}^n,
\end{equation}
где $V_{m \pm 1/2}^n$~--- объемы фигур, образованных гранями $S_{m \pm 1/2}$ при их движении за время 
$\Delta t$. Подставим соотношение \eqref{29} в уравнение \eqref{28}
\begin{gather}
  \frac{3V_{i + 1/2}^n  - V_{i + 1/2}^{n - 1}}{2\Delta t} - 
  \frac{3V_{i - 1/2}^n  - V_{i - 1/2}^{n - 1}}{2\Delta t} +
  \frac{3V_{j + 1/2}^n  - V_{j + 1/2}^{n - 1}}{2\Delta t} - 
  \frac{3V_{j - 1/2}^n  - V_{j - 1/2}^{n - 1}}{2\Delta t} +
  \notag \\ + \frac{3V_{k + 1/2}^n  - V_{k + 1/2}^{n - 1}}{2\Delta t} -  
  \frac{3V_{k - 1/2}^n  - V_{k - 1/2}^{n - 1}}{2\Delta t} =
  \left( {{\bf{x}}_t  \cdot {\bf{S}}} \right)_{i + 1/2} - 
  \left( {{\bf{x}}_t  \cdot {\bf{S}}} \right)_{i - 1/2} +\notag \\ + 
  \left( {{\bf{x}}_t  \cdot {\bf{S}}} \right)_{j + 1/2} - 
  \left( {{\bf{x}}_t  \cdot {\bf{S}}} \right)_{j - 1/2} +  
  \left( {{\bf{x}}_t  \cdot {\bf{S}}} \right)_{k + 1/2} - 
  \left( {{\bf{x}}_t  \cdot {\bf{S}}} \right)_{k - 1/2}.
  \label{30}
\end{gather}
Из \eqref{30} заключаем, что для точного выполнения УГК достаточно положить
\begin{equation}
  \label{31} 
  \left( {{\bf{x}}_t \cdot {\bf{S}}} \right)_{m \pm 1/2} = 
  \frac{\left(3V^n  - V^{n - 1} \right)_{m \pm 1/2}}{{2\Delta t}},\quad m = i,j,k.
\end{equation}

\subsection{Метод вычисления объемов \boldmath{$V^n$} и \boldmath{$V^n_{m\pm 1/2}$},
            обусловливающий точное выполнение УГК}
\label{s:124}
Комплексы $({\bf{x}}_t  \cdot {\bf{S}})_{m + 1/2}$, вычисленные по формуле \eqref{31}, будут давать точное 
выполнение дискретного УГК \eqref{eq1:17}, если соотношение \eqref{29} выполнено точно, для чего необходимо 
определенным образом вычислять объемы $V^n_{m\pm 1/2}$, заметаемые гранями ячеек при движении сетки. 
Очевидно, способ их вычисления зависит от способа расчета самих объемов $V^n$. Расчет объемов криволинейных 
шестигранников требует аккуратного отношения. Например, способ вычисления, используемый 
в работе~\cite{RizziErikson} приводит к тому, что в зависимости от положений узлов сетки некоторые части 
расчетной области учитываются в объемах ячеек дважды, а другие вообще не учитываются. В нестационарной 
постановке и на движущихся сетках такой способ вычисления объемов ячеек приводит к появлению нефизических  
источников массы и импульса внутри расчетной области, накоплению ошибки с течением времени, и как следствие, 
неверным результатам расчета. В \cite{volkov} сформулированы общие принципы выполнения \eqref{29} 
на основе метода неопределенных коэффициентов, однако предложенная там методика применима только к одномерным 
и двумерным задачам.

В настоящей работе предлагается следующий метод вычисления объемов, входящих в равенство \eqref{29}.

Пусть объем $V^n_{ijk}$ образован вершинами с радиус-векторами
\begin{gather}
  {\mathbf{x}}^n_{i,j,k}, \quad {\mathbf{x}}^n_{i,j+1,k},\quad {\mathbf{x}}^n_{i,j+1,k+1},
  \quad {\mathbf{x}}^n_{i,j,k+1}, \notag\\
  {\mathbf{x}}^n_{i+1,j,k},\quad {\mathbf{x}}^n_{i+1,j+1,k}, 
  \quad {\mathbf{x}}^n_{i+1,j+1,k+1},\quad {\mathbf{x}}^n_{i+1,j,k+1}.
  \label{x_Vijk}
\end{gather}
Объем ячейки $V^n_{ijk}$ вычислим как сумму объемов шести треугольных пирамид, имеющих общее ребро, 
соединяющее вершины ${\mathbf{x}}^n_{i,j,k}$ и ${\mathbf{x}}^n_{i+1,j+1,k+1}$ (рисунок~\ref{fig1:1},\,\emph{а}):
\begin{gather}
  V^n_{ijk}={\displaystyle{\frac{1}{6}}}\big({\bf{x}}^n_{i+1,j+1,k+1}-{\bf{x}}^n_{i,j,k}\big)
  \Big(
  \left[({\bf{x}}^n_{i,j+1,k}-{\bf{x}}^n_{i,j,k+1})\times
       ({\bf{x}}^n_{i,j+1,k+1}-{\bf{x}}^n_{i,j,k})\right]+ \notag\\[2mm]
 +\left[({\bf{x}}^n_{i,j,k+1}-{\bf{x}}^n_{i+1,j,k})\times
       ({\bf{x}}^n_{i+1,j,k+1}-{\bf{x}}^n_{i,j,k})\right]+\notag \\[2mm]
 +\left[({\bf{x}}^n_{i+1,j,k}-{\bf{x}}^n_{i,j+1,k})\times
       ({\bf{x}}^n_{i+1,j+1,k}-{\bf{x}}^n_{i,j,k})\right] \Big).
  \label{Vijk}
\end{gather}
Согласно введенным обозначениям, объем $V^n_{i-1/2}$, заметаемый гранью $S_{i-1/2}$ при ее движении за 
время $\Delta t=t^{n+1}-t^{n}$, образован вершинами
\begin{gather}
  {\mathbf{x}}^n_{i,j,k},  \quad  {\mathbf{x}}^n_{i,j+1,k},\quad {\mathbf{x}}^n_{i,j+1,k+1}, \quad  
  {\mathbf{x}}^n_{i,j,k+1}, \quad \notag \\
  {\mathbf{x}}^{n+1}_{i,j,k}, \quad  {\mathbf{x}}^{n+1}_{i,j+1,k}, \quad
  {\mathbf{x}}^{n+1}_{i,j+1,k+1},\quad {\mathbf{x}}^{n+1}_{i,j,k+1}.
  \label{x_Vi12}
\end{gather}
Заметим, что обозначения вершин объема $V^n_{i-1/2}$ совпадают с обозначениями (\ref{x_Vijk}) вершин 
объема $V^n_{ijk}$ с точностью до замены индексов $(\cdot)^{n}_{i+1}$ на индексы $(\cdot)^{n+1}_{i}$ 
(рисунок~\ref{fig1:1},\,{\it б}). Объемы $V^n_{i-1/2}$ вычисляются по той же формуле~\eqref{Vijk}, в которой  
индексы $(\cdot)^{n}_{i+1}$ заменены на $(\cdot)^{n+1}_{i}$:
\begin{gather}
  V^n_{i-1/2}={\displaystyle{\frac{1}{6}}}\big({\bf{x}}^{n+1}_{i,j+1,k+1}-{\bf{x}}^n_{i,j,k}\big)
  \Big(
  \left[({\bf{x}}^n_{i,j+1,k}-{\bf{x}}^n_{i,j,k+1})\times
       ({\bf{x}}^n_{i,j+1,k+1}-{\bf{x}}^n_{i,j,k})\right]+ \nonumber\\[2mm]
  \label{Vi12}
  +\left[({\bf{x}}^n_{i,j,k+1}-{\bf{x}}^{n+1}_{i,j,k})\times
       ({\bf{x}}^{n+1}_{i,j,k+1}-{\bf{x}}^n_{i,j,k})\right]+\nonumber \\[2mm]
  +\left[({\bf{x}}^{n+1}_{i,j,k}-{\bf{x}}^n_{i,j+1,k})\times
       ({\bf{x}}^{n+1}_{i,j+1,k}-{\bf{x}}^n_{i,j,k})\right] \Big).
  \label{Vni12}
\end{gather}
Аналогично с использованием формулы \eqref{Vijk} вычисляются объемы $V^n_{j-1/2}$ и $V^n_{k-1/2}$. 
Для нахождения $V^n_{j-1/2}$ индексы $(\cdot)^{n}_{j+1}$ заменяются на $(\cdot)^{n+1}_{j}$, для
нахождения $V^n_{k-1/2}$~--- $(\cdot)^{n}_{k+1}$ заменяются на $(\cdot)^{n+1}_{k}$.

\begin{figure}[ht!]
  \label{fig1:1}
  \centering\small
  \rule{-15mm}{0mm}{\it a}\rule{80mm}{0mm}{\it б}\\[1.5mm]
  \includegraphics[width=16.0cm]{ris1.png}%  \\
  \caption{Объем $V^n_{ijk}$ (\emph{а}) и объем $V^n_{i-1/2}$, заметаемый гранью
           $S_{i-1/2}$ при движении сетки за время $\Delta t=t^{n+1}-t^{n}$, (\emph{б})}
\end{figure}

С помощью элементарных преобразований показано, что при вычислении объемов $V^n_{ijk}$ и $V^{n+1}_{ijk}$ по 
формуле~\eqref{Vijk}, а объемов $V^n_{i\pm 1/2}$, $V^n_{j\pm 1/2}$, $V^n_{k\pm 1/2}$ по формулам 
вида~\eqref{Vni12} равенство~\eqref{29} выполнено тождественно. При проведении преобразований в правых 
частях~\eqref{Vijk} и~\eqref{Vni12} предварительно приведены подобные слагаемые. Так, ~\eqref{Vijk}
эквивалентно равенству
\begin{gather}
  V^n_{ijk}={\displaystyle{\frac{1}{6}}}\big({\bf{x}}^n_{i+1,j+1,k+1}-{\bf{x}}^n_{i,j,k}\big)
  \Big(
  \left[({\bf{x}}^n_{i,j+1,k}-{\bf{x}}^n_{i,j,k+1})\times
        {\bf{x}}^n_{i,j+1,k+1}\right]+ \nonumber\\[2mm]
  +\left[({\bf{x}}^n_{i,j,k+1}-{\bf{x}}^n_{i+1,j,k})\times
        {\bf{x}}^n_{i+1,j,k+1}\right]+ \nonumber\\[2mm]
  +\left[({\bf{x}}^n_{i+1,j,k}-{\bf{x}}^n_{i,j+1,k})\times
        {\bf{x}}^n_{i+1,j+1,k}\right] \Big).  \nonumber
\end{gather}
Аналогично упрощается выражение~\eqref{Vni12}.

\subsection{Обобщение метода искусственной сжимаемости на подвижные сетки}
\label{s:125}
\subsubsection{Метод искусственной сжимаемости}
\label{s:1251}
Метод искусственной сжимаемости заключается во введении в уравнение неразрывности производной по 
псевдовремени от давления, а в уравнения количества движения~--- производных по псевдовремени
от соответствующих компонент скорости. Модифицированное уравнение~\eqref{11} запишется в виде
\begin{equation}
  \label{32} 
  \left( {{\bf{R}}^\tau  \frac{\partial }{{\partial \tau }} + {\bf{R}}^t \frac{\partial }{{\partial t}}}
  \right)\int\limits_{V(t)} {{\bf{Q}}dV} + \displaystyle\oint\limits_{\partial V(t)} {{\bf{K}}^t_{\beta} dS}= \int\limits_{V(t)} {{\bf{F}}dV},
\end{equation}
где ${\bf{R}}^\tau = {\rm{diag}}(1,1,1,1)$,
\begin{equation}
  \label{33}
  {\bf{K}}^t_{\beta}  = \left( {\begin{array}{*{20}c}
  \begin{array}{l}
  \beta w_1  \\
  w_1^2  + p - \tau _{11}  - w_1 x_t  \\
  w_1 w_2  - \tau _{21}  - w_2 x_t  \\
  w_1 w_3  - \tau _{31}  - w_3 x_t  \\
  \end{array} & \begin{array}{l}
  \beta w_2  \\
  w_1 w_2  - \tau _{12}  - w_1 y_t  \\
  w_2^2  + p - \tau _{22}  - w_2 y_t  \\
  w_2 w_3  - \tau _{32}  - w_3 y_t  \\
  \end{array} & \begin{array}{l}
  \beta w_3  \\
  w_1 w_3  - \tau _{13}  - w_1 z_t  \\
  w_2 w_3  - \tau _{23}  - w_2 z_t  \\
  w_3^2  + p - \tau _{33}  - w_3 z_t  \\
  \end{array}  \\
  \end{array}} \right).
\end{equation}
В~\eqref{32} $\beta$~--- коэффициент искусственной сжимаемости. Уравнение \eqref{32} решается численно. При 
этом на каждом шаге по физическому времени $t$ проводится установление решения по псевдовремени $\tau$.

\subsubsection{Неявная конечно"=объемная аппроксимация}
\label{s:1252}
Дискретизация уравнения \eqref{32} дает
\begin{gather}
  {\bf{R}}^\tau  \frac{{{\bf{Q}}^{n + 1,\ s + 1}  -
  {\bf{Q}}^{n + 1,\ s} }}{{\Delta \tau }}V^{n + 1}  + 
  {\bf{R}}^t \frac{{3({\bf{Q}}V)^{n + 1,\ s + 1}  - 4\left( {{\bf{Q}}V}
  \right)^n  + \left( {{\bf{Q}}V} \right)^{n - 1} }}{{2\Delta t}} = \notag \\
  = ({\bf{RHS}}^{t} )^{n + 1,\ s + 1},
  \label{38}
\end{gather}
где $\Delta \tau, \Delta t$~--- шаги по псевдовремени и физическому времени соответственно, $s$~--- номер 
итерации по псевдовремени, $n$~--- номер слоя по времени. Дискретизация \eqref{38} имеет первый порядок 
аппроксимации по псевдовремени $\tau$ и второй порядок аппроксимации по физическому времени $t$. Правая часть
имеет структуру
\begin{equation*}
  \begin{array}{c}
  {\bf{RHS}}^{\,t}  =  - (({\bf{K}}^t_{\beta}  \cdot {\bf{S}})_{i + 1/2}  - ({\bf{K}}^t_{\beta}  
  \cdot {\bf{S}})_{i - 1/2} + ({\bf{K}}^t_{\beta}  \cdot {\bf{S}})_{j + 1/2} -
  ({\bf{K}}^t_{\beta} \cdot {\bf{S}})_{j - 1/2}  + \\ +
  ({\bf{K}}^t_{\beta} \cdot {\bf{S}})_{k + 1/2}  -
  ({\bf{K}}^t_{\beta} \cdot {\bf{S}})_{k - 1/2} ) + {\bf{F}}V,
\end{array}
\end{equation*}
где разностные потоки через грани ячеек представляются в виде суммы конвективного (невязкого) и вязкого 
потоков:
\begin{equation}
  \label{39} 
  \left( {{\bf{K}}^t_{\beta}  \cdot {\bf{S}}} \right)_{m + 1/2} = \left( {{\bf{K}}_{\beta,\, inv}^t  
  \cdot {\bf{S}}} \right)_{m + 1/2}  + \left( {{\bf{K}}_{\beta,\, vis}^t  
  \cdot {\bf{S}}} \right)_{m + 1/2}.
\end{equation}
Из \eqref{33} следует, что движение узлов сетки влияет на вычисление только невязких потоков. Невязкие потоки
имеют вид
\begin{equation}
  \label{40} 
  {{\bf{K}}_{\beta,\, inv}^t  \cdot {\bf{S}}} = \left(
  {\begin{array}{*{20}c}
   \beta U  \\
   {\left( {U - U_g } \right)w_1 + pS_x }  \\
   {\left( {U - U_g } \right)w_2 + pS_y }  \\
   {\left( {U - U_g } \right)w_3 + pS_z }  \\
\end{array}} \right),
\end{equation}
где $U = {\bf{w}} \cdot {\bf{S}},\ U_g  = {\bf{x}}_t  \cdot {\bf{S}}$. Для вычисления разностных невязких 
потоков используется MUSCL-схема~\cite{anders}, обеспечивающая третий порядок аппроксимации по пространству
\begin{equation}
  \label{41} 
  \left( {{\bf{K}}_{\beta,\, inv}^t  \cdot {\bf{S}}} \right)_{m + \frac{1}{2}} = 
  \frac{1}{2}\left[ {\left( {{\bf{K}}_{\beta,\, inv}^t \left( {{\bf{Q}}_L } \right) +
  {\bf{K}}_{\beta,\, inv}^t \left( {{\bf{Q}}_R } \right)} \right) \cdot {\bf{S}}_{m + 1/2} - 
  \left| {{\bf{A}}^t } \right|\left({{\bf{Q}}_R  - {\bf{Q}}_L } \right)} \right],
\end{equation}
где
\begin{gather}
  \label{42}
  {\bf{Q}}_L  = {\bf{Q}}_m  + \frac{1}{4} \left[ {\frac{2}{3}\left( {{\bf{Q}}_m  - {\bf{Q}}_{m - 1} } 
  \right) + \frac{4}{3}\left( {{\bf{Q}}_{m + 1}  - {\bf{Q}}_m } \right)} \right],\notag \\
  {\bf{Q}}_R  = {\bf{Q}}_{m + 1}  - \frac{1}{4} \left[ {\frac{2}{3}\left( {{\bf{Q}}_{m + 1}  - {\bf{Q}}_m } 
  \right) + \frac{4}{3}\left( {{\bf{Q}}_{m + 2}  - {\bf{Q}}_{m + 1} } \right)} \right].
\end{gather}
Под модулем матрицы $|{\bf{A}}^t|$ в \eqref{41} подразумевается разность
\begin{equation}
  |{\bf{A}}^t|={\bf{A}}^+-{\bf{A}}^-
\end{equation}
матриц ${\bf{A}}^+$ и ${\bf{A}}^-$ из разложения матрицы Якоби невязкого потока
\begin{equation}
  {\bf{A}}^t ({\bf{Q}}) = \frac{{\partial ({\bf{K}}_{\beta,\, inv}^t ({\bf{Q}}) \cdot {\bf{S}})}}{{\partial
  {\bf{Q}}}} \!=\! {\bf{R}}{\bf{D}}{\bf{L}}
  \label{43} 
\end{equation}
на сумму
\begin{equation}
  \label{431} 
  {\bf{A}}^t={\bf{A}}^++{\bf{A}}^-.
\end{equation}
Здесь ${\bf{R}},\ {\bf{L}}$ --- матрицы правых и левых собственных векторов матрицы ${\bf{A}}^t$ такие, 
что ${\bf{R}}{\bf{L}}={\bf{I}}$ (где ${\bf{I}}$ -- единичная матрица). Вид 
матриц ${\bf{A}}^t,\ {\bf{R}},\ {\bf{D}},\ {\bf{L}}$ приведен в приложении~А. 
Расщепление матрицы ${\bf{A}}^t$ на сумму матриц ${\bf{A}}^+$ и ${\bf{A}}^-$ \eqref{431} осуществляется на 
основе наличия у ${\bf{A}}^t$ действительных собственных значений
\begin{equation}
  \label{44}
  \lambda _{1,2}  = U_{}  - U_g , \quad
  \lambda _{3,4}  = U_{}  - \frac{1}{2}U_g \pm c,
\end{equation}
где $c = \sqrt {\left({U_{} -\cfrac{1}{2}U_g} \right)^2 + \beta \left( {S_x^2  + S_y^2  + S_z^2 } \right)}$.
Матрицы ${\bf{A}}^+$ и ${\bf{A}}^-$ вычисляются по формулам
\begin{gather}
  {\bf{A}}^+ = {\bf{R}}{\bf{D}}^+{\bf{L}}, \notag \\
  {\bf{A}}^- = {\bf{R}}{\bf{D}}^-{\bf{L}} \notag
\end{gather}
где ${\bf{D}}^{\pm}=Diag(\lambda_{1}^{\pm},\lambda_{2}^{\pm},\lambda_{3}^{\pm},\lambda_{4}^{\pm})=0.5({\bf{D}}
\pm |{\bf{D}}|)$, $|{\bf{D}}|=Diag(|\lambda_{1}|,|\lambda_{3}|,|\lambda_{3}|,|\lambda_{4}|)$ имеют 
неотрицательные и неположительные собственные значения. 

Вычисление вязких потоков ${\bf{K}}_{\beta,\, vis}({\bf{Q}}) \cdot {\bf{S}}$ производится так же, как и 
в исходном методе для фиксированных сеток~\cite{Cher}. Это связано с тем, что дополнительный член в 
уравнениях, связанный с движением узлов сетки, входит только в невязкие потоки. 

\subsubsection{Линеаризация и $LU$-факторизация}
\label{s:1253}
Неявная схема \eqref{38} представляет из себя систему нелинейных разностных уравнений и поэтому не может быть 
решена в неизменном виде. Для решения производится линеаризация с использованием метода Ньютона
\begin{gather}
  \left[ {\left( {\frac{1}{{\Delta \tau }}{\bf{R}}^\tau   + \frac{3}{{2\Delta t}}{\bf{R}}^t } \right)V^{n+1}- 
  \left( {\frac{\partial }{{\partial {\bf{Q}}}}{\bf{RHS}}^t} \right)^s } \right]
  \left( {\bf{Q}}^{n + 1,\,s + 1}  - {\bf{Q}}^{n + 1,\,s} \right) \! = \notag \\ \!=
  - {\bf{R}}^t \frac{{3({\bf{Q}}V)^{n + 1,\,s} - 4({\bf{Q}}V)^n + ({\bf{Q}}V)^{n - 1} }}
  {{2\Delta t}}V + ({\bf{RHS}}^t)^{n + 1,\,s}. 
  \label{Newton}
\end{gather}
При построении неявного оператора в левой части \eqref{Newton} предполагается, что на всех гранях 
невязкий поток аппроксимируется с первым порядком:
\begin{gather}
  \left( {{\bf{K}}_{\beta,\, inv}^t  \cdot {\bf{S}}} \right)_{m + \frac{1}{2}} = \notag \\ =
  \frac{1}{2}\left[ \left( {{\bf{K}}_{\beta,\, inv}^t \left( {{\bf{Q}}_m } \right) +
  {\bf{K}}_{\beta,\, inv}^t \left( {{\bf{Q}}_{m+1}} \right)} \right) \cdot {\bf{S}}_{m + 1/2} - 
  \left| {{\bf{A}}^t } \right|_{m+1/2} \Delta_{m+1/2} {\bf{Q}} \right],  
  \label{Kinv1} 
\end{gather}
а в вязком потоке оставляются производные только по той координате, поверхностью уровня которой является 
рассматриваемая грань. Таким образом вычисляя матрицу $\frac{\partial}{\partial {\bf{Q}}}{\bf{RHS}}^t$ и 
подставляя ее в \eqref{Newton}, получим следующую систему линейных уравнений:
\begin{gather}
  \left[ \left( {\frac{1}{{\Delta \tau }}{\bf{R}}^\tau + \frac{3}{{2\Delta t}}{\bf{R}}^t }\right)V^{n+1} + 
  {\bf{C}}_{i + 1/2}^- \Delta_{i + 1/2} + {\bf{C}}_{i - 1/2}^+ \Delta_{i - 1/2} +
  {\bf{C}}_{j + 1/2}^- \Delta_{j + 1/2} + \right. \notag \\ \left.
  + {\bf{C}}_{j - 1/2}^+ \Delta_{j - 1/2} +   {\bf{C}}_{k + 1/2}^- \Delta_{k + 1/2} + 
  {\bf{C}}_{k - 1/2}^+ \Delta_{k - 1/2} \right] \Delta^{s+1}{\bf{Q}} = \notag \\ =
  - {\bf{R}}^t \frac{3({\bf{Q}}V)^{n + 1,\,s} - 4({\bf{Q}}V)^n + 
  ({\bf{Q}}V)^{n - 1}}{2\Delta t} + ({\bf{RHS}}^t)^{n + 1,\,s}, 
  \label{discret} 
\end{gather}
где 
\begin{gather}
  \Delta^{s + 1}{\bf{Q}} = {\bf{Q}}^{n+1,\,s+1} - {\bf{Q}}^{n + 1,\,s}, \notag \\
  {\bf{C}}_{m + 1/2}^{\pm} = {\bf{A}}_{m + 1/2}^{\pm} \pm (\nu _{\text{eff}}{\bf{R}})_{m+1/2}.
\end{gather}
Матрицы ${\bf{R}}_{i+1/2},\ {\bf{R}}_{j+1/2},\ {\bf{R}}_{k+1/2}$ есть 
матрицы ${\bf{R}}^1,\ {\bf{R}}^2,\ {\bf{R}}^3$, вычисленные в приложении А и взятые на 
соответствующих гранях.

Система уравнений~\eqref{discret} переписывается в виде
\begin{gather}
  \left[{\bf{B}}+{\bf{C}}_{i + 1/2}^- T_i^+ - {\bf{C}}_{i - 1/2}^+ T_i^-
                +{\bf{C}}_{j + 1/2}^- T_j^+ - {\bf{C}}_{j - 1/2}^+ T_j^- \right. \notag \\
        \left.  +{\bf{C}}_{k + 1/2}^- T_k^+ - {\bf{C}}_{k - 1/2}^+ T_k^- \right] \Delta^{s+1}{\bf{Q}} 
        = \label{LU1}\\
        = - {\bf{R}}^t \frac{{3({\bf{Q}}V)^{n + 1,\,s} - 4({\bf{Q}}V)^n + ({\bf{Q}}V)^{n - 1} }}{{2\Delta t}}
        + ({\bf{RHS}}^t )^{n + 1,\,s}, \notag
\end{gather}
где $T_m^{\pm}$ --- оператор сдвига на один узел вперед ($+$) или назад ($-$) по индексу $m$,
\begin{equation}
  {\bf{B}}= \left({\frac{1}{{\Delta t}}{\bf{R}}^\tau + \frac{3}{{2\Delta t}}{\bf{R}}^t}\right)V^{n+1}
   + {\bf{C}}_{i -1/2}^+ - {\bf{C}}_{i +1/2}^- + {\bf{C}}_{j -1/2}^+ - {\bf{C}}_{j +1/2}^- 
   + {\bf{C}}_{k -1/2}^+ - {\bf{C}}_{k +1/2}^-. 
\end{equation}
После этого оператор в левой части \eqref{LU1} может быть приближенно представлен в виде произведения 
\begin{gather}
  \left[{\bf{B}}-{\bf{C}}_{i-1/2}^+T_i^- - {\bf{C}}_{j-1/2}^+T_j^- -{\bf{C}}_{k-1/2}^+T_k^- \right]
  {\bf{B}}^{-1} \times \notag \\ \times
  \left[{\bf{B}}+{\bf{C}}_{i+1/2}^-T_i^+ + {\bf{C}}_{j+1/2}^-T_j^+ +{\bf{C}}_{k+1/2}^-T_k^+ \right]
  \Delta ^{s + 1}{\bf{Q}} = \\ =
  - {\bf{R}}^t \frac{{3({\bf{Q}}V)^{n + 1,\,s} - 4({\bf{Q}}V)^n + ({\bf{Q}}V)^{n - 1} }}{{2\Delta t}}
        + ({\bf{RHS}}^t )^{n + 1,\,s}, \notag
\end{gather}
и последовательно обращен
\begin{gather}
  \Delta^{s+1/2} {\bf{Q}}_{ijk} = {\bf{B}}^{-1} \left[-{\bf{R}}^t 
  \frac{{3({\bf{Q}}V)^{n + 1,\,s} - 4({\bf{Q}}V)^n + ({\bf{Q}}V)^{n-1}}}{{2\Delta t}} + 
  ({\bf{RHS}}^t)^{n+1,\,s} + \right. \notag \\ \left.
  + {\bf{C}}_{i-1/2}^+ \Delta^{s+1/2}{\bf{Q}}_{i - 1jk} + {\bf{C}}_{j-1/2}^+ 
  \Delta^{s+1/2} {\bf{Q}}_{ij - 1k} + {\bf{C}}_{k -1/2}^+ \Delta^{s+1/2} {\bf{Q}}_{ijk-1} \right],
  \label{1step}
\end{gather}
\begin{gather}
  \Delta^{s+1} {\bf{Q}}_{ijk} = \Delta^{s+1/2} {\bf{Q}}_{ijk} - {\bf{B}}^{-1} 
  \left[ {\bf{C}}_{i + 1/2}^- \Delta^{s+1} {\bf{Q}}_{i+1jk} + \notag \right. \\ + \left.
  {\bf{C}}_{j+1/2}^- \Delta^{s+1} {\bf{Q}}_{ij+1k} + {\bf{C}}_{k+1/2}^- \Delta^{s+1} {\bf{Q}}_{ijk+1}\right]
  \label{2step}
\end{gather}
На первом шаге по формулам бегущего счета~\eqref{1step} совершается разовый обход области в направлении 
возрастания всех индексов и определяется вспомогательная величина $\Delta ^{s + 1/2} {\bf{Q}}_{ijk}$. 
На втором шаге по формулам бегущего счета~\eqref{2step} в направлении убывания индексов определяется 
величина $\Delta ^{s + 1} {\bf{Q}}_{ijk}$, по которой находится вектор неизвестных на $s+1$ итерации.
\begin{equation}
  ({\bf{Q}}_{ijk}^{n + 1} )^{s + 1}  = ({\bf{Q}}_{ijk}^{n + 1} )^s  + \Delta ^{s + 1} {\bf{Q}}_{ijk}. 
\end{equation}
Итерации по $s$ повторяются до достижения сходимости $({\bf{Q}}^{n + 1} )^s  \to {\bf{Q}}^{n + 1}$.

\section{Краевые условия на подвижной твердой границе}
\label{s:13}
Для скорости на подвижной границе $\Gamma$ задается условие прилипания
\begin{equation}
  \label{451}
  \left.{\bf{w}}\right|_\Gamma={\bf{v}}_\Gamma,
\end{equation}
где ${\bf{v}}_\Gamma$~--- скорость движения твердой границы.

Для выяснения вопроса о необходимости модификации условий для давления на твердой стенке в случае ее движения 
рассмотрим простейшую модельную ситуацию. Так как для расчета давления на поверхности тела используется 
уравнение количества движения в проекции на нормаль ${\bf{n}}$, то рассмотрим локально одномерную
задачу, в~которой твердой стенкой является плоскость $x=$~const. Нормальная составляющая уравнения количества 
движения к этой стенке есть
\begin{equation}
  \label{45} 
  u_t+uu_x+\frac{1}{\rho}p_x=\nu u_{xx}.
\end{equation}
Пусть стенка движется со скоростью $U$ в положительном направлении оси $Ox$. Введем систему координат
\begin{equation}
  x'=x-\int\limits_0^t Udt,
\end{equation}
движущуюся вместе с границей, и перепишем в ней уравнение \eqref{45}:
\begin{equation}
  \label{46} 
  u_t+(u-U)u_{x'}+\frac{1}{\rho}p_{x'}=\nu u_{x'x'}.
\end{equation}
Отнесем это уравнение к движущейся стенке. В силу условия прилипания \eqref{451} на стенке $u=U$ получим 
условие для давления в общем виде
\begin{equation}
  \label{47} 
  p_{x'}=\rho\left(-U_t+\nu u_{x'x'}\right).
\end{equation}
Производная $u_{x'x'}$ рассчитывается от скорости жидкости на стенке по нормали к ней. При $U=$~const и 
течении при больших числах Рейнольдса из \eqref{47} следует приближение пограничного слоя $p_{x'}=0,$
которое совпадает с условием для давления и в случае неподвижных сеток. Поэтому в настоящей работе для 
давления на подвижной твердой границе используется также приближение пограничного слоя.

\section{Стандартная $k-\boldmath{\varepsilon}$ модель турбулентности на подвижных сетках}
\label{s:14}
Выпишем модификацию метода решения уравнений $k-\varepsilon$ модели \cite{Cher} на подвижные сетки.

Каждое из уравнений $k-\varepsilon$ модели может быть записано в общем виде
\begin{equation}
  \label{50} 
  \frac{{\partial \varphi }}{{\partial t}} + \frac{\partial }{{\partial x_j }}\left( {\varphi w_j - 
  \left( {\nu  + \frac{{\nu _t }}{{\sigma _\varphi  }}} 
  \right)\frac{{\partial \varphi }}{{\partial x_j }}} \right) = H_\varphi.
\end{equation}
Интегрируя соотношение \eqref{50} по подвижному объему и применяя формулу
\begin{equation}
  \label{51} 
  \int\limits_{V(t)} {\frac{{\partial \varphi }}{{\partial t}}} dV = \frac{\partial }
  {{\partial t}}\int\limits_{V(t)} \varphi  dV - \displaystyle\oint\limits_{\partial V} 
  {\varphi {\bf{x}}_t  \cdot d{\bf{S}}},
\end{equation}
получим уравнение в виде интегрального закона сохранения
\begin{equation}
  \label{52} 
  \frac{\partial }{{\partial t}}\int\limits_{V(t)} \varphi dV = - 
  \displaystyle\oint\limits_{\partial V(t)} {\varphi \left( {{\bf{w}} - {\bf{x}}_t } 
  \right)\cdot d{\bf{S}}}  + \displaystyle\oint\limits_{\partial V(t)} {\left( {\nu + 
  \frac{{\nu _t }}{{\sigma _\varphi  }}} \right)\nabla \varphi d{\bf{S}}}  + 
  \int\limits_{V(t)} {H_\varphi dV}.
\end{equation}
Неявная аппроксимация интегрального уравнения~\eqref{52} приводит к системе нелинейных уравнений
\begin{equation}
  \label{53} 
  \frac{{3\left( {\varphi V} \right)_{ijk}^{n + 1} - 4\left( {\varphi V} \right)_{ijk}^n  + 
  \left( {\varphi V} \right)_{ijk}^{n - 1} }}{{2\Delta t}} = {\bf{RHS}}_{ijk}^{n + 1},
\end{equation}
где значения  $\varphi_{ijk}$ отнесены к центру ячейки и
\begin{gather}
  \label{54}
  {\bf{RHS}}_{ijk}^{n + 1} = - \sum\limits_{m = i,j,k} {\left[ {(\varphi {\bf{w}} 
  \cdot {\bf{S}})_{m + 1/2} - (\varphi {\bf{w}} \cdot {\bf{S}})_{m - 1/2} } \right]^{n + 1} }+ \notag\\
  + \sum\limits_{m = i,j,k} {\left[ {(\varphi {\bf{x}}_t  \cdot {\bf{S}})_{m + 1/2}  - (
  \varphi {\bf{x}}_t  \cdot {\bf{S}})_{m - 1/2} } \right]^{n + 1} } +\notag \\
  + \sum\limits_{m = i,j,k} {\left( {\left[ {\left( {\nu  + \frac{{\nu _t }}{{\sigma_\varphi  }}} \right)
  \left( {S_x \frac{{\partial \varphi }}{{\partial x}} + S_y \frac{{\partial \varphi }}{{\partial y}} + S_z 
  \frac{{\partial \varphi }}{{\partial z}}} \right)} \right]_{m + 1/2} } \right.}  - \notag \\
  \quad \quad \left. { - \left[ {\left( {\nu  + \frac{{\nu _t }}{{\sigma _\varphi  }}} \right)
  \left( {S_x \frac{{\partial \varphi }}{{\partial x}} + S_y \frac{{\partial \varphi }}{{\partial y}} + S_z 
  \frac{{\partial \varphi }}{{\partial z}}} \right)} \right]_{m - 1/2} } \right)^{n + 1}  + 
  \left( {H_\varphi V} \right)_{ijk}^{n + 1}.
\end{gather}
Схема для вычисления невязкого потока приобретает вид
\begin{gather}
  {\bf{F}}_{m + 1/2}^{inv}  = \frac{1}{2}\left[ \left(
  {\left( {\varphi {\bf{w}}} \right)_m  + \left( {\varphi {\bf{w}}}
  \right)_{m + 1} } \right) \cdot {\bf{S}}_{m + 1/2}-\right. \notag \\
  \left. - \left({\varphi _m  + \varphi _{m + 1} } \right) \cdot U_{g\, m + 1/2} -
  \left| {U - U_g } \right|_{m + 1/2} \left( {\varphi _{m+1} - \varphi _{m} } \right)  \right],
  \label{55}
\end{gather}
где $U = {\bf{w}} \cdot {\bf{S}},\ U_g = {\bf{x}}_t \cdot {\bf{S}}$. Скорость движения грани 
ячейки $U_{g\, m + 1/2}$ находится по формулам \eqref{31},~\eqref{Vni12}.

\section{Верификация и валидация предложенного метода}
\label{s:15}
\subsection{Расчет однородного потока на подвижной сетке}
\label{s:151}
\begin{figure}[b!]
  \label{fig1:2}
  \centering{
  \includegraphics[width=5cm]{Set_0.png}
  \includegraphics[width=5cm]{Set_2.png}
  \includegraphics[width=5cm]{Set_4.png}}
  \caption{Деформация сетки в кубической области течения однородного потока для трех шагов по $t$} 
\end{figure}

Целью численного эксперимента, описанного в настоящем разделе, является подтверждение точного выполнения 
дискретного УГК в предложенном численном алгоритме. Для этого рассматривается однородный поток несжимаемой 
жидкости с единичной скоростью через куб. Поскольку компоненты скорости 
потока и давление в нем постоянны, то погрешность аппроксимации разностной задачи будет определяться только 
погрешностью выполнения УГК~\eqref{eq1:17}. При этом не важен размер ячеек сетки. Необходимо 
продемонстрировать, что на существенно неравномерной сетке (рисунок~\ref{fig1:2}) погрешность аппроксимации 
остается равной нулю. В качестве краевых условий задавались параметры самого потока.
\begin{table}[t!]
  \centering
  \caption{Результаты расчета однородного потока на подвижной сетке}
  \vspace{2mm}
  \begin{tabular}{|c|c|c|c|}
  \hline
  &  &\multicolumn{2}{c|}{Предложенный метод} \\
  \cline{3-4}
  {Время $t$, c}&{Исходный метод  \cite{Cher}}  & с несогласованным & c согласованным\\
  &  & расчетом объемов& расчетом объемов\\
  \hline
  0.0 & 1.0000000 & 1.0000000 & 1.0000000  \\
  0.4 & 0.8785331 & 1.0025764 & 1.0000000  \\
  0.8 & 0.7646654 & 1.0056505 & 1.0000000  \\
  \hline
  \end{tabular}
  \label{tab:1}
\end{table}

Решение находилось тремя способами: исходным методом~\cite{Cher}, не учитывающим движение сетки, 
предложенным методом с несогласованным и согласованным расчетом 
объемов $V^n$ и $V^n_{m\pm 1/2}$ (см. раздел~\ref{s:124}). 
С помощью каждого способа было осуществлено пять шагов по времени с $\Delta t=0.2$~с. При 
этом на каждом новом шаге проводилась дополнительная деформация 
сетки (см. рисунок~\ref{fig1:2}). В таблице~\ref{tab:1} представлены рассчитанные скорости потока 
для каждого из примененных методов в разные моменты времени. Видно, что предложенный в диссертации 
метод с машинной точностью сохраняет однородный поток во времени.

\subsection{Движение кругового цилиндра в покоящейся несжимаемой вязкой жидкости}
\label{s:152}
В данном разделе исследуются свойства и возможности построенного метода расчета течений на подвижных сетках 
на задаче ламинарного обтекания кругового цилиндра, решение которой содержит большой набор достаточно сложных 
гидродинамических явлений. Адекватное предсказание последних является важным требованием, предъявляемым
к численным методам. Особенность проводимых исследований~---  нестационарная постановка задачи, в которой в
отличие от большинства используемых не жидкость обтекает неподвижный цилиндр, а цилиндр движется в 
покоящейся жидкости. Отметим, что важность этого теста состоит  в демонстрации эффективности и
надежности созданного программного инструментария.

\subsubsection{Постановка задачи}
\label{s:1521}
Движение цилиндра диаметра $d$ по области с покоящейся жидкостью задается посредством перемещения со временем 
части границы расчетной области, совпадающей с контуром цилиндра. Для построения сетки используется 
криволинейная система координат, нормально связанная с поверхностью цилиндра. Таким образом, вместе с
цилиндром движется и сетка (рисунок~\ref{fig1:3}). В качестве неподвижной внешней границы расчетной области 
принимается окружность радиуса $R$ с центром в начале декартовой системы координат $x,y$. Цилиндр
движется вдоль оси $Ox$ справа налево. В момент $t=0$ центр цилиндра находится в точке $(0.8R, 0)$, в 
конечный момент времени $t=T$~--- в точке $(-0.8R, 0)$. Движение начинается мгновенно с~постоянной 
скоростью $U$.

Пусть $0\leq \xi\leq 1$ есть продольная, а $0\leq \eta\leq 1$~--- поперечная координаты криволинейной системы 
координат, нормально связанной с цилиндром. Связь между декартовой и~криволинейной системами координат 
задается формулами
\begin{equation}
  \left\{
  \begin{array}{l}
  x\left( {\xi ,\eta ,t} \right) =  - r\left( {\xi ,\eta ,t} \right)\cos 2\pi\xi  + x_o \left( t \right), \\
  y\left( {\xi ,\eta ,t} \right) =    r\left( {\xi ,\eta ,t} \right)\sin 2\pi\xi , \\
  \end{array} \right.
\end{equation}
где $x_o \left( t \right) = 0.8\emph{R} - U  t$~--- координата двигающегося по оси $Ox$ центра цилиндра;
\begin{gather*}
  r\left( {\xi ,\eta ,t} \right) = \frac{\emph{d}}{2} + \left(
  {L\left( {\xi ,t} \right) - \frac{\emph{d}}{2}}
  \right)\frac{{\left( {1 + a} \right)^\eta - 1}}{a}, \\
  L\left( {\xi ,t} \right) = \sqrt {\emph{R}^2  - \left( {x_o
  \left( t \right)\sin 2\pi\xi } \right)^2 }  + x_o \left( t
  \right)\cos 2\pi\xi.
\end{gather*}
Кроме рассмотренных выше краевых условий на подвижной поверхности цилиндра, задаются все параметры 
невозмущенного потока на внешней границе расчетной области. На совпадающих границах $\Gamma^+$ и
$\Gamma^-$ (см. рисунок~\ref{fig1:3}) ставятся условия периодичности.
\begin{figure}[!t]
  \label{fig1:3}
  \centering\small \emph{а}\hspace*{81mm}\emph{б}\\
  \includegraphics[width=6.5cm]{set_r.png}\hspace*{20mm}\includegraphics[width=6.5cm]{set_l.png}
  \caption{Схема движения кругового цилиндра в области покоящейся несжимаемой вязкой жидкости: \emph{а}~--- 
           начальное, \emph{б}~--- конечное положения цилиндра}
\end{figure}

\subsubsection{Физические и схемные параметры задачи}
\label{s:1522}
Для расчета приняты следующие значения физических параметров задачи: $d=1$~м, $U=0.125$~м/с, коэффициент 
кинематической вязкости $\nu=0.003125$~м$^2$/с, давление в~покоящейся жидкости $p=0$. Этому режиму обтекания 
цилиндра соответствовало число Рейнольдса
\begin{equation*}
\mathrm{Re}=\frac{U  d}{\nu}=40.
\end{equation*}
К схемным параметрам относятся радиус внешней границы $R=200$~м, время движения цилиндра $T=2560$~c, параметр 
сгущения сетки по нормальному к цилиндру направлению $a=50$, шаг по физическому времени $\Delta t=0.1$~с, 
коэффициент искусственной сжимаемости $\beta=4$, шаг по псевдовремени $\Delta\tau=1$~с.

Сетка имеет 500 узлов в окружном и 500 узлов в нормальном к цилиндру направлениях. Заданное значение 
параметра сгущения $a$ обеспечивало отношение нормального размера ячейки у внешней границы к размеру ячейки у 
поверхности цилиндра, равное~50.

\subsubsection{Результаты расчета}
\label{s:1523}
\begin{figure}[b!]
  \label{fig1:4}
  \centering\includegraphics[width=10cm]{Phi_L_tec.png}  \\
  \caption{Картина установившегося обтекания цилиндра (линии тока): $\theta$~--- угол отрыва потока, 
           $L$~--- длина рециркуляционной зоны}
\end{figure}

Реализуемому в расчете числу Рейнольдса $\mathrm{Re}=40$ соответствует стационарный режим ламинарного 
обтекания цилиндра. В то же время расчет проводится в рамках нестационарной постановки задачи, в которой на 
каждом слое по времени $t$ решение устанавливается по псевдовремени $\tau$. Критерием сходимости
итераций $s$ по псевдовремени было условие
\begin{equation}
  \label{56} 
  \left\| {\bf{R}}^t\frac{3({\bf{Q}}V)^{n + 1, s} - 4\left( {{\bf{Q}}V} \right)^n  + \left( {{\bf{Q}}V} 
  \right)^{n - 1}}{2\Delta t}-\left({\bf{RHS}}^{\,t}\right)^{n+1, s}\right\|
  \leq10^{-6},
\end{equation}
где норма определяется как
\begin{equation*}
  \parallel {\mathbf F}\parallel=\max_{i,j,k,m}|F^m_{ijk}|.
\end{equation*}
Здесь $m=1,\ldots,4$ --- номер уравнения (соответствует номеру координаты вектора $\mathbf F$), $i,j,k$~--- 
номера ячеек по соответствующим координатным направлениям.

При выполнении условия \eqref{56} осуществляется переход на следующий слой по физическому времени и  
продвижение цилиндра по оси $Ox$ влево. В процессе движения цилиндра в его окрестности формируется и 
устанавливается поле течения, характерное для обтекания неподвижного цилиндра потоком жидкости с 
соответствующими параметрами. По достижении цилиндром центра расчетной области решение становится стационарным
и картина обтекания цилиндра приобретает вид, изображенный на рисунке~\ref{fig1:4}, где хорошо видны две 
сформировавшиеся симметричные рециркуляционные зоны на подветренной стороне цилиндра.

Полярный угол $\theta$ точки отрыва потока с цилиндра и длина рециркуляционной зоны~$L$ в~зависимости от 
положения цилиндра показаны на рисунке~\ref{fig1:5}.
\begin{figure}[t!]
  \label{fig1:5}
  \centering\small \emph{а}\hspace*{79mm}\emph{б}\\
  {\includegraphics[width=8.15cm]{Phi_sep_n.png}}\hfill
  {\includegraphics[width=8.15cm]{L_sep_n.png}}
  \caption{Зависимости $\theta$ (\emph{а}) и $L$ (\emph{б}) от положения центра цилиндра: серым 
           цветом выделены интервалы изменения значений параметров в сводных 
           данных~\cite{takami,dennis,belocirk,tuan,braza,coutan,tritt}, 
           пунктир~--- расчет в стационарной постановке \cite{Cher}}
\end{figure}
\begin{figure}[t!]\vspace*{2mm}
  \label{fig1:6}
  \centering\small \emph{а}\hspace*{79mm}\emph{б}\\
  {\includegraphics[width=8.15cm]{CD_n.png}}\hfill
  {\includegraphics[width=8.15cm]{Cp_n.png}}
  \caption{Зависимости $C_D$ (\emph{а}) и $C_p$ (\emph{б}) от положения центра цилиндра: серые 
           полосы~--- интервалы изменения значений параметров в сводных 
           данных~\cite{takami,dennis,belocirk,tuan,braza,coutan,tritt},
           штрих~--- расчет в стационарной постановке \cite{Cher}}
\end{figure}
Коэффициент сопротивления
\begin{equation*}
  C_{D} = \frac{2}{{\rho U^2}}\int\limits_S {\left( {p + \frac{2}{{{\mathop{\rm Re}\nolimits} }}
          \frac{{\partial u}}{{\partial n}}} \right)} n_x dS
\end{equation*}
и коэффициент давления  в передней критической точке
\begin{equation*}
  C_p=\frac{2(p-P)}{\rho U^2}
\end{equation*}
показаны в процессе движения цилиндра на рисунке~\ref{fig1:6}. Там же серым цветом отмечены интервалы 
изменения указанных параметров в сводных 
данных~\cite{takami,dennis,belocirk,tuan,braza,coutan,tritt}, а пунктиром~--- значения этих 
параметров, рассчитанные в постановке <<неподвижный цилиндр в потоке жидкости>> в стационарном приближении на 
неподвижной сетке \cite{Cher}. Видно хорошее совпадение параметров потока, полученных в нестационарной 
постановке на движущейся сетке, с экспериментальными данными и с результатами расчета стационарного обтекания 
цилиндра на неподвижной сетке.

Необходимо отметить, что рассмотренный подход  при решении данной задачи является более затратным по времени 
по сравнению с традиционным. Кроме того, как следует из рисунка~\ref{fig1:5}-\ref{fig1:6}, движение цилиндра 
сопровождается осцилляциями параметров рассчитываемого потока, природа которых связана с изменением на каждом 
шаге по времени аппроксимирующей поверхность цилиндра ломаной кривой~--- многогранника. Вершины этого 
многогранника меняют свое положение относительно сформировавшегося на предыдущем шаге по времени потока, что 
в свою очередь меняет значения наблюдаемых параметров. Возникающие осцилляции являются нефизическими~---  
обтекаемая поверхность колеблется как бы искусственно. В то же время в задачах с  подвижными границами 
подобное поведение параметров потока будет соответствовать физике явления и в задаче обтекания цилиндра в 
такой постановке продемонстрирована хорошая точность моделирования течения на подвижной сетке.


\subsection{Моделирование переходных процессов в гидротурбинах}
\label{s:153}
Для валидации разработанного численного метода было проведено сравнение результатов расчетов ряда 
переходных режимов с данными, полученными в натурных экспериментах. Результаты сравнения приводятся в 
разделе~\ref{s:24}. Анализ результатов решений показал, что метод корректно 
воспроизводит все рассмотренные характеристики и может быть использован для моделирования течений
в областях с подвижными границами в различных задачах вычислительной гидродинамики.

\section{Ускорение решения нестационарных задач динамики несжимаемой жидкости}
\label{s:16}
\subsection{Подходы к ускорению решения нестационарных задач}
\label{s:161}
Наиболее общей является постановка задачи гидродинамики турбин, в которой моделирование 
нестационарного течения проводится во всем проточном тракте. Нестационарная постановка позволяет моделировать 
весь диапазон режимов работы гидротурбин, в том числе и режимы неполной загрузки, учитывать взаимодействие 
ротора и статора турбины, описывать пульсации сил и моментов на лопатках, моделировать прецессирующий 
вихревой жгут за ротором и т.д. Для расчета потока строится многосвязная блочно-структурированная сетка, 
покрывающая весь проточный тракт турбины (рисунок~\ref{fig1:12}).  

\begin{figure}[ht!]
  \label{fig1:12}
  \centering                                                                                   
  \includegraphics[width=12cm]{SAKDO_domain.png} \\
  \caption{Блоки сеток, покрывающие проточный тракт гидротурбины}
\end{figure}

При этом нестационарная полная постановка требует значительных вычислительных 
ресурсов, а повышение точности расчетов за счет увеличения количества ячеек сетки приводит к нехватке 
оперативной памяти персонального компьютера. Например, расчет периодически нестационарного течения с 
прецессирующим вихревым жгутом на одном периоде (около трех оборотов рабочего колеса) с использованием сетки, 
содержащей  1 млн. узлов, требует 10 дней работы процессора Core2Duo 2.6 ГГц.

Подходы к ускорению нестационарных расчетов можно разделить на две группы. К первой отнесем все подходы, 
направленные на более быстрое решение системы линейных алгебраических уравнений (СЛАУ) 
\eqref{discret} на одном процессоре. Ко второй -- подходы, использующие 
параллельные алгоритмы расчета. В работе \cite{universe} 
проведено сравнение некоторых методов обращения матрицы СЛАУ, показавшее, что 
существенного уменьшения времени расчета относительно метода, использованного в диссертации, добиться не 
удается. Поэтому наиболее перспективным подходом к ускорению нестационарных расчетов является 
распараллеливание метода. 

Основной характеристикой, определяющей эффективность 
параллельной программы, является относительное 
сокращение времени счета, получаемое при использовании нескольких процессоров. 
Ускорение счета $S_N$ определяются следующим образом:
\begin{equation}
  S_N = \frac{T_1}{T_N},
\end{equation}
где $T_1$ -- время счета на одном процессоре, $T_N$ -- на $N$ процессорах. В идеальной ситуации ускорение счета равняется числу 
используемых процессоров.

В действительности же время, затрачиваемое на обмен данными между процессорами, неравномерная загрузка 
процессоров препятствуют идеальной ситуации и уменьшают получаемое ускорение счета.
При этом согласно закону Амдала \cite{amdal} ускорение работы программы на $N$ процессорах
\begin{equation}
  S_N \leq \frac{1}{f+\frac{1-f}{N}},
\end{equation}
где $f$ -- доля последовательного кода в программе. То есть, если например доля последовательного кода 
составляет 2~\%, то более чем 50-кратное ускорение в принципе получить невозможно. Таким образом, применение 
многопроцессорных ЭВМ является наиболее целесообразным подходом для создания модулей оперативного расчета 
динамических нагрузок, и преодоления ограничений, свойственных однопроцессорным машинам в терминах времени 
счета. Более того, лучше всего использовать геометрическое распараллеливание, заключающееся в декомпозиции
всей расчетной области на блоки, каждый из которых рассчитывается на отдельном ядре многопроцессорной 
вычислительной системы. В этом случае доля последовательного кода $f \sim 0$.

\subsection{Реализация геометрического распараллеливания}
\label{s:162}
В настоящей работе распараллеливание счета осуществляется распределением блоков расчетной сетки на процессоры 
кластера. Расчетная область естественным образом разбивается на 5 элементов -- спиральная камера, 
направляющий аппарат, рабочее колесо, диффузор и отсасывающая труба. Каждый элемент в свою очередь 
разбивается на односвязные блоки, топологически эквивалентные параллелепипедам. НА, РК, диффузор и бетонная 
спиральная камера представляют собой кольца из блоков, межпроцессорные обмены в них устроены одинаково. 
Для стальной спиральной камеры и ОТ были разработаны отдельные процедуры обмена полями течения, т.к. они 
имеют другую топологию, отличную от кольца. Вначале происходит параллельный расчет течения внутри блоков и 
обмены полями между блоками. Обмены между элементами расчетной области производятся после завершения расчета 
во всех блоках перед переходом на следующую итерацию по псевдовремени.

Коммуникации между процессорами осуществляются с использованием стандарта MPI~\cite{MPI}. 
Для посылки сообщений использовалась неблокирующая процедура MPI\_ISEND, для 
получения -- блокирующая MPI\_RECV. 
Если расчет блоков, имеющих общую границу, проводится на одном процессоре, то межпроцессорных обменов не 
проводится.

\subsection{Результаты численных экспериментов}
\label{s:163}
\subsubsection{Многопроцессорные системы}
\label{s:1631}
Решение задачи с использованием разработанного программного комплекса проводилось на многопроцессорных 
системах, представленных в таблице~\ref{tab:2}.
\begin{table}[h!]
  \caption{Рассмотренные многопроцессорные системы}
  \center
  \begin{tabular}{|c|c|c|c|c|}
  \hline
  Кластер & $N_{proc}$ (узлы$\times$ & Название проц.,    & RAM         &  Компилятор \\
          & $\times$проц.$\times$ядра) & размер кэш-памяти L3 & на узел &
  \\ \hhline{=|=|=|=|=} 
  ИВЦ НГУ & 1152  & Intel Xeon X5670  & 24 ГБ & Intel Fortran \\
          & (96 $\times$ 2 $\times$ 6)& 2.93 ГГц, 12 MБ & & 10.1.015
  \\ \hline
  ССКЦ  & 1152  & Intel Xeon X5670  & 24 ГБ & Intel Fortran \\
  НКС-30Т & (96 $\times$ 2 $\times$ 6)& 2.93 ГГц, 12 MБ & & 13.1.000
  \\ \hline
  Персональ- & 8  & Intel Core i7 950  & 12 ГБ & Intel Fortran \\
  ный комп.  & (1 $\times$ 2 $\times$ 4)& 3.06 ГГц, 8 MБ & & 11.1.051
  \\ \hline
  \end{tabular}
  \label{tab:2}
\end{table}
\begin{table}[h!]
  \caption{Число ячеек в расчетной сетке, покрывающей проточный тракт}
  \center
  \begin{tabular}{|c|c|c|c|c|c|c|}
  \hline
    & СК & НА & РК & ДФ & ОТ & Всего 
  \\ \hhline{=|=|=|=|=|=|=} 
   Число & 18  & 32 & 4  & 4 & 3 & 61 \\
   блоков&     &    &    &   &   & 
  \\ \hline
   Число  &         &        &         &       &         &   \\
   ячеек  &  29 232 & 44 573 &  64 757 & 8624  & 28 594  & - \\
   в блоке&         &        &         &       &         & 
  \\ \hline
   Общее  &          &           &         &        &         &           \\
   число  &  526 176 & 1 426 336 & 259 028 & 34 496 & 85 782  & 2 331 818 \\
   ячеек  &          &           &         &        &         & 
  \\ \hline
  \end{tabular}
  \label{tab:3}
\end{table}

\subsubsection{Расчетная область и ее размеры}
\label{s:1632}
Проточный тракт гидротурбины включает следующие элементы: спиральную камеру (СК), направляющий 
аппарат (НА), рабочее колесо (РК), конус отсасывающей трубы (диффузор, ДФ) и отсасывающую трубу (ОТ). 
Направляющий аппарат имеет 32 лопатки,
РК -- 4 лопасти. В таблице~\ref{tab:3} приведены количество блоков в каждом элементе гидротурбины и число ячеек 
в каждом блоке. В принятом подходе к распараллеливанию максимальное число процессоров, которое может 
использоваться  для расчета течения в элементе гидротурбины, не может превышать числа блоков, на которые 
разбит этот элемент.
\begin{figure}[h!]
  \label{fig1:13}
  \centering\small{\it а}\hspace*{90mm}{\it б}\\
  {\includegraphics[width=8.5cm]{speed_Up_HA.png}}\hfill
  {\includegraphics[width=8.5cm]{speed_Up_HA_eff.png}}
  \caption{Ускорения (\it{а}) и эффективности (\it{б}) распараллеливания НА, полученные на различных 
           многопроцессорных системах: \it{1} -- идеальное ускорение, \it{2} -- ССКЦ, 
           \it{3} -- ИВЦ НГУ, \it{4} -- ПК}
\end{figure}

\subsubsection{Расчет потока в направляющем аппарате}
\label{s:1633}
Проведены расчеты течения в направляющем аппарате на различном числе процессоров 
(1, 2, 3, 4, 6, 8, 12, 16, 24, 32). На рисунке~\ref{fig1:13} представлены полученные ускорения и 
эффективности счета $E_N = \dfrac{S_N}{N}$ на различных кластерах. 
Для сравнения там же приведено идеальное ускорение, равное числу используемых процессоров и идеальная
эффективность, тождественно равная 1. 
Видно, что на кластере НГУ наиболее оптимально задача решается при использовании 4 процессоров, 
при этом получено сверхлинейное ускорение счета. 
Такая особенность объясняется <<эффектом кэша>>, когда блоки расчетной сетки начинают 
полностью помещаться в кэш-память процессора. Дальнейшее увеличение числа используемых процессоров $N$ 
приводит к снижению эффективности $E_N$. Наибольшая эффективность счета достигается, когда число используемых 
процессоров является делителем числа блоков в элементе, так как в этом случае нагрузка на каждый процессор 
одинакова.

Двукратное отставание реального ускорения от идеального объясняется медленной скоростью обмена данными между
процессорами. Измерения времени в расчете показали, что время расчета одного блока НА на одном 
процессоре $t_\text{расчета}$ примерно равно времени обмена полями течения на границах между двумя 
процессорами $t_\text{обмена}$. Таким образом, 
\begin{equation}
  T_{32} = t_\text{расчета} + t_\text{обмена} \sim 2 t_\text{расчета},
\end{equation}
что и подтверждает рисунок~\ref{fig1:13}.

\subsubsection{Расчет потока во всем проточном тракте}
\label{s:1634}
Расчетная область включает спиральную камеру (18 блоков), направляющий аппарат (32 блока), рабочее колесо 
(4 блока), диффузор (4 блока) и отсасывающую трубу (3 блока). 
Число блоков и ячеек в расчетной сетке указаны в таблице~\ref{tab:3}.
Расчеты проводились на 1, 5, 6, 7, 9, 11, 13, 15, 17, 19, 21, 23, 25, 28, 36, 45 и 61 процессорах. 

Для получения наибольшего ускорения необходимо добиваться равномерности числа расчетных ячеек на каждый 
процессор. Внутри одного элемента гидротурбины, лучше брать число процессоров, являющееся делителем числа 
блоков этого элемента. Тогда для аппарата число процессоров может принимать значения степеней двойки (т.к. 
$32=2^5$), для рабочего колеса и диффузора -- 1, 2, 4, для трубы -- 1 или 3. 

\begin{figure}[ht]
  \label{fig1:14}
  \centering\small{\it а}\hspace*{90mm}{\it б}\\
  {\includegraphics[width=8.5cm]{speed_Up_all.png}}\hfill
  {\includegraphics[width=8.5cm]{speed_Up_all_eff.png}}
  \caption{Ускорения (\emph{а}) и эффективности (\emph{б}) распараллеливания всей гидротурбины, 
           полученные на различных многопроцессорных системах: \emph{1} -- идеальное ускорение, 
           \emph{2} -- ССКЦ, \emph{3} -- ИВЦ НГУ}
\end{figure}

Например, если взять максимально возможное число процессоров, равное 61, то вычислительная нагрузка для блока
рабочего колеса будет самой большой и составит 64 757 ячеек на процессор. На рисунке~\ref{fig1:14} 
приведены результаты распараллеливания расчета в полной постановке, полученные на кластерах ИВЦ НГУ и ССКЦ. 
Четырехкратное отставание от идеального значения обусловлено неравномерностью распределения ячеек расчетной 
сетки по процессорам и низкой скоростью обмена данными между процессорами. Из проведенных исследований 
следует, что при использовании 61 счетного процессора время решения задачи моделирования трехмерного 
турбулентного течения во всей гидротурбине на сетке с общим количеством ячеек около 2.5 млн сокращается более 
чем в 15 раз и составляет от 1 до 2 дней, что вполне приемлемо для практического применения.
