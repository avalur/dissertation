%!TEX root = dissertation.tex
%\pagestyle{plain}
\chapter*{Введение}
\addcontentsline{toc}{chapter}{Введение}%
\setcounter{section}{0}

Диссертационная работа посвящена разработке новых численных моделей для адекватного описания трехмерных 
течений в гидротурбинах при переходных режимах, в уплотнениях и полостях гидротурбин и в проточных частях с 
нестандартными элементами.

\textbf{Актуальность темы исследований.} 
Технологии новых и возобновляемых источников энергии включены в перечень критических технологий 
Российской Федерации~\cite{prt_RF}. В 2010 г. гидроэнергетика обеспечивала около 20\% мирового конечного 
потребления энергии от возобновляемых источников, и этот процент неуклонно растет~\cite{GSR_12}.

В связи с этим исследование динамики вязкой несжимаемой жидкости в проточном тракте (ПТ) 
гидроэлектростанции (ГЭС) и ее воздействия на элементы гидротурбины (ГТ) представляет значительный интерес. 
Как правило, лабораторные и натурные экспериментальные исследования сложны и дорогостоящи, поэтому задача 
построения эффективных численных моделей и изучения на их основе нестационарных турбулентных течений в ПТ ГЭС
является актуальной. Создание численной модели подразумевает математическую формулировку задачи, 
разработку, обоснование и тестирование численных алгоритмов и их реализацию в виде комплекса программ.

Течения жидкости в ПТ ГЭС подразделяются на установившиеся и переходные.
К установившимся относятся течения при неизменных значениях расхода жидкости, частоты
вращения рабочего колеса (РК) и нагрузки на вал РК. Эти течения являются стационарными или
периодически нестационарными и достаточно точно моделируются в настоящее 
время~\cite{Cher,Rupr,Vunen1,Smir,Kuro}.

Переходные течения возникают при переходе ГТ из одного фиксированного режима работы в другой, 
вызванном регулировкой открытия направляющего аппарата (НА) или изменением нагрузки на вал РК. 
Такие течения являются существенно нестационарными и характеризуются сильными колебаниями расхода во 
времени. Это приводит к динамическому изменению давления в проточном тракте, проявляющемуся в форме 
гидравлического удара. Моделирование таких течений актуально в связи с частыми изменениями
мощности электросети (как суточными, так и в течение дня), приводящими к необходимости 
изменения режима работы ГЭС. 

До последнего времени для исследования переходных процессов в ГЭС использовалась одномерная 
гидроакустическая теория. Она основана на гиперболической системе уравнений сохранения массы и импульса для 
сжимаемой жидкости \cite{Jmud,krivch,Nicol}.
В рамках этого подхода водовод и элементы гидротурбины
моделируются набором осесимметричных труб заданной геометрии, переменными являются скорость и 
давление жидкости. В России такие модели получили развитие в работах 
\cite{Popov,AleksMon,KuiPilZah,Samoil,GorKurBook,GorKurAtlas,KurTurb,Kur93,OkulPil,ArkOkulPil,Kur2013}. 
В \cite{Popov} рассмотрены вопросы теории и расчета нестационарных гидромеханических процессов, 
изложены методы построения одномерных моделей неустановившегося движения рабочих жидкостей в напорных 
трактах, приведены примеры приложения предлагаемых методов к расчету переходных процессов, колебаний 
и устойчивости гидро- и пневмосистем. В монографии \cite{AleksMon} систематизированы и обобщены сведения 
о концентрированных вихрях, наблюдаемых в 
природе и технике. Описаны модели вихревых структур, применяемые при интерпретации экспериментальных данных и
являющиеся базовыми для развития теоретических и численных подходов к изучению вихревых течений. Применение 
представленных в \cite{AleksMon} подходов привело к совершенствованию моделей для описания нестационарных 
явлений в гидротурбинах. В работе \cite{KuiPilZah} оценено влияние характера распределения завихренности и 
размера кавитационной полости за РК ГТ на коэффициенты характеристического уравнения, решение которого
дает собственные частоты и инкременты колебаний давления. 

Первые исследования вопроса о собственных акустических колебаниях в решетчатых областях 
в нестационарном потоке представлены в \cite{Samoil,GorKurBook,GorKurAtlas}. В работах \cite{KurTurb,Kur93}
разработанные подходы были применены к моделированию акустических колебаний в ПТ ГТ. Выводы, сделанные 
в работе \cite{Kur93} практически совпали с экспериментальными и теоретическими исследованиями уровня 
пульсаций давления в ПТ ГТ в зависимости от режима эксплуатации, представленными в \cite{OkulPil,ArkOkulPil}.
Согласно работе \cite{Kur2013} это привело к запрету эксплуатации турбоагрегатов на форсированных 
режимах работы.

Тем не менее существенным недостатком одномерного подхода является 
необходимость {\it априори} знать универсальную характеристику гидротурбины (УХ ГТ), используемую 
для задания условий на границах труб и коэффициентов уравнений.
Сложная геометрия спиральной камеры, решеток статора, НА, РК и отсасывающей трубы (ОТ) представляется 
в таком подходе косвенно через УХ ГТ, также в нем невозможно учесть особенности трехмерного нестационарного
турбулентного потока (вихри, рециркуляционные и кавитационные зоны).

В последние годы появляются работы, в которых переходные течения моделируются в трехмерных 
постановках. В \cite{Li,Liu1,Liu2} с использованием пакета FLUENT 6.3 
проведено моделирование выхода в разгон, сброса нагрузки в турбинном режиме и сброса нагрузки в 
насосном режиме работы гидротурбины, но без учета явления гидроудара. 
В \cite{Nicolle} проведено моделирование переходного процесса пуска в турбинный режим 
с использованием пакета CFX 13.0, также без учета гидроудара. 

В настоящей работе для решения задачи моделирования течения при переходных режимах работы ГЭС 
предлагается оригинальная гибридная модель, в которой прохождение гидроудара в длинном водоводе описывается 
одномерной моделью упругого гидроудара, а в области турбины -- в <<жестком>> приближении, но в 
аккуратной пространственной геометрической и гидродинамической постановке.

Следующая проблема, рассмотренная в диссертации, связана с неизбежными зазорами между вращающимися 
(рабочее колесо) и неподвижными частями ГТ (рисунок~\ref{fig:01}), приводящими к протечкам жидкости и 
потерям энергии. Для снижения объемных потерь в зазорах устанавливаются уплотнения. 
Кольца уплотнений крепятся на верхнем и нижнем ободах или на камере РК. Ступица 
РК имеет разгрузочные отверстия, расположенные за выходными кромками лопастей. Через них полость
над РК сообщается с основным проточным трактом.

\begin{figure}[ht!]
  \label{fig:01}
  \centering
  \includegraphics[width=14.0cm]{fig_RK.png}%  \\
  \caption{Рабочее колесо с зазорами, полостями и разгрузочными отверстиями}
\end{figure}

Кроме уменьшения потерь мощности в турбине за счет ограничения величин протечек воды между вращающимися и 
неподвижными частями, уплотнения обеспечивают также снижение осевой и радиальной нагрузок (ОРН), 
действующих на РК~\cite{granovsky}. В настоящее время для предсказания ОРН и потерь мощности в уплотнениях 
и полостях в основном используются приближённые инженерно-эмпирические методики, в которых грубо
представлена сложная трехмерная конфигурация зазоров и не учитываются особенности трёхмерного 
нестационарного потока жидкости в гидротурбине. 
Поскольку величины зазоров и характерные размеры областей протечек на три порядка меньше характерных
размеров РК, совместный расчет трехмерного течения во всей гидротурбине в полной постановке с учетом зазоров 
и областей протечек потребует большого объема вычислений и как следствие, приведет к абсолютно 
нереалистичным временам счета. Поэтому в диссертации разработан комбинированный метод 
моделирования трехмерных течений в полостях и уплотнениях, сопряженный с расчетом в основном проточном тракте 
гидротурбины посредством инженерно-эмпирической методики. 

Третьей задачей, поставленной и решенной в диссертационной работе, было создание численной модели течения в
гидротурбине с кольцевым затвором. Для обеспечения надежности работы гидроэнергетического оборудования 
высоконапорных ГЭС, а также для повышения эффективности работы гидроаккумулирующих электростанций, 
работающих при больших циклических нагрузках, вместо предтурбинных дисковых или шаровых затворов устанавливают 
кольцевые затворы. 
Затвор встроен в проточную часть между колоннами статора и направляющим аппаратом гидротурбины. 
Он предназначен для работы в оперативном и аварийном режимах и приводится в действие сервомоторами, 
расположенными на крышке турбины.

Одна из серьезных проблем, возникающих при эксплуатации кольцевого затвора -- его заклинивание. 
Поэтому при его проектировании важным является учет осевых $F_z$ и радиальных $(F_x, F_y)$ сил, 
действующих на затвор со стороны нестационарного потока воды. 
От их величин зависит расчет мощности сервомоторов и принятие мер по 
предотвращению заклинивания затвора~\cite{shevch}. Важным является также расчет зависимости расхода от степени 
закрытия затвора. От него, в частности, зависит повышение давления в спиральной камере вследствие гидроудара. 
При этом известно, что кольцевой затвор, в отличие от классических дисковых и шаровых 
затворов, обладает рабочим процессом пониженной нестационарности.

Таким образом, несмотря на многолетние исследования, в моделировании течений в ПТ ГЭС указанные проблемы 
остаются недостаточно изученными. Не выяснено влияние гидроудара и изменения скорости вращения рабочего 
колеса на структуру потока. Недостаточно исследованы ОРН на РК и потери энергии, 
возникающие из-за протечек воды через зазоры между вращающимися и неподвижными частями гидротурбины, 
а также динамические нагрузки на элементы гидротурбины и характеристики нестационарного потока при 
наличии кольцевого затвора. 

Существующие численные трехмерные модели гидродинамики турбин не позволяют одновременно учитывать 
движение лопаток направляющего аппарата, изменение скорости вращения рабочего колеса и расхода жидкости, 
явление гидроудара. Также отсутствуют развитые трехмерные модели течений жидкости в зазорах и при наличии 
кольцевого затвора. В связи с этим разработка новых математических моделей течений в ПТ ГЭС, 
учитывающих такие особенности, актуальна как в научном, так и в 
практическом плане. Решение этой задачи позволит сформировать более полные представления 
о структуре исследуемого потока и возможностях того или иного метода изменения характера течения, 
повышения коэффициента полезного действия или улучшения прочностных качеств гидротурбины.

\textbf{Цель исследования}~--- создание новых численных моделей для адекватного описания 
течений при переходных режимах работы ГТ, в уплотнениях и полостях между 
вращающимися и неподвижными частями, при наличии нестандартных элементов (кольцевой затвор).

\textbf{Объектом исследований} являются методы моделирования течений в проточных частях гидротурбин при 
переходных режимах работы, в полостях и уплотнениях, при наличии кольцевого затвора.

\textbf{Предметом исследований} являются нестационарные процессы и особенности течений в гидротурбинах и 
методов их моделирования в новых постановках.

\textbf{Основные задачи, решенные в ходе достижения поставленной цели.}
\begin{enumerate}
  \setlength{\itemsep}{1pt} \setlength{\parskip}{0pt} \setlength{\parsep}{0pt}
  \item[1.] Обобщен на подвижные сетки метод решения трехмерных уравнений Рейнольдса движения несжимаемой 
            жидкости, замкнутых $k-\varepsilon$ моделью турбулентности.
  \item[2.] Разработан метод совместного решения \\
            --- трехмерных уравнений Рейнольдса на подвижных сетках, \\
            --- уравнения вращения рабочего колеса, \\
            --- одномерных уравнений упругого гидроудара.
  \item[3.] Предложена новая постановка для численного моделирования течений в переходных
            режимах работы гидротурбин.
  \item[4.] Разработан метод моделирования течений в полостях и уплотнениях, сопряженный с методом 
            моделирования течения в основном проточном тракте гидротурбины.
  \item[5.] Создан метод расчета течения при наличии цилиндрического кольцевого затвора.
  \item[6.] Создан программный комплекс, реализующий построенные численные алгоритмы на многопроцессорных 
            вычислительных системах.
  \item[7.] Проведены верификация и валидация разработанных численных моделей.
  \item[8.] В новых постановках решены практически важные задачи определения динамических нагрузок 
            на элементы гидротурбины и давлений в нестационарном потоке.
\end{enumerate}

\textbf{На защиту выносятся} следующие результаты, соответствующие четырем пунктам паспорта 
специальности 05.13.18~--- математическое моделирование, численные методы и комплексы программ 
по физико"=математическим наукам.

\noindent
\textit{Пункт 1: Разработка новых математических методов моделирования объектов и явлений.}
\vspace{-2mm}
\begin{enumerate}
  \setlength{\itemsep}{1pt} \setlength{\parskip}{0pt} \setlength{\parsep}{0pt}
  \item[1.] Новый математический метод моделирования течений при переходных режимах работы 
            гидротурбин, объединяющий решение нестационарных трехмерных уравнений Рейнольдса на подвижных 
            сетках, уравнения вращения рабочего колеса и одномерных уравнений упругого гидроудара. 
            Новый метод моделирования течений воды через полости и уплотнения гидротурбин. 
\end{enumerate}

\noindent
\textit{Пункт 3: Разработка, обоснование и тестирование эффективных вычислительных методов с 
применением современных компьютерных технологий.}
\vspace{-2mm}
\begin{enumerate}
  \setlength{\itemsep}{1pt} \setlength{\parskip}{0pt} \setlength{\parsep}{0pt}
  \item[2.] Численный метод решения уравнений Рейнольдса на подвижных сетках и численный метод совместного 
            решения уравнений, описывающих переходные процессы в гидротурбинах, реализованные на 
            многопроцессорных вычислительных системах с применением технологии MPI.
\end{enumerate}

\noindent
\textit{Пункт 4: Реализация эффективных численных методов и алгоритмов в виде комплексов 
проблемно"=ориентированных программ для проведения вычислительного эксперимента.}
\vspace{-2mm}
\begin{enumerate}
  \setlength{\itemsep}{1pt} \setlength{\parskip}{0pt} \setlength{\parsep}{0pt}
  \item[3.] Программные комплексы CADRUN/2013 и CADRUN2/2013, 
            предназначенные для расчетов нестационарных трехмерных 
            турбулентных течений в гидротурбинах, созданные на основе предложенных методов и используемые для 
            проведения вычислительных экспериментов в ОАО~<<Силовые машины>> <<ЛМЗ>> в г.~Санкт-Петербурге.
\end{enumerate}

\noindent
\textit{Пункт 5: Комплексные исследования научных и технических проблем с применением современной 
технологии математического моделирования и вычислительного эксперимента.}
\vspace{-2mm}
\begin{enumerate}
  \setlength{\itemsep}{1pt} \setlength{\parskip}{0pt} \setlength{\parsep}{0pt}
  \item[4.] Численные исследования течений в переходных режимах: пуска в турбинный режим, регулировки 
            мощности и мгновенного сброса нагрузки. Основные закономерности формирования и 
            развития структуры нестационарных турбулентных течений несжимаемой вязкой среды при 
            обтекании подвижных лопаток НА и лопастей РК в переходных режимах работы радиально-осевых 
            гидротурбин. 

            С использованием нового комбинированного метода: 
            
            $\bullet$ обнаружено влияние высоты верхней области протечки на течение в разгрузочном 
            отверстии --- при высоте менее 10 мм в разгрузочном отверстии 
            формируется закрученное вихревое течение, гидравлическое сопротивление при этом увеличивается 
            в 4 раза; 
            
            $\bullet$ найдено соотношение радиальных сил --- при относительном эксцентриситете 
            рабочего колеса $\varepsilon > 0.6$ радиальные силы, действующие на лабиринтные уплотнения, 
            дают более 50~\% общей радиальной нагрузки, действующей на всё рабочее колесо гидротурбины. 
\end{enumerate}

\noindent
Таким образом, в соответствии с формулой специальности 05.13.18 в диссертации представлены 
оригинальные результаты одновременно из трех областей: математического моделирования, 
численных методов и комплексов программ.

\textbf{Научная новизна} выносимых на защиту результатов заключается в следующем. 

Впервые рассмотрена модель переходных процессов в проточной части ГЭС, одновременно 
учитывающая нестационарное трехмерное турбулентное течение 
несжимаемой жидкости в гидротурбине, движение лопаток НА, изменение скорости 
вращения РК и явление гидравлического удара в водоводе. 

Предложен новый численный метод совместного решения нелинейных систем уравнений, образующих
модель переходных процессов.

На основе разработанных эффективных численных методов создан оригинальный программный комплекс, 
допускающий использование в расчетах современных многопроцессорных систем.

Новым является разработанный комбинированный метод моделирования течений воды через 
полости и уплотнения гидротурбин, позволяющий определять величины протечек и осевых и радиальных нагрузок на 
рабочее колесо. 

\textbf{Достоверность и обоснованность} полученных результатов обеспечивается использованием в 
качестве основы моделирования фундаментальных законов механики жидкости и газа, динамики твердых тел и выбором 
теоретически обоснованных численных методов, а также подтверждается хорошим согласованием полученных 
численных результатов с экспериментальными данными и расчетными данными других исследователей.

\textbf{Практическая ценность} исследования заключается в возможности использования 
полученных результатов в ряде прикладных областей наукоемкого машиностроения и электроэнергетики для 
моделирования течений жидкости в проточных частях различных типов гидротурбин или других аэрогидродинамических 
установках (программные комплексы CADRUN/2013 и CADRUN2/2013, зарегистрированные в 
Роспатенте 25 января 2013 г., рег.~\No~2013611576 и \No~2013611580). 
Результаты диссертационной работы используются в исследованиях в ОАО~<<Силовые машины>> <<ЛМЗ>> 
в г.~Санкт-Петербурге, что подтверждает приложенный в конце диссертации акт об использовании научных 
результатов в практической деятельности.

Основные положения и результаты диссертации \textbf{докладывались и обсуждались} 
на объединенном научном семинаре ИВТ~СО~РАН <<Информационно"=вычислительные технологии 
(численные методы механики сплошной среды)>> под руководством академика~РАН Ю.\,И.~Шокина и 
профессора В.\,М.~Ковени, на объединенном научном семинаре ИВМиМГ и ССКЦ~СО~РАН <<Архитектура, системное и 
прикладное программное обеспечение кластерных суперкопьютеров>> под руководством профессора Б.\,М.~Глинского, 
на научном семинаре института теплофизики, а также на 11 всероссийских и международных конференциях:
\begin{enumerate}
  \item[1.] III~Международная конференция <<High Performance Computing 2013>> 
            (Киев, Украина, октябрь 2013).
  \item[2.] Международная конференция <<Вычислительные и информационные технологии в науке, 
            технике и образовании ВИТ-2013>> (Усть-Каменогорск, Казахстан, сентябрь 2013).
  \item[3.] XIII~Всероссийская конференция молодых ученых по математическому моделированию и информационным 
            технологиям (Новосибирск, октябрь 2012).
  \item[4.] XI~Всероссийская конференция «Краевые задачи и математическое моделирование» 
            (Новокузнецк, октябрь 2012).
  \item[5.] XXVI~Международная конференция <<IAHR Symposium on hydraulic machinery and systems>>
            (Пекин, Китай, август 2012).
  \item[6.] Международная конференция <<Современные проблемы прикладной математики и механики: теория, 
            эксперимент и практика>>, посвященная 90-летию Н.\,Н.~Яненко (Новоcибирск, июнь 2011).
  \item[7.] XVII~Международная конференция по вычислительной механике и современным прикладным 
            программным системам <<ВМСППС’2011>> (Алушта, май 2011).
  \item[8.] XXV~Международная конференция <<IAHR Symposium on hydraulic machinery and systems>> (Тимишоара, 
            Румыния, сентябрь 2010).
  \item[9.] VIII~Международная конференция по неравновесным процессам в соплах и струях <<NPNJ`10>> (Алушта, 
            май 2010).
  \item[10.]XLVII~Международная научная студенческая конференция <<Студент и научно-технический прогресс>>:
            Математика (Новосибирск, апрель 2009).
  \item[11.]XLVI~Международная научная студенческая конференция <<Студент и научно-технический прогресс>>:
            Математика (Новосибирск, апрель 2008).
\end{enumerate} 

Основные результаты диссертации \textbf{опубликованы} в 19 научных 
работах~\cite{my1,my1_2,my2,my3,my4,my5,my6,my7,my8,my9,my10,my11,my12,my13,my14,my15,my16,my17,my18}~(в 
скобках в числителе указан общий объем этого типа публикаций в печатных листах, в знаменателе~--- объем 
принадлежащий лично автору), в том числе  4~статьи~\cite{my1,my1_2,my2,my3}~(9.02/5.4) в 
периодических изданиях, рекомендованных ВАК для представления основных научных результатов диссертаций на 
соискание ученой степени доктора или кандидата наук, 8 публикаций в трудах международных и всероссийских
конференций~\cite{my4,my5,my6,my7,my8,my9,my10,my11}~(5.47/2.9), 2~свидетельства государственной 
регистрации программ для ЭВМ в Роспатенте~\cite{my12,my13}, 5~публикаций в тезисах международных и 
всероссийских конференций~\cite{my14,my15,my16,my17,my18}~(0.14/0.1).

Работа выполнена при финансовой поддержке Российского фонда фундаментальных исследований 
(грант \No 11-01-00475-а), Президиума СО РАН (Интеграционный проект СО РАН \No 130).

\textbf{Личный вклад соискателя.} Основные результаты диссертационной работы получены автором 
самостоятельно. Во всех работах~\cite{my1,my1_2,my2,my3,my4,my5,my6,my8,my9,my10,my11,my12,my13,my14,my16}, 
опубликованных в соавторстве, постановка задач выполнена совместно;
соискатель самостоятельно обобщил на подвижные сетки и реализовал в виде комплекса программ 
численный метод решения трехмерных уравнений Рейнольдса движения несжимаемой жидкости, 
разработал метод совместного решения уравнений Рейнольдса, уравнения вращения рабочего колеса 
и одномерных уравнений упругого гидроудара. Также автор 
принимал непосредственное участие в разработке комбинированного метода моделирования течения в полостях и 
уплотнениях, сопряженного с методом моделирования в основном проточном тракте гидротурбины, и в создании 
метода расчета течения при наличии цилиндрического кольцевого затвора. 
Адаптация численных алгоритмов для работы на многопроцессорных системах, их верификация и валидация, анализ 
полученных результатов выполнены автором самостоятельно. 

\textbf{Структура и объем диссертации.}
Диссертация состоит из введения, четырех глав, заключения, списка цитируемой литературы 
и двух приложений. Диссертация изложена на 145 страницах машинописного 
текста, включая 63 иллюстрации и 10 таблиц. Список цитируемой литературы 
содержит 88 наименований.

\section*{Содержание диссертации}

Во \textbf{Введении} обоснована актуальность выбранной темы исследований, сформулированы
цель работы и задачи, которые необходимо решить для ее достижения. 

Отмечены работы С.\,Г.~Черного, 
Е.\,М.~Смирнова, В.\,В.~Риса, И.\,М.~Пылева, А.\,В.~Захарова, A.~Ruprecht, T.\,C.~Vu, S.~Kurosawa и др., 
посвященные численному  моделированию установившихся течений в гидротурбинах.  
Проводится обзор подходов к построению одномерных моделей неустановившегося движения рабочей жидкости в 
напорных трактах, представленных в работах Г.\,И.~Кривченко, В.\,Б.~Курзина, С.\,В.~Алексеенко, 
П.\,А.~Куйбина, В.\,Л.~Окулова, Д.\,Н.~Горелова и др.

Приведены основные результаты и положения выносимые на защиту. 
Также во введении представлены сведения о научной новизне, практической значимости, 
апробации результатов и основных публикациях. Затем кратко изложено содержание диссертации по главам.

В \textbf{Главе~1} строится численный метод расчета трехмерных турбулентных течений несжимаемой 
жидкости на подвижных структурированных сетках.

В разделе~\ref{s:11} приведены нестационарные трехмерные уравнения Рейнольдса, записанные как в 
дифференциальной форме, так и в форме интегральных законов сохранения массы и количества движения для 
подвижного объема. Далее формулируется условие геометрической консервативности (УГК), являющееся одним из 
главных при дискретизации уравнений, записанных в криволинейных координатах, зависящих от времени. Если 
при дискретизации уравнений Рейнольдса не удовлетворить УГК, 
то на задаче с решением в виде постоянного однородного потока численный метод на подвижной сетке даст 
возмущенное, отличное от начальных данных решение.

В разделе~\ref{s:12} предлагается метод решения уравнений Рейнольдса на подвижных сетках, основанный на 
концепции искусственной сжимаемости, неявной конечно-объемной аппроксимации и приближенной $LU$-факторизации
линеаризованной системы разностных уравнений.  
Параллельно с проводимой дискретизацией уравнений рассматриваются известные подходы 
построения численных методов на подвижных сетках,  представленные в работах J.\,G.~Trulio, 
P.\,D.~Thomas, I.~Demirdzic, M.~Peric, W.~Shyy, M.~Lesoinne, C.~Farhat, B.~Koobus, C.~Forster, 
M.~Engel, M.~Griebel, Н.\,О.~Зайцева, Н.\,А.~Щура, К.\,Н.~Волкова и др.
Отмечается, что предложенный в диссертации метод, 
в отличие от имеющихся в литературе, приводит к точному выполнению УГК 
в трехмерном случае на дискретном уровне. Это достигается аккуратным разбиением всей расчетной области на 
каждом шаге по времени на непересекающиеся тетраэдры и аппроксимацией скоростей движения граней 
ячеек, специально согласованным с разбиениями способом.

В разделе~\ref{s:13} обосновывается постановка краевых условий на подвижной твердой границе. 
В разделе~\ref{s:14} представлены уравнения стандартной $k - \varepsilon $ модели турбулентности и их 
дискретизация на подвижных сетках. 

Далее в разделе~\ref{s:15} приводятся результаты решения двух модельных задач с подвижными границами: 
расчет однородного потока на подвижной сетке и движение кругового цилиндра в покоящемся однородном поле 
несжимаемой вязкой жидкости. Движение цилиндра задано посредством перемещения со временем 
его границы и нормально связанной с ней системы координат. 
Анализ результатов показал, что метод корректно 
воспроизводит все характеристики и может быть использован для моделирования течений
в областях с подвижными границами в различных задачах вычислительной гидродинамики.

В разделе~\ref{s:16} строится алгоритм параллельной реализации разработанного численного метода
на многопроцессорных вычислительных системах. 
Обосновывается использование геометрического распараллеливания, 
заключающегося в декомпозиции всей расчетной области на блоки, каждый из которых рассчитывается на отдельном 
ядре многопроцессорной вычислительной системы. Коммуникации между процессорами осуществляются с использованием 
стандарта MPI. Приводятся результаты распараллеливания расчёта в полной постановке, 
полученные на кластерах ИВЦ НГУ и ССКЦ. Из проведенных исследований 
следует, что при использовании 61 счетного ядра время решения задачи моделирования нестационарного 
трехмерного турбулентного течения во всей гидротурбине на сетке с общим количеством ячеек 
около 2.5~млн сокращается более чем в 15~раз и составляет от 1 до 2 дней, что вполне приемлемо 
для практического применения.

В \textbf{Главе~2} для расчета течений в переходном режиме работы ГЭС 
предлагается оригинальная гибридная численная модель, объединяющая решение нестационарных 
уравнений Рейнольдса на подвижных сетках, уравнения вращения РК 
как твёрдого тела и одномерных уравнений распространения упругого гидроудара в водоводе. 

В разделе~\ref{s:20} описаны проблемы, возникающие при моделировании переходных процессов в гидротурбинах.  

В разделе~\ref{s:21} приведена система уравнений гибридной модели переходных процессов, отмечены связывающие
их переменные. 

В разделе~\ref{s:22} рассмотрены краевые условия при совместном расчете в новой постановке 
водовод-гидротурбина. Особое внимание уделено взаимообмену параметрами течения между одномерной областью 
водовода и трехмерной областью гидротурбины.

В разделе~\ref{s:23} представлена численная реализация разработанной модели. 
Уравнения каждой из подмоделей решаются эффективными устойчивыми численными методами 
и замыкаются корректными краевыми условиями.  
В результате разработан надежный инструмент моделирования переходных трехмерных течений в гидротурбинах, 
позволяющий проводить вычисления за 1-2 дня на сетках, содержащих около 2.5 млн ячеек.

В разделе~\ref{s:24} приводятся результаты расчетов переходных процессов: пуска в турбинный режим, уменьшения 
мощности, мгновенного сброса нагрузки. Показано хорошее согласование с имеющимися экспериментальными данными 
по частоте вращения рабочего колеса, величине возникающего гидроудара и давлению в точках проточного тракта.
Применение разработанной гибридной модели позволило проанализировать структуру потока в ПТ ГТ 
при переходных режимах работы.

\textbf{Глава~3} посвящена построению метода моделирования течений в полостях и уплотнениях, сопряженному с 
методом моделирования течения в основном проточном тракте гидротурбины. 

В разделе~\ref{s:31} дана общая постановка задачи. В разделе ~\ref{s:32} приведен обзор существующих методик 
определения осевых и радиальных нагрузок на РК гидротурбины, вызванных нестационарным течением рабочей 
жидкости в его межлопастных каналах, а также протечками в лабиринтных уплотнениях, полостях и разгрузочных 
отверстиях. Отмечено, что для простых геометрий приближенные инженерно-эмпирические
методики дают приемлемые результаты при минимальных вычислительных затратах.

В разделе~\ref{s:34} строится комбинированный метод определения нестационарных осевых и радиальных 
нагрузок на рабочее колесо, 
основанный на расчетах трехмерного течения несжимаемой жидкости в основной проточной части и областях протечек 
гидротурбины. Метод позволяет рассчитывать осевые и радиальные нагрузки, вызванные следующими факторами: 
неравномерностью потока в спиральной камере и статоре, ротор-статор взаимодействием, неравномерностью потока 
за РК вследствие нестационарности потока в конусе ОТ (вихревой жгут и т.д.), несоосностью статора и ротора, 
изгибом вала ротора. Кроме того, метод позволяет рассчитывать расход жидкости через зазоры между вращающимися
и неподвижными частями гидротурбины. Потери давления в лабиринтных уплотнениях, щелях и разгрузочных 
отверстиях вычисляются непосредственно с использованием турбулентной модели, тем самым не требуется 
эмпирическая информация о гидравлических сопротивлениях этих элементов.

В разделе~\ref{s:35} усовершенствована инженерно-эмпирическая методика~\cite{lomakin} для расчета радиальных
нагрузок, действующих на лабиринтное уплотнение вследствие смещения оси вращения ротора. В модификации 
учтены влияние вращения ротора на коэффициент сопротивления узкой части лабиринта, сопротивление 
ячеек расширения, зависимость коэффициента сопротивления узкой части и ячеек расширения от переменного зазора 
между статором и ротором. Это позволило получить количественное соответствие результатов 
расчетов по модифицированной методике с расчетом по трехмерной модели течения жидкости в ЛУ.

В разделе~\ref{s:36} представлены результаты расчетов осевых и радиальных нагрузок на РК по предложенной 
методике. Показано, что при относительном эксцентриситете $\varepsilon < 0.5$ прецессия ротора не оказывает 
заметного влияния на модуль радиальной силы. Влияние прецессии растет по мере увеличения эксцентриситета. 
Так, при $\varepsilon = 0.9$ в случае $\Omega = \omega$ радиальная сила на 30\% больше, чем в случае простой 
несоосности ротора и статора. Показано, что при $\varepsilon > 0.6$ радиальные силы, действующие на 
лабиринтные уплотнения, дают более 50\% итоговой радиальной нагрузки, действующей на все рабочее колесо 
гидротурбины.

В \textbf{Главе~4} строится численная модель течения воды в гидротурбине с затвором. 

В разделе~\ref{s:41} приведена постановка задачи. Выделена особенность геометрии 
спиральной камеры -- <<зуб>>, приводящая к сильной окружной неравномерности потока. 
Затвор, перекрывающий ПТ ГТ, задается как твердая стенка в выходном сечении статора (модель тонкого затвора). 
В этом случае не учитывается форма оголовка затвора и область течения под ним. Другой подход задания 
затвора заключается в добавлении нового элемента ПТ --- кольцевой области между оголовком затвора 
и нижней крышкой НА. В таком случае область течения под затвором моделируется точно. 

В разделе~\ref{s:42} представлены численные реализации обеих моделей кольцевого затвора: 
бесконечно тонкого и затвора реальной толщины. Для реализации второй модели в программный комплекс 
добавлены процедуры построения сеток в новой области под затвором и передачи данных из нее 
в каналы статора и НА.

В разделе~\ref{s:43} описаны входные и выходные условия. Неизвестная априори зависимость расхода от 
степени закрытия затвора рассчитывается в экономичной циклической постановке, в которой расчетная область 
состоит из одного канала статора, одного межлопаточного и одного межлопастного канала. На входе и выходе из 
расчетной области фиксировались полные энергии потока. В расчетах в полной постановке, содержащей все каналы 
элементов ГТ, задавался расход, полученный в циклической постановке.

В разделе~\ref{s:44} приведены зависимости выталкивающей силы $F_z$ и расхода от степени закрытия затвора $s$.
Отмечено, что при $s \geqslant 0.9$ в модели затвора реальной толщины выталкивающая сила становится 
отрицательной, т.е. затвор втягивается в проточную часть.

В разделе~\ref{s:45} приведены результаты расчета крутящих моментов лопаток НА 
и радиальных сил $(F_x,\ F_y)$, действующих на затвор. 
Показано, что амплитуды пульсаций сил резко возрастают, при $s \geqslant 0.7$. 
Обнаружено сильное влияние окружной неравномерности потока, вызванной <<зубом>> спиральной камеры, на моменты 
лопаток НА, расположенных в окрестности <<зуба>>, при небольших степенях закрытия 
затвора. При $s \geqslant 0.7$ влияние <<зуба>> на моменты исчезает.

В \textbf{заключении} сформулированы основные результаты диссертационной работы.

В \textbf{приложении А} приведены матрицы Якоби невязкого и вязкого потоков, матрицы правых  
собственных векторов, а также $RDL$-разложение матрицы Якоби невязкого потока. В \textbf{приложении В} 
представлены два свидетельства о государственной регистрации программ для ЭВМ и акт об использовании научных 
результатов диссертационной работы в филиале ОАО <<Силовые машины>> <<ЛМЗ>> в г.~Санкт-Петербурге.  

Автор выражает глубокую благодарность и признательность своему научному руководителю Черному
Сергею Григорьевичу за постановки интересных задач, 
всестороннюю поддержку и постоянное внимание в ходе выполнения работы. 
Успешному выполнению работы способствовали неоценимая 
помощь Чиркова Дениса Владимировича, ценные критические замечания Банникова Дениса Викторовича, 
Ешкуновой Ирины Федоровны и Лапина Василия Николаевича. 
Автор отдельно благодарит Есипова Дениса Викторовича, Астракову Анну Сергеевну и Горобчука Алексея 
Геннадьевича за создание творческой атмосферы и возможность постоянного обсуждения научных задач. 
В заключение хочется поблагодарить весь коллектив Института вычислительных технологий СО РАН за поддержку
и благоприятную рабочую обстановку.
