%!TEX root = dissertation.tex
\chapter*{Заключение}
\addcontentsline{toc}{chapter}{Заключение}

Основные научные и практические результаты проведенных исследований заключаются в следующем.

\begin{enumerate}
  \item[1.] Обобщен на задачи с подвижными границами метод расчета течений несжимаемой жидкости в 
            неподвижных областях \cite{Cher}. Предложен подход, обеспечивающий точное выполнение 
            условия геометрической консервативности на дискретном уровне. Построенный метод применен 
            для решения тестовых и практических задач. Анализ результатов показал, что метод 
            корректно воспроизводит все рассмотренные характеристики и может быть использован для 
            моделирования течений в областях с~подвижными границами в различных задачах 
            вычислительной гидродинамики. 
  \item[2.] Разработана модель нестационарного трехмерного потока в 
            переходных режимах работы гидротурбин. Построенная модель учитывает явление гидроудара, 
            переменную частоту вращения РК и меняющийся расход воды, проходящей через водовод и гидротурбину. 
            Представлены результаты расчетов процессов пуска в турбинный режим, уменьшения мощности и 
            мгновенного сброса нагрузки. Исследованы структуры и особенности трехмерных течений при переходных
            режимах работы ГТ: развитие прецессирующего вихревого жгута за РК при уменьшении мощности; срыв 
            вихря с входной кромки лопасти РК и насосный вихрь, возникающие при пуске турбины. 
  \item[3.] Разработан комбинированный метод определения нестационарных осевых и радиальных 
            нагрузок (ОРН) на рабочее колесо, основанный на расчетах трехмерного течения 
            несжимаемой жидкости в основной проточной части и в 
            областях протечек гидротурбины, позволяющий рассчитывать осевые и радиальные нагрузки, 
            вызванные следующими факторами: неравномерностью потока в спиральной камере и статоре, 
            ротор-статор взаимодействием, влиянием вверх по потоку нестационарности 
            в конусе ОТ (вихревой жгут), несоосностью статора и ротора, изгибом вала ротора. Кроме 
            того, метод позволяет рассчитывать расход жидкости через зазоры между вращающимися и 
            неподвижными частями гидротурбины. Потери давления в лабиринтных уплотнениях, щелях и разгрузочных 
            отверстиях вычисляются непосредственно с использованием турбулентной модели, тем самым не 
            требуется эмпирическая информация о гидравлических сопротивлениях этих элементов. 
            С использованием нового метода обнаружено влияние высоты верхней области протечки на течение в 
            разгрузочном отверстии: при высоте менее 10 мм в разгрузочном отверстии формируется закрученное 
            вихревое течение, гидравлическое сопротивление при этом увеличивается в 4 раза. 
  \item[4.] Усовершенствована инженерно-эмпирическая методика~\cite{lomakin} для расчета радиальных нагрузок, 
            действующих на лабиринтное уплотнение вследствие смещения оси вращения ротора. В предложенной 
            методике учтены влияние вращения ротора на коэффициент сопротивления узкой части лабиринта, 
            сопротивление ячеек расширения, зависимость коэффициента сопротивления узкой части и ячеек 
            расширения от переменного зазора между статором и ротором. Усовершенствованная методика применима 
            для расчета радиальных нагрузок в случае несоосности статора и ротора и отсутствия 
            прецесии последнего. Результаты расчетов радиальной силы по предложенной методике хорошо совпадают 
            с данными трехмерных расчетов течения во всем ЛУ для диапазона рассмотренных значений 
            относительного эксцентриситета $\varepsilon$ от 0 до 1, что позволяет рекомендовать ее 
            для оценки возникающих радиальных сил, действующих на РК, при проектировании геометрии 
            областей протечек. Показано, что при относительном эксцентриситете $\varepsilon>0.6$ радиальные 
            силы, действующие на лабиринтные уплотнения, дают более 50\,\% итоговой радиальной нагрузки, 
            действующей на все рабочее колесо гидротурбины. 
  \item[5.] Создан метод расчета течения при наличии цилиндрического кольцевого затвора. С его помощью решены 
            практически важные задачи определения динамических нагрузок на элементы гидротурбины и давлений в 
            нестационарном потоке. Проведено сопоставление зависимостей сил, действующих на затвор, крутящих 
            моментов лопаток, полученных в полной и циклической постановках с затворами реальной толщины и 
            бесконечно тонкими. Установлены интервалы наиболее достоверных значений этих параметров при 
            различных положениях затвора. Показано, что амплитуды пульсаций резко возрастают, начиная от 
            степени закрытия 0.7 и большей. 
  \item[6.] Проведены верификация и валидация разработанных методов. Создан программный комплекс, реализующий 
            построенные численные алгоритмы на многопроцессорных вычислительных системах. Результаты 
            расчетов, выполненных с помощью разработанного программного комплекса, используются в проектных 
            исследованиях филиала ОАО <<Силовые машины>> <<ЛМЗ>> в г.~Санкт-Петербурге.
\end{enumerate}
%%%%%%%%%%%%%%%%%%%%%%%%%%%%%%%%%%%%%%%%%%%%%%%%%%%%%%%%%%%%%%%%%%%%%%%%%%%%%%%%%%%%%%%%%%%%%%%%%%%%

%
%
