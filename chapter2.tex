%!TEX root = dissertation.tex
% \pagestyle{plain}
\chapter*{Глава 2. Численная модель переходных процессов в гидротурбинах}
\label{s:2}
\setcounter{chapter}{2}
\addcontentsline{toc}{chapter}{Глава~\thechapter~ Численная модель переходных процессов в гидротурбинах}
\setcounter{section}{0}

\section{Проблемы, возникающие при моделировании переходных процессов в гидротурбинах}
\label{s:20}
Течения в проточном тракте (ПТ) гидротурбины (ГТ) подразделяются на установившиеся и переходные.
К установившимся относятся течения при неизменных значениях расхода воды $Q$ через ПТ, частоты
вращения рабочего колеса, нагрузки на вал РК. Эти течения являются стационарными или
периодически нестационарными и достаточно адекватно моделируются исследователями в полной
трехмерной постановке \cite{Cher,Rupr,Vunen1,Vunen2,Vu}.

Переходные режимы течений возникают при переводе работы ГТ из одного состояние в другое путем регулировки 
открытия направляющего аппарата или при увеличении (уменьшении) нагрузки на вал РК. Такие течения
являются существенно нестационарными и характеризуются резкими колебаниями расхода воды во времени. Это 
приводит к динамическому изменению давления в ПТ, проявляющемуся в форме гидравлического удара 
$\Delta H({\bf x}, t)$~\cite{jukovskii}. Гидравлический удар $\Delta H({\bf x}, t)$ может как повышать
общий напор на турбине
\begin{equation}
H({\bf x}, t)=H_0+\Delta H({\bf x}, t)
\end{equation}
в случае, если он положительный, так и понижать, когда он отрицательный. Здесь $H_0$~–-~разность полных 
энергий потоков между уровнями верхнего и нижнего бьефов гидроэлектростанции.
Следовательно, при расчете переходных режимов трехмерных течений возникают новые проблемы при создании для них 
численных моделей, не имевшиеся в случае моделирования установившихся течений. 

Первая проблема -- описание величины гидравлического удара $\Delta H({\bf x}, t)$. 
Явление гидравлического удара можно было бы
непосредственно моделировать, если учитывать упругие деформации аэрированной воды и стенок ПТ. При этом 
пришлось бы рассчитывать течение во всем ПТ ГТ, включая напорный водовод. Это привело бы к колоссальным 
вычислительным затратам и поэтому в настоящее время недостижимо. Если отказаться от учета деформации стенок 
ПТ, а моделирование течения воды проводить в приближении несжимаемой жидкости, то, рассчитав поток во всем ПТ, 
можно было бы получить картину <<жесткого>> гидравлического удара. Однако, найденное в такой постановке 
решение в случае длинного водовода и короткого времени изменения режима будет сильно отличаться от реального
переходного процесса в силу неограниченного возрастания величины гидравлического удара $\Delta H({\bf x}, t)$.

Вторая проблема -- использование на входной и выходной границах ПТ краевых условий, не фиксирующих расход воды 
$Q$ через них. Численное моделирование течений в ГТ осуществляется, как правило, не во всем ПТ, а в основных 
его элементах -- НА, РК и отсасывающей трубе (ОТ) \cite{Cher}. Одной из наиболее распространенных постановок 
условий во входном и выходном сечениях ПТ указанной цепочки элементов является задание распределения вектора 
скорости на входе и давления на выходе. Однако из-за изменения во времени расхода воды эта постановка при 
моделировании переходных процессов становится неприемлемой. Предложенная в \cite{bannikov} альтернативная 
постановка позволяет находить расход в процессе расчета течения. В этой постановке в обоих сечениях задаются 
полные энергии, кроме того, во входном сечении требуется информация о направлении потока, а в
выходном -- о профиле давления. Задаваемые в сечениях энергии могут меняться во времени. Такая постановка 
больше отвечает задачам о переходных процессах и будет использована при их решении.

При моделировании переходных течений в каналах НА с открывающимися или закрывающимися лопатками форма ячеек 
сетки не является фиксированной и подстраивается под положение поверхностей лопаток, являющихся границами 
расчетной области. Это приводит к необходимости использования подвижных, меняющихся со временем сеток. 
Распределение узлов сетки внутри расчетной области определяется распределением узлов на поверхностях лопаток. 
Движение поверхностей контролируется законом изменения регулирующего органа ГТ. Таким образом, третьей 
проблемой является обобщение разработанного ранее авторами численного метода решения трехмерных уравнений 
несжимаемой жидкости на фиксированных сетках на подвижные сетки. Особо важным требованием, которому
должен удовлетворять численный метод решения уравнений движения жидкости на подвижной сетке, является 
выполнение условия геометрической консервативности (УГК). Суть его состоит в том, что если решением 
рассматриваемой задачи является однородный поток, то метод с двигающейся сеткой должен выдавать этот же поток 
без какого-либо возмущения.

Наконец, четвертой проблемой, возникающей при обращении от моделирования установившихся течений к 
переходным, становится определение скорости вращения РК, которая в процессе переходного течения изменяется, 
подчиняясь закону вращения твердого тела под действием гидродинамического момента со стороны жидкости.

Изучение переходных процессов и правильный их учет при проектировании силового узла ГЭС представляют большой 
практический интерес. Задача анализа переходных процессов при изменении мощности состоит в том, чтобы найти 
оптимальные режимы регулирования, при которых изменение момента турбины будет происходить с наибольшей 
скоростью при соблюдении ограничений на возникающие динамические воздействия, в том числе от гидравлического 
удара.

\begin{figure}[h]
  \centering
  \includegraphics[width = 13cm]{VOD_GT_new.png}
  \caption{Схема проточной части ГЭС и геометрия турбины}
  \label{fig2:1}
\end{figure}

Предложенный гибридный подход включает в себя модель переходного процесса, 
состоящую из нестационарных усредненных по Рейнольдсу трехмерных уравнений Навье-Стокса, 
замкнутых $k-\varepsilon $ моделью турбулентности Кима-Чена, 
решаемых в областях с меняющимися во времени границами \cite{my1}; уравнения вращения РК как твердого целого и 
одномерных уравнений распространения упругого гидроудара в водоводе (рисунок~\ref{fig2:1}). 
Уравнения модели переходного процесса 
замыкаются новой постановкой краевых условий во входном и выходном сечениях ПТ ГТ, предложенной в 
\cite{bannikov}, и условиями сопряжения на границе водовод-гидротурбина.

Представлены результаты моделирования ряда переходных процессов в ГТ: пуска в турбинный режим, уменьшения 
мощности посредством закрытия лопаток НА, мгновенного сброса нагрузки. Проводится сравнение полученных 
результатов с экспериментальными данными.

\section{Основные уравнения гибридной модели}
\label{s:21}
\subsection{Уравнения Рейнольдса в форме интегральных законов сохранения для движущегося объема} Поскольку 
переходные процессы сопряжены с изменением границ расчетной области -- поворотом лопаток направляющего 
аппарата -- для моделирования пространственного течения несжимаемой жидкости используются
осредненные по Рейнольдсу уравнения Навье-Стокса, записанные в виде интегральных законов сохранения \eqref{6} 
для движущегося объема $V(t)$ \cite{my1}.

Величины $\nu_t$ и $k$ определяются по  $k-\varepsilon$ модели турбулентности Кима-Чена \cite{kimchen} с 
логарифмической пристеночной функцией вблизи твердых стенок.

\subsection{Уравнения $k-\varepsilon$ модели турбулентности Кима-Чена в интегральной форме для 
            движущегося объема} 
Каждое из уравнений $k-\varepsilon$ модели Кима-Чена \cite{kimchen} также может быть записано в виде
интегрального закона сохранения для движущегося объема $V(t)$
\begin{equation}
  \label{eq2:5} 
  \frac{\partial }{{\partial t}}\int\limits_{V\left( t \right)} {\varphi dV =  - } 
  \oint\limits_{\partial V\left( t \right)} {\varphi \left( {{\bf{w}} - {\bf{x}}_t } \right) 
  \cdot d{\bf{S}}} + \oint\limits_{\partial V\left( t \right)} {\left( {\nu  + \frac{{\nu _t }}
  {{\sigma _\varphi  }}} \right)\nabla \varphi \cdot d{\bf{S}}}  + \int\limits_{V\left( t \right)} 
  {H_\varphi dV},
\end{equation}
где $\varphi$ и $H_\varphi$ приведены в таблице~\ref{tab2:1}, 
${\bf{w}}=\left( {w_1 ,w_2 ,w_3 } \right)$, $\nu _t = C_\mu k^2/{\varepsilon }$.
\begin{table}[ht!]
  \label{tab2:1}
  \center
  \caption{Значения $\varphi$ и $H_\varphi$ в уравнении \eqref{eq2:5}}
  \begin{tabular}{|c|c|c|}
  \hline
   Уравнение & $\varphi$ & $H_\varphi$ \\
  \hline
  Для турбулентной  & $k$ & $G-\varepsilon$  \\
  кинетической энергии & &
  \\ \hline 
  Для скорости диссипации & $\varepsilon$ & 
  $C_{\varepsilon 1}\varepsilon G/k-C_{\varepsilon 2}\varepsilon^2/k + C_{\varepsilon 3} G^2 /{k}$ \\
  турб. кин. энергии & & \\ \hline 
  \end{tabular}
\end{table}

В таблице~\ref{tab2:1} $G = \tau _{ij} \dfrac{{\partial w_i }}{{\partial x_j }}$, значения констант
\begin{equation*}
C_\mu=0.09,\ C_{\varepsilon 1}=1.15,\ C_{\varepsilon 2}=1.9,\ C_{\varepsilon 3}=0.25,\ {{\sigma _k = 0.75}},
\ {{\sigma _\varepsilon = 1.15}}.
\end{equation*}
\subsection{Уравнение вращения рабочего колеса}
Большинство переходных процессов сопровождается изменением частоты вращения РК. Заранее эта зависимость не 
известна, поэтому одновременно с решением уравнений \eqref{6},\eqref{eq2:5} решается уравнение вращения РК 
как твердого целого
\begin{equation}
  \label{eq2:6} 
  I_z \frac{{d\omega }}{{dt}} = M_R (t) - M_{gen} (t) - sgn(\omega) M_{\text{тр}},
\end{equation}
где $I_z$ -- суммарный момент инерции РК и генератора, $M_R$ -- крутящий момент, обусловленный действием 
потока на РК, $M_{gen}$ -- момент полезной нагрузки, приложенный к валу электрогенератора,
$M_{\text{тр}}$ -- результирующий момент трения в электромеханической системе агрегата. Зависимость 
$M_{gen}(t)$ как правило, известна, в то время как $M_R(t)$ определяется гидродинамикой потока, т.е. находится 
из решения уравнений \eqref{6},\eqref{eq2:5}.

\subsection{Модель упругого гидравлического удара}
Для моделирования гидроакустических колебаний в напорном водоводе используется хорошо зарекомендовавшая себя 
одномерная модель упругого гидроудара \cite{jukovskii,krivch}. В случае, если трением жидкости о стенки
пренебрегается, эта модель может быть записана в виде системы дифференциальных уравнений
\begin{equation}
  \label{eq2:7}
  \left\{ \begin{aligned}
  \frac{{\partial m}}{{\partial t}} + \frac{{c^2 }}{{gS}}\frac{{\partial Q}}{{\partial \xi}} = 0 \\
  \frac{{\partial Q}}{{\partial t}} + gS\frac{{\partial m}}{{\partial \xi}} = 0 \\
  \end{aligned} \right.,\
  \xi \in [0,L],
\end{equation}
где $m(\xi,\ t) = \dfrac{p(\xi,\ t)}{{\rho g}} - z(\xi)$ -- потенциальный напор, $Q(\xi,\ t)$ -- расход 
жидкости, $S(\xi)$ -- площадь сечения водовода, $c(\xi)$ -- скорость распространения упругой волны удара, 
$L$ -- длина водовода. Скорость $c$ определяется концентрацией нерастворимого газа в воде и упругостью
стенок водовода \cite{krivch}. Для реальных водоводов $c= 1000\div 1450$ м/с.

\section{Краевые условия при совместном расчете течения в области водовод-гидротурбина}
\label{s:22}
\subsection{Входная и выходная границы} 
При расчете течений в гидротурбинах общеупотребительной постановкой краевых условий на входе и выходе области 
является постановка <<расход-давление>>, в которой во входном сечении задается расход жидкости и угол входа 
потока или распределение вектора скорости, а в выходном сечении -- распределение давления и касательные
составляющие скорости ${\bf w}\cdot\tau_i|_{\text{вых}},\ i=1,\,2$ ($\tau_1,\ \tau_2$ -- линейно 
независимые векторы, касательные к выходному сечению) \cite{AntKajMon}.

Однако в задачах моделирования переходных процессов расход жидкости меняется во времени и эта зависимость 
заранее не известна. В то же время полный напор, равный разности энергий на входе в водовод и на выходе из ОТ, 
остается неизменным даже в ходе переходного процесса. Поэтому в настоящей работе предлагается использовать 
краевые условия, которые позволяют по заданному напору $H_0$ определять расход жидкости одновременно с 
нахождением поля течения. Во входном сечении водовода $\xi=0$ (рисунок~\ref{fig2:1}) задается полная энергия потока
\begin{equation}
  \label{eq2:8}
  E_{\text{\it ВВ, вх}}\equiv m(0,t) + \frac{{Q^2 \left( {0,\,\,t} \right)}}{{2gS^2 }} = H_0.
\end{equation}
В выходном сечении ОТ задаются значение усредненной по расходу полной энергии
\begin{equation}
  \label{eq2:9}
  E_{\text{\it ОТ, вых}}\equiv \frac{1}{Q}\int\limits_{S_{\text{\it ОТ, вых}}}\left(\frac{p}
  {\rho g}-z+\frac{{\bf w}^2_{\text{\it ср}}}{2g}\right)\left({\bf w}d{\bf S}\right)=0,
\end{equation}
где ${\bf w}_{\text{\it ср}}=\frac{Q}{S_{\text{\it ОТ, вых}}} $, $S_{\text{\it ОТ, вых}}$ -- площадь выходного 
сечения ОТ, и условие на профиль статического давления
\begin{equation}
  p = p_0  + \rho g(z - z_0 ).
\end{equation}
Отметим, что значение $p_0$ не фиксируется априори, а определяется в процессе решения уравнений движения 
жидкости так, чтобы выполнялось равенство \eqref{eq2:9}.

Такая постановка входных и выходных условий отвечает физике переходного процесса. Фактически 
условия \eqref{eq2:8}-\eqref{eq2:9} означают, что уровни верхнего и нижнего бьефов не меняются.

\subsection{Граница обмена между водоводом и НА} 
При совместном расчете течения в области <<водовод-гидротурбина>> необходимо обеспечить корректную передачу 
параметров из одной области в другую. Трудность состоит в том, что в рассматриваемой экономичной циклической 
постановке спиральная камера (СК) и статор (СТ) не входят в расчетную область. Поэтому обмен параметрами 
течения нужно произвести между выходной границей водовода $\xi=L$ и входным сечением 
НА (рисунок~\ref{fig2:1}). Расход воды в 
этих сечениях одинаков, поэтому на входе в НА задаются значение расхода $Q_{\text{\it ВВ, вых}}$, полученное в 
ходе решения системы уравнений \eqref{eq2:7} в водоводе, и угол входа потока $\delta_{sp}=const$. Давление на 
входе в НА экстраполируется изнутри расчетной области.

Опишем более подробно передачу давления из области НА на выходную границу водовода. Выведем соотношение между 
давлением $p_{\text{\it ВВ, вых}}$ на выходе из водовода и давлением $p_{\text{\it НА, вх}}$ на входе в НА при 
условии, что суммарные потери энергии в СК и СТ оцениваются по формуле
\begin{equation}
  \Delta h_{\text{\it СК}}+\Delta h_{\text{\it СТ}} = \zeta_s H_0,
\end{equation}
где $\zeta_s$ -- коэффициент потерь в спиральной камере и статоре. Вообще, $\zeta_s\sim Q^2$~\cite{topaj}, 
но на практике для простоты можно принять, что $\zeta_s=const\sim 0.01$. Полная энергия на выходе из водовода
\begin{equation}
  E_{\text{\it ВВ, вых}}=(\frac{p}{\rho g}-z)_{\text{\it ВВ, вых}}+\frac{|{\bf w}|^2_{\text{\it ВВ, вых}}}{2g},
\end{equation}
где $|{\bf w}|_{\text{\it ВВ, вых}}=\frac{Q}{S_{\text{\it ВВ, вых}}}$, так как в модели предполагается, что 
скорость потока на выходе из водовода перпендикулярна сечению. На входе в НА
\begin{equation}
  E_{\text{\it НА, вх}}=(\frac{p}{\rho g}-z)_{\text{\it НА, вх}}+\frac{|{\bf w}|^2_{\text{\it НА, вх}}}{2g}.
\end{equation}
В цилиндрической системе координат вектор скорости на входе в НА ${\bf w}_{\text{\it НА, вх}}$ имеет 
компоненты $(c_r,c_z,c_u)$, при этом осевая компонента скорости $c_z=0$. Радиальная и окружная
связаны соотношением $\dfrac{c_r}{c_u}={\text tg }\,\delta_{sp}$. Отсюда
\begin{equation}
  |{\bf w}|_{\text{\it НА, вх}}=\sqrt{c_r^2+c_u^2}=\frac{c_r}{\sin\delta_{sp}}=\frac{Q}
  {S_{\text{\it НА, вх}}\sin \delta_{sp}}.
\end{equation}
Таким образом, учитывая
\begin{equation}
  E_{\text{\it ВВ, вых}}=E_{\text{\it НА, вх}}+\zeta_s H_0
\end{equation}
и равенство уровней $z_{\text{\it ВВ, вых}}=z_{\text{\it НА, вх}}$, получаем соотношение для статических 
давлений
\begin{equation}
  \label{eq2:16} 
  \frac{p_{\text{\it ВВ, вых}}}{\rho g}=\frac{p_{\text{\it НА, вх}}}{\rho g}+\frac{Q^2}{2g}
  \left(\frac{1}{S_{\text{\it НА, вх}}^2\, \sin^2\delta_{sp}}-\frac{1}{S_{\text{\it ВВ, вых}}^2 }\right)+
  \zeta_s H_0.
\end{equation}

\subsection{Остальные границы} 
Расчет трехмерного течения в НА и РК проводится в циклической постановке -- в одном межлопаточном канале НА, 
одном межлопастном канале РК и всей ОТ. На границах протекания жидкости из одного канала в другой ставится 
условие периодичности. На твердых стенках ставится условие прилипания потока. При передаче параметров на 
границах обмена между НА и РК, РК и ОТ производится осреднение потока в окружном направлении.

\section{Численный метод совместного решения уравнений модели переходного течения}
\label{s:23}
\subsection{Решение уравнений Рейнольдса}
Численный метод решения нестационарных уравнений \eqref{6} описан в п.~\ref{s:12}.

\subsection{Решение уравнений упругого гидроудара}
Система \eqref{eq2:7} -- одномерная гиперболическая система с постоянными коэффициентами. В векторном виде 
система записывается следующим образом
\begin{equation}
  \label{eq2:19} 
  \frac{{\partial {\bf{f}}}}{{\partial t}} +{\bf{A}}
  \frac{{\partial {\bf{f}}}}{{\partial \xi}} = 0,
\end{equation}
где ${\bf{f}} = \left( {\begin{array}{*{20}c} m \\ Q \\ \end{array}}\right),\ 
{\bf{A}} = \left( {\begin{array}{*{20}c} 0 & {c^2 /gS}  \\ {gS} & 0 \\ \end{array}} \right)$. Введем на отрезке
 $[0,L]$ равномерную сетку с шагом $\Delta \xi: \xi_1=0,\,\xi_2=\Delta \xi,\,\xi_3=2\Delta \xi,...,\,\xi_J=L.$
Уравнение \eqref{eq2:19} решается численно по неявной разностной схеме с использованием аппроксимаций против 
потока 1-го порядка по пространству и времени
\begin{equation}
  \label{eq2:20} 
  \frac{{{\bf{f}}_j^{s+1}  - {\bf{f}}_j^n }}{{\Delta t}} + {\bf{A}}^ +  \frac{{{\bf{f}}_j^{s+1} - 
  {\bf{f}}_{j - 1}^{s+1} }}{{\Delta \xi}} + {\bf{A}}^ -  
  \frac{{{\bf{f}}_{j + 1}^{s+1} - {\bf{f}}_j^{s+1} }}{{\Delta \xi}} = 0,\quad j=2,...,J-1,
\end{equation}
где ${\bf{A}}^+ +{\bf{A}}^-={\bf{A}}$, матрицы ${\bf{A}}^+=\left( {
\begin{array}{*{20}c}
   {c/2} & {c^2 /2gS}  \\
   {gS/2} & {c/2}  \\
\end{array}} \right)$ и ${\bf{A}}^-= \left( {\begin{array}{*{20}c}
   { - c/2} & {c^2 /2gS}  \\
   {gS/2} & { - c/2}  \\
\end{array}} \right)$ имеют только неотрицательные $(c,0)$ и неположительные $(-c,0)$ собственные значения, 
соответственно. Уравнение \eqref{eq2:20} итерируется по $s$, при этом ${\bf{f}}^{n+1}=\lim\limits_{s\to
\infty}{\bf{f}}^{s}$. Итерации по $s$ нужны из-за нелинейных краевых условий на входе в водовод \eqref{eq2:8}, 
а также для обмена параметрами с областью НА. Одна итерация по $s$ в области водовода соответствует одному 
шагу по псевдовремени в гидротурбине.

\subsection{Численная реализация краевых условий}
Граничное условие \eqref{eq2:8} во входном сечении водовода ($j=1$) реализуется следующим образом. Пусть
${\bf{f}}_2^{s}=(m_2^s,Q_2^s)$, тогда полагается ${\bf{f}}_1^{s+1}=(H_0-(Q_2^s)^2/{2gS^2},Q_2^s)$. Таким 
образом, во входном сечении значение расхода $Q$ экстраполируется изнутри области с предыдущей итерации, а 
величина потенциального напора $m$ вычисляется так, чтобы полная энергия, вычисленная в узле $j=1$, была равна 
$H_0$.

На границе обмена водовод-НА ($j=J$) полагается
\begin{equation}
  {\bf{f}}_J^{s+1} = \left( {
  \begin{array}{*{20}c}
   {m_{HA} }  \\
   Q^s_{J-1} \\
  \end{array}} \right),
\end{equation}
где $m_{HA}=p_{\text{\it ВВ, вых}}/{\rho g}-z_{HA}$, давление $p_{\text{\it ВВ, вых}}$ определяется 
соотношением \eqref{eq2:16}, в котором $p_{\text{\it НА, вх}}$~--~среднее по сечению статическое давление на 
входе в НА.

На выходе из ОТ гидростатическое распределение давления $p$ подбирается так, чтобы полная энергия потока на 
выходе из ОТ была равна нулю
\begin{equation}
  E_{\text{\it ОТ, вых}}=0,
\end{equation}
а значения скоростей экстраполируются изнутри расчетной области. Необходимо отметить, что при решении 
уравнений \eqref{6} давление определяется с точностью до константы, поэтому в проведенных расчетах энергия 
на выходе полагалась равной нулю. В реальности для рассматриваемой турбины энергия потока на выходе есть
\begin{equation}
  E_{\text{\it ОТ, вых}}=13.532 \text{м вод. ст}.
\end{equation}
Соответственно, для получения абсолютных значений давления в проточном тракте ГТ нужно увеличить приводимые рассчитанные величины на 13.532 м вод. ст.

\section{Результаты расчетов}
\label{s:24}
Разработанный метод применен для расчета основных переходных процессов в натурной радиально-осевой турбине 
с напором $ H_0=73.5$~м, диаметром РК $D_1=3.15$~м и номинальной частотой 
вращения $n_{\text{ном}}=200$ об/мин. 
Момент трения в \eqref{eq2:6} принят равным $M_{\text{тр}}=2$~т*м. Расчеты проведены в циклической
постановке в области, состоящей из водовода, одного межлопаточного канала НА, одного межлопастного канала РК и 
всей ОТ (рисунок~\ref{fig2:2}). Структурированная сетка в подвижной области НА строится автоматически на каждом 
шаге по времени. Расчетная сетка для всей области содержит суммарно около 100 тыс. ячеек.  Для области
водовода используется равномерная сетка с шагом по пространству $\Delta \xi=L/1000$. Во всех расчетах шаг по 
времени $\Delta t$ = 0.01~c, что соответствует повороту РК на $12^{\circ}$. На каждом шаге по времени $t$ 
проводилось 1500 итераций для установления по псевдовремени $\tau$. Характерное время проведения одного
нестационарного расчета с использованием 3 процессоров составляет 2 суток.

Все расчеты проведены в натурных параметрах, однако далее в тексте значения открытия НА $a_0$ (минимальное расстояние между соседними лопатками НА) указаны для соответствующей модели с диаметром РК $D_1=0.46$~м.
\begin{figure}[h]
  \centering
  \includegraphics[width = 9cm]{VV_OT.png}
  \caption{Расчетная область в циклической постановке: ВВ-НА-РК-ОТ}
  \label{fig2:2}
\end{figure}

\subsection{Моделирование переходного режима пуска турбины}
Пуск агрегата -- процесс, при котором РК из состояния покоя переводится в режим холостого хода с частотой 
$n_{\text{ном}}$ с последующей синхронизацией и включением генератора в сеть. При этом открытие НА меняется по 
закону, показанному на рисунке~\ref{fig2:3}, \emph{а}. Процесс пуска обычно имеет два пусковых открытия НА. 
Для рассмотренной турбины первое пусковое открытие НА $ a_{0, \text{по1}}=0.25a_{0,\max }$, где $a_{0,\max }$ 
-- максимальное открытие НА. При достижении РК частоты вращения, равной 90-95\% от номинальной 
$n_{\text{ном}}$, НА прикрывается на второе пусковое открытие $ a_{0, \text{по2}}=0.15a_{0,\max }$. Далее НА 
выходит на открытие холостого хода $a_{0,\text{хх}}=0.10a_{0,\max }$. Весь процесс пуска для рассматриваемой 
турбины в реальности занимает 30~с.

\begin{figure}[htb]
  \centering \small \rule{0mm}{0mm}\emph{a}\rule{90mm}{0mm}\emph{б}\\[1.5mm]
  {\includegraphics[width=8.5cm]{alfa0_a0_ot_article.png}}\hfill
  {\includegraphics[width=8.5cm]{n_ot_t.png}}
  \caption{Пуск в турбинный режим. \emph{а}) закон изменения открытия НА $a_0(t)$ в эксперименте ($\bullet$) 
  и принятый в расчете (\textcolor{blue}{$\genfrac{}{}{2pt}{2}{\quad}{  }$}); \emph{б}) 
  рассчитанная (\textcolor{blue}{$\genfrac{}{}{2pt}{2}{\quad}{  }$}) и 
  экспериментальная ($\bullet$) зависимости скорости вращения РК от времени}
  \label{fig2:3}
\end{figure}

Отметим, что требование невырожденности ячеек сетки не позволяет закрыть лопатки НА до 0~мм, поэтому в 
начальный момент времени $t=0$ задается поле течения, полученное в стационарной постановке при открытии 
$a_0=1$~мм и нулевой частоте вращения РК. Далее открытие лопаток НА за 2.6~с линейно растет до 
$a_{0,\text{по1}}$, при этом в первые 2.1~c частота вращения РК $n=0$~об/мин фиксирована (как в эскперименте). 
Начиная с момента времени $t=2.1$~c, частота $n$ находится в результате решения уравнения \eqref{eq2:6}.

На рисунке~\ref{fig2:3}, \emph{б} приведено сравнение рассчитанной и экспериментальной скоростей вращения РК. 
Получено хорошее качественное и количественное совпадение. При этом в первые 30~с наблюдается небольшое 
(меньше 7\%) отставание рассчитанной скорости вращения РК от экспериментальной. 
Возможно, это связано с неизбежным при 
использовании циклической постановки усреднением по окружности параметров течения при обмене на границе НА-РК.

\begin{figure}[ht!]
  \centering \small \rule{0mm}{0mm}\emph{a}\rule{95mm}{0mm}\emph{б}\\[1.5mm]
  {\includegraphics[width=8.5cm]{Qin_M_pusk_ppt.png}}\hfill
  {\includegraphics[width=8.5cm]{P_HA_exp_comp.png}}
  \caption{Рассчитанные зависимости расхода $Q$, момента $M_R$ (\emph{a}) и давления в НА (\emph{б}) от 
  времени, пуск в турбинный режим: \textcolor{blue}{$\genfrac{}{}{2pt}{2}{\quad}{  }$} 
  расчет, $\bullet$ эксперимент}
  \label{fig2:4}
\end{figure}

\begin{table}[!t]
  \label{tab2:2}
  \centering
  \caption{Параметры точек рассмотрения давления на лопасти РК в процессе пуска} 
  \begin{tabular}{|c|r|r|r|r|}
  \hline
  \No~п/п & время $t$, с & расход, м${^3}/$с & частота $n$, мин$^{-1}$ & момент сил $M_R$, тс$\cdot$м \\
  \hline
  1 & $0.5$ & $3.02$ & 0.00 & 13.34 \\
  2 & $3.0$ & $24.53$ & 12.23 & 142.41 \\
  3 & $15.0$ & $23.35$ & 162.78 & 73.19 \\
  4 & $30.0$ & $9.58$ & 192.08 & 8.49 \\
  \hline
  \end{tabular}
\end{table}

На рисунке~\ref{fig2:4}, \emph{а} изображены зависимости расхода воды $Q$ и момента $M_R$  турбины от времени. 
Зависимость расхода практически повторяет форму зависимости открытия 
НА от времени (рисунок~\ref{fig2:3}, \emph{а}). 
В первые 5~с гидравлический момент, действующий на лопасть РК быстро возрастает и достигает 55\% от
величины момента, действующего на лопасть в режиме оптимального КПД. После достижения максимума величина 
момента плавно снижается, при этом скорость вращения рабочего колеса продолжает возрастать.
При достижении режима холостого хода значение момента $M_R\sim M_{\text{тр}}$, что соответствует постоянному 
значению $n$. Параметры точек рассмотрения давления на лопасти РК приведены в таблице~\ref{tab2:2}. 
\begin{figure}[!h]
  \centering
  \begin{tabular}{cc}
    \includegraphics[natwidth=3702, natheight=2646,trim=  0 0 0 0,clip=true,scale=0.075]{k01_all_005_n.png}&
    \includegraphics[natwidth=3694, natheight=2634,trim=425 0 0 0,clip=true,scale=0.075]{k01_all_300_n.png} \\
    $t_1$=0.5 с&
    $t_2$=3.0 с \\
    \includegraphics[natwidth=3706, natheight=2638,trim=  0 0 0 0,clip=true,scale=0.075]{k01_all_1500_n.png}&
    \includegraphics[natwidth=3678, natheight=2658,trim=425 0 0 0,clip=true,scale=0.075]{k01_all_3000_n.png} \\
    $t_3$=15 с&
    $t_4$=30 с
  \end{tabular} \par
  \caption{Изменение давления на лопасти рабочего колеса при пуске в турбинный режим
  (для каждого момента времени рабочая сторона слева, тыльная -- справа)}
  \label{fig2:5}
\end{figure}

\begin{figure}[hb!]
  \includegraphics[width=8.5cm]{Cz_Cr_Cu.png} \hfill\includegraphics[width=8.5cm]{k01_vihr_article_2.png}\\
  \parbox{8.5cm} {\caption{Профили скорости на выходе из РК в режиме холостого хода}\label{fig2:6}}\hfill
  \parbox{8.5cm} {\caption{Срыв вихря с входной кромки лопасти РК в режиме холостого хода}\label{fig2:7}}
\end{figure}
\begin{figure}[hb!]
  \centering
  \includegraphics[width = 12cm]{z_vortexes.png}
  \caption{Линии тока и изолинии модуля скорости в двух сечениях РК}
  \label{fig2:8}
\end{figure}

На рисунке~\ref{fig2:4}, \emph{б} представлено сравнение рассчитанного давления во входном сечении НА с 
экспериментальным. Построенная модель правильно отражает величину гидроудара, возникающего 
при быстрых изменениях открытия НА.

Распределения давления на лопасти РК в выбранные моменты времени представлены на рисунке~\ref{fig2:5}. В первые 
3~с давление на рабочей стороне вблизи входной кромки лопасти резко возрастает, т.к. жидкость разгоняет РК. 
После этого распределение давления становится более равномерным -- давление на рабочей стороне лопасти 
становится меньше и компенсируется повышающимся давлением на тыльной стороне лопасти. Это приводит к 
уменьшению момента РК $M_R$ до величины момента трения $M_{\text{тр}}$.

На рисунке~\ref{fig2:6} изображены зависимости осевой $C_z$, радиальной $C_r$ и окружной $C_u$ компонент 
скорости от радиуса $R$ в сечении $z=2$~м отсасывающей трубы сразу за РК. При достижении режима холостого 
хода поток собирается и закручивается вблизи стенок ОТ, а в центре наблюдается обширная зона возвратного 
течения. Срыв вихря с входной кромки лопасти РК в режиме холостого хода представленный на 
рисунке~\ref{fig2:7} посредством линий тока, помогает понять структуру потока в канале РК. 
На рабочей стороне лопасти в ее нижней части 
формируется сильно закрученное течение, уходящее к стенкам ОТ. Вблизи обода РК наблюдается насосный вихрь 
(рисунок~\ref{fig2:8}), поток в котором идет вверх.

\subsection{Моделирование течения в гидротурбине при закрытии лопаток направляющего аппарата}
Ниже представлены результаты моделирования процесса уменьшения мощности турбины. 
В начальный момент времени $t=0$ задается стационарное поле течения, полученное при расчете режима 
максимального КПД. Далее открытие лопаток НА уменьшается линейно от $27.5$~мм (соответствует режиму 
максимального КПД) до $18$~мм (режим неполной загрузки) за $10$~c, начиная с момента времени $t=2.5$~c. 
Шаг по времени $\Delta t=0.01$~c соответствует повороту РК на $12^{\circ}$. Подвижная сетка в канале НА на 
каждом шаге по времени строилась алгебраически.

Известно, что в режиме максимального КПД поток за РК имеет слабую закрутку, вследствие чего течение в конусе 
отсасывающей трубы приобретает практически стационарный осесимметричный характер. В режиме неполной загрузки, 
напротив, окружная скорость потока за РК существенна, что приводит к формированию прецессирующего вихревого 
жгута в конусе ОТ.
\begin{figure}[b!]
  \label{fig1:9}
  \centering                                                                                   
  \includegraphics[width=12cm]{Q_M_new_i.png} \\
  \caption{Поведение момента сил $M_R$ и расхода $Q$ в процессе уменьшения мощности}
\end{figure}

\begin{figure}[ht]
  \label{fig1:10}
  \centering\small{\it а}\hspace*{78mm}{\it б}\\
  {\includegraphics[width=7.8cm]{pressureNA_1n.png}}\hfill
  {\includegraphics[width=7.8cm]{pressureOT_n1.png}}
  \caption{Эволюция давления перед НА (\emph{а}) и на стенке в конусе ОТ (\emph{б})}
\end{figure}

\begin{figure}[t!]
  \label{fig1:11}
  \centering\small
  \begin{tabular}{p{7.cm}p{7.cm}}
  \centering\small $t_1=6.02$~c
  \par \includegraphics[width=6.5cm]{vihr_0602.png}&
  \centering $t_2=8.06$~c \par
  \includegraphics[width=6.5cm]{vihr_0806.png}
  \end{tabular}\vspace*{5mm}
  \begin{tabular}{p{7.cm}p{7.cm}}
  \centering $t_3=10.02$~c \par
  \includegraphics[width=6.5cm]{vihr_1002.png}&
  \centering $t_4=12.80$~c \par
  \includegraphics[width=6.5cm]{vihr_1280.png}
  \end{tabular}\\[3mm]
  \caption{Рост интенсивности вихревого жгута (показаны изоповерхности давления в моменты 
           времени $t_1, t_2, t_3, t_4$, отмеченные на рисунке~\ref{fig1:10}, \emph{б})}
\end{figure}

На рисунке~\ref{fig1:9} представлены полученные в расчете изменения во времени момента сил $M_R$, 
действующего на РК, и расхода $Q$. На рисунке~\ref{fig1:10} показана эволюция давления перед НА 
и на стенке конуса ОТ. 
Снижение расхода при закрытии лопаток приводит к положительному гидроудару, который проявляется в повышении 
давления перед НА (рисунок~\ref{fig1:10},\,\emph{a}). Зародышевый вихревой жгут, наблюдающийся в начальный 
момент времени, по мере закрытия НА увеличивает свой размер и радиус винта (рисунок~\ref{fig1:11}), что 
вызвано ростом остаточной закрутки за рабочим колесом. 
После окончания движения лопаток~НА, т.\,е. при $t>12.5$~c, течение имеет периодически 
нестационарный характер. Вращение вихревого жгута вызывает пульсации расхода и момента сил на
валу гидротурбины~(рисунок~\ref{fig1:9}). Амплитуда пульсаций давления в конусе ОТ достигает 2.4\,\% от 
действующего напора, период прецессии жгута $T=1.41$~с, что согласуется со значениями, измеренным в 
эксперименте: 2.3\,\% и $T_{\text{exp}}=1.3$ с.

\subsection{Сброс нагрузки с последующим выходом на холостой ход}
Сброс нагрузки (отключение нагруженного генератора) -- процесс аварийный. Расчетный процесс сброса нагрузки с 
выходом на холостой ход определяется приведенным на рисунке~\ref{fig2:9}, \emph{а} законом закрытия лопаток.
В начальный момент времени $t=0$ задается стационарное поле течения, полученное при расчете режима 
максимальной мощности: $a_0=35$~мм, $n=200$~об/мин. Момент полезной нагрузки (см. уравнение \eqref{eq2:6}) 
становится нулевым при $t=2.5$~с, и тогда же начинают закрываться лопатки НА. В расчете процесса невозможно
закрыть НА до $a_0=0$ мм, поэтому в расчете на интервале $t\in(13.5;24.5)$ задавалось минимальное открытие 
$a_0=1$ мм. На рисунке~\ref{fig2:10} приведены сетки при максимальном и минимальном открытии НА, используемые 
в расчете процесса сброса нагрузки.

\begin{figure}[htb]
  \centering \small \rule{0mm}{0mm}\emph{a}\rule{95mm}{0mm}\emph{б}\\[1.5mm]
  {\includegraphics[width=8.5cm]{alfa0_a0_ot_t_sbros.png}}\hfill
  {\includegraphics[width=8.5cm]{n_lay_sbros.png}}
  \caption{Cброс нагрузки с выходом на холостой ход.
   \emph{а}) закон изменения открытия НА $a_0(t)$ в эксперименте ($\bullet$) и принятый в 
   расчете (\textcolor{blue}{$\genfrac{}{}{2pt}{2}{\quad}{  }$});
   \emph{б}) рассчитанная (\textcolor{blue}{$\genfrac{}{}{2pt}{2}{\quad}{  }$}) и 
   экспериментальная ($\bullet$) зависимости скорости вращения РК от времени}
  \label{fig2:9}
\end{figure}

\begin{figure}[t!]
  \centering
  \includegraphics[width = 16cm]{HA_set_35_1.png}
  \caption {Сетки в НА для двух положений лопаток процесса сброса нагрузки:
   $a_0(2.5 c)=35$~мм (слева) и $a_0(12.5 c)=1$~мм (справа)}
  \label{fig2:10}
\end{figure}
\begin{figure}[hb!]
  \centering \small \rule{0mm}{0mm}\emph{a}\rule{90mm}{0mm}\emph{б}\\[1.5mm]
  {\includegraphics[width=8.5cm]{Qin_sbros.png}}\hfill
  {\includegraphics[width=8.5cm]{M_H_sbros.png}}
  \caption{Рассчитанные зависимости расхода $Q$ (\emph{a}), момента $M_R$ и
   напора $H$ (\emph{б}) от времени, мгновенный сброс нагрузки}
  \label{fig2:11}
\end{figure}

На рисунках~\ref{fig2:9}, \emph{б} и \ref{fig2:11} приведены зависимости от времени 
в процессе сброса нагрузки частоты 
вращения РК $n$, расхода $Q$, гидродинамического момента РК $M_R$ и напора на турбине $H$. После отключения 
генератора частота вращения РК быстро увеличивается. Уменьшение открытия НА приводит к уменьшению расхода воды.
Момент РК падает до нуля и при $a_0<a_{0,\;\text{хх}}$ становится отрицательным. Частота вращения при этом 
проходит максимум и постепенно снижается, но пока она выше номинальной $n_{\text{ном}}$, направляющий аппарат 
продолжает закрываться и достигает своего минимального значения (1 мм в расчете, 0 мм в действительности), 
которое держится до тех пор, пока $n$ не приблизится к $n_{\text{ном}}$. После этого НА плавно открывается до 
$a_{0,\;\text{хх}}$ и ГТ переходит тем самым в режим холостого хода. Отметим, что рассчитанное увеличение
скорости вращения РК в первые 25~с вновь отстает от экспериментально полученного.
\begin{table}[h!]
  \label{tab2:3}
  \centering
  \caption{Параметры в процессе сброса нагрузки}
  \begin{tabular}{|c|r|r|r|r|}
  \hline
  \No~п/п & время $t$, с & расход, м${^3}/$с & частота $n$, мин$^{-1}$ & момент сил $M_R$, тс$\cdot$м \\
  \hline
  1 & $3.0$ & $84.42$ & 214.67 & 257.45 \\
  2 & $10.0$ & $5.35$ & 286.00 & -49.91 \\
  3 & $24.0$ & $2.68$ & 216.48 & -28.80 \\
  4 & $50.0$ & $10.38$ & 208.56 & 5.29 \\
  \hline
  \end{tabular}
\end{table}
\begin{figure}[h!]
  \centering
  \begin{tabular}{ccc} 
    \includegraphics[natwidth=3730, natheight=2674,trim=  0 0 0 0,clip=true,scale=0.08]{k01_0300_sbros_n.png}&
    \includegraphics[natwidth=3722, natheight=2638,trim=425 0 0 0,clip=true,scale=0.08]{k01_1000_sbros_n.png} 
     \vspace{-5mm} \\ 
    $t_1$=3.0 с&
    $t_2$=10.0 с \\ \vspace{-5mm}
    \includegraphics[natwidth=3734, natheight=2650,trim=  0 0 0 0,clip=true,scale=0.08]{k01_2400_sbros_n.png}&
    \includegraphics[natwidth=3702, natheight=2670,trim=425 0 0 0,clip=true,scale=0.08]{k01_5000_sbros_n.png} 
    \\ 
    $t_3$=24.0 с&
    $t_4$=50.0 с
  \end{tabular} \par \vspace{-5mm}
  \caption{Изменение давления на лопасти рабочего колеса при мгновенном сбросе нагрузки (для каждого момента 
  времени рабочая сторона слева, тыльная -- справа)}
  \label{fig2:12} \vspace{-5mm}
\end{figure}

Распределения давления на лопасти РК в выбранные моменты времени (таблица~\ref{tab2:3}) 
представлены на рисунке~\ref{fig2:12}. 
Вначале давление на рабочей стороне лопасти уменьшается, на тыльной возрастает. При $t=10$~с момент РК 
достигает своего минимального значения, после чего плавно возрастает до значения $M_{\text{тр}}$.
Так же, как и в процессе пуска, распределение давления жидкости по лопасти РК сильно зависит от времени. 
\vspace{-5mm}
\subsection{Траектории мгновенных режимов при различных переходных процессах}
Имея кривые изменения натурных параметров гидротурбины для различных переходных процессов, можно построить 
траекторию последовательного перемещения режимной точки в координатах приведенных к $H=1$~м, $D_1=1$~м расхода 
и частоты \vspace{-3mm}
$$
  Q_{11}=\frac{Q}{D_1^2\sqrt{H}},\ n_{11}=n\frac{D_1}{\sqrt{H}},
$$
т.е. изобразить переходной процесс в поле главной универсальной характеристики. Это позволяет наглядно 
представить диапазон изменения режимов и определить условия работы гидротурбины при различных переходных 
процессах. На рисунке~\ref{fig2:15}, \emph{а} приведена типичная универсальная характеристика РО турбины и линии 
мгновенных режимов, соответствующие различным типам переходных процессов, заимствованная из \cite{krivch}. На 
линии разгона $II$ гидродинамический момент РК $M_R=0$ и соответственно КПД 
$\eta = 0$. Эта линия делит универсальную характеристику на зону $I$ турбинных режимов ($M_R>0$) и зону $III$ 
тормозных режимов работы ($M_R<0$). Пунктирной линией ограничена область рабочих режимов при нормальной 
эксплуатации турбины. Рассчитанные в настоящей работе траектории представлены на 
рисунке~\ref{fig2:15}, \emph{б}. 
Получено хорошее качественное соответствие результатов расчетов экспериментальным данным. \vspace{-3mm}
\begin{figure}[!h]
  \begin{tabular}{p{8.0cm}p{8.0cm}}
  \centering
  \includegraphics[width=8.0cm]{UX_result_2ppt_1.png}\par \vspace{-4mm}
  \centering \emph{а}& \centering
  \includegraphics[width=8.0cm]{UX_count_2ppt_2.png}\par \vspace{-4mm}
  \centering \emph{б}
  \end{tabular} \vspace{-5mm}
  \caption{\emph{а}) типичная УХ РО турбины \cite{krivch}. I -- турбинные режимы, II -- разгонные режимы, 
  III -- тормозные режимы. \emph{б}) рассчитанные траектории движения режимных точек: 1~--~пуск в турбинный 
  режим, 2~--~снятие нагрузки, 3~--~сброс нагрузки} \vspace{-5mm}
  \label{fig2:15}
  % \rule{0mm}{12cm}
\end{figure}
%
 