%!TEX root = dissertation.tex
% \pagestyle{plain}
\chapter*{Глава 3. Метод расчета осевых и радиальных нагрузок на рабочее колесо гидротурбины 
в нестационарном потоке}
\label{s:3}
\setcounter{chapter}{3}
\addcontentsline{toc}{chapter}{Глава~\thechapter~ Метод расчета осевых и радиальных нагрузок на рабочее 
колесо гидротурбины в нестационарном потоке}
\setcounter{section}{0}

% 
% \begin{abstract}
% 
% {\it Ключевые слова}: гидротурбина, нестационарные течения,
% лабиринтные уплотнения, радиальные и осевые нагрузки.
% \end{abstract}
%

\section{Постановка задачи}
\label{s:31}
Между вращающимися и неподвижными частями гидротурбины (ГТ) имеются зазоры протечек жидкости 
(рисунок~\ref{fig:01}). Для снижения объемных потерь в зазорах устанавливаются лабиринтные
уплотнения (ЛУ). Кольца лабиринтных уплотнений крепятся на верхнем и нижнем ободах или на камере рабочего 
колеса (РК). Ступица РК имеет разгрузочные отверстия, расположенные за выходными кромками лопастей, через 
которые полость над РК сообщается с основным проточным трактом (ПТ). На рисунке~\ref{fig3:2} приведена
схема течения жидкости в проточном тракте ГТ, лабиринтных уплотнениях, полости между ступицей и крышкой 
турбины, разгрузочных отверстиях, полости между ободом и камерой РК.

Кроме уменьшения потерь мощности в турбине за счет ограничения величин протечек воды между вращающимися и 
неподвижными частями, уплотнения обеспечивают также снижение осевых и радиальных нагрузок, действующих 
на РК~\cite{granovsky}.

Полная гидравлическая нагрузка $\textbf{F}$, действующая на РК, складывается из трех составляющих:
\begin{equation}
  \textbf{F} = \textbf{F}_1 + \textbf{F}_2 + \textbf{F}_3,
  \label{eq3:1}
\end{equation}
где $\textbf{F}_1$~--- нагрузка от основного потока на поверхность проточной части РК, $\textbf{F}_2$~--- 
нагрузка от течения через верхнюю область протечки, $\textbf{F}_3$~--- нагрузка от течения
через нижнюю область протечки (см.~рисунок~\ref{fig3:2}). В полной гидравлической 
нагрузке $\textbf{F}=(F_x,F_y,F_z)$ выделяются осевые $F_z$ и радиальные $\textbf{F}_R=(F_x,F_y)$ усилия. 
Индексами $x,\, y,\, z$ обозначены компоненты сил, действующие по соответствующим осям декартовой системы 
координат. Ось $Oz$ совпадает с осью РК.
\begin{figure}[t!]
  \centering
  \includegraphics[width=12cm]{fig2_geom.png}
  \caption{Основной проточный тракт и области протечек за ступицей и ободом рабочего колеса гидротурбины}
  \label{fig3:2}
\end{figure}

Основными источниками осевых и радиальных сил в нагрузке $\textbf{F}_1$ являются
неравномерность потока на выходе из спиральной камеры, ротор-статор взаимодействие между лопастями
РК и лопатками направляющего аппарата, прецессирующий вихревой жгут в конусе отсасывающей трубы,
гидроудар.

Существенный вклад в осевые $F_z$ и радиальные $\textbf{F}_R$ усилия
вносят также составляющие $\textbf{F}_2$ и $\textbf{F}_3$, вызванные течениями в ЛУ, полостях и
разгрузочных отверстиях.

Таким образом, для наиболее точного определения осевых и радиальных нагрузок (ОРН) следует проводить
совместные расчеты полей во всех указанных областях течений основного ПТ и протечек за ступицей и ободом РК и
определять по ним нагрузку $\textbf{F}$~\eqref{eq3:1}. 
Однако такой расчет потребует огромных затрат вычислительных ресурсов из-за необходимости 
его совместного проведения в областях, имеющих значительно отличающиеся характерные размеры. 
Так, диаметр РК натурной гидротурбины составляет
$D_1\sim 6$~м, а ширина зазоров $d$ при этом варьируется в пределах 2--3~мм. 

В настоящее время для предсказания ОРН используются приближенные методики, которые можно разделить на
инженерно-эмпирические~\cite{lomakin, mak_pilev, kuzminsk} и основанные на методах вычислительной 
гидродинамики~\cite{staubli, roy_vu, xi_rhode}.

\section{Обзор существующих методик}
\label{s:32}
\subsection{Инженерно-эмпирическая методика определения осевых нагрузок}
\label{s:321}
В этой методике расходы жидкости через верхнее и нижнее уплотнения находятся по формулам
\begin{gather*}
 q_B  = f_1 (\xi_B ,\Delta h_B ), \\
 q_H  = f_2 (\xi_H ,\Delta h_H ),
\end{gather*}
где  $\xi_B,\xi_H$~--- коэффициенты гидравлического сопротивления верхнего и нижнего уплотнений, приведенные 
к  входному сечению уплотнения, $\Delta h_B, \Delta h_H$~--- напоры, срабатываемые в уплотнениях. 
Конкретный вид зависимостей $f_1, f_2$ приведен в~\cite{mak_pilev}.

Коэффициент гидравлического сопротивления для уплотнения ступицы складывается из коэффициентов сопротивления 
лабиринта и разгрузочных устройств
\begin{equation*}
  \xi _B = \xi _{\text{вх}} \frac{{S_1^2}}{{S_{\text{вх}}^2}} + 
  \xi _{\text{ЛУ}} + \xi_p \frac{{S_1^2 }}{{\left( {N_p S_p } \right)^2 }}+
  \xi_{\text{щ}}\frac{{S_1^2}}{{S_{\text{щ}}^2 }},
\end{equation*}
где $S_1$~--- площадь сечения узкой части зазора ЛУ; $\xi_{\text{вх}}$~--- коэффициент гидравлического 
сопротивления щели входа в уплотнительный тракт ступицы в случае установки уплотнения не на 
периферии; $\xi_{\text{ЛУ}}$ ~--- коэффициент гидравлического сопротивления~ЛУ;
 $\xi_p$~--- коэффициент гидравлического сопротивления разгрузочных 
 отверстий; $N_p$~--- число разгрузочных отверстий; $S_p$~--- площадь сечения разгрузочного 
 отверстия, $\xi_{\text{щ}}$~--- коэффициент гидравлического сопротивления выходной щели.

Коэффициент гидравлического сопротивления $\xi_{\text{ЛУ}} = \xi_{\text{ЛУ}}(\lambda ,\xi_0),$ ЛУ вычисляется 
по эмпирической формуле, например из \cite{idelchik}. Здесь
где $\lambda = \lambda (\omega ,q)$, $\omega$~--- частота вращения РК; $\xi_0$~--- коэффициент 
гидравлического сопротивления ячейки расширения, зависящий от геометрии ячеек и числа Рейнольдса.

Коэффициент гидравлического сопротивления вращающихся разгрузочных отверстий определяется соотношением
\begin{equation*}
  \xi_p = \xi _p (W_0 ,V_p ,S'_p ),
\end{equation*}
где $W_0$~--- относительная скорость жидкости перед разгрузочным отверстием, $V_p$~--- среднерасходная 
скорость воды в разгрузочном отверстии, $S'_p$~--- относительная ширина полости над отверстием.

Коэффициент гидравлического сопротивления уплотнения обода определяется сопротивлением ЛУ, 
входа в ЛУ, если таковой имеется, и выхода из полости.

Напор  в уплотнении
\begin{equation*}
  \Delta h = h_{in} - h_{out} - h_\omega,
\end{equation*}
где $h_{in}$~--- напор на входе в уплотнение, $h_{out}$~--- напор на выходе из 
уплотнения, $h_\omega$~--- напор, создаваемый вращающимся объемом жидкости,
заключенным между ступицей и ободом и неподвижными крышкой турбины и камерой 
РК, $h_\omega$~--- функция  частоты вращения РК $\omega$ и геометрических размеров проточного тракта.

Так как коэффициенты гидравлического сопротивления лабиринтных уплотнений $\xi_{\text{ЛУ}}$ и 
разгрузки $\xi_p$ определяются расходом жидкости через уплотнения,
который в свою очередь зависит от коэффициентов $\xi_B$ и $\xi_H$, то расчет протечек ведется 
итерационным методом последовательных приближений.

Полная гидравлическая нагрузка на РК рассчитывается по формуле~\eqref{eq3:1}. Нагрузка воздействия потока на 
внутреннюю полость РК $F_{1,z}$ вычисляется по инженерно-эмпирическим формулам.
Нагрузка $F_{2,z}$, действующая на внешнюю часть ступицы РК, в общем случае складывается из трех составляющих
\begin{equation*}
  F_{2,z} = F'_{2\text{ст}} + F_{2\text{л}} + F''_{2\text{ст}},
\end{equation*}
где $F'_{2\text{ст}}$~--- сила давления жидкости на часть поверхности ступицы, расположенную между 
входом в полость <<ступица\,---\,крышка>> и ЛУ
(если ЛУ находится на периферии ступицы, то $F'_{2\text{ст}}=0$); $F_{2\text{л}}$~--- сила давления 
жидкости на ЛУ; $F''_{2\text{ст}}$~--- сила давления жидкости на часть поверхности ступицы, 
расположенную между ЛУ и уплотнением вала.
Для однощелевых уплотнений с односторонней и двусторонней ячейкой расширения $F_{2\text{л}}=0$,
а для уплотнений елочного и гребенчатого типа эта сила определяется по формуле~\cite{mak_pilev}
\begin{equation*}
  F_{2\text{л}} = f(\xi_B ,q_B ).
\end{equation*}
Силы $F'_{2\text{ст}}$ и $F''_{2\text{ст}}$ находятся интегрированием пьезометрического напора по 
поверхности ступицы. Приближенно можно положить
\begin{gather*}
 F'_{2\text{ст}}=\frac{1}{4}\pi\rho g\left( {D_{\text{вх}}^2 - D_{\text{л}}^2}\right)\left[{h_{\text{вх}} 
      -\xi_{\text{вх}}
       \left({\frac{{q_B}}{{S_{\text{вх}} }}} \right)^2  - \frac{\tilde{\omega}^2}{16g}
       \left( {D_{\text{вх}}^2  - D_{\text{л}}^2 } \right)} \right], \\
 F''_{2\text{ст}} = \frac{1}{4}\pi \rho g\left( {D_{\text{вых}}^2  - D_{\text{вал}}^2 } \right)
       \left[{h_{\text{вых}} - \frac{\tilde{\omega}^2}{16g}
       \left( {D_{\text{вых}}^2 - D_{\text{вал}}^2} \right)} \right],
\end{gather*}
где $D_{\text{вх}}$~--- диаметр входа в полость; $D_{\text{л}}$~--- диаметр ЛУ; $D_{\text{вал}}$~--- 
диаметр уплотнения вала;
$D_{\text{вых}}$~--- диаметр выхода из уплотнения;
$\tilde{\omega}=\tilde{\omega}(\omega)$~--- угловая скорость вращения 
жидкости; $h_{\text{вх}},~h_{\text{вых}}$~--- пьезометрический напор на входе и выходе.

Аналогично по инженерно-эмпирическим формулам определяется нагрузка на внешнюю часть обода РК $F_{3,\,z}$.

Отметим, что в силу предположения об осесимметричности потока в уплотнениях и~повторяемости
течения в каналах РК определение радиальных нагрузок по данной методике невозможно.

\subsection{Инженерно-эмпирическая методика определения радиальных нагрузок}
\label{s:322}
Для оценки радиальных сил, вызванных несоосностью ротора и статора для однощелевых уплотнений без ячеек 
расширения и без учета вращения смещенного ротора, в~\cite{lomakin} предложена инженерно-эмпирическая методика.
 В ней, как и в~\cite{idelchik}, предполагается, что падение давления на входе в уплотнение определяется 
выражением
\begin{equation}
  \Delta p_{in} = p_A -p_{in}  = \left( {1 + \xi _{in} } \right)\frac{{V_{in}^2 \left( \theta  \right)}}{{2g}},
  \label{eq3:pin}
\end{equation}
а изменение давления по всей длине уплотнения равно сумме потерь на входе и потерь 
от гидравлического трения по всей длине
\begin{gather}
  p_A  - p_{out}  = p_A  - p_{in}  + p_{in}  - p_{out}  =  \Delta p_{in}  + p_{in}  - p_{out}  = \notag \\
  = \left( 1 + \xi _{in}  + \frac{\lambda l}{2d^* \left(\theta \right)} \right)
  \frac{{V_{in}^2 \left( \theta  \right)}}{{2g}},
  \label{eq3:pall}
\end{gather}
где $\xi_{in}=0.5$~--- коэффициент гидравлического сопротивления входа; $V_{in}(\theta)$ ~--- скорость на 
входе в ЛУ, зависящая от окружной координаты $\theta$; $\lambda$~--- коэффициент гидравлического трения; 
$l$~--- длина уплотнения; $d^* \left(\theta \right)$~--- переменный зазор. Авторы~\cite{lomakin} не учитывали 
в \eqref{eq3:pall} зависимости~$\lambda$ от частоты вращения ротора $\omega$ и наличия ячеек расширения с их 
гидравлическим сопротивлением. Эти упрощения были сделаны с целью существенного сокращения времени расчета 
нагрузки. При известном перепаде давления скорость $V_{in}$ выражается из уравнения~\eqref{eq3:pall} в виде
\begin{equation*}
  V_{in}\left(\theta \right) = \sqrt{\frac{2g (p_A -p_{out})}{1+\xi_{in}+\cfrac{\lambda l}{2d^* 
        \left(\theta \right)}}}
\end{equation*}
и подставляется в уравнение~\eqref{eq3:pin}. Следовательно, перепад давлений на входе
\begin{equation*}
  \Delta p_{in}(\theta) = \frac{1+\xi_{in}}{1+\xi_{in}+\cfrac{\lambda l}{2d^* 
                \left(\theta \right)}}(p_A -p_{out})
\end{equation*}
меняется по окружности. Принимается, что давление $p$ по длине щели $z$ изменяется по линейному закону
\begin{equation*}
  p(z,\theta) = p_A - \Delta p_{in} - \left(p_A -p_{out}-\Delta p_{in} \right)\frac{z}{l}.
\end{equation*}
Наличие в уплотнении изменяющегося по окружности давления вызывает силу, действующую на 
ротор в радиальном направлении. Она в силу симметрии величины зазоров будет направлена по линии смещения 
центра ротора от оси уплотнительного кольца и равна
\begin{equation*}
  F = -\int\limits_0^l {\int\limits_0^{2\pi } {p\left( {z,\theta } \right)R_{\,r} \cos \theta } } dzd\theta,
\end{equation*}
где $R_r$~--- радиус ротора. При определении $F$ не учитывается возможность частичного 
выравнивания неравномерности распределения давления $p(z,\theta)$ по окружности из-за перетекания жидкости.

В настоящей работе данная методика усовершенствована путем учета вращения ротора и ячеек 
расширения лабиринтных уплотнений (см. ниже раздел~\ref{s:35}).

Существенный недостаток инженерно-эмпирических методик --- грубое представление в них потока жидкости в 
основном проточном тракте, не учитывающее нестационарность потока и форму лопасти, следствием чего становится 
неточное задание давлений на входе и выходе зазоров, приводящее к неправильному определению осевой силы, 
действующей на~РК.

В связи с этим в последнее время стали появляться работы, в которых моделирование течений в зазорах и полостях 
ГТ осуществляется на основе более совершенных методов вычислительной гидродинамики.

\subsection{Подходы, основанные на методах вычислительной гидродинамики}
\label{s:323}
В работе~\cite{staubli} в полной постановке рассчитывается трехмерное течение в 
области РК\,---\,ЛУ\,---\,по\-лос\-ть над ступицей\,---\,разгрузочные отверстия. Во входном сечении направляющего аппарата задается расход, в выходном сечении РК~--- давление. 
Области РК, полости за ступицей и разгрузочных отверстий рассчитываются во вращающейся системе координат, 
а область НА~--- в неподвижной. На стыке вращающихся и неподвижных областей осуществляется обмен всеми 
параметрами течения.
\begin{figure}[b!]
  \centering
  \includegraphics[width=12cm]{fig2_5_Vu_1_ru.png}
  \caption{Экспериментальные данные о зависимости типа течения в  прямом ЛУ от величины осевого числа 
  Рейнольдса $\mathrm{Re}$ и числа Тейлора $\mathrm{Ta}$~\cite{roy_vu}}
  \label{fig3:2.5}
\end{figure}

В рассматриваемых областях решаются нестационарные уравнения Рейнольдса с замыканием SST-моделью 
турбулентности. SST-модель турбулентности является синтезом моделей $k-\varepsilon$ и~$k-\omega$. 
Она основана на том, что модели типа $k-\varepsilon$ лучше описывают свойства свободных сдвиговых течений, 
а модель $k-\omega$ имеет преимущество при моделировании пристеночных течений. Плавный переход 
от $k-\omega$-модели в пристеночной области к $k-\varepsilon$-модели вдали от твердых стенок обеспечивается 
введением весовой эмпирической функции. Средняя величина нормализованного расстояния до стенки в
расчетах~\cite{staubli} равна $y_+=70$.
  
Приводятся распределения давления в ЛУ и по верхней части ступицы для трех режимов работы ГТ: неполной 
загрузки, оптимального КПД и максимальной мощности, и распределения давления по верхней части ступицы при 
наличии разгрузочного устройства для тех же режимов. Полученные зависимости качественно адекватно отражают  
моделируемые явления.

В~\cite{roy_vu} исследуется применимость различных вычислительных моделей и сеток для расчета течения в 
лабиринтных уплотнениях радиально-осевых турбин.
Осевое число Рейнольдса для рассмотренных в~\cite{roy_vu} конфигураций ЛУ и режимов течения варьируется 
в~диапазоне Re$_{axial}=\dfrac{U_{axial}d}{\nu}=2\cdot10^3\div6.4\cdot10^3$,
число Тейлора Ta$~=\dfrac{\omega\sqrt{R_1d^3}}{\nu}=300\div962$, в связи с чем на основании 
рисунка~\ref{fig3:2.5} делается вывод, что течение в зазоре является 
турбулентным. Утверждается также~\cite{roy_vu}, что при ширине зазора лабиринтного уплотнения $d$ 
много меньшей радиуса рабочего колеса $R_1$ вихри Тейлора~--- источник нестационарности~--- в нем не 
возникают.
Поэтому течение в зазоре ЛУ моделируется стационарными уравнениями Рейнольдса в двумерном осесимметричном 
приближении. На входе в ЛУ задается либо расход, либо полное давление, на выходе~--- статическое давление, 
которые берутся из эксперимента. Для замыкания уравнений используется Low-Re-SST-модель
турбулентности без пристеночных функций с использованием подробной сетки, разрешающей пограничный 
слой ($y_+\sim 1$). При  низких числах Рейнольдса толщина логарифмического пограничного слоя сопоставима 
с шириной зазора $d$. Поэтому метод пристеночных функций, требующий расположения
первого слоя сетки на расстоянии $y_+\sim 30\div 300$ от стенки, должен применяться с осторожностью. 
В~\cite{roy_vu} исследуется влияние сгущения сетки и режима течения, подбираются сетка, модель турбулентности 
и краевые условия для адекватного количественного описания течения в~ЛУ.
Проводится сравнение результатов расчетов с экспериментом и результатами, полученными по эмпирической модели.

Использование подходов работ~\cite{staubli, roy_vu} позволяет построить вычислительные модели для расчета 
объемных потерь через уплотнения, осевых и радиальных нагрузок на РК при условии совпадения оси вращения 
колеса и оси симметрии неподвижной камеры РК. Однако эти подходы не годятся для моделирования радиальных сил, 
связанных с несимметричностью расположения РК относительно статора.

\subsection{Моделирование течения в зазорах между РК и неподвижными 
            частями гидротурбины с учетом биения РК} \label{s:324}
При наличии эксцентриситета оси РК (рисунок~\ref{fig3:6}) область течения в зазорах на ступице и ободе 
становится областью с подвижными границами, что при расчете этих течений требует использования подвижных 
сеток, перестраивающихся на каждом шаге по времени. Это обстоятельство усложняет расчет осевых
и радиальных нагрузок. Однако в ряде случаев постановка задачи может быть упрощена.
\begin{figure}[b!]
  \centering
  \includegraphics[width=10cm]{fig_4_eks.png}
  \caption{Кольцевая щель лабиринтного уплотнения c осью вращения РК, смещенной относительно центра полости}
  \label{fig3:6}
\end{figure}

В работе~\cite{xi_rhode} численно моделируются течения в зазорах ЛУ паровой турбины. 
При этом рассчитываются радиальные силы, действующие на ротор со стороны потока при малых биениях ротора в 
радиальном направлении ($e \ll d$). Предполагается, что биение оси ротора периодическое по времени и вместе с
тем она перемещается в окружном направлении (прецессирует) с угловой скоростью $\Omega$ 
(см.~рисунок~\ref{fig3:6}). В этом случае течение приближенно представимо в виде
суперпозиции стационарного осесимметричного течения при фиксированном положении ротора в центре 
и малого периодически нестационарного несимметричного возмущения, 
порождаемого движением центра ротора относительно центра полости
\begin{equation*}
  \Phi {\rm{(}}x,r,\theta ,t{\rm{)}} = \Phi_0 {\rm{(}}x{\rm{,}}r{\rm{)}} +
  \frac{e}{d}\left( {\Phi _{1c} {\rm{(}}x{\rm{,}}r{\rm{)cos(}}\Omega t{\rm{ - }}
  \theta {\rm{)}}\,{\kern 1pt}  - \;\Phi _{1s} {\rm{(}}x{\rm{,}}r{\rm{)sin(}}\Omega t{\rm{ - }}
  \theta {\rm{)}}} \right).
\end{equation*}
Для нахождения первой, осесимметричной, составляющей 
течения $\Phi _0 {\rm{(}}x{\rm{,}}r{\rm{)}}$ используются уравнения Рейнольдса, замкнутые
стандартной $k-\varepsilon$ моделью турбулентности. Для нахождения второй, несимметричной, 
составляющей применяются уравнения Рейнольдса для возмущенных течений.
На основе полученной несимметричной составляющей поля давления вычисляется радиальная сила, действующая 
на ротор. Далее найденная сила используется для вычисления коэффициентов динамической устойчивости ротора, 
которые показывают, насколько лабиринтное уплотнение демпфирует колебания вала в радиальном
направлении. Расщепление на осесимметричную невозмущенную и несимметричную возмущенную части позволило 
в~\cite{xi_rhode} находить эти составляющие решением двумерных осесимметричных задач.

Подход~\cite{xi_rhode} может быть применен для вычисления радиальных гидродинамических сил, действующих 
на РК при наличии эксцентриситета и неподвижной поверхности уплотнений.
В более общем случае можно использовать трехмерное моделирование течения во всей кольцевой щели 
переменной ширины в неинерциальной системе отcчета, вращающейся со~скоростью прецессии 
вала $\Omega$ относительно центра камеры РК. При этом предполагается, что разгрузочные отверстия 
заменяются на кольцевую щель. В~данном случае не требуется перестроения сетки.

\section{Методика определения полной гидравлической нагрузки на РК}
\label{s:34}
В диссертации предложена методика определения осевых и радиальных нагрузок на
рабочее колесо гидротурбины, вызванных нестационарным течением рабочей жидкости в его межлопастных каналах, 
а также протечками в лабиринтных уплотнениях, полостях и разгрузочных отверстиях. Данная методика превосходит 
по точности инженерно-эмпирические подходы, так как основана на моделях трехмерных нестационарных
турбулентных течений и позволяет учитывать окружную неравномерность потока, прецессию вихревого жгута за 
рабочим колесом и другие особенности трехмерных нестационарных течений. Также проведено усовершенствование 
инженерно-эмпирического метода расчета радиальных нагрузок в лабиринтных уплотнениях, вызванных смещением
оси вращения рабочего колеса. При этом учтены влияние вращения ротора на коэффициент сопротивления узкой 
части лабиринта, сопротивления ячеек расширения, зависимости коэффициента сопротивления узкой части лабиринта 
и ячеек расширения от переменного зазора между статором и ротором.

\subsection{Определение нагрузки $\textbf{F}_1$ от основного потока}
\label{s:341}
Для определения гидравлической нагрузки $\textbf{F}_1$, действующей на РК, рассчитывается нестационарное 
течение в основной проточной части гидротурбины, включающей спиральную камеру, все каналы направляющего 
аппарата, все межлопастные каналы РК и отсасывающую трубу. В качестве граничного условия во входном сечении 
спиральной камеры задается равномерное распределение скорости, перпендикулярной сечению
\begin{equation*}
  \left| {\textbf{V}_{in}} \right| = \frac{Q}{{S_{in} }},
\end{equation*}
где $Q$~--- фиксированный расход жидкости, $S_{in}$~--- площадь входного сечения.

Для определения абсолютных значений давления в точках $A,\ B,\ C,\ D$ (см.~рисунок~\ref{fig3:2}) рабочего 
колеса, требуемых для постановки граничных условий при расчете течения жидкости через области протечки, 
нужно задать абсолютное давление в выходном сечении ОТ
\begin{equation}
  p(z) = p_{atm} + (z-z_\text{н.б}),
  \label{eq3:5}
\end{equation}
где $p_{atm}$~--- атмосферное давление, $z_\text{н.б}$~--- уровень нижнего бьефа. Полная 
нагрузка $\textbf{F}_1$, действующая на рабочее колесо, не зависит от константы, входящей в 
распределение давления (атмосферное давление). Но мы не можем посчитать вклад в $\textbf{F}_1$ атмосферного 
давления, действующего сверху на вал гидротурбины.
Поэтому для корректного расчета силы $\textbf{F}_1$ необходимо проводить вычисления при $p_{atm}=0$. Этот 
же подход использован в работе~\cite{mak_pilev}.

\subsection{Определение осевой составляющей нагрузки $\textbf{F}_2$}
\label{s:342}
Область протечки над ступицей включает (рисунок~\ref{fig3:4}) верхнее лабиринтное уплотнение,
полость между крышкой турбины и внешней поверхностью ступицы, разгрузочные отверстия (РО), кольцевую полость 
и щель, через которую жидкость попадает за лопасти~РК. Разгрузочные отверстия служат для снижения давления
на крышку турбины.

Из-за разгрузочных отверстий течение в полости не является осесимметричным. Поэтому все течение в верхней 
области протечки разбивается на два. Течение в самом ЛУ считается в осесимметричной постановке.
Область ЛУ содержит один слой сетки по окружному направлению. Трехмерное течение в остальной части 
предполагается циклически повторяющимся по числу разгрузочных отверстий и рассчитывается в секторе полости
между крышкой и ступицей с одним РО и в секторе кольцевой полости. При передаче параметров потока между этой 
областью и областью ЛУ проводится их усреднение по окружному направлению.

Расход воды через области протечки {\it априори} не известен и должен быть найден в~ходе решения задачи.
\begin{figure}[t!]
  \centering \small \emph{a}\rule{52mm}{0mm}\emph{б}\rule{55mm}{0mm}\emph{в}\\[1.5mm]
  {\includegraphics[width=10cm]{fig4a_1.png}}\hfill
  {\includegraphics[width=6cm]{fig4b_all_1.png}}
  \caption{Верхняя область протечки: \emph{а}~--- расчетная область периодического трехмерного течения в полости и осесимметричного течения в ЛУ;
  \emph{б}~--- сетка в одной ячейке ЛУ для расчета в ней осесимметричного течения;
  \emph{в}~--- $rz$-проекция верхней области протечки}
  \label{fig3:4}
\end{figure}

Поскольку в предлагаемом методе определения нагрузки на РК мы отказались от одновременного
совместного расчета течений в основном проточном тракте и в областях протечек, то возникает проблема обмена
между ними параметрами течений в точке $A$, расположенной у входного сечения $S_{in}$ в ЛУ, и в точке
$B$ у выходного сечения $S_{out}$ из кольцевой щели (см.~рисунок~\ref{fig3:4},\,{\it в}). Во-первых, 
в этом обмене исключено влияние течения в зазорах на течение в основном ПТ. В построенной методике течение 
в зазорах определяется течением в основном ПТ. Для сопряжения этих течений необходимо учесть входные потери 
энергии в точке~$A$ и выходные в точке~$B$. Согласно~\cite{idelchik}, потери полной энергии на входе в
лабиринтное уплотнение могут быть найдены по формуле
\begin{equation}
  \left({\Delta E}\right)_{in} = E_A  - E_{in} = \xi _{in} \frac{{\left| {{\bf{V}}_{in}} \right|^2 }}{{2g}},
  \label{eq3:6}
\end{equation}
где $\xi _{in}$~--- коэффициент гидравлического сопротивления входа в ЛУ полагается равным $\xi _{in}=0.5$. 
Поскольку полная энергия
\begin{equation}
  E = p - z + \frac{{\left| {\bf{V}} \right|^2 }}{{2g}},
  \label{eq3:7}
\end{equation}
то из~\eqref{eq3:6} следует
\begin{equation}
  p_A -z_A +\frac{{\left| {{\bf{V}}_A }\right|^2 }}{{2g}} -\left({p_{in} -z_{in} +
  \frac{{\left| {{\bf{V}}_{in}}\right|^2 }}{{2g}}}\right) =
  \xi _{in} \frac{{\left| {{\bf{V}}_{in}^{} } \right|^2 }}{{2g}}.
  \label{eq3:8}
\end{equation}
Считается~\cite{mak_pilev}, что течение в лабиринтном уплотнении не зависит от скорости жидкости в
основной части ГТ, а зависит только от давления. Поэтому в~\eqref{eq3:8} полагаем
${\bf{V}}_A  = 0$, а также считаем, что $z_A=z_{in}$. Тогда
\begin{equation}
  p_A  -p_{in} -\frac{{\left| {{\bf{V}}_{in}} \right|^2 }}{{2g}}=
  \xi _{in}\frac{{\left| {{\bf{V}}_{in}}\right|^2 }}{{2g}},
  \label{eq3:9}
\end{equation}
откуда получаем
\begin{equation}
  \left| {{\bf{V}}_{in}}\right| =\sqrt{2g\left({\frac{{p_A -p_{in}}}{{1+\xi_{in}}}}\right)}.
  \label{eq3:10}
\end{equation}
\begin{figure}[ht!]
  \centering
  \includegraphics[width=4.5cm]{fig5_Sout.png}
  \caption{К заданию давления в выходном сечении}
  \label{fig3:5}
\end{figure}

Таким образом, во входном сечении $S_{in}$ задается условие на модуль скорости, из которого находится 
расходная $z$-я составляющая, при этом тангенциальные составляющие скорости полагаются нулевыми. 
Давление $p_{in}$ в соотношении~\eqref{eq3:10} находится в процессе решения задачи.

Течение в выходной щели $K$ (рисунок~\ref{fig3:5}) не рассчитывается. Поэтому для задания
условия в сечении $S_{out}$ необходимо связать параметры потока в этом сечении с параметрами потока 
в точке $B$. Потери энергии на выходе из кольцевой полости (сечение $S_{out}$) складываются из
потерь на входе в кольцевую щель
\begin{equation}
  E_{out} -E_K =\xi _{in} \frac{{\left| {{\bf{V}}_{out}^{}} \right|^2 }}{{2g}}
  \label{eq3:11}
\end{equation}
и потерь на выходе из нее в область основного ПТ (см.~рисунок~\ref{fig3:5})
\begin{equation}
  E_K  -E_B  = \xi _{out} \frac{{\left| {{\bf{V}}_{out}^{} } \right|^2 }}{{2g}},
  \label{eq3:12}
\end{equation}
где, согласно~\cite{idelchik}, коэффициент гидравлического сопротивления входа $\xi_{in}=0.5$, коэффициент 
гидравлического сопротивления на выходе из кольцевой щели $\xi_{out}=1$. Тогда,
складывая соотношения~\eqref{eq3:11},~\eqref{eq3:12} и подставляя выражения для полной энергии, получим
\begin{equation}
  p_{out} -z_{out} +\frac{{\left| {{\bf{V}}_{out} }\right|^2 }}{{2g}} -\left( {p_B - z_B +\frac{{\left|{{\bf{V}}_B }\right|^2}}{{2g}}}\right) =
  \left( {\xi _{in} + \xi _{out}}\right)\frac{{\left| {{\bf{V}}_{out}^{}}\right|^2 }}{{2g}}.
  \label{eq3:13}
\end{equation}
Положив в~\eqref{eq3:13} ${\bf{V}}_B = 0$ и $z_B=z_{out}$, находим
\begin{equation}
  p_{out} = p_B + \xi _{in}\frac{{\left| {{\bf{V}}_{out}^{}}\right|^2 }}{{2g}}.
  \label{eq3:14}
\end{equation}
Таким образом, скорость $V_{out}$ находится из основных уравнений в процессе их решения, а~давление 
вычисляется по формуле~\eqref{eq3:14}, где $p_B$ предварительно определено из расчета потока 
в основной части ГТ.

\subsection{Определение радиальных составляющих нагрузок $\textbf{F}_2$ и $\textbf{F}_3$}
\label{s:343}
Силы $\textbf{F}_{2,R}$ и $\textbf{F}_{3,R}$ возникают главным образом в силу окружной
неравномерности потока в лабиринтных уплотнениях при смещении центра вращения РК относительно 
геометрического центра камеры рабочего колеса.

В общем случае полагаем, что имеются смещение оси РК на эксцентриситет $e$ и ее прецессия вокруг центра 
статорной части с угловой скоростью $\Omega$ (см. рисунок~\ref{fig3:6}). Течение в кольцевом зазоре
переменной ширины приводит к появлению радиальных сил, действующих со стороны протекающей в зазоре жидкости. 
Используются две методики расчета радиальных составляющих нагрузок $\textbf{F}_2$ и $\textbf{F}_3$ в модели с
эксцентриситетом. Первая основана на расчете трехмерного течения в несимметричной щели, 
вторая~--- усовершенствованный инженерно-эмпирический подход~\cite{lomakin}.

\bigskip
\noindent{\it Расчет трехмерного течения в ЛУ с эксцентриситетом}.
\label{s:3431}
\medskip

\noindent Если для описания течения в кольцевой щели лабиринтного уплотнения с осью вращения РК, 
смещенной относительно центра полости, перейти во вращающуюся вокруг центра полости систему 
координат $(x',y',z)$ (см.~рисунок~\ref{fig3:6}),
то относительно этой системы   течение будет стационарным. В самом деле, форма области зазора в 
координатах $(x',y')$ не меняется, сетку можно считать неподвижной, и необходимо  учесть только вращение 
ротора с угловой скоростью $\Omega - \omega$ и статора~--- со скоростью $\Omega$. Компоненты скорости 
движения точки $(x'_{\text{PK}},y'_{\text{PK}})$ на роторе будут следующими:
\begin{equation}
%  \begin{array}{l}
    u = y'_{\text{PK}}\left( {\Omega  - \omega } \right), \quad
    v =  - \left( {x'_{\text{PK}} - e} \right)\left( {\Omega  - \omega } \right), \quad
    w = 0,
%   \end{array}
   \label{eq3:15}
\end{equation}
а точки $(x'_{\text{ст}},y'_{\text{ст}})$ на статоре~---
\begin{equation}
  \begin{array}{l}
    u = y'_{\text{ст}}\Omega , \quad
    v =  - x'_{\text{ст}}\Omega , \quad
    w = 0. \\
  \end{array}
  \label{eq3:16}
\end{equation}
При этом в уравнениях количества движения~\eqref{1}--\eqref{2} в источниковых членах 
величину~$\omega$ следует заменить на $\Omega$.
\begin{figure}[t!]
  \centering \includegraphics[width=5.5cm]{fig7_in_out_1.png}
  \caption{К определению давления $p_{out}$}
  \label{fig3:7}
\end{figure}
Условия для входного и выходного сечений выводятся с
учетом входных и выходных потерь в полном соответствии с данными раздела~\ref{s:342}.
Связь скорости и давления на входе задает соотношение~\eqref{eq3:10}, из которого
 находится расходная $z$-компонента скорости $\textbf{V}_{in}$, остальные
компоненты скорости полагаются нулевыми.

Выражение для давления в выходном сечении (рисунок~\ref{fig3:7}) выводится из потерь энергии
\begin{equation}
  E_{out} - E_L = \xi _{out} \frac{{\left| {{\textbf{V}}_{out}} \right|^2}}{{2g}}.
  \label{eq3:17}
\end{equation}
С учетом того что $\textbf{V}_L=0$, $z_L=z_{out}$ и $\xi_{out}=1$, находим $p_{out}=p_L$.
Давление $p_L$ берется из расчета течения в верхней области протечки (см. раздел~\ref{s:342}).

\section{Улучшенная инженерно-эмпирическая методика определения радиальных 
составляющих нагрузок $\textbf{F}_2$ и $\textbf{F}_3$}
\label{s:35}
Методика~\cite{lomakin} позволяет вычислять действующие на ось РК радиальные нагрузки, вызванные
наличием эксцентриситета (см. рисунок~\ref{fig3:6}). Она применяется
только для щелевых уплотнений (без ячеек расширения) для
неподвижного РК и нулевой скорости прецессии вала $\Omega =0$. Ниже этот подход распространяется на случай 
лабиринтных уплотнений с ячейками расширения и вращающимся рабочим колесом.

\begin{figure}[h]
  \centering
  \includegraphics[width=5.5cm]{fig9_lu.png}
  \caption{Давления $ p_A,\,p_{in},\, p_{out}$  и коэффициенты сопротивления верхнего ЛУ}
  \label{fig3:9}
\end{figure}

Как и в~\cite{lomakin}, предполагается, что падение давления на входе в ЛУ (рисунок~\ref{fig3:9}) 
определяется выражением
\begin{equation}
  \Delta p_{in} = p_A -p_{in}  = \left({1 + \xi _{in}} \right)\frac{{V_{in}^2 \left( \theta  \right)}}{{2g}}.
  \label{eq3:18}
\end{equation}

Для изменения давления по всей длине ЛУ вместо~\eqref{eq3:pall} используем формулу из~\cite{idelchik},
в~которой $\lambda$ зависит от частоты вращения ротора $\omega$, учитываются ячейки расширения и~коэффициенты 
их гидравлического сопротивления. Тогда
\begin{gather}
  p_A  - p_{out}  = p_A  - p_{in}  + p_{in}  - p_{out}  = \notag \\
  = \Delta p_{in}  + p_{in}  - p_{out}   = \left( {1 + \xi _{in}  + 
  \frac{{\lambda \left( {\theta ,\omega } \right)l_{\text{уз}}}}
  {{2d^* \left( \theta  \right)}} + n\xi _0 \left( \theta  \right)} \right)
  \frac{{V_{in}^2 \left( \theta  \right)}}{{2g}},
  \label{eq3:19}
\end{gather}
где $\xi_{in}=0.5$~--- коэффициент гидравлического сопротивления входа; $V_{in}(\theta)$ ~--- 
скорость на входе в ЛУ, зависящая от окружной координаты $\theta$; 
$\lambda \left({\theta ,\omega}\right)$~--- коэффициент гидравлического трения;
$l_{\text{уз}}$~--- суммарная длина узкой части (щели без ячеек) уплотнения; 
$d^* \left(\theta \right)$~--- переменный зазор;
$n$~--- число ячеек расширения; $\xi_0\left(\theta\right)$~--- коэффициент гидравлического 
сопротивления одной ячейки.

Найдем зависимость величины зазора от окружного угла $d^* \left(\theta \right)$. Пусть $R_s$~--- радиус 
внутренней стенки статора, $R_r$~--- радиус ротора. Тогда (см.~рисунок~\ref{fig3:6})
\begin{equation*}
  R_s  = e\cos \left( {\theta  - \gamma } \right) + \left( {R_{\,r}  + d^* } \right)\cos \gamma.
\end{equation*}
При эксцентриситете $e<<R_r$ угол $\gamma <<1$, поэтому можно считать
\begin{equation*}
  R_s  = e\cos \theta  + R_{\,r}  + d^*.
\end{equation*}
Тогда
\begin{equation*}
  d^*  = R_s  - R_{\,r}  - e\cos \theta  = d - e\cos \theta.
\end{equation*}
Окончательно переменный зазор ЛУ
\begin{equation*}
  d^* \left( \theta  \right) = d\left( {1 - \varepsilon \cos \theta } \right),
\end{equation*}
где $d=R_s -R_{\,r}$~--- радиальный зазор, $\varepsilon=e/d$~--- относительный эксцентриситет,
отсчет угла $\theta$ идет от самой узкой части зазора.
Из соотношения~\eqref{eq3:19} находим
\begin{equation}
  V_{in}^2 \left( \theta  \right) = \frac{{2g\left( {p_A  - p_{out,\,l} } \right)}}
  {{1 + \xi _{in}  + \cfrac{{\lambda \left( {\theta ,\omega } \right)l_{\text{уз}} }}{{2d^* 
  \left( \theta  \right)}} + n\xi _0 \left( \theta  \right)}}.
  \label{eq3:20}
\end{equation}
В уравнении~\eqref{eq3:20} коэффициент гидравлического трения кольцевой 
щели $\lambda \left( {\theta ,\omega } \right)$ определяется по формуле~\cite{idelchik}
\begin{equation}
  \lambda \left({\theta ,\omega }\right) =\frac{{0.316}}{{{\mathop{\rm Re}\nolimits} ^{0.25} 
  \left( \theta  \right)}}
  \left[ {1 + \left({\frac{{{\mathop{\rm Re}\nolimits} _u \left( {\theta ,\omega } \right)}}
  {{{\mathop{\rm Re}\nolimits} \left( \theta  \right)}}} \right)^2 } \right]^{0.375},
  \label{eq3:21}
\end{equation}
где
\begin{equation}
  {\mathop{\rm Re}\nolimits} = V_{in} \left( \theta  \right)\frac{{2d^* \left( \theta  \right)}}{\nu },\quad
  {\mathop{\rm Re}\nolimits} _u = \frac{\omega }{2}  \frac{{2d^* \left( \theta  \right)}}{\nu },
  \label{eq3:22}
\end{equation}
$\nu$~--- коэффициент кинематической вязкости.
\begin{figure}[b!]
  \centering
  \includegraphics[width=9cm]{fig11_ab.png}
  \caption{Определение коэффициентов $a$ и $b$~\eqref{eq3:23} при $h_k>d_{\text{гр}}$}
  \label{fig3:11}
\end{figure}

Коэффициент гидравлического сопротивления ячеек расширения лабиринта $\xi_0 \left(\theta \right)$
определяется по формуле~\cite{idelchik}
\begin{equation}
  \xi _0  = a\left( \theta  \right) + 0.5b\left( \theta  \right),
  \label{eq3:23}
\end{equation}
где $a$~--- коэффициент потерь энергии постоянной массы в начале и в конце ячейки, 
$b$~--- коэффицент потерь энергии от внезапного сжатия ядра постоянной массы при втекании в щель.
Значения этих коэффициентов определяются в зависимости от соотношения величин
\begin{equation}
  d_{\text{гр}} = d^* \left(\theta \right)+0.24  S
  \label{eq3:24}
\end{equation}
и глубины камеры ячейки $h_k$ (см. рисунок~\ref{fig3:9}). Если $h_k>d_{\text{гр}}$, то $a$ и $b$ находятся 
из эмпирических зависимостей
(рисунок~\ref{fig3:11}), если $h_k<d_{\text{гр}}$, то по формулам
\begin{equation}
  a=1-\frac{d_{\text{гр}}}{h_k},\quad b=\left(1-\frac{d_{\text{гр}}}{h_k}\right)^2.
  \label{eq3:241}
\end{equation}

Далее, так как коэффициент гидравлического трения кольцевой 
щели $\lambda \left( {\theta ,\omega } \right)$ зависит от расходной составляющей
скорости в щели $V_{in} \left( \theta  \right)$~\eqref{eq3:21}--\eqref{eq3:22}, которая в свою 
очередь зависит от коэффициента гидравлического трения $\lambda \left( {\theta ,\omega } \right)$, то 
расчет этих двух величин ведется итерационным методом последовательных приближений.

Таким образом, в соответствии с~\eqref{eq3:19} распределение давления $p(z,\theta)$ 
по длине ЛУ $z\in[0,l]$ состоит из потерь давления на трение
$\left({\displaystyle\frac{{\lambda \left({\theta ,\omega}\right)l_{\text{уз}} }}
{{2d^* \left( \theta  \right)}}} \right)\displaystyle\frac{{V_{in}^2 \left(\theta \right)}}{{2g}}$
на каждом участке узкой части уплотнения и скачкообразных 
потерь $\xi _0 \left( \theta  \right)\displaystyle\frac{{V_{in}^2 \left( \theta  \right)}}{{2g}}$
на входе и выходе каждой ячейки расширения. На рисунке~\ref{fig3:12} приведена зависимость $p(z, 0)$ 
для значения относительного эксцентриситета
$\varepsilon=0.5$. Конфигурация ячеек расширения показана над осью абсцисс. При построении 
зависимости скачкообразные потери на каждой ячейке расширения разделены на входные и выходные
$$
a(\theta)\frac{V^2_{in}(\theta)}{2g}~~
\text{и}~~ 0.5b(\theta)\frac{V^2_{in}(\theta)}{2g}
$$
составляющие и отнесены к началу и концу ячейки соответственно. Поэтому на внутренней части 
ячейки давление не меняется.
\begin{figure}[t!]
  \centering
  \includegraphics[width=9cm]{fig12_p_1.png}
  \caption{Распределение давления вдоль длины ЛУ}
  \label{fig3:12}
\end{figure}

Радиальную составляющую нагрузки $\textbf{F}_2$ находим численным интегрированием найденного 
распределения давления $p(z,\theta)$ по поверхности ротора:
\begin{equation}
  |\textbf{F}_{2,\,R}| = -\int\limits_0^l {\int\limits_0^{2\pi } 
  {p\left( {z,\theta } \right)R_{\,r} \cos \theta } } dzd\theta.
  \label{eq3:242}
\end{equation}
Эта радиальная сила при $\omega= 0$ в силу симметрии зазоров будет направлена по линии смещения центра 
ротора к оси статора.

Радиальная составляющая нагрузки $\textbf{F}_3$ определяется аналогично, отличаются только
геометрические размеры и число ячеек расширения нижнего лабиринтного уплотнения.

В случае прецессии вала РК с ненулевой скоростью $\Omega$ этот подход определения радиальной силы не применим.

\section{Результаты расчетов осевых и радиальных нагрузок}
\label{s:36}
Разработанный метод использован для расчета основных радиальных нагрузок на РК в~двух режимах 
работы гидротурбины,
приведенных в таблице~\ref{tab3:1}. Для режима частичной нагрузки характерно наличие прецессирующего 
вихревого жгута в конусе
отсасывающей трубы, влияние которого на гидродинамические силы в проточном тракте представляет 
практический интерес. Расчеты проведены
в натурных параметрах, шаг по физическому времени $\Delta t=0.003126$ с соответствует повороту 
РК на угол $\theta = 3.75^\circ$.
Расчетная сетка содержит суммарно по рассмотренным спиральной камере, НА, РК и ОТ 2\,615\,925 внутренних 
ячеек.
\begin{table}[t!]
  \centering\small\caption{Режимы работы гидротурбины, в которых рассчитывались 
  основные радиальные нагрузки}\vspace*{2mm}

  \begin{tabular}{|c|c|c|c|c|} \hline
  Режим             & Открытие НА& Частота вращения РК & Напор   & Расход       \\
                    &  $a_0$, мм &   $n$, об/мин &  $H$, м & $Q$, м$^3$/с \\ \hline
  Максимальная мощность    &   219.13   &     200       &  73.5   &    87.980    \\ % \hline
  Частичная нагрузка&   123.26   &     200       &  73.5   &    50.615    \\ \hline
  \end{tabular}
\label{tab3:1}
\end{table}
\begin{figure}[!b]
  \centering
  {\includegraphics[width=8.2cm]{fig13_F1z_1.png}}\hfill
  {\includegraphics[width=8.2cm]{fig13_F1xy_1.png}}
  \caption{Пульсации компонент нагрузки $\textbf{F}_1$ в режиме максимальной мощности:
  {\sl 1}~--- $F_{1,\,z}$, {\sl 2}~--- $F_{1,\,x}$, {\sl 3}~--- $F_{1,\,y}$}
  \label{fig3:13}
\end{figure}

\vspace*{-1mm}

\subsection{Режим максимальной мощности}
\label{s:361}
\subsubsection{Сила $\textbf{F}_1$}
\label{s:3611}
На рисунке~\ref{fig3:13} представлены рассчитанные в полной постановке зависимости компонент 
силы $\textbf{F}_{1}$ от времени.
Период пульсаций $F_{1,\,z}$   $T = 0.01875$~с, их частота $f_b ={1}/{T} = 53.28$~Гц.
При этом частота вращения РК  $f_n  = 3.33$~Гц. Отношение  ${f_b}/{f_n}=16$  совпадает с количеством 
лопастей РК, т.\,е. наблюдаемые пульсации обусловлены лопастной системой РК и имеют в своей основе 
продольную (вдоль потока) природу. Та же самая частота наблюдается
для пульсаций радиальных компонент $F_{1,\,x}$ и $F_{1,\,y}$ нагрузки $\textbf{F}_1$. Среднее значение 
модуля радиальной силы
$|{\bf{F}}_{1,\,R}| =\sqrt {F_{1,\,x}^2 +F_{1,\,y}^2}= 3077$ кГс. Амплитуда колебаний в данном 
режиме пренебрежимо мала и составляет 1\,\%.

\subsubsection{Осевая составляющая $F_{2,\,z}$}
\label{s:3612}
\begin{table}[b!]
  \centering\small\caption{Давления $p_A$ и $p_B$ для расчета течения в верхней области протечки}\vspace*{2mm}

  \begin{tabular}{|c|c|c|} \hline
Давление  & Инженерно-эмпирическая & Расчет  \\
  & методика~\cite{mak_pilev} & в полной постановке  \\ \hline
    $p_A$   &                 54.570              &       52.465                \\ % \hline
    $p_B$   &                 -4.100              &       -0.535                \\ % \hline
  $p_A-p_B$ &                 58.670              &       53.000                \\ \hline
    \end{tabular}
  \label{tab3:2}
\end{table}
\begin{table}[b!]
  \centering\small  \caption{Характеристики течения в верхней области протечки}\vspace*{2mm}

  \begin{tabular}{|p{4cm}|c|c|c|} \hline
  &                            &                                          & Расчет с $p_A$ и $p_B$, \\
  Параметр & Данные~\cite{mak_pilev}    & Расчет с $p_A$ и $p_B$ из~\cite{mak_pilev} & полученными  \\
  &                            &                                         & в полной постановке      \\ \hline
Расход через уплотнение $Q$, м$^3$/с  &    0.295     &    0.258    &  0.239     \\ \hline
Осевая сила $F_{2,\,z}$, кГс          &    34546     &    33113    &  47694     \\ \hline
  \end{tabular}
  \label{tab3:3}
\end{table}
\begin{figure}[b!]\vspace*{4mm}
\centering \small \rule{0mm}{0mm}\emph{a}\rule{95mm}{0mm}\emph{б} \\[1.5mm]
  {\includegraphics[width=8.2cm]{fig14_a1_1.png}}\hfill
  {\includegraphics[width=8.2cm,trim = 7 7 7 7,clip=true]{fig14_b1_1.png}}
  \caption{Распределение давления в верхней области протечки ({\it а}) и линии тока вблизи разгрузочного 
  отверстия в относительном движении ({\it б})}
  \label{fig3:14}
\end{figure}

Расчет осевой составляющей   $F_{2,\,z}$ проводился в соответствии с постановкой, описанной в 
разделе~\ref{s:342}.
В частности, давления $p_A$ перед входной границей расчетной области и~$p_B$ за выходной границей 
(см. рисунок~\ref{fig3:4}) взяты из результатов расчета потока в полной постановке. С целью сравнения с 
результатами, полученными по инженерно-эмпирической методике~\cite{mak_pilev}, проведен расчет с 
давлениями $p_A$  и $p_B$, взятыми из этой работы (таблица~\ref{tab3:2}).

Расход и осевая сила $F_{2,\,z}$, найденные в расчетах, сравниваются с данными работы~\cite{mak_pilev} 
в таблице~\ref{tab3:3}. Видно, что в нашем расчете при одинаковых с [3] давлениях $p_A$ и $p_B$ расход 
получается на 12\,\% ниже, а сила $F_{2,\,z}$ хорошо совпадает с результатом~\cite{mak_pilev}. На 
рисунке~\ref{fig3:14},\,{\it а}  показано распределение давления во всей расчетной 
области, {\it б}~--- представлена картина течения в полости над ступицей и в разгрузочных отверстиях. 
На рисунке~\ref{fig3:15} приведены распределения давления~({\it а}) и отношения окружной
компоненты скорости $c_u$ к $\omega R_r$~({\it б}) вдоль средней линии полости. Результаты, полученные 
в расчете сравниваются с данными, полученными по~[3]. Видно, что предположение о квадратичной зависимости 
давления от расстояния до оси РК, принятое в инженерно-эмпирической методике~[3], не выполняется.
\begin{figure}[t!]%[htb]
\centering \small \emph{a}\rule{90mm}{0mm}\emph{б} \\[1.5mm]
  {\includegraphics[width=8.5cm]{fig15a_p_2.png}}\hfill
  {\includegraphics[width=8.5cm]{fig15b_w_1.png}}
  \caption{Давление ({\it а}) и распределение отношения $c_u/\omega R_r$ ({\it б}) в полости между 
  ступицей и крышкой турбины в режиме максимальной мощности:
  {\sl 1}~--- расчет при $p_A$ и $p_B$, взятых из~\cite{mak_pilev};
  {\sl 2}~--- расчет при $p_A$ и $p_B$ из полной постановки; $\Box$ и серый цвет~--- данные~\cite{mak_pilev}}
  \label{fig3:15}
\end{figure}

\subsubsection{Радиальные составляющие $\textbf{F}_{2,R}$ и $\textbf{F}_{3,R}$}
\label{s:3613}
Рассмотрена несоосность статора и ротора без прецессии ($\Omega=0$) и с 
прецессией (с~заданной частотой $\Omega$) вала ротора. В последнем случае исследован наиболее 
вероятный вариант~--- со скоростью вращения рабочего колеса  $\Omega=\omega$, соответствующий изгибу
вала ротора.

При расчете радиальной составляющей $\textbf{F}_{2,R}$ проведено сравнение
подхода (см. раздел~\ref{s:343}), основанного на трехмерном расчете всего ЛУ, и
подхода (см. раздел~\ref{s:35}), реализующего улучшенную инже\-нер\-но-эм\-пи\-рическую
методику. Результаты расчетов  $\textbf{F}_{2,R}$ в~случае
непрециссирующего ротора ($\Omega = 0$) представлены на рисунке~\ref{fig3:16}.
Видно, что модифицированная методика и  трехмерная модель
течения жидкости дают близкие между собой результаты, отличающиеся от данных
расчета по методике~\cite{lomakin}.
\begin{figure}[!ht]
  \centering\small \emph{a}\rule{90mm}{0mm}\emph{б} \\[1.5mm]
  {\includegraphics[width=8.2cm]{fig16_1.png}}\hfill
  {\includegraphics[width=8.2cm]{fig17_1.png}}
  \caption{Зависимости модуля радиальной силы $|\textbf{F}_{2,R}|$ от относительного 
  эксцентриситета $\varepsilon$ в случае непрециссирующего ротора с
  невращающимся РК ({\it а}) и вращающимся РК ({\it б}):
  $\circ$~--- ин\-же\-нер\-но-эмпирическая методика~\cite{lomakin}, $\triangle$~--- улучшенная 
  инженерно-эмпирическая методика,
  $\square$~--- трехмерный расчет}
  \label{fig3:16}
\end{figure}

Давления перед входом в верхнее ЛУ $p_A=54.57$~м, за выходом из ЛУ $p_{out}=12.16$~м взяты
из расчета в верхней области протечки. Дополнительные входные данные, необходимые
для определения радиальной нагрузки, следующие: радиус ротора  $R_r=1.38225$~м,
постоянный зазор щели ЛУ $d=0.002$~м, длина ЛУ $l=0.11$~м, число ячеек расширения $n=4$.

Трехмерный расчет требует 2--3 ч времени, а улучшенная методика --- 2--3 мин, что позволяет рекомендовать 
последнюю для практической оценки модуля радиальной силы.

Результаты трехмерных расчетов показали, что вращение вала РК и скорость его прецессии оказывают
существенное влияние на направление радиальной силы. Этот факт находится в соответствии с результатами 
работы\,\cite{marcinkovskii}.
Если РК не вращается ($\omega\,=\,0$), то ${\textbf{F}}_R =\left({F_x ,\,0}\right)$. Однако при вращении 
РК с угловой скоростью $\omega$, согласно~\cite{marcinkovskii}, в~уплотнении помимо компоненты $F_x$ 
радиальной силы, пропорциональной перепаду давления, возникает дополнительная
гидродинамическая сила, перпендикулярная оси $Ox$ и пропорциональная угловой скорости вращения 
РК (компонента $F_{2,\,y}$, рисунок~\ref{fig3:20}).
Кроме того, вращение РК приводит к появлению радиальной силы, направленной в сторону увеличения 
эксцентриситета, уменьшающей компоненту $F_{2,\,x}$ радиальной силы.

Наличие прецессии $\Omega$ существенно меняет направление радиальной силы $\textbf{F}_{2,\,R}$. 
За счет возникающего при $\Omega=\omega$ вязкого сопротивления поступательному перемещению центра вала 
компонента $F_{2,\,y}$ меняет знак (см. рисунок~\ref{fig3:20}).
\begin{figure}[!hb]
  \centering
  \includegraphics[width=8.2cm]{fig20_F2x_1.png}\hfill
  \includegraphics[width=8.2cm]{fig20_F2y_1.png}
  \caption{Зависимости $F_{2,\,x}$ и $F_{2,\,y}$ от $\varepsilon$ в расчете трехмерного течения в ЛУ:
  $\triangle$~--- неподвижное РК ($\omega=0,\ \Omega=0$), $\Box$~--- РК, вращающееся 
  со скоростью $\omega$ без прецессии вала ($\Omega=0$),
  $\circ$~--- РК, вращающееся со скоростью $\omega$ с прецессией вала ($\Omega=\omega$)}
  \label{fig3:20}
\end{figure}

\begin{figure}[!b]
  \includegraphics[width=8.2cm]{fig23_F1_reg2.png}\hfill \includegraphics[width=8.2cm]{pA_pB_1.png}\\[-5mm]
\parbox[t]{0.48\textwidth}{\caption{Пульсации осевой $F_{1,\,z}$ и 
радиальных $F_{1,\,x}$, $F_{1,\,y}$ составляющих нагрузки $\textbf{F}_1$ \label{fig3:23}}}\hfill
\parbox[t]{0.48\textwidth}{\caption{Пульсации давления в точках $A$ и~$B$}\label{fig3:25}}
\end{figure}

\subsection{Режим частичной нагрузки}
\label{s:362}
Для данного режима характерно наличие прецессирующего вихревого жгута в конусе отсасывающей трубы, 
оказывающего влияние на ГТ вверх по потоку. Частота прецессии вихря $f_\nu=0.74$ Гц явно выделяется
в пульсациях осевой и радиальной нагрузок, действующих на РК (рисунок~\ref{fig3:23}).
В режиме неполной нагрузки присутствует сильная динамическая составляющая 
сил $F_{1,\,z}$, $\textbf{F}_{1,\,R}$, вызванная прецессией вихревого жгута. Она имеет
для  $F_{1,\,z}$ величину порядка 10\,\%, для $\textbf{F}_{1,\,R}$~--- 100\,\% от среднего значения. 
Частота динамических составляющих равна $0.222 f_n$.
На рисунке~\ref{fig3:25} показаны пульсации давления в точках $A$ и $B$ в режиме частичной нагрузки.
При расчете нагрузок $\textbf{F}_2$ и $\textbf{F}_3$ давления $p_A$ и $p_B$ усредняются по времени.

\section{Исследование течения в полости над ступицей и в разгрузочных отверстиях}
\label{s:37}
Осевая сила $F_{2,\,z}$ определяется распределением давления в полости над ступицей. 
Распределение давления на внешней поверхности ступицы зависит в основном от давления $p_B$, 
сопротивления разгрузочного устройства и закона движения жидкости в полости над ступицей. Это давление 
практически не зависит от сопротивления в верхнем ЛУ, так как гидравлическое сопротивление разгрузочных 
устройств мало по сравнению с сопротивлением лабиринтного уплотнения. Таким образом, при заданном 
давлении $p_B$ точность определения осевой силы определяется точностью нахождения угловой скорости 
жидкости над ступицей $\tilde{\omega}$ и точностью определения сопротивления разгрузочных отверстий. 
Согласно экспериментальным данным, рассмотренным в \cite{makar}, коэффициент сопротивления разгрузочных 
отверстий зависит главным образом от 
соотношения $W_0 / V_p$, где $W_0 =\sqrt{c_r^2 + c_{u,rel}^2}$ -- относительная скорость потока 
перед входом в отверстие, $V_p$ -- скорость в самом отверстии. Кроме этого, при малом относительном осевом 
зазоре над  отверстием ($S_p / d_p < 1$) отношение $S_p / d_p$ также влияет на сопротивление (здесь 
$S_p$ -- высота верхней области протечки над разгрузочным отверстием, $d_p$ -- диаметр 
разгрузочного отверстия). Экспериментальные данные, проанализированные в \cite{makar}, обобщены на 
рисунке~\ref{fig3:26}. 
\begin{figure}[!t]
  \centering
  \includegraphics[width=12cm]{eksp_ot_Makar.jpg}\\[-5mm]
  \caption{Обобщенная экспериментальная зависимость из \cite{makar}}
  \label{fig3:26}
\end{figure}
\begin{figure}[!t]
  \centering
  \includegraphics[width=12cm]{geom_UpDomain.png}\\[-5mm]
  \caption{Рассмотренные варианты полости между ступицей и крышкой турбины. Серым цветом показана 
           полость над ступицей натурной турбины ГЭС Платановрисси ($S_p = 80$~мм). Темным выделена 
           область над ступицей, соответствующая варианту $S_p = 10$~мм}
  \label{fig3:27}
\end{figure}
С целью сравнения расчетных значений сопротивления разгрузочных отверстий с обобщенными экспериментальными 
данными, проведены расчеты течения в верхней области протечки гидротурбины ГЭС Платановрисси, в которых 
варьировалась высота $S_p$ полости между ступицей и крышкой турбины при неизменном диаметре $d_p$ и 
числе $z_p$ разгрузочных отверстий. Это позволило рассмотреть различные режимы натекания жидкости 
на разгрузочные отверстия, соответствующие различным отношениям $W_0/V_p$  и $S_p/d_p$. Рассмотрены 
варианты $S_p = 10,\ 20,\ 40,\ 60,\ 80,\ 120,\ 160$~мм (рисунок~\ref{fig3:27}). В расчетах анализировались 
также распределение закрутки потока $\tilde{\omega}$ по радиусу и картина течения над 
разгрузочными отверстиями.
\begin{figure}[!b]
  \includegraphics[width=8.2cm]{RO_p_distribution.png}\hfill 
  \includegraphics[width=8.2cm]{omega1_omegaR.png}\\[-5mm]
  \parbox[t]{0.48\textwidth}{\caption{Распределение давления вдоль линии $MM^\prime$, 
            проходящей через центр разгрузочного отверстия} \label{fig3:28}}\hfill
  \parbox[t]{0.48\textwidth}{\caption{Распределение угловой скорости потока над ступицей, 
            отнесенной к угловой скорости РК} \label{fig3:29}}
\end{figure}

Расчеты показали (рисунок~\ref{fig3:28}), что для всех значений $S_p$ основное падение давления происходит 
на входе в разгрузочное отверстие, что согласуется с заключением работы \cite{makar}. Резкое 
снижение давления в центре разгрузочного отверстия для варианта $S_p = 10$~мм связано с формированием 
вихревого жгута в разгрузочном отверстии (рисунок~\ref{fig3:31}, слева). 
На рисунке~\ref{fig3:29} показано распределение угловой скорости потока над ступицей $\tilde{\omega}$, к угловой 
скорости вращения РК $\omega_R$. Угловая скорость измерялась вдоль линии $m$ на рисунке~\ref{fig3:32}. 
На рисунке~\ref{fig3:29} серой полосой показан интервал для отношения $\tilde{\omega}/\omega_R$, 
который принимается для определения $W_0$ в инженерно-эмпирической методике.  
В таблице~\ref{tab3:4} представлены рассчитанные значения расхода 
и осевой силы $F_{2,\,z}$, действующей на внешнюю поверхность ступицы. 
Сравнение сопротивлений разгрузочных отверстий, рассчитанных по 
трехмерной модели с инженерно-эмпирической методикой затруднено тем, что течение в окрестности входа 
в разгрузочное отверстие и выхода из него имеет существенно трехмерный характер (рисунок~\ref{fig3:30}). 
Кроме этого, все параметры потока в полости существенно меняются в зависимости от радиуса. Поэтому имеется 
некий произвол в выборе характерных величин $c_r$, $\tilde{\omega}$ перед разгрузочным отверстием, 
а также перепада давления $\Delta p$, которые подставляются в инженерные формулы. 

В расчетах коэффициент сопротивления разгрузочных отверстий находился из соотношения 
\begin{equation}
  \Delta p = \xi _p \frac{{V_p^{\rm{2}} }}{{{\rm{2g}}}},
\end{equation}
где $\Delta p$ -- перепад давлений в точках 1 и 2 
(рисунок~\ref{fig3:32}), $ V_p = \frac{{4q_B}}{{z_p \pi d_p^2 }}$ -- средняя расходная скорость в 
разгрузочном отверстии. Для сравнения с экспериментальными данными (рисунок~\ref{fig3:26}) в таблице~\ref{tab3:4} 
представлено также рассчитанное значение коэффициента расхода $\mu _p  = 1/\sqrt {\xi _p } $. 
Для позиционирования значения $\mu _p$ на графике рисунка~\ref{fig3:26}, необходимо вычислить значение 
модуля относительной скорости $W_0  = \sqrt {c_r^2  + c_{u,rel}^2 }$ над разгрузочным отверстием. 
Для определения $c_r$ использовалась формула $c_r  = \frac{{q_B }}{{2\pi R_p S_p }}$, в то время как 
относительная окружная составляющая скорости $c_{u,rel}  = R_p (\omega _1  - \omega _R )$ полагалась равной 
нулю. На рисунке~\ref{fig3:33} показано сравнение полученной в результате зависимости $\mu _p (W_0 /V_p )$ с 
эмпирическими данными \cite{makar}. Наблюдается хорошее качественное и количественное совпадение с 
обобщенными экспериментальными данными. В настоящих расчетах коэффициент расхода выше примерно на 15~\%. 
\begin{table}[t!]
  \centering\small\caption{Интегральные параметры течения в верхней области протечки}\vspace*{2mm}
  \begin{tabular}{|c|c|c|c|c|c|c|c|c|c|} \hline
  $S_p$, м & $S_p / d_p$ & $q_B$, м$^3$/с & $F_{2,\,z}$, тс & $\Delta p$, м & $V_p$, м/с & 
  $\xi_{p,\,\text{расч}}$ & $\mu_{p,\,\text{расч}}$ & $C_r$, м/с & $W_0/V_p$ \\ \hline
    0.01  & 0.182 & 0.232 & 68.89 & 10.17 & 4.88 & 8.37 & 0.35 & 5.43 & 1.11 \\ % \hline
    0.02  & 0.364 & 0.257 & 38.83 & 3.79  & 5.41 & 2.54 & 0.63 & 3.01 & 0.56 \\ % \hline
    0.04  & 0.727 & 0.259 & 32.5  & 2.65  & 5.45 & 1.75 & 0.76 & 1.52 & 0.28 \\ % \hline
    0.06  & 1.091 & 0.259 & 32.02 & 2.58  & 5.45 & 1.70 & 0.77 & 1.01 & 0.19 \\ % \hline
    0.078 & 1.418 & 0.258 & 33.11 & 2.50  & 5.43 & 1.66 & 0.78 & 0.77 & 0.14 \\ % \hline
    0.12  & 2.182 & 0.257 & 33.06 & 2.43  & 5.41 & 1.63 & 0.78 & 0.50 & 0.09 \\ % \hline
    0.16  & 2.909 & 0.254 & 35.73 & 2.24  & 5.35 & 1.54 & 0.81 & 0.37 & 0.07 \\   \hline
  \end{tabular}
  \label{tab3:4}
\end{table}

\clearpage
\thispagestyle{empty}
\begin{figure}[!hb]
  \centering
  \text{внешняя пов. ступицы} \rule{15mm}{0mm} \text{средняя линия} 
  \rule{20mm}{0mm} \text{крышка турбины} \rule{5mm}{0mm} \\
  \includegraphics[width=5cm]{sp10l.png}\hfil\includegraphics[width=5cm]{sp10m.png}\hfil
  \includegraphics[width=5cm]{sp10r.png} \\
  $S_p = 10$~мм \\
  \includegraphics[width=5cm]{sp20l.png}\hfil\includegraphics[width=5cm]{sp20m.png}\hfil
  \includegraphics[width=5cm]{sp20r.png} \\
  $S_p = 20$~мм \\
  \includegraphics[width=5cm]{sp40l.png}\hfil\includegraphics[width=5cm]{sp40m.png}\hfil
  \includegraphics[width=5cm]{sp40r.png} \\
  $S_p = 40$~мм \\
  \includegraphics[width=5cm]{sp80l.png}\hfil\includegraphics[width=5cm]{sp80m.png}\hfil
  \includegraphics[width=5cm]{sp80r.png} \\
  $S_p = 80$~мм \\
  \includegraphics[width=5cm]{sp120l.png}\hfil\includegraphics[width=5cm]{sp120m.png}\hfil
  \includegraphics[width=5cm]{sp120r.png} \\
  $S_p = 120$~мм \\
  \caption{Картина относительного движения в полости над ступицей в окрестности входа в разгрузочное 
  отверстие. Показаны векторы скорости и линии тока вблизи внешней поверхности ступицы, в среднем по высоте сечении полости, вблизи крышки турбины}
  \label{fig3:30}
\end{figure}
 
\begin{figure}[!b]
  \centering
  \includegraphics[width=8.2cm]{sp10_flow.png}\hfill 
  \includegraphics[width=8.2cm]{sp80_flow.png}\\
  \text{$S_p = 10$ мм} \rule{55mm}{0mm} \text{$S_p = 80$ мм} \\
  \caption{Линии тока и изолинии давления изоповерхности давления на внешней поверхности 
  ступицы и в разгрузочном отверстии}
  \label{fig3:31}
\end{figure}

\begin{figure}[!b]
  \centering
  \includegraphics[width=12cm]{geom_p12.png}\\[-5mm]
  \caption{Точки 1 и 2 для расчета $\Delta p$}
  \label{fig3:32}
\end{figure}

\clearpage
\begin{figure}[!t]
  \centering
  \includegraphics[width=12cm]{count_eksp_UpDomain.png}\\[-5mm]
  \caption{Сравнение рассчитанного коэффициента расхода (\textcolor{red}{$\circ$}) с экспериментальными 
  данными ({\textbf{---}}). Цифрами показано отношение $S_p/d_p$}
  \label{fig3:33}
\end{figure}
\rule{0mm}{20mm}

