\documentclass[14pt]{extreport} % , draft
\usepackage[a4paper, mag=1000, 
left=23mm, right=12mm, top=20mm, bottom=20mm]{geometry}    % Отступы страниц по ГОСТ Р 7.0.11-2011
\usepackage[utf8]{inputenc}                                % Кодировка
\usepackage[english,russian]{babel}                        % Переносы
\usepackage{cmap}                                          % Функция поиска по кириллице
\usepackage[pdftex,unicode,colorlinks,linkcolor=black,citecolor=black!60!green,urlcolor=blue]{hyperref}
% Гиперссылки, убрал все рамки.
\usepackage[all]{hypcap}                                   % Гиперссылки на caption
\usepackage{indentfirst}                                   % Отступ первого абзаца
\usepackage{amsmath,amssymb,amsfonts}                      % Символы АМО
\usepackage{cite}                                          % Цитирование
\usepackage{graphicx}                                      % Графика
\usepackage{ccaption}                                      % Подписи к рисункам
\usepackage{color}                                         % Выделение текста цветом
\usepackage{xcolor}                                        % Работа с цветом
\usepackage{hhline}                                        % Для верстки линий в таблицах

\usepackage{remreset}
\makeatletter
\@removefromreset{equation}{chapter}
\@removefromreset{figure}{chapter}
\@removefromreset{table}{chapter}
\renewcommand{\chapter}{
\cleardoublepage
% \thispagestyle{plain}%
\global\@topnum=0
\@afterindentfalse
\secdef\@chapter\@schapter}
\makeatother 

\usepackage{fancyhdr} % пакет для установки колонтитулов
\pagestyle{fancy} % смена стиля оформления страниц
\fancyhf{} % очистка текущих значений
\fancyhead[C]{\thepage} % установка верхнего колонтитула
\renewcommand{\headrulewidth}{0pt} % убрать разделительную линию

\renewcommand{\theequation}{\arabic{equation}}             % Сквозная нумерация формул
\renewcommand{\thefigure}{\arabic{figure}}                 % Сквозная нумерация рисунков
\renewcommand{\thetable}{\arabic{table}}                   % Сквозная нумерация таблиц

\graphicspath{{figures/}}                                  % Путь до рисунков
\captiondelim{~---~}                                       % Разделитель в подписях к рисункам и таблицам
\renewcommand{\baselinestretch}{1.31} %1.31                % Интервал
\sloppy                                                    % Запрет переполнений

\author{Авдюшенко А.\,Ю.}
\begin{document}
  \renewcommand{\contentsname}{Оглавление}
  \renewcommand{\chaptername}{Глава}
  \renewcommand{\bibname}{Список литературы}
  \renewcommand{\figurename}{Рисунок}
  \renewcommand{\tablename}{Таблица}

  \renewcommand{\leq}{\leqslant} % красивый знак <=
  \renewcommand{\geq}{\geqslant} % красивый знак >=

\parindent=1.25cm                                          % Абзацный отступ равен пяти знакам

%\renewcommand{\rmdefault}{ftm} % Переключение на шрифт документа Times New Roman (пакет pscyr)

  % Диссертация
  %!TEX root = dissertation.tex
\begin{titlepage}

\ \vspace{-5mm}
\begin{center}
  ФЕДЕРАЛЬНОЕ ГОСУДАРСТВЕННОЕ БЮДЖЕТНОЕ УЧРЕЖДЕНИЕ \\
  НАУКИ ИНСТИТУТ ВЫЧИСЛИТЕЛЬНЫХ ТЕХНОЛОГИЙ  \\
  СИБИРСКОГО ОТДЕЛЕНИЯ РОССИЙСКОЙ АКАДЕМИИ НАУК \\
\end{center}
\large
\vspace{10mm}
\begin{flushright}
  \textit{На правах рукописи}
\end{flushright}

\vspace{10mm}
\begin{center}
\textbf{ Авдюшенко Александр Юрьевич}

\vspace{10mm}
\textbf{НОВЫЕ ЧИСЛЕННЫЕ МОДЕЛИ \\ \vspace{-3mm}
        ГИДРОДИНАМИКИ ТУРБОМАШИН}

\vspace{10mm}
05.13.18 --- <<Математическое моделирование, \\ \vspace{-3mm}
численные методы и комплексы программ>>

\vspace{10mm}
Диссертация на соискание ученой степени \\ \vspace{-3mm}
кандидата физико-математических наук
\end{center}

\vspace{10mm}
\begin{flushright}
  \vbox{Научный руководитель:\\ \vspace{-3mm}
        доктор физ.-мат. наук, профессор\\ \vspace{-3mm}
        Черный Сергей Григорьевич}
\end{flushright}

\vfill
\begin{center}
  Новосибирск --- 2014
\end{center}
\end{titlepage}
  \setcounter{page}{2}
  \tableofcontents
  %!TEX root = dissertation.tex
%\pagestyle{plain}
\chapter*{Введение}
\addcontentsline{toc}{chapter}{Введение}%
\setcounter{section}{0}

Диссертационная работа посвящена разработке новых численных моделей для адекватного описания трехмерных 
течений в гидротурбинах при переходных режимах, в уплотнениях и полостях гидротурбин и в проточных частях с 
нестандартными элементами.

\textbf{Актуальность темы исследований.} 
Технологии новых и возобновляемых источников энергии включены в перечень критических технологий 
Российской Федерации~\cite{prt_RF}. В 2010 г. гидроэнергетика обеспечивала около 20\% мирового конечного 
потребления энергии от возобновляемых источников, и этот процент неуклонно растет~\cite{GSR_12}.

В связи с этим исследование динамики вязкой несжимаемой жидкости в проточном тракте (ПТ) 
гидроэлектростанции (ГЭС) и ее воздействия на элементы гидротурбины (ГТ) представляет значительный интерес. 
Как правило, лабораторные и натурные экспериментальные исследования сложны и дорогостоящи, поэтому задача 
построения эффективных численных моделей и изучения на их основе нестационарных турбулентных течений в ПТ ГЭС
является актуальной. Создание численной модели подразумевает математическую формулировку задачи, 
разработку, обоснование и тестирование численных алгоритмов и их реализацию в виде комплекса программ.

Течения жидкости в ПТ ГЭС подразделяются на установившиеся и переходные.
К установившимся относятся течения при неизменных значениях расхода жидкости, частоты
вращения рабочего колеса (РК) и нагрузки на вал РК. Эти течения являются стационарными или
периодически нестационарными и достаточно точно моделируются в настоящее 
время~\cite{Cher,Rupr,Vunen1,Smir,Kuro}.

Переходные течения возникают при переходе ГТ из одного фиксированного режима работы в другой, 
вызванном регулировкой открытия направляющего аппарата (НА) или изменением нагрузки на вал РК. 
Такие течения являются существенно нестационарными и характеризуются сильными колебаниями расхода во 
времени. Это приводит к динамическому изменению давления в проточном тракте, проявляющемуся в форме 
гидравлического удара. Моделирование таких течений актуально в связи с частыми изменениями
мощности электросети (как суточными, так и в течение дня), приводящими к необходимости 
изменения режима работы ГЭС. 

До последнего времени для исследования переходных процессов в ГЭС использовалась одномерная 
гидроакустическая теория. Она основана на гиперболической системе уравнений сохранения массы и импульса для 
сжимаемой жидкости \cite{Jmud,krivch,Nicol}.
В рамках этого подхода водовод и элементы гидротурбины
моделируются набором осесимметричных труб заданной геометрии, переменными являются скорость и 
давление жидкости. В России такие модели получили развитие в работах 
\cite{Popov,AleksMon,KuiPilZah,Samoil,GorKurBook,GorKurAtlas,KurTurb,Kur93,OkulPil,ArkOkulPil,Kur2013}. 
В \cite{Popov} рассмотрены вопросы теории и расчета нестационарных гидромеханических процессов, 
изложены методы построения одномерных моделей неустановившегося движения рабочих жидкостей в напорных 
трактах, приведены примеры приложения предлагаемых методов к расчету переходных процессов, колебаний 
и устойчивости гидро- и пневмосистем. В монографии \cite{AleksMon} систематизированы и обобщены сведения 
о концентрированных вихрях, наблюдаемых в 
природе и технике. Описаны модели вихревых структур, применяемые при интерпретации экспериментальных данных и
являющиеся базовыми для развития теоретических и численных подходов к изучению вихревых течений. Применение 
представленных в \cite{AleksMon} подходов привело к совершенствованию моделей для описания нестационарных 
явлений в гидротурбинах. В работе \cite{KuiPilZah} оценено влияние характера распределения завихренности и 
размера кавитационной полости за РК ГТ на коэффициенты характеристического уравнения, решение которого
дает собственные частоты и инкременты колебаний давления. 

Первые исследования вопроса о собственных акустических колебаниях в решетчатых областях 
в нестационарном потоке представлены в \cite{Samoil,GorKurBook,GorKurAtlas}. В работах \cite{KurTurb,Kur93}
разработанные подходы были применены к моделированию акустических колебаний в ПТ ГТ. Выводы, сделанные 
в работе \cite{Kur93} практически совпали с экспериментальными и теоретическими исследованиями уровня 
пульсаций давления в ПТ ГТ в зависимости от режима эксплуатации, представленными в \cite{OkulPil,ArkOkulPil}.
Согласно работе \cite{Kur2013} это привело к запрету эксплуатации турбоагрегатов на форсированных 
режимах работы.

Тем не менее существенным недостатком одномерного подхода является 
необходимость {\it априори} знать универсальную характеристику гидротурбины (УХ ГТ), используемую 
для задания условий на границах труб и коэффициентов уравнений.
Сложная геометрия спиральной камеры, решеток статора, НА, РК и отсасывающей трубы (ОТ) представляется 
в таком подходе косвенно через УХ ГТ, также в нем невозможно учесть особенности трехмерного нестационарного
турбулентного потока (вихри, рециркуляционные и кавитационные зоны).

В последние годы появляются работы, в которых переходные течения моделируются в трехмерных 
постановках. В \cite{Li,Liu1,Liu2} с использованием пакета FLUENT 6.3 
проведено моделирование выхода в разгон, сброса нагрузки в турбинном режиме и сброса нагрузки в 
насосном режиме работы гидротурбины, но без учета явления гидроудара. 
В \cite{Nicolle} проведено моделирование переходного процесса пуска в турбинный режим 
с использованием пакета CFX 13.0, также без учета гидроудара. 

В настоящей работе для решения задачи моделирования течения при переходных режимах работы ГЭС 
предлагается оригинальная гибридная модель, в которой прохождение гидроудара в длинном водоводе описывается 
одномерной моделью упругого гидроудара, а в области турбины -- в <<жестком>> приближении, но в 
аккуратной пространственной геометрической и гидродинамической постановке.

Следующая проблема, рассмотренная в диссертации, связана с неизбежными зазорами между вращающимися 
(рабочее колесо) и неподвижными частями ГТ (рисунок~\ref{fig:01}), приводящими к протечкам жидкости и 
потерям энергии. Для снижения объемных потерь в зазорах устанавливаются уплотнения. 
Кольца уплотнений крепятся на верхнем и нижнем ободах или на камере РК. Ступица 
РК имеет разгрузочные отверстия, расположенные за выходными кромками лопастей. Через них полость
над РК сообщается с основным проточным трактом.

\begin{figure}[ht!]
  \label{fig:01}
  \centering
  \includegraphics[width=14.0cm]{fig_RK.png}%  \\
  \caption{Рабочее колесо с зазорами, полостями и разгрузочными отверстиями}
\end{figure}

Кроме уменьшения потерь мощности в турбине за счет ограничения величин протечек воды между вращающимися и 
неподвижными частями, уплотнения обеспечивают также снижение осевой и радиальной нагрузок (ОРН), 
действующих на РК~\cite{granovsky}. В настоящее время для предсказания ОРН и потерь мощности в уплотнениях 
и полостях в основном используются приближённые инженерно-эмпирические методики, в которых грубо
представлена сложная трехмерная конфигурация зазоров и не учитываются особенности трёхмерного 
нестационарного потока жидкости в гидротурбине. 
Поскольку величины зазоров и характерные размеры областей протечек на три порядка меньше характерных
размеров РК, совместный расчет трехмерного течения во всей гидротурбине в полной постановке с учетом зазоров 
и областей протечек потребует большого объема вычислений и как следствие, приведет к абсолютно 
нереалистичным временам счета. Поэтому в диссертации разработан комбинированный метод 
моделирования трехмерных течений в полостях и уплотнениях, сопряженный с расчетом в основном проточном тракте 
гидротурбины посредством инженерно-эмпирической методики. 

Третьей задачей, поставленной и решенной в диссертационной работе, было создание численной модели течения в
гидротурбине с кольцевым затвором. Для обеспечения надежности работы гидроэнергетического оборудования 
высоконапорных ГЭС, а также для повышения эффективности работы гидроаккумулирующих электростанций, 
работающих при больших циклических нагрузках, вместо предтурбинных дисковых или шаровых затворов устанавливают 
кольцевые затворы. 
Затвор встроен в проточную часть между колоннами статора и направляющим аппаратом гидротурбины. 
Он предназначен для работы в оперативном и аварийном режимах и приводится в действие сервомоторами, 
расположенными на крышке турбины.

Одна из серьезных проблем, возникающих при эксплуатации кольцевого затвора -- его заклинивание. 
Поэтому при его проектировании важным является учет осевых $F_z$ и радиальных $(F_x, F_y)$ сил, 
действующих на затвор со стороны нестационарного потока воды. 
От их величин зависит расчет мощности сервомоторов и принятие мер по 
предотвращению заклинивания затвора~\cite{shevch}. Важным является также расчет зависимости расхода от степени 
закрытия затвора. От него, в частности, зависит повышение давления в спиральной камере вследствие гидроудара. 
При этом известно, что кольцевой затвор, в отличие от классических дисковых и шаровых 
затворов, обладает рабочим процессом пониженной нестационарности.

Таким образом, несмотря на многолетние исследования, в моделировании течений в ПТ ГЭС указанные проблемы 
остаются недостаточно изученными. Не выяснено влияние гидроудара и изменения скорости вращения рабочего 
колеса на структуру потока. Недостаточно исследованы ОРН на РК и потери энергии, 
возникающие из-за протечек воды через зазоры между вращающимися и неподвижными частями гидротурбины, 
а также динамические нагрузки на элементы гидротурбины и характеристики нестационарного потока при 
наличии кольцевого затвора. 

Существующие численные трехмерные модели гидродинамики турбин не позволяют одновременно учитывать 
движение лопаток направляющего аппарата, изменение скорости вращения рабочего колеса и расхода жидкости, 
явление гидроудара. Также отсутствуют развитые трехмерные модели течений жидкости в зазорах и при наличии 
кольцевого затвора. В связи с этим разработка новых математических моделей течений в ПТ ГЭС, 
учитывающих такие особенности, актуальна как в научном, так и в 
практическом плане. Решение этой задачи позволит сформировать более полные представления 
о структуре исследуемого потока и возможностях того или иного метода изменения характера течения, 
повышения коэффициента полезного действия или улучшения прочностных качеств гидротурбины.

\textbf{Цель исследования}~--- создание новых численных моделей для адекватного описания 
течений при переходных режимах работы ГТ, в уплотнениях и полостях между 
вращающимися и неподвижными частями, при наличии нестандартных элементов (кольцевой затвор).

\textbf{Объектом исследований} являются методы моделирования течений в проточных частях гидротурбин при 
переходных режимах работы, в полостях и уплотнениях, при наличии кольцевого затвора.

\textbf{Предметом исследований} являются нестационарные процессы и особенности течений в гидротурбинах и 
методов их моделирования в новых постановках.

\textbf{Основные задачи, решенные в ходе достижения поставленной цели.}
\begin{enumerate}
  \setlength{\itemsep}{1pt} \setlength{\parskip}{0pt} \setlength{\parsep}{0pt}
  \item[1.] Обобщен на подвижные сетки метод решения трехмерных уравнений Рейнольдса движения несжимаемой 
            жидкости, замкнутых $k-\varepsilon$ моделью турбулентности.
  \item[2.] Разработан метод совместного решения \\
            --- трехмерных уравнений Рейнольдса на подвижных сетках, \\
            --- уравнения вращения рабочего колеса, \\
            --- одномерных уравнений упругого гидроудара.
  \item[3.] Предложена новая постановка для численного моделирования течений в переходных
            режимах работы гидротурбин.
  \item[4.] Разработан метод моделирования течений в полостях и уплотнениях, сопряженный с методом 
            моделирования течения в основном проточном тракте гидротурбины.
  \item[5.] Создан метод расчета течения при наличии цилиндрического кольцевого затвора.
  \item[6.] Создан программный комплекс, реализующий построенные численные алгоритмы на многопроцессорных 
            вычислительных системах.
  \item[7.] Проведены верификация и валидация разработанных численных моделей.
  \item[8.] В новых постановках решены практически важные задачи определения динамических нагрузок 
            на элементы гидротурбины и давлений в нестационарном потоке.
\end{enumerate}

\textbf{На защиту выносятся} следующие результаты, соответствующие четырем пунктам паспорта 
специальности 05.13.18~--- математическое моделирование, численные методы и комплексы программ 
по физико"=математическим наукам.

\noindent
\textit{Пункт 1: Разработка новых математических методов моделирования объектов и явлений.}
\vspace{-2mm}
\begin{enumerate}
  \setlength{\itemsep}{1pt} \setlength{\parskip}{0pt} \setlength{\parsep}{0pt}
  \item[1.] Новый математический метод моделирования течений при переходных режимах работы 
            гидротурбин, объединяющий решение нестационарных трехмерных уравнений Рейнольдса на подвижных 
            сетках, уравнения вращения рабочего колеса и одномерных уравнений упругого гидроудара. 
            Новый метод моделирования течений воды через полости и уплотнения гидротурбин. 
\end{enumerate}

\noindent
\textit{Пункт 3: Разработка, обоснование и тестирование эффективных вычислительных методов с 
применением современных компьютерных технологий.}
\vspace{-2mm}
\begin{enumerate}
  \setlength{\itemsep}{1pt} \setlength{\parskip}{0pt} \setlength{\parsep}{0pt}
  \item[2.] Численный метод решения уравнений Рейнольдса на подвижных сетках и численный метод совместного 
            решения уравнений, описывающих переходные процессы в гидротурбинах, реализованные на 
            многопроцессорных вычислительных системах с применением технологии MPI.
\end{enumerate}

\noindent
\textit{Пункт 4: Реализация эффективных численных методов и алгоритмов в виде комплексов 
проблемно"=ориентированных программ для проведения вычислительного эксперимента.}
\vspace{-2mm}
\begin{enumerate}
  \setlength{\itemsep}{1pt} \setlength{\parskip}{0pt} \setlength{\parsep}{0pt}
  \item[3.] Программные комплексы CADRUN/2013 и CADRUN2/2013, 
            предназначенные для расчетов нестационарных трехмерных 
            турбулентных течений в гидротурбинах, созданные на основе предложенных методов и используемые для 
            проведения вычислительных экспериментов в ОАО~<<Силовые машины>> <<ЛМЗ>> в г.~Санкт-Петербурге.
\end{enumerate}

\noindent
\textit{Пункт 5: Комплексные исследования научных и технических проблем с применением современной 
технологии математического моделирования и вычислительного эксперимента.}
\vspace{-2mm}
\begin{enumerate}
  \setlength{\itemsep}{1pt} \setlength{\parskip}{0pt} \setlength{\parsep}{0pt}
  \item[4.] Численные исследования течений в переходных режимах: пуска в турбинный режим, регулировки 
            мощности и мгновенного сброса нагрузки. Основные закономерности формирования и 
            развития структуры нестационарных турбулентных течений несжимаемой вязкой среды при 
            обтекании подвижных лопаток НА и лопастей РК в переходных режимах работы радиально-осевых 
            гидротурбин. 

            С использованием нового комбинированного метода: 
            
            $\bullet$ обнаружено влияние высоты верхней области протечки на течение в разгрузочном 
            отверстии --- при высоте менее 10 мм в разгрузочном отверстии 
            формируется закрученное вихревое течение, гидравлическое сопротивление при этом увеличивается 
            в 4 раза; 
            
            $\bullet$ найдено соотношение радиальных сил --- при относительном эксцентриситете 
            рабочего колеса $\varepsilon > 0.6$ радиальные силы, действующие на лабиринтные уплотнения, 
            дают более 50~\% общей радиальной нагрузки, действующей на всё рабочее колесо гидротурбины. 
\end{enumerate}

\noindent
Таким образом, в соответствии с формулой специальности 05.13.18 в диссертации представлены 
оригинальные результаты одновременно из трех областей: математического моделирования, 
численных методов и комплексов программ.

\textbf{Научная новизна} выносимых на защиту результатов заключается в следующем. 

Впервые рассмотрена модель переходных процессов в проточной части ГЭС, одновременно 
учитывающая нестационарное трехмерное турбулентное течение 
несжимаемой жидкости в гидротурбине, движение лопаток НА, изменение скорости 
вращения РК и явление гидравлического удара в водоводе. 

Предложен новый численный метод совместного решения нелинейных систем уравнений, образующих
модель переходных процессов.

На основе разработанных эффективных численных методов создан оригинальный программный комплекс, 
допускающий использование в расчетах современных многопроцессорных систем.

Новым является разработанный комбинированный метод моделирования течений воды через 
полости и уплотнения гидротурбин, позволяющий определять величины протечек и осевых и радиальных нагрузок на 
рабочее колесо. 

\textbf{Достоверность и обоснованность} полученных результатов обеспечивается использованием в 
качестве основы моделирования фундаментальных законов механики жидкости и газа, динамики твердых тел и выбором 
теоретически обоснованных численных методов, а также подтверждается хорошим согласованием полученных 
численных результатов с экспериментальными данными и расчетными данными других исследователей.

\textbf{Практическая ценность} исследования заключается в возможности использования 
полученных результатов в ряде прикладных областей наукоемкого машиностроения и электроэнергетики для 
моделирования течений жидкости в проточных частях различных типов гидротурбин или других аэрогидродинамических 
установках (программные комплексы CADRUN/2013 и CADRUN2/2013, зарегистрированные в 
Роспатенте 25 января 2013 г., рег.~\No~2013611576 и \No~2013611580). 
Результаты диссертационной работы используются в исследованиях в ОАО~<<Силовые машины>> <<ЛМЗ>> 
в г.~Санкт-Петербурге, что подтверждает приложенный в конце диссертации акт об использовании научных 
результатов в практической деятельности.

Основные положения и результаты диссертации \textbf{докладывались и обсуждались} 
на объединенном научном семинаре ИВТ~СО~РАН <<Информационно"=вычислительные технологии 
(численные методы механики сплошной среды)>> под руководством академика~РАН Ю.\,И.~Шокина и 
профессора В.\,М.~Ковени, на объединенном научном семинаре ИВМиМГ и ССКЦ~СО~РАН <<Архитектура, системное и 
прикладное программное обеспечение кластерных суперкопьютеров>> под руководством профессора Б.\,М.~Глинского, 
на научном семинаре института теплофизики, а также на 11 всероссийских и международных конференциях:
\begin{enumerate}
  \item[1.] III~Международная конференция <<High Performance Computing 2013>> 
            (Киев, Украина, октябрь 2013).
  \item[2.] Международная конференция <<Вычислительные и информационные технологии в науке, 
            технике и образовании ВИТ-2013>> (Усть-Каменогорск, Казахстан, сентябрь 2013).
  \item[3.] XIII~Всероссийская конференция молодых ученых по математическому моделированию и информационным 
            технологиям (Новосибирск, октябрь 2012).
  \item[4.] XI~Всероссийская конференция «Краевые задачи и математическое моделирование» 
            (Новокузнецк, октябрь 2012).
  \item[5.] XXVI~Международная конференция <<IAHR Symposium on hydraulic machinery and systems>>
            (Пекин, Китай, август 2012).
  \item[6.] Международная конференция <<Современные проблемы прикладной математики и механики: теория, 
            эксперимент и практика>>, посвященная 90-летию Н.\,Н.~Яненко (Новоcибирск, июнь 2011).
  \item[7.] XVII~Международная конференция по вычислительной механике и современным прикладным 
            программным системам <<ВМСППС’2011>> (Алушта, май 2011).
  \item[8.] XXV~Международная конференция <<IAHR Symposium on hydraulic machinery and systems>> (Тимишоара, 
            Румыния, сентябрь 2010).
  \item[9.] VIII~Международная конференция по неравновесным процессам в соплах и струях <<NPNJ`10>> (Алушта, 
            май 2010).
  \item[10.]XLVII~Международная научная студенческая конференция <<Студент и научно-технический прогресс>>:
            Математика (Новосибирск, апрель 2009).
  \item[11.]XLVI~Международная научная студенческая конференция <<Студент и научно-технический прогресс>>:
            Математика (Новосибирск, апрель 2008).
\end{enumerate} 

Основные результаты диссертации \textbf{опубликованы} в 19 научных 
работах~\cite{my1,my1_2,my2,my3,my4,my5,my6,my7,my8,my9,my10,my11,my12,my13,my14,my15,my16,my17,my18}~(в 
скобках в числителе указан общий объем этого типа публикаций в печатных листах, в знаменателе~--- объем 
принадлежащий лично автору), в том числе  4~статьи~\cite{my1,my1_2,my2,my3}~(9.02/5.4) в 
периодических изданиях, рекомендованных ВАК для представления основных научных результатов диссертаций на 
соискание ученой степени доктора или кандидата наук, 8 публикаций в трудах международных и всероссийских
конференций~\cite{my4,my5,my6,my7,my8,my9,my10,my11}~(5.47/2.9), 2~свидетельства государственной 
регистрации программ для ЭВМ в Роспатенте~\cite{my12,my13}, 5~публикаций в тезисах международных и 
всероссийских конференций~\cite{my14,my15,my16,my17,my18}~(0.14/0.1).

Работа выполнена при финансовой поддержке Российского фонда фундаментальных исследований 
(грант \No 11-01-00475-а), Президиума СО РАН (Интеграционный проект СО РАН \No 130).

\textbf{Личный вклад соискателя.} Основные результаты диссертационной работы получены автором 
самостоятельно. Во всех работах~\cite{my1,my1_2,my2,my3,my4,my5,my6,my8,my9,my10,my11,my12,my13,my14,my16}, 
опубликованных в соавторстве, постановка задач выполнена совместно;
соискатель самостоятельно обобщил на подвижные сетки и реализовал в виде комплекса программ 
численный метод решения трехмерных уравнений Рейнольдса движения несжимаемой жидкости, 
разработал метод совместного решения уравнений Рейнольдса, уравнения вращения рабочего колеса 
и одномерных уравнений упругого гидроудара. Также автор 
принимал непосредственное участие в разработке комбинированного метода моделирования течения в полостях и 
уплотнениях, сопряженного с методом моделирования в основном проточном тракте гидротурбины, и в создании 
метода расчета течения при наличии цилиндрического кольцевого затвора. 
Адаптация численных алгоритмов для работы на многопроцессорных системах, их верификация и валидация, анализ 
полученных результатов выполнены автором самостоятельно. 

\textbf{Структура и объем диссертации.}
Диссертация состоит из введения, четырех глав, заключения, списка цитируемой литературы 
и двух приложений. Диссертация изложена на 145 страницах машинописного 
текста, включая 63 иллюстрации и 10 таблиц. Список цитируемой литературы 
содержит 88 наименований.

\section*{Содержание диссертации}

Во \textbf{Введении} обоснована актуальность выбранной темы исследований, сформулированы
цель работы и задачи, которые необходимо решить для ее достижения. 

Отмечены работы С.\,Г.~Черного, 
Е.\,М.~Смирнова, В.\,В.~Риса, И.\,М.~Пылева, А.\,В.~Захарова, A.~Ruprecht, T.\,C.~Vu, S.~Kurosawa и др., 
посвященные численному  моделированию установившихся течений в гидротурбинах.  
Проводится обзор подходов к построению одномерных моделей неустановившегося движения рабочей жидкости в 
напорных трактах, представленных в работах Г.\,И.~Кривченко, В.\,Б.~Курзина, С.\,В.~Алексеенко, 
П.\,А.~Куйбина, В.\,Л.~Окулова, Д.\,Н.~Горелова и др.

Приведены основные результаты и положения выносимые на защиту. 
Также во введении представлены сведения о научной новизне, практической значимости, 
апробации результатов и основных публикациях. Затем кратко изложено содержание диссертации по главам.

В \textbf{Главе~1} строится численный метод расчета трехмерных турбулентных течений несжимаемой 
жидкости на подвижных структурированных сетках.

В разделе~\ref{s:11} приведены нестационарные трехмерные уравнения Рейнольдса, записанные как в 
дифференциальной форме, так и в форме интегральных законов сохранения массы и количества движения для 
подвижного объема. Далее формулируется условие геометрической консервативности (УГК), являющееся одним из 
главных при дискретизации уравнений, записанных в криволинейных координатах, зависящих от времени. Если 
при дискретизации уравнений Рейнольдса не удовлетворить УГК, 
то на задаче с решением в виде постоянного однородного потока численный метод на подвижной сетке даст 
возмущенное, отличное от начальных данных решение.

В разделе~\ref{s:12} предлагается метод решения уравнений Рейнольдса на подвижных сетках, основанный на 
концепции искусственной сжимаемости, неявной конечно-объемной аппроксимации и приближенной $LU$-факторизации
линеаризованной системы разностных уравнений.  
Параллельно с проводимой дискретизацией уравнений рассматриваются известные подходы 
построения численных методов на подвижных сетках,  представленные в работах J.\,G.~Trulio, 
P.\,D.~Thomas, I.~Demirdzic, M.~Peric, W.~Shyy, M.~Lesoinne, C.~Farhat, B.~Koobus, C.~Forster, 
M.~Engel, M.~Griebel, Н.\,О.~Зайцева, Н.\,А.~Щура, К.\,Н.~Волкова и др.
Отмечается, что предложенный в диссертации метод, 
в отличие от имеющихся в литературе, приводит к точному выполнению УГК 
в трехмерном случае на дискретном уровне. Это достигается аккуратным разбиением всей расчетной области на 
каждом шаге по времени на непересекающиеся тетраэдры и аппроксимацией скоростей движения граней 
ячеек, специально согласованным с разбиениями способом.

В разделе~\ref{s:13} обосновывается постановка краевых условий на подвижной твердой границе. 
В разделе~\ref{s:14} представлены уравнения стандартной $k - \varepsilon $ модели турбулентности и их 
дискретизация на подвижных сетках. 

Далее в разделе~\ref{s:15} приводятся результаты решения двух модельных задач с подвижными границами: 
расчет однородного потока на подвижной сетке и движение кругового цилиндра в покоящемся однородном поле 
несжимаемой вязкой жидкости. Движение цилиндра задано посредством перемещения со временем 
его границы и нормально связанной с ней системы координат. 
Анализ результатов показал, что метод корректно 
воспроизводит все характеристики и может быть использован для моделирования течений
в областях с подвижными границами в различных задачах вычислительной гидродинамики.

В разделе~\ref{s:16} строится алгоритм параллельной реализации разработанного численного метода
на многопроцессорных вычислительных системах. 
Обосновывается использование геометрического распараллеливания, 
заключающегося в декомпозиции всей расчетной области на блоки, каждый из которых рассчитывается на отдельном 
ядре многопроцессорной вычислительной системы. Коммуникации между процессорами осуществляются с использованием 
стандарта MPI. Приводятся результаты распараллеливания расчёта в полной постановке, 
полученные на кластерах ИВЦ НГУ и ССКЦ. Из проведенных исследований 
следует, что при использовании 61 счетного ядра время решения задачи моделирования нестационарного 
трехмерного турбулентного течения во всей гидротурбине на сетке с общим количеством ячеек 
около 2.5~млн сокращается более чем в 15~раз и составляет от 1 до 2 дней, что вполне приемлемо 
для практического применения.

В \textbf{Главе~2} для расчета течений в переходном режиме работы ГЭС 
предлагается оригинальная гибридная численная модель, объединяющая решение нестационарных 
уравнений Рейнольдса на подвижных сетках, уравнения вращения РК 
как твёрдого тела и одномерных уравнений распространения упругого гидроудара в водоводе. 

В разделе~\ref{s:20} описаны проблемы, возникающие при моделировании переходных процессов в гидротурбинах.  

В разделе~\ref{s:21} приведена система уравнений гибридной модели переходных процессов, отмечены связывающие
их переменные. 

В разделе~\ref{s:22} рассмотрены краевые условия при совместном расчете в новой постановке 
водовод-гидротурбина. Особое внимание уделено взаимообмену параметрами течения между одномерной областью 
водовода и трехмерной областью гидротурбины.

В разделе~\ref{s:23} представлена численная реализация разработанной модели. 
Уравнения каждой из подмоделей решаются эффективными устойчивыми численными методами 
и замыкаются корректными краевыми условиями.  
В результате разработан надежный инструмент моделирования переходных трехмерных течений в гидротурбинах, 
позволяющий проводить вычисления за 1-2 дня на сетках, содержащих около 2.5 млн ячеек.

В разделе~\ref{s:24} приводятся результаты расчетов переходных процессов: пуска в турбинный режим, уменьшения 
мощности, мгновенного сброса нагрузки. Показано хорошее согласование с имеющимися экспериментальными данными 
по частоте вращения рабочего колеса, величине возникающего гидроудара и давлению в точках проточного тракта.
Применение разработанной гибридной модели позволило проанализировать структуру потока в ПТ ГТ 
при переходных режимах работы.

\textbf{Глава~3} посвящена построению метода моделирования течений в полостях и уплотнениях, сопряженному с 
методом моделирования течения в основном проточном тракте гидротурбины. 

В разделе~\ref{s:31} дана общая постановка задачи. В разделе ~\ref{s:32} приведен обзор существующих методик 
определения осевых и радиальных нагрузок на РК гидротурбины, вызванных нестационарным течением рабочей 
жидкости в его межлопастных каналах, а также протечками в лабиринтных уплотнениях, полостях и разгрузочных 
отверстиях. Отмечено, что для простых геометрий приближенные инженерно-эмпирические
методики дают приемлемые результаты при минимальных вычислительных затратах.

В разделе~\ref{s:34} строится комбинированный метод определения нестационарных осевых и радиальных 
нагрузок на рабочее колесо, 
основанный на расчетах трехмерного течения несжимаемой жидкости в основной проточной части и областях протечек 
гидротурбины. Метод позволяет рассчитывать осевые и радиальные нагрузки, вызванные следующими факторами: 
неравномерностью потока в спиральной камере и статоре, ротор-статор взаимодействием, неравномерностью потока 
за РК вследствие нестационарности потока в конусе ОТ (вихревой жгут и т.д.), несоосностью статора и ротора, 
изгибом вала ротора. Кроме того, метод позволяет рассчитывать расход жидкости через зазоры между вращающимися
и неподвижными частями гидротурбины. Потери давления в лабиринтных уплотнениях, щелях и разгрузочных 
отверстиях вычисляются непосредственно с использованием турбулентной модели, тем самым не требуется 
эмпирическая информация о гидравлических сопротивлениях этих элементов.

В разделе~\ref{s:35} усовершенствована инженерно-эмпирическая методика~\cite{lomakin} для расчета радиальных
нагрузок, действующих на лабиринтное уплотнение вследствие смещения оси вращения ротора. В модификации 
учтены влияние вращения ротора на коэффициент сопротивления узкой части лабиринта, сопротивление 
ячеек расширения, зависимость коэффициента сопротивления узкой части и ячеек расширения от переменного зазора 
между статором и ротором. Это позволило получить количественное соответствие результатов 
расчетов по модифицированной методике с расчетом по трехмерной модели течения жидкости в ЛУ.

В разделе~\ref{s:36} представлены результаты расчетов осевых и радиальных нагрузок на РК по предложенной 
методике. Показано, что при относительном эксцентриситете $\varepsilon < 0.5$ прецессия ротора не оказывает 
заметного влияния на модуль радиальной силы. Влияние прецессии растет по мере увеличения эксцентриситета. 
Так, при $\varepsilon = 0.9$ в случае $\Omega = \omega$ радиальная сила на 30\% больше, чем в случае простой 
несоосности ротора и статора. Показано, что при $\varepsilon > 0.6$ радиальные силы, действующие на 
лабиринтные уплотнения, дают более 50\% итоговой радиальной нагрузки, действующей на все рабочее колесо 
гидротурбины.

В \textbf{Главе~4} строится численная модель течения воды в гидротурбине с затвором. 

В разделе~\ref{s:41} приведена постановка задачи. Выделена особенность геометрии 
спиральной камеры -- <<зуб>>, приводящая к сильной окружной неравномерности потока. 
Затвор, перекрывающий ПТ ГТ, задается как твердая стенка в выходном сечении статора (модель тонкого затвора). 
В этом случае не учитывается форма оголовка затвора и область течения под ним. Другой подход задания 
затвора заключается в добавлении нового элемента ПТ --- кольцевой области между оголовком затвора 
и нижней крышкой НА. В таком случае область течения под затвором моделируется точно. 

В разделе~\ref{s:42} представлены численные реализации обеих моделей кольцевого затвора: 
бесконечно тонкого и затвора реальной толщины. Для реализации второй модели в программный комплекс 
добавлены процедуры построения сеток в новой области под затвором и передачи данных из нее 
в каналы статора и НА.

В разделе~\ref{s:43} описаны входные и выходные условия. Неизвестная априори зависимость расхода от 
степени закрытия затвора рассчитывается в экономичной циклической постановке, в которой расчетная область 
состоит из одного канала статора, одного межлопаточного и одного межлопастного канала. На входе и выходе из 
расчетной области фиксировались полные энергии потока. В расчетах в полной постановке, содержащей все каналы 
элементов ГТ, задавался расход, полученный в циклической постановке.

В разделе~\ref{s:44} приведены зависимости выталкивающей силы $F_z$ и расхода от степени закрытия затвора $s$.
Отмечено, что при $s \geqslant 0.9$ в модели затвора реальной толщины выталкивающая сила становится 
отрицательной, т.е. затвор втягивается в проточную часть.

В разделе~\ref{s:45} приведены результаты расчета крутящих моментов лопаток НА 
и радиальных сил $(F_x,\ F_y)$, действующих на затвор. 
Показано, что амплитуды пульсаций сил резко возрастают, при $s \geqslant 0.7$. 
Обнаружено сильное влияние окружной неравномерности потока, вызванной <<зубом>> спиральной камеры, на моменты 
лопаток НА, расположенных в окрестности <<зуба>>, при небольших степенях закрытия 
затвора. При $s \geqslant 0.7$ влияние <<зуба>> на моменты исчезает.

В \textbf{заключении} сформулированы основные результаты диссертационной работы.

В \textbf{приложении А} приведены матрицы Якоби невязкого и вязкого потоков, матрицы правых  
собственных векторов, а также $RDL$-разложение матрицы Якоби невязкого потока. В \textbf{приложении В} 
представлены два свидетельства о государственной регистрации программ для ЭВМ и акт об использовании научных 
результатов диссертационной работы в филиале ОАО <<Силовые машины>> <<ЛМЗ>> в г.~Санкт-Петербурге.  

Автор выражает глубокую благодарность и признательность своему научному руководителю Черному
Сергею Григорьевичу за постановки интересных задач, 
всестороннюю поддержку и постоянное внимание в ходе выполнения работы. 
Успешному выполнению работы способствовали неоценимая 
помощь Чиркова Дениса Владимировича, ценные критические замечания Банникова Дениса Викторовича, 
Ешкуновой Ирины Федоровны и Лапина Василия Николаевича. 
Автор отдельно благодарит Есипова Дениса Викторовича, Астракову Анну Сергеевну и Горобчука Алексея 
Геннадьевича за создание творческой атмосферы и возможность постоянного обсуждения научных задач. 
В заключение хочется поблагодарить весь коллектив Института вычислительных технологий СО РАН за поддержку
и благоприятную рабочую обстановку.

  %!TEX root = dissertation.tex
%\pagestyle{plain}
\chapter*{Глава 1. Метод численного решения трехмерных задач динамики несжимаемой жидкости 
на подвижных сетках}
\label{s:1}
\setcounter{chapter}{1}
\addcontentsline{toc}{chapter}{Глава~\thechapter~ Метод численного решения трехмерных задач 
динамики несжимаемой жидкости на подвижных сетках}
\setcounter{section}{0}

В~\cite{Cher} для моделирования установившихся (стационарных или периодически нестационарных) течений в 
проточном тракте (ПТ) гидротурбины при неподвижных лопатках направляющего аппарата предложен 
численный метод решения трехмерных уравнений несжимаемой жидкости, хорошо зарекомендовавший себя в работе на 
фиксированных сетках. В случае переходных течений, возникающих вследствие изменения положения
лопаток направляющего аппарата (НА), требуется обобщение метода на подвижные сетки, 
подстраивающиеся со временем под положения жестких 
границ расчетных областей. Выписать основные уравнения движения несжимаемой жидкости в криволинейных 
координатах, зависящих от времени, не представляет трудности. Однако проведение правильной
дискретизации этих уравнений для построения эффективного численного метода требует выполнения определенных 
условий. Главным из них является условие геометрической консервативности (УГК), которое заключается в 
сохранении без какого-либо возмущения постоянного однородного потока при расчете его на двигающейся сетке.

В диссертации метод~\cite{Cher} обобщается на задачи с подвижными сетками. Параллельно с проводимой
дискретизацией уравнений рассматриваются известные подходы построения численных методов на подвижных сетках, 
предложенные в \cite{trulio,thomas,demper,shyy,lesoin,koobus,forst,engel,zaits,volkov}. Осуществляется их 
сравнительный анализ и сопоставление с предлагаемым автором методом.

Отметим преимущества представленного метода при решении задач на подвижных сетках. 
Во-первых, в методе сохранены все важные свойства 
исходного численного алгоритма моделирования течений на фиксированных сетках. Для высокой 
экономичности численного решения трехмерных нестационарных уравнений Рейнольдса несжимаемой жидкости в 
реальных аэрогидродинамических установках, имеющих сложные многосвязные проточные части, в алгоритме 
использован максимально устойчивый и в то же время достаточно просто реализуемый неявный метод конечных 
объемов. Экономичность алгоритма повышается тем, что сегменты, в которых находится решение, строятся 
топологически эквивалентными параллелепипедам с регулярными сетками в них. Благодаря этому, а также введению 
фиктивных слоев сетки алгоритмы остаются однородными во всем сегменте, в том числе и у его границ. Обращение 
неявных операторов посредством попеременно-треугольного метода становится в данном случае крайне
простым. Сложная область проточного тракта с помощью созданного аппарата геометрического моделирования 
покрывается указанными сегментами. При этом используется экономичный метод задания и хранения в 
памяти компьютера типов граничных условий для каждого сегмента. Метод позволяет задавать произвольные 
граничные условия на различных участках границы и организовывать оперативный обмен данными между сегментами 
при минимальных затратах памяти и времени выполнения операций. Проводимая естественным образом сегментация 
проточного тракта аэрогидродинамической установки позволяет, с одной стороны, эффективно находить решения 
уравнений в каждом сегменте, с другой~--- строить в каждом сегменте достаточно качественные сетки, 
независимые от сеток соседних сегментов.

Помимо сохранения свойств исходного алгоритма, новый метод дополнен 
важным для работы на подвижных сетках свойством точного выполнения УГК.

\section{Основные уравнения в случае подвижного конечного объема}
\label{s:11}
\subsection{Дифференциальная форма записи в декартовой системе координат}
\label{s:111}
Течение вязкой несжимаемой жидкости в проточном тракте ГТ, как и в~\cite{Cher}, описывается нестационарными 
трехмерными уравнениями Рейнольдса, имеющими в декартовой системе координат
$\left( {x,\,y,\,z} \right) = \left( {x_1 ,\,x_2 ,\,x_3 } \right)$ следующий вид (по повторяющимся индексам 
предполагается суммирование $\sum\limits_{j=1}^3$):
\begin{equation}
  \label{1} 
  \frac{{\partial w_j }}{{\partial x_j }} = 0,
\end{equation}
\begin{equation}
  \label{2} 
  \frac{{\partial w_i }}{{\partial {\kern 1pt} t}} +
  \frac{{\partial {\kern 1pt} w_i w_j }}{{\partial {\kern 1pt} x_j}} + 
  \frac{{\partial {\kern 1pt} p}}{{\partial {\kern 1pt} x_i }} = 
  \frac{{\partial {\kern 1pt} }}{{\partial {\kern 1pt} x_j}}
  \left[ {\nu _{\text{eff}} \left( {\frac{{\partial {\kern 1pt} w_i }}{{\partial {\kern 1pt} x_j }} + 
  \frac{{\partial {\kern 1pt} w_j }}{{\partial {\kern 1pt} x_i }}} \right)} \right] + f_i,
  \ \ \ i = 1,2,3,
\end{equation}
где $ w_1 ,\ w_2 ,\ w_3$~--- компоненты вектора скорости; $p = p_c + \cfrac{2}{3}k$;  $p_c$~--- 
гидростатическое давление, деленное на плотность жидкости; $k$~--- кинетическая энергия турбулентности;
$f_1=x_1\omega^2+2w_2\omega$; $f_2=x_2\omega^2-2w_1\omega$; $f_3=g$~--- ускорение свободного падения;
$\omega$~--- угловая скорость вращения рабочего колеса вокруг оси $x_3$;  $\omega=0$ в остальных элементах 
проточной части.

Величина $\nu _{\text{eff}}$\, есть сумма молекулярной $\nu$ и турбулентной $\nu_t$ вязкостей
\begin{equation}
  \label{3} 
  \nu _{\text{eff}} = \nu  + \nu _t.
\end{equation}
Турбулентная вязкость $\nu_t$ определяется по стандартной $k - \varepsilon $ модели 
турбулентности~\cite{sharma}.

Для удобства дальнейшей работы с уравнениями запишем их в виде векторного уравнения
\begin{equation}
  \label{4} 
  {\bf{R}}^t {\bf{Q}}_t  + {\bf{E}}_x  + {\bf{G}}_y  + {\bf{H}}_z  = {\bf{F}},
\end{equation}
где ${\bf{Q}} = \left( {p,w_1 ,w_2 ,w_3 } \right)^T$, ${\bf{R}}^t =
\text{diag}\left( {0,1,1,1} \right), {\bf{F}} = \left( {0,\, f_1 ,\, f_2 ,\, f_3 } \right)^T$,
\begin{gather}
  \label{5} 
  {\bf{E}} = \left( \begin{array}{l}
   w_1  \\
   w_1^2  + p - \tau _{11}  \\
   w_1 w_2  - \tau _{21}  \\
   w_1 w_3  - \tau _{31}  \\
  \end{array} \right),\ {\bf{G}} = \left( \begin{array}{l}
   w_2  \\
   w_1 w_2  - \tau _{12}  \\
   w_2^2  + p - \tau _{22}  \\
   w_2 w_3  - \tau _{32}  \\
  \end{array} \right),\ {\bf{H}} = \left( \begin{array}{l}
   w_3  \\
   w_1 w_3  - \tau _{13}  \\
   w_2 w_3  - \tau _{23}  \\
   w_3^2  + p - \tau _{33}  \\
  \end{array} \right),\  \\
  \tau _{ij}  = \nu _{\text{eff}} \left( {\frac{{\partial w_i }}{{\partial x_j }} + \frac{{\partial w_j }}{{\partial x_i }}}
  \right).
\end{gather}

Перейдем от дифференциального векторного уравнения \eqref{4} к интегральному, выполняющемуся на произвольном 
подвижном объеме.

\subsection{Интегральные законы сохранения для подвижного объема}
\label{s:112}
Интегрируя уравнение \eqref{4} по произвольному подвижному объему $V(t)$  с границей $\partial V(t)$,
ориентация которой задается внешней единичной нормалью ${\bf{n}}$, и применяя теорему 
Гаусса"=Остроградского, получим
\begin{equation}
  \label{6} 
  {\bf{R}}^t \int\limits_{V(t)}
  {\frac{{\partial {\bf{Q}}}}{{\partial t}}dV}  +
  \displaystyle\oint\limits_{\partial V(t)} {\bf{K}} d{\bf{S}} =
  \int\limits_{V(t)} {{\bf{F}}dV},
\end{equation}
где $ d{\bf{S}} = {\bf{n}}dS$;  $dS$~-- элемент поверхности $\partial V (t )$; матрица ${\bf{K}}$ имеет 
структуру ${\bf{K}} = ( {{\bf{E}},\, {\bf{G}},\, {\bf{H}}} ).$ Интеграл в первом слагаемом левой части 
уравнения \eqref{6} преобразуется по правилу Лейбница для производной от интеграла с переменными 
пределами~\cite{fiht}
\begin{equation}
  \label{7} 
  \int\limits_{V(t)} {\frac{{\partial
  {\bf{Q}}}}{{\partial t}}dV} = \frac{\partial }{{\partial t}}\int\limits_{V(t)} {{\bf{Q}}dV} -
  \displaystyle\oint\limits_{\partial V(t)} {{\bf{Q}}
  {\frac{{\partial {\bf{x}}_{} }}{{\partial t}} \cdot d{\bf{S}}} },
\end{equation}
где ${\bf{x}}$~--- радиус-вектор элемента поверхности $dS$;
\begin{equation*}
  \cfrac{{\partial {\bf{x}}}}{{\partial t}} \cdot d{\bf{S}} = {\bf{x}}_t  \cdot d{\bf{S}} =
  ( {x_t ,\,y_t ,\, z_t })  \left(
  \begin{array}{l}
   dS_x  \\
   dS_y  \\
   dS_z  \\
  \end{array} \right)~\text{---}
\end{equation*}
нормальная скорость перемещения этого элемента. Подставляя \eqref{7} в \eqref{6}, получим
\begin{equation}
  \label{8} 
  {\bf{R}}^t \frac{\partial }{{\partial t}}\int\limits_{V(t)} {{\bf{Q}}dV} +
  \displaystyle\oint\limits_{\partial V(t)} {\left(
  {{\bf{K}}d{\bf{S}} - {\bf{R}}^t {\bf{Qx}}_t  \cdot d{\bf{S}}}
  \right)}  = \int\limits_{V(t)} {{\bf{F}}dV}.
\end{equation}
Преобразуем второй член в круглых скобках \eqref{8} под поверхностным интегралом
\begin{equation*}
  \label{9}
  \begin{array}{l}
   {\bf{R}}^t {\bf{Qx}}_t  \cdot d{\bf{S}} = {\bf{R}}^t \left( {\begin{array}{*{20}c}
     {p{\bf{x}}_t  \cdot d{\bf{S}}}  \\
     {w_1 {\bf{x}}_t  \cdot d{\bf{S}}}  \\
     {w_2 {\bf{x}}_t  \cdot d{\bf{S}}}  \\
     {w_3 {\bf{x}}_t  \cdot d{\bf{S}}}  \\
  \end{array}} \right) = {\bf{R}}^t \left( {\begin{array}{*{20}c}
     {p\left( {x_t dS_x  + y_t dS_y  + z_t dS_z } \right)}  \\
     {w_1 \left( {x_t dS_x  + y_t dS_y  + z_t dS_z } \right)}  \\
     {w_2 \left( {x_t dS_x  + y_t dS_y  + z_t dS_z } \right)}  \\
     {w_3 \left( {x_t dS_x  + y_t dS_y  + z_t dS_z } \right)}  \\
  \end{array}} \right) =  \\
    = {\bf{R}}^t \left( {\begin{array}{*{20}c}
     {px_t } & {py_t } & {pz_t }  \\
     {w_1 x_t } & {w_1 y_t } & {w_1 z_t }  \\
     {w_2 x_t } & {w_2 y_t } & {w_2 z_t }  \\
     {w_3 x_t } & {w_3 y_t } & {w_3 z_t }  \\
  \end{array}} \right)\left( {\begin{array}{*{20}c}
     {dS_x }  \\
     {dS_y }  \\
     {dS_z }  \\
  \end{array}} \right) = \left( {\begin{array}{*{20}c}
     0 & 0 & 0  \\
     {w_1 x_t } & {w_1 y_t } & {w_1 z_t }  \\
     {w_2 x_t } & {w_2 y_t } & {w_2 z_t }  \\
     {w_3 x_t } & {w_3 y_t } & {w_3 z_t }  \\
  \end{array}} \right)\left( {\begin{array}{*{20}c}
     {dS_x }  \\
     {dS_y }  \\
     {dS_z }  \\
  \end{array}} \right), \\
 \end{array}
\end{equation*}
сгруппируем его с матрицей потоков ${\bf{K}}$ и обозначим объединенную матрицу через ${\bf{K}}^t $
\begin{equation}
  \label{10} 
  {\bf{K}}^t  = \left( {\begin{array}{*{20}c}
   \begin{array}{l}
   w_1  \\
   w_1^2  + p - \tau _{11}  - w_1 x_t  \\
   w_1 w_2  - \tau _{21}  - w_2 x_t  \\
   w_1 w_3  - \tau _{31}  - w_3 x_t  \\
   \end{array} & \begin{array}{l}
   w_2  \\
   w_1 w_2  - \tau _{12}  - w_1 y_t  \\
   w_2^2  + p - \tau _{22}  - w_2 y_t  \\
   w_2 w_3  - \tau _{32}  - w_3 y_t  \\
   \end{array} & \begin{array}{l}
   w_3  \\
   w_1 w_3  - \tau _{13}  - w_1 z_t  \\
   w_2 w_3  - \tau _{23}  - w_2 z_t  \\
   w_3^2  + p - \tau _{33}  - w_3 z_t  \\
   \end{array}  \\
  \end{array}} \right).
\end{equation}
Окончательно интегральные законы сохранения массы и количества движения для подвижного объема $V(t)$ примут 
вид
\begin{equation}
  \label{11} 
  {\bf{R}}^t \frac{\partial }{{\partial t}}\int\limits_{V(t)} {{\bf{Q}}dV +
  \displaystyle\oint\limits_{\partial V(t)} {{\bf{K}}^t d{\bf{S}}} =
  \int\limits_{V(t)} {{\bf{F}}dV} }.
\end{equation}
\subsection{Условие геометрической консервативности в интегральной форме}
\label{s:113}
Уравнения, описывающие течение жидкости, должны сохранять со временем постоянный однородный поток в 
отсутствие внешних сил. Поэтому равенства ${\bf{F}} = 0,\ {\bf{Q}} =$~const и условие
замкнутости $\displaystyle\oint\limits_{\partial V(t)} {d{\bf{S}}} = 0$ выделенного объема $V(t)$, 
подставленные в уравнение \eqref{11}, дают тождественное скалярное уравнение
\begin{equation}
  \label{12} 
  \frac{\partial }{{\partial t}}V(t) =
  \displaystyle\oint\limits_{\partial V(t)} {{\bf{x}}_t  \cdot d{\bf{S}}},
\end{equation}
которое также является следствием правила Лейбница \eqref{7} и называется условием геометрической 
консервативности. Оно имеет простую геометрическую интерпретацию~--- изменение
выделенного объема во времени равно сумме объемов, образованных геометрическими элементами поверхности объема 
при их движении. Впервые понятие УГК было введено в \cite{trulio} для конечно-разностного метода, а позднее 
распространено на метод конечных объемов \cite{thomas,demper}. В работе~\cite{demper} было предложено 
вычислять скорости движения граней ячеек исходя из известных перемещений узлов так, чтобы условие 
геометрической консервативности автоматически выполнялось. При этом в~\cite{demper} рассмотрен только случай 
двумерных течений. В настоящей работе указанный подход получения метода, удовлетворяющего УГК, распространен 
на случай трехмерных течений.

В~\cite{demper,shyy,lesoin} показано, что нарушение УГК приводит к возникновению осцилляций решения, 
вызываемых только движением сетки. В работах~\cite{lesoin,koobus} предложены удовлетворяющие УГК схемы
соответственно первого и второго порядка по времени для двумерного и трехмерного случаев движения сетки.

Если в дискретном аналоге уравнения \eqref{11} не удовлетворить условие геометрической консервативности, то 
на задаче с решением в виде постоянного однородного потока численный метод на подвижной
сетке даст возмущенное, отличное от начальных данных решение. Ниже в разделе~\ref{s:12} проводится обобщение 
метода~\cite{Cher} на случай подвижных сеток. Для полноты изложения приведен также геометрический закон 
сохранения в дифференциальной форме.

\subsection{Условие геометрической консервативности в дифференциальной форме}
\label{s:114}
Наряду с физической областью $(x,y,z)$ с подвижной сеткой и подвижным элементом объема физического 
пространства $dV$ рассмотрим вычислительную область $(\xi,\eta,\zeta)$ с~неподвижным элементом объема 
вычислительного пространства $dV_0$. Между декартовыми координатами физической области и криволинейными
координатами вычислительной области устанавливается взаимно-однозначное отображение
\begin{gather}
  \label{xyzt}
  \xi=\xi(x,y,z,t),\notag \\
  \eta=\eta(x,y,z,t),\notag \\
  \zeta=\zeta(x,y,z,t),\notag \\
  t'=t.
\end{gather}
Рассмотрим матрицу
\begin{equation}
  \label{Amatr}
   {\bf{A}} = \left( {\begin{array}{*{20}c}
   {\xi _x } & {\xi _y } & {\xi _z } & {\xi _t } \\
   {\eta _x } & {\eta _y } & {\eta _z } & {\eta _t } \\
   {\zeta _x } & {\zeta _y } & {\zeta _y } & {\zeta _t } \\
   0 & 0 & 0 & 1 \\
   \end{array}} \right)
\end{equation}
и определим величину
\begin{equation}
  \label{JdetA}
  J=\text{det}\,{\bf{A}}.
\end{equation}
Тогда связь между элементами объемов в физическом и вычислительном пространствах задается равенством
\begin{equation}
  \label{JdV}
  J\,dV=dV_0.
\end{equation}
Следует отметить, что сетка и объемы в вычислительном пространстве фиксированы и не зависят от времени. 
Из-за подвижности сетки и объемов в физическом пространстве преобразование координат (13) и соответствующая 
ему матрица Якоби \eqref{Amatr} зависят от времени. В силу теоремы Гаусса"=Остроградского УГК
\eqref{12} переписывается в виде
\begin{equation}
  \label{difUGK} 
  \frac{\partial}{{\partial t}}\int\limits_V dV = \int\limits_V \text{div}\left({\bf{x}}_t\right) dV,
\end{equation}
с учетом \eqref{JdV} и независимости $dV_0$ от времени получаем
\begin{equation}
  \label{difUGK_V0} 
  \int\limits_{V_0}\frac{\partial }{{\partial t}}\left(\frac{1}{J}\right)dV_0 = \int\limits_{V_0}\frac{1}{J}
  \text{div}\left({\bf{x}}_t\right)dV_0.
\end{equation}
Из \eqref{difUGK_V0} следует дифференциальная форма геометрического закона сохранения
\begin{equation}
  \label{difUGK_1}
  J\frac{\partial }{{\partial t}}\frac{1}{J} - \text{div}\,{\bf{x}}_t  = 0.
\end{equation}

\section{Численный метод}
\label{s:12}
\subsection{Дискретизация уравнений для подвижного объема}
\label{s:121}
Расчетная физическая область разбивается на элементарные ячейки в виде криволинейных 
шестигранников. Центрам ячеек приписываются осредненные по их объемам $V^n_{ijk}$ значения переменных и
массовых сил
\begin{equation}
  \label{13} 
  \left( {{\bf{Q}}_{ijk} V_{ijk} } \right)^n  = \int\limits_{V_{ijk}^n } {{\bf{Q}}^n dV} ,\quad   \left(
  {{\bf{F}}_{ijk} V_{ijk} } \right)^n  = \int\limits_{V_{ijk}^n }
  {{\bf{F}}^n dV}.
\end{equation}

Неявная конечно"=объемная аппроксимация уравнения \eqref{11} на ячейке ${ijk}$ дает
\begin{equation}
  \label{14} 
  {\bf{R}}^t \frac{{3\left( {{\bf{Q}}V} \right)^{n + 1} - 4\left( {{\bf{Q}}V} \right)^n  + \left( {{\bf{Q}}V}
  \right)^{n - 1} }}{{2\Delta t}} = \left( {{\bf{RHS}}^{\,t} } \right)^{n + 1},
\end{equation}
где $\Delta t$~--- шаг по времени, $n$~--- номер слоя по времени, $V^n$~--- объем ячейки на $n$-м слое по 
времени. Правая часть \eqref{14} имеет структуру
\begin{equation*}
  \begin{array}{c}
  {\bf{RHS}}^t  =  - \left( {\left( {{\bf{K}}^t {\bf{S}}} \right)_{i + 1/2}  - \left( {{\bf{K}}^t {\bf{S}}} 
  \right)_{i - 1/2}  + \left( {{\bf{K}}^t {\bf{S}}} \right)_{j + 1/2}  - \left( {{\bf{K}}^t {\bf{S}}} 
  \right)_{j - 1/2}  + } \right. \\
  \left. { + \left( {{\bf{K}}^t {\bf{S}}} \right)_{k + 1/2}  - \left( {{\bf{K}}^t {\bf{S}}} \right)_{k - 1/2}}
  \right) + {\bf{F}}V_{ijk}, \\
  \end{array}
\end{equation*}
где выражения $ \left( {{\bf{K}}^t {\bf{S}}} \right)_{i + 1/2},\ \left( {{\bf{K}}^t {\bf{S}}} 
\right)_{j + 1/2} ,\ \left({{\bf{K}}^t {\bf{S}}} \right)_{k + 1/2}$ представляют собой
разностные потоки через грани $S_{i + 1/2},\, S_{j + 1/2},\, S_{k + 1/2}$ ячейки $ijk$ и объемом 
$V_{ijk} $. Определим векторы
\begin{gather}
  {\bf{S}}_{i + 1/2}  = \left( {{{S}}_x ,\ {{S}}_y ,\ {{S}}_z } \right)_{i + 1/2} ,\notag \\
  {\bf{S}}_{j + 1/2}  = \left( {{{S}}_x ,\ {{S}}_y ,\ {{S}}_z } \right)_{j + 1/2} ,\notag \\
  {\bf{S}}_{k + 1/2}  = \left( {{{S}}_x ,\ {{S}}_y ,\ {{S}}_z } \right)_{k + 1/2}
  \label{15}
\end{gather}
как нормали к граням $S_{i + 1/2},\, S_{j + 1/2},\, S_{k + 1/2}$ ячейки $ijk$. Длины этих векторов 
равны площадям соответствующих граней данной ячейки. Направления нормалей выбираются таким образом, чтобы к 
граням $S_{i + 1/2},\, S_{j + 1/2},\, S_{k + 1/2}$ они были внешними, а к граням $S_{i - 1/2},\,
S_{j - 1/2},\, S_{k - 1/2}$~--- внутренними по отношению к ячейке сетки с~индексом $ijk$. Такой выбор 
нормалей упрощает их построение, поскольку при этом выделяются одни глобальные для всех ячеек сетки, 
положительные по каждой из координат направления. Нормали к граням ячейки определяются таким образом, чтобы
выполнялось соотношение
\begin{equation}
  \label{16} 
  {\bf{S}}_{i + 1/2}  - {\bf{S}}_{i - 1/2}  + {\bf{S}}_{j + 1/2}  - {\bf{S}}_{j - 1/2}  + {\bf{S}}_{k + 1/2}
  - {\bf{S}}_{k - 1/2}  = 0,
\end{equation}
являющееся следствием интегрального условия замкнутости $\displaystyle\oint\limits_{\partial V(t)} 
{d{\bf{S}}} = 0$.

\subsection{Дискретное условие геометрической консервативности}
\label{s:122}
Подставив в \eqref{14} значения ${\bf{F}} = 0,\ {\bf{Q}} =$~const и учитывая дискретное условие замкнутости 
ячейки \eqref{16}, получим дискретное условие геометрической консервативности
\begin{gather}
  \frac{{3 V^{n + 1}  - 4V^n  + V^{n - 1} }}{{2\Delta t}} = \left( {{\bf{x}}_t  \cdot {\bf{S}}} 
  \right)_{i + 1/2}  - \left( {{\bf{x}}_t  \cdot {\bf{S}}} \right)_{i - 1/2} +
  \left( {{\bf{x}}_t  \cdot {\bf{S}}} \right)_{j + 1/2}  - \notag \\ -
  \left( {{\bf{x}}_t  \cdot {\bf{S}}} \right)_{j - 1/2} + 
  \left( {{\bf{x}}_t  \cdot {\bf{S}}} \right)_{k + 1/2}  - \left( {{\bf{x}}_t  \cdot {\bf{S}}} 
  \right)_{k - 1/2}.   
  \label{eq1:17}
 \end{gather}
Соотношение (\ref{eq1:17}) связывает способы вычисления объема ячейки и 
нормальной составляющей скорости движения грани ${\bf{x}}_t \cdot {\bf{S}}$.

\subsection{Скорость движения грани ячейки}
\label{s:123}
Выведем из \eqref{eq1:17} конкретные выражения для скорости движения грани ячейки ${\bf{x}}_t \cdot {\bf{S}}$.

\subsubsection{Одномерный случай}
\label{s:1231}
В одномерном случае скорости движения граней и нормали вырождаются в скалярные величины и УГК \eqref{eq1:17} 
принимает вид
\begin{equation}
  \label{18} 
  \frac{{3V_i^{n + 1}  - 4V_i^n + V_i^{n - 1} }}{{2\Delta t}} = \left( {x_t } \right)_{i + 1/2} - 
  \left( {x_t } \right)_{i - 1/2}.
\end{equation}
Объемы ячеек вычисляются как $V_i = x_{i + 1/2} - x_{i - 1/2}$, где $x_{i \pm 1/2}$~--- координаты 
краев $i$-й ячейки. Левая часть уравнения \eqref{18} может быть преобразована как
\begin{gather}
  \frac{{3 {V_i}^{n + 1}  - 4{V_i}^n  + {V_i}^{n - 1} }}{{2\Delta t}} =\notag  \\ = 
  \frac{1}{{2\Delta t}}\left[ {3\left( {x_{i + 1/2}  - x_{i - 1/2} } \right)^{n + 1}  - 4
  \left( {x_{i + 1/2}  - x_{i - 1/2} } \right)^n  + \left( {x_{i + 1/2}  - x_{i - 1/2} } 
  \right)^{n - 1} } \right] = \notag  \\ = 
  \frac{{3x_{i + 1/2}^{n + 1}  - 4x_{i + 1/2}^n  + x_{i + 1/2}^{n - 1} }}{{2\Delta t}} - 
  \frac{{3x_{i - 1/2}^{n + 1}  - 4x_{i - 1/2}^n  + x_{i - 1/2}^{n - 1} }}{{2\Delta t}} = \notag \\ =
  \left( {\frac{{3x_{}^{n + 1}  - 4x_{}^n  + x_{}^{n - 1} }}{{2\Delta t}}} \right)_{i + 1/2}  - 
  \left( {\frac{{3x_{}^{n + 1}  - 4x_{}^n  + x_{}^{n - 1} }}{{2\Delta t}}} \right)_{i - 1/2}.
  \label{19}
\end{gather}
Из \eqref{19} следует, что для тождественного выполнения УГК \eqref{18} скорости движения граней ячеек в 
одномерном случае должны рассчитываться по направленной назад разности второго порядка
\begin{equation}
  \label{20} 
  x_t  = \frac{{3x^{n + 1}  - 4x^n  + x^{n - 1} }}{{2\Delta t}}.
\end{equation}

\subsubsection{Двумерный случай}
\label{s:1232}
В двумерном случае ячейками сетки являются многоугольники, грани ячеек~--- отрезки прямых, объемы 
ячеек~--- это площади многоугольников. За основу построения 
выражения для скорости движения грани ячейки в этом случае взята методика, предложенная для двуслойной схемы 
в работе \cite{demper}. Для двуслойной схемы УГК имеет вид
\begin{equation}
  \label{21} 
  \frac{V_{ij}^{n + 1}  - V_{ij}^n }{\Delta t} = \left(
  {{\bf{x}}_t  \cdot {\bf{S}}} \right)_{i + 1/2}  - \left(
  {{\bf{x}}_t  \cdot {\bf{S}}} \right)_{i - 1/2}  + \left(
  {{\bf{x}}_t  \cdot {\bf{S}}} \right)_{j + 1/2}  - \left(
  {{\bf{x}}_t  \cdot {\bf{S}}} \right)_{j - 1/2},
\end{equation}
что можно переписать как
\begin{equation}
  \label{22} 
  \frac{\Delta V_{ij}^n}{\Delta t}= \left( {{\bf{x}}_t \cdot {\bf{S}}} \right)_{i + 1/2} - 
  \left( {{\bf{x}}_t  \cdot {\bf{S}}} \right)_{i - 1/2}  + \left( {{\bf{x}}_t  \cdot {\bf{S}}}
  \right)_{j + 1/2}  - \left( {{\bf{x}}_t  \cdot {\bf{S}}}
  \right)_{j - 1/2} ,
\end{equation}
где
\begin{equation}
  \label{23} 
  \Delta V_{ij}^n  \equiv V_{ij}^{n + 1} - V_{ij}^n
\end{equation}
есть изменение площади ячейки при переходе с $n$-го слоя на $n+1$-й слой.

В работе \cite{lesoin} показано, что
\begin{equation}
  \label{24} 
  \Delta V_{}^n  = V_{i + 1/2}^n  - V_{i - 1/2}^n  + V_{j + 1/2}^n - V_{j - 1/2}^n,
\end{equation}
где
\begin{equation*}
  V_{m \pm 1/2}^n  = \left( \left( {{\bf{x}}^{n + 1}  - {\bf{x}}^n } \right) {\frac{{{\bf{S}}^{n + 1}
  + {\bf{S}}^n }}{2}} \right)_{m \pm 1/2},\quad m=i,j,
\end{equation*}
представляют собой площади фигур, образованных гранями-отрезками ${{S}}_{m \pm 1/2}$ при их движении за время 
$\Delta t$. Плюсы при $V_{i+1/2},\  V_{j+1/2}$ и минусы при $V_{i-1/2},\  V_{j-1/2}$ в правой части \eqref{24}
обусловлены тем, что направления нормалей к отрезкам ${{S}}_{i+1/2},\  {{S}}_{j+1/2}$ являются внешними, а к
${{S}}_{i-1/2},\  {{S}}_{j-1/2}$~--- внутренними по отношению к ячейке сетки с~индексом $ij$.

Подставляя \eqref{24} в \eqref{22}, получим 
\begin{gather}
  \label{26}
  \frac{V^n_{i + 1/2}}{\Delta t} -  \frac{V^n_{i - 1/2}}{\Delta t} +  \frac{V^n_{j + 1/2}}{\Delta t}- 
  \frac{V^n_{j - 1/2}}{\Delta t} =\notag\\ = 
  \left( {{\bf{x}}_t  \cdot {\bf{S}}} \right)_{i + 1/2}  - \left( {{\bf{x}}_t  \cdot {\bf{S}}} 
  \right)_{i - 1/2}  + \left( {{\bf{x}}_t  \cdot {\bf{S}}} \right)_{j + 1/2}  - \left( {{\bf{x}}_t  
  \cdot {\bf{S}}} \right)_{j - 1/2}.
\end{gather}
Следовательно, если положить
\begin{equation}
  \label{27}
  \left( {{\bf{x}}_t  \cdot {\bf{S}}} \right)_{m \pm 1/2} = 
  \frac{{V^n_{m \pm 1/2}}}{{\Delta t}},\quad m = i,j,
\end{equation}
то УГК будет выполняться точно.

В случае трехслойной схемы УГК имеет вид
\begin{equation}
  \label{121} 
  \frac{{3V_{ij}^{n + 1}  - 4V_{ij}^n  + V_{ij}^{n - 1} }}{{2\Delta t}} = \left( {{\bf{x}}_t  \cdot {\bf{S}}}
  \right)_{i + 1/2}  - \left( {{\bf{x}}_t  \cdot {\bf{S}}} \right)_{i - 1/2} +
  \left( {{\bf{x}}_t  \cdot {\bf{S}}} \right)_{j + 1/2} - 
  \left( {{\bf{x}}_t  \cdot {\bf{S}}} \right)_{j - 1/2},
\end{equation}
что можно переписать как
\begin{equation}
  \label{122} 
  \frac{{3\Delta V_{}^n - \Delta V_{}^{n - 1} }}{{2\Delta t}}= \left( {{\bf{x}}_t  \cdot {\bf{S}}} 
  \right)_{i + 1/2}  - \left( {{\bf{x}}_t  \cdot {\bf{S}}} \right)_{i - 1/2} +
  \left( {{\bf{x}}_t  \cdot {\bf{S}}} \right)_{j + 1/2} - 
  \left( {{\bf{x}}_t  \cdot {\bf{S}}} \right)_{j - 1/2} ,
\end{equation}
где величина $\Delta V_{}^n$ определена в \eqref{23} и вычисляется также по формуле \eqref{24}.
Подставляя \eqref{24} в \eqref{122}, получим
\begin{gather}
  \left(\frac{3V^n  - V^{n - 1}}{2\Delta t}\right)_{i + 1/2}\! - \! 
  \left(\frac{3V^n  - V^{n - 1}}{2\Delta t}\right)_{i - 1/2}\! + \notag\\ +\! 
  \left(\frac{3V^n  - V^{n - 1}}{2\Delta t}\right)_{j + 1/2}\! -\! 
  \left(\frac{3V^n  - V^{n - 1}}{2\Delta t}\right)_{j - 1/2}\! =\notag\\ = 
  \left( {{\bf{x}}_t  \cdot {\bf{S}}} \right)_{i + 1/2}  - 
  \left( {{\bf{x}}_t  \cdot {\bf{S}}} \right)_{i - 1/2}  + 
  \left( {{\bf{x}}_t  \cdot {\bf{S}}} \right)_{j + 1/2}  - 
  \left( {{\bf{x}}_t  \cdot {\bf{S}}} \right)_{j - 1/2}.
  \label{126}
\end{gather}
Следовательно, если положить
\begin{equation}
 \label{127}
 \left( {{\bf{x}}_t  \cdot {\bf{S}}} \right)_{m \pm 1/2} = 
 \frac{{{\left(3V^n - V^{n - 1}\right)}_{m \pm 1/2}}}{{2\Delta t}},\quad m = i,j,
\end{equation}
то УГК будет выполняться точно.

\subsubsection{Трехмерный случай}
\label{s:1233}
В трехмерном случае автором развита идея работы \cite{demper} для двумерного случая. Дискретное условие
геометрической консервативности~\eqref{eq1:17} перепишем как
\begin{gather}
  \frac{{3\Delta V^n  - \Delta V^{n - 1} }}{{2\Delta t}} = 
  \left({{\bf{x}}_t  \cdot {\bf{S}}} \right)_{i + 1/2} - 
  \left({{\bf{x}}_t  \cdot {\bf{S}}} \right)_{i - 1/2} + 
  \left({{\bf{x}}_t  \cdot {\bf{S}}} \right)_{j + 1/2} - 
  \left({{\bf{x}}_t  \cdot {\bf{S}}} \right)_{j - 1/2} + \notag\\ + 
  \left({{\bf{x}}_t  \cdot {\bf{S}}} \right)_{k + 1/2} - 
  \left({{\bf{x}}_t  \cdot {\bf{S}}} \right)_{k - 1/2} ,
  \label{28}
\end{gather}
где $\Delta V_{}^n \equiv V^{n + 1} - V^n$. По аналогии с двумерным случаем изменение объема ячейки 
представим в виде
\begin{equation}
  \label{29} 
  \Delta V_{}^n = V_{i + 1/2}^n - V_{i - 1/2}^n + V_{j + 1/2}^n - V_{j - 1/2}^n 
  + V_{k + 1/2}^n - V_{k - 1/2}^n,
\end{equation}
где $V_{m \pm 1/2}^n$~--- объемы фигур, образованных гранями $S_{m \pm 1/2}$ при их движении за время 
$\Delta t$. Подставим соотношение \eqref{29} в уравнение \eqref{28}
\begin{gather}
  \frac{3V_{i + 1/2}^n  - V_{i + 1/2}^{n - 1}}{2\Delta t} - 
  \frac{3V_{i - 1/2}^n  - V_{i - 1/2}^{n - 1}}{2\Delta t} +
  \frac{3V_{j + 1/2}^n  - V_{j + 1/2}^{n - 1}}{2\Delta t} - 
  \frac{3V_{j - 1/2}^n  - V_{j - 1/2}^{n - 1}}{2\Delta t} +
  \notag \\ + \frac{3V_{k + 1/2}^n  - V_{k + 1/2}^{n - 1}}{2\Delta t} -  
  \frac{3V_{k - 1/2}^n  - V_{k - 1/2}^{n - 1}}{2\Delta t} =
  \left( {{\bf{x}}_t  \cdot {\bf{S}}} \right)_{i + 1/2} - 
  \left( {{\bf{x}}_t  \cdot {\bf{S}}} \right)_{i - 1/2} +\notag \\ + 
  \left( {{\bf{x}}_t  \cdot {\bf{S}}} \right)_{j + 1/2} - 
  \left( {{\bf{x}}_t  \cdot {\bf{S}}} \right)_{j - 1/2} +  
  \left( {{\bf{x}}_t  \cdot {\bf{S}}} \right)_{k + 1/2} - 
  \left( {{\bf{x}}_t  \cdot {\bf{S}}} \right)_{k - 1/2}.
  \label{30}
\end{gather}
Из \eqref{30} заключаем, что для точного выполнения УГК достаточно положить
\begin{equation}
  \label{31} 
  \left( {{\bf{x}}_t \cdot {\bf{S}}} \right)_{m \pm 1/2} = 
  \frac{\left(3V^n  - V^{n - 1} \right)_{m \pm 1/2}}{{2\Delta t}},\quad m = i,j,k.
\end{equation}

\subsection{Метод вычисления объемов \boldmath{$V^n$} и \boldmath{$V^n_{m\pm 1/2}$},
            обусловливающий точное выполнение УГК}
\label{s:124}
Комплексы $({\bf{x}}_t  \cdot {\bf{S}})_{m + 1/2}$, вычисленные по формуле \eqref{31}, будут давать точное 
выполнение дискретного УГК \eqref{eq1:17}, если соотношение \eqref{29} выполнено точно, для чего необходимо 
определенным образом вычислять объемы $V^n_{m\pm 1/2}$, заметаемые гранями ячеек при движении сетки. 
Очевидно, способ их вычисления зависит от способа расчета самих объемов $V^n$. Расчет объемов криволинейных 
шестигранников требует аккуратного отношения. Например, способ вычисления, используемый 
в работе~\cite{RizziErikson} приводит к тому, что в зависимости от положений узлов сетки некоторые части 
расчетной области учитываются в объемах ячеек дважды, а другие вообще не учитываются. В нестационарной 
постановке и на движущихся сетках такой способ вычисления объемов ячеек приводит к появлению нефизических  
источников массы и импульса внутри расчетной области, накоплению ошибки с течением времени, и как следствие, 
неверным результатам расчета. В \cite{volkov} сформулированы общие принципы выполнения \eqref{29} 
на основе метода неопределенных коэффициентов, однако предложенная там методика применима только к одномерным 
и двумерным задачам.

В настоящей работе предлагается следующий метод вычисления объемов, входящих в равенство \eqref{29}.

Пусть объем $V^n_{ijk}$ образован вершинами с радиус-векторами
\begin{gather}
  {\mathbf{x}}^n_{i,j,k}, \quad {\mathbf{x}}^n_{i,j+1,k},\quad {\mathbf{x}}^n_{i,j+1,k+1},
  \quad {\mathbf{x}}^n_{i,j,k+1}, \notag\\
  {\mathbf{x}}^n_{i+1,j,k},\quad {\mathbf{x}}^n_{i+1,j+1,k}, 
  \quad {\mathbf{x}}^n_{i+1,j+1,k+1},\quad {\mathbf{x}}^n_{i+1,j,k+1}.
  \label{x_Vijk}
\end{gather}
Объем ячейки $V^n_{ijk}$ вычислим как сумму объемов шести треугольных пирамид, имеющих общее ребро, 
соединяющее вершины ${\mathbf{x}}^n_{i,j,k}$ и ${\mathbf{x}}^n_{i+1,j+1,k+1}$ (рисунок~\ref{fig1:1},\,\emph{а}):
\begin{gather}
  V^n_{ijk}={\displaystyle{\frac{1}{6}}}\big({\bf{x}}^n_{i+1,j+1,k+1}-{\bf{x}}^n_{i,j,k}\big)
  \Big(
  \left[({\bf{x}}^n_{i,j+1,k}-{\bf{x}}^n_{i,j,k+1})\times
       ({\bf{x}}^n_{i,j+1,k+1}-{\bf{x}}^n_{i,j,k})\right]+ \notag\\[2mm]
 +\left[({\bf{x}}^n_{i,j,k+1}-{\bf{x}}^n_{i+1,j,k})\times
       ({\bf{x}}^n_{i+1,j,k+1}-{\bf{x}}^n_{i,j,k})\right]+\notag \\[2mm]
 +\left[({\bf{x}}^n_{i+1,j,k}-{\bf{x}}^n_{i,j+1,k})\times
       ({\bf{x}}^n_{i+1,j+1,k}-{\bf{x}}^n_{i,j,k})\right] \Big).
  \label{Vijk}
\end{gather}
Согласно введенным обозначениям, объем $V^n_{i-1/2}$, заметаемый гранью $S_{i-1/2}$ при ее движении за 
время $\Delta t=t^{n+1}-t^{n}$, образован вершинами
\begin{gather}
  {\mathbf{x}}^n_{i,j,k},  \quad  {\mathbf{x}}^n_{i,j+1,k},\quad {\mathbf{x}}^n_{i,j+1,k+1}, \quad  
  {\mathbf{x}}^n_{i,j,k+1}, \quad \notag \\
  {\mathbf{x}}^{n+1}_{i,j,k}, \quad  {\mathbf{x}}^{n+1}_{i,j+1,k}, \quad
  {\mathbf{x}}^{n+1}_{i,j+1,k+1},\quad {\mathbf{x}}^{n+1}_{i,j,k+1}.
  \label{x_Vi12}
\end{gather}
Заметим, что обозначения вершин объема $V^n_{i-1/2}$ совпадают с обозначениями (\ref{x_Vijk}) вершин 
объема $V^n_{ijk}$ с точностью до замены индексов $(\cdot)^{n}_{i+1}$ на индексы $(\cdot)^{n+1}_{i}$ 
(рисунок~\ref{fig1:1},\,{\it б}). Объемы $V^n_{i-1/2}$ вычисляются по той же формуле~\eqref{Vijk}, в которой  
индексы $(\cdot)^{n}_{i+1}$ заменены на $(\cdot)^{n+1}_{i}$:
\begin{gather}
  V^n_{i-1/2}={\displaystyle{\frac{1}{6}}}\big({\bf{x}}^{n+1}_{i,j+1,k+1}-{\bf{x}}^n_{i,j,k}\big)
  \Big(
  \left[({\bf{x}}^n_{i,j+1,k}-{\bf{x}}^n_{i,j,k+1})\times
       ({\bf{x}}^n_{i,j+1,k+1}-{\bf{x}}^n_{i,j,k})\right]+ \nonumber\\[2mm]
  \label{Vi12}
  +\left[({\bf{x}}^n_{i,j,k+1}-{\bf{x}}^{n+1}_{i,j,k})\times
       ({\bf{x}}^{n+1}_{i,j,k+1}-{\bf{x}}^n_{i,j,k})\right]+\nonumber \\[2mm]
  +\left[({\bf{x}}^{n+1}_{i,j,k}-{\bf{x}}^n_{i,j+1,k})\times
       ({\bf{x}}^{n+1}_{i,j+1,k}-{\bf{x}}^n_{i,j,k})\right] \Big).
  \label{Vni12}
\end{gather}
Аналогично с использованием формулы \eqref{Vijk} вычисляются объемы $V^n_{j-1/2}$ и $V^n_{k-1/2}$. 
Для нахождения $V^n_{j-1/2}$ индексы $(\cdot)^{n}_{j+1}$ заменяются на $(\cdot)^{n+1}_{j}$, для
нахождения $V^n_{k-1/2}$~--- $(\cdot)^{n}_{k+1}$ заменяются на $(\cdot)^{n+1}_{k}$.

\begin{figure}[ht!]
  \label{fig1:1}
  \centering\small
  \rule{-15mm}{0mm}{\it a}\rule{80mm}{0mm}{\it б}\\[1.5mm]
  \includegraphics[width=16.0cm]{ris1.png}%  \\
  \caption{Объем $V^n_{ijk}$ (\emph{а}) и объем $V^n_{i-1/2}$, заметаемый гранью
           $S_{i-1/2}$ при движении сетки за время $\Delta t=t^{n+1}-t^{n}$, (\emph{б})}
\end{figure}

С помощью элементарных преобразований показано, что при вычислении объемов $V^n_{ijk}$ и $V^{n+1}_{ijk}$ по 
формуле~\eqref{Vijk}, а объемов $V^n_{i\pm 1/2}$, $V^n_{j\pm 1/2}$, $V^n_{k\pm 1/2}$ по формулам 
вида~\eqref{Vni12} равенство~\eqref{29} выполнено тождественно. При проведении преобразований в правых 
частях~\eqref{Vijk} и~\eqref{Vni12} предварительно приведены подобные слагаемые. Так, ~\eqref{Vijk}
эквивалентно равенству
\begin{gather}
  V^n_{ijk}={\displaystyle{\frac{1}{6}}}\big({\bf{x}}^n_{i+1,j+1,k+1}-{\bf{x}}^n_{i,j,k}\big)
  \Big(
  \left[({\bf{x}}^n_{i,j+1,k}-{\bf{x}}^n_{i,j,k+1})\times
        {\bf{x}}^n_{i,j+1,k+1}\right]+ \nonumber\\[2mm]
  +\left[({\bf{x}}^n_{i,j,k+1}-{\bf{x}}^n_{i+1,j,k})\times
        {\bf{x}}^n_{i+1,j,k+1}\right]+ \nonumber\\[2mm]
  +\left[({\bf{x}}^n_{i+1,j,k}-{\bf{x}}^n_{i,j+1,k})\times
        {\bf{x}}^n_{i+1,j+1,k}\right] \Big).  \nonumber
\end{gather}
Аналогично упрощается выражение~\eqref{Vni12}.

\subsection{Обобщение метода искусственной сжимаемости на подвижные сетки}
\label{s:125}
\subsubsection{Метод искусственной сжимаемости}
\label{s:1251}
Метод искусственной сжимаемости заключается во введении в уравнение неразрывности производной по 
псевдовремени от давления, а в уравнения количества движения~--- производных по псевдовремени
от соответствующих компонент скорости. Модифицированное уравнение~\eqref{11} запишется в виде
\begin{equation}
  \label{32} 
  \left( {{\bf{R}}^\tau  \frac{\partial }{{\partial \tau }} + {\bf{R}}^t \frac{\partial }{{\partial t}}}
  \right)\int\limits_{V(t)} {{\bf{Q}}dV} + \displaystyle\oint\limits_{\partial V(t)} {{\bf{K}}^t_{\beta} dS}= \int\limits_{V(t)} {{\bf{F}}dV},
\end{equation}
где ${\bf{R}}^\tau = {\rm{diag}}(1,1,1,1)$,
\begin{equation}
  \label{33}
  {\bf{K}}^t_{\beta}  = \left( {\begin{array}{*{20}c}
  \begin{array}{l}
  \beta w_1  \\
  w_1^2  + p - \tau _{11}  - w_1 x_t  \\
  w_1 w_2  - \tau _{21}  - w_2 x_t  \\
  w_1 w_3  - \tau _{31}  - w_3 x_t  \\
  \end{array} & \begin{array}{l}
  \beta w_2  \\
  w_1 w_2  - \tau _{12}  - w_1 y_t  \\
  w_2^2  + p - \tau _{22}  - w_2 y_t  \\
  w_2 w_3  - \tau _{32}  - w_3 y_t  \\
  \end{array} & \begin{array}{l}
  \beta w_3  \\
  w_1 w_3  - \tau _{13}  - w_1 z_t  \\
  w_2 w_3  - \tau _{23}  - w_2 z_t  \\
  w_3^2  + p - \tau _{33}  - w_3 z_t  \\
  \end{array}  \\
  \end{array}} \right).
\end{equation}
В~\eqref{32} $\beta$~--- коэффициент искусственной сжимаемости. Уравнение \eqref{32} решается численно. При 
этом на каждом шаге по физическому времени $t$ проводится установление решения по псевдовремени $\tau$.

\subsubsection{Неявная конечно"=объемная аппроксимация}
\label{s:1252}
Дискретизация уравнения \eqref{32} дает
\begin{gather}
  {\bf{R}}^\tau  \frac{{{\bf{Q}}^{n + 1,\ s + 1}  -
  {\bf{Q}}^{n + 1,\ s} }}{{\Delta \tau }}V^{n + 1}  + 
  {\bf{R}}^t \frac{{3({\bf{Q}}V)^{n + 1,\ s + 1}  - 4\left( {{\bf{Q}}V}
  \right)^n  + \left( {{\bf{Q}}V} \right)^{n - 1} }}{{2\Delta t}} = \notag \\
  = ({\bf{RHS}}^{t} )^{n + 1,\ s + 1},
  \label{38}
\end{gather}
где $\Delta \tau, \Delta t$~--- шаги по псевдовремени и физическому времени соответственно, $s$~--- номер 
итерации по псевдовремени, $n$~--- номер слоя по времени. Дискретизация \eqref{38} имеет первый порядок 
аппроксимации по псевдовремени $\tau$ и второй порядок аппроксимации по физическому времени $t$. Правая часть
имеет структуру
\begin{equation*}
  \begin{array}{c}
  {\bf{RHS}}^{\,t}  =  - (({\bf{K}}^t_{\beta}  \cdot {\bf{S}})_{i + 1/2}  - ({\bf{K}}^t_{\beta}  
  \cdot {\bf{S}})_{i - 1/2} + ({\bf{K}}^t_{\beta}  \cdot {\bf{S}})_{j + 1/2} -
  ({\bf{K}}^t_{\beta} \cdot {\bf{S}})_{j - 1/2}  + \\ +
  ({\bf{K}}^t_{\beta} \cdot {\bf{S}})_{k + 1/2}  -
  ({\bf{K}}^t_{\beta} \cdot {\bf{S}})_{k - 1/2} ) + {\bf{F}}V,
\end{array}
\end{equation*}
где разностные потоки через грани ячеек представляются в виде суммы конвективного (невязкого) и вязкого 
потоков:
\begin{equation}
  \label{39} 
  \left( {{\bf{K}}^t_{\beta}  \cdot {\bf{S}}} \right)_{m + 1/2} = \left( {{\bf{K}}_{\beta,\, inv}^t  
  \cdot {\bf{S}}} \right)_{m + 1/2}  + \left( {{\bf{K}}_{\beta,\, vis}^t  
  \cdot {\bf{S}}} \right)_{m + 1/2}.
\end{equation}
Из \eqref{33} следует, что движение узлов сетки влияет на вычисление только невязких потоков. Невязкие потоки
имеют вид
\begin{equation}
  \label{40} 
  {{\bf{K}}_{\beta,\, inv}^t  \cdot {\bf{S}}} = \left(
  {\begin{array}{*{20}c}
   \beta U  \\
   {\left( {U - U_g } \right)w_1 + pS_x }  \\
   {\left( {U - U_g } \right)w_2 + pS_y }  \\
   {\left( {U - U_g } \right)w_3 + pS_z }  \\
\end{array}} \right),
\end{equation}
где $U = {\bf{w}} \cdot {\bf{S}},\ U_g  = {\bf{x}}_t  \cdot {\bf{S}}$. Для вычисления разностных невязких 
потоков используется MUSCL-схема~\cite{anders}, обеспечивающая третий порядок аппроксимации по пространству
\begin{equation}
  \label{41} 
  \left( {{\bf{K}}_{\beta,\, inv}^t  \cdot {\bf{S}}} \right)_{m + \frac{1}{2}} = 
  \frac{1}{2}\left[ {\left( {{\bf{K}}_{\beta,\, inv}^t \left( {{\bf{Q}}_L } \right) +
  {\bf{K}}_{\beta,\, inv}^t \left( {{\bf{Q}}_R } \right)} \right) \cdot {\bf{S}}_{m + 1/2} - 
  \left| {{\bf{A}}^t } \right|\left({{\bf{Q}}_R  - {\bf{Q}}_L } \right)} \right],
\end{equation}
где
\begin{gather}
  \label{42}
  {\bf{Q}}_L  = {\bf{Q}}_m  + \frac{1}{4} \left[ {\frac{2}{3}\left( {{\bf{Q}}_m  - {\bf{Q}}_{m - 1} } 
  \right) + \frac{4}{3}\left( {{\bf{Q}}_{m + 1}  - {\bf{Q}}_m } \right)} \right],\notag \\
  {\bf{Q}}_R  = {\bf{Q}}_{m + 1}  - \frac{1}{4} \left[ {\frac{2}{3}\left( {{\bf{Q}}_{m + 1}  - {\bf{Q}}_m } 
  \right) + \frac{4}{3}\left( {{\bf{Q}}_{m + 2}  - {\bf{Q}}_{m + 1} } \right)} \right].
\end{gather}
Под модулем матрицы $|{\bf{A}}^t|$ в \eqref{41} подразумевается разность
\begin{equation}
  |{\bf{A}}^t|={\bf{A}}^+-{\bf{A}}^-
\end{equation}
матриц ${\bf{A}}^+$ и ${\bf{A}}^-$ из разложения матрицы Якоби невязкого потока
\begin{equation}
  {\bf{A}}^t ({\bf{Q}}) = \frac{{\partial ({\bf{K}}_{\beta,\, inv}^t ({\bf{Q}}) \cdot {\bf{S}})}}{{\partial
  {\bf{Q}}}} \!=\! {\bf{R}}{\bf{D}}{\bf{L}}
  \label{43} 
\end{equation}
на сумму
\begin{equation}
  \label{431} 
  {\bf{A}}^t={\bf{A}}^++{\bf{A}}^-.
\end{equation}
Здесь ${\bf{R}},\ {\bf{L}}$ --- матрицы правых и левых собственных векторов матрицы ${\bf{A}}^t$ такие, 
что ${\bf{R}}{\bf{L}}={\bf{I}}$ (где ${\bf{I}}$ -- единичная матрица). Вид 
матриц ${\bf{A}}^t,\ {\bf{R}},\ {\bf{D}},\ {\bf{L}}$ приведен в приложении~А. 
Расщепление матрицы ${\bf{A}}^t$ на сумму матриц ${\bf{A}}^+$ и ${\bf{A}}^-$ \eqref{431} осуществляется на 
основе наличия у ${\bf{A}}^t$ действительных собственных значений
\begin{equation}
  \label{44}
  \lambda _{1,2}  = U_{}  - U_g , \quad
  \lambda _{3,4}  = U_{}  - \frac{1}{2}U_g \pm c,
\end{equation}
где $c = \sqrt {\left({U_{} -\cfrac{1}{2}U_g} \right)^2 + \beta \left( {S_x^2  + S_y^2  + S_z^2 } \right)}$.
Матрицы ${\bf{A}}^+$ и ${\bf{A}}^-$ вычисляются по формулам
\begin{gather}
  {\bf{A}}^+ = {\bf{R}}{\bf{D}}^+{\bf{L}}, \notag \\
  {\bf{A}}^- = {\bf{R}}{\bf{D}}^-{\bf{L}} \notag
\end{gather}
где ${\bf{D}}^{\pm}=Diag(\lambda_{1}^{\pm},\lambda_{2}^{\pm},\lambda_{3}^{\pm},\lambda_{4}^{\pm})=0.5({\bf{D}}
\pm |{\bf{D}}|)$, $|{\bf{D}}|=Diag(|\lambda_{1}|,|\lambda_{3}|,|\lambda_{3}|,|\lambda_{4}|)$ имеют 
неотрицательные и неположительные собственные значения. 

Вычисление вязких потоков ${\bf{K}}_{\beta,\, vis}({\bf{Q}}) \cdot {\bf{S}}$ производится так же, как и 
в исходном методе для фиксированных сеток~\cite{Cher}. Это связано с тем, что дополнительный член в 
уравнениях, связанный с движением узлов сетки, входит только в невязкие потоки. 

\subsubsection{Линеаризация и $LU$-факторизация}
\label{s:1253}
Неявная схема \eqref{38} представляет из себя систему нелинейных разностных уравнений и поэтому не может быть 
решена в неизменном виде. Для решения производится линеаризация с использованием метода Ньютона
\begin{gather}
  \left[ {\left( {\frac{1}{{\Delta \tau }}{\bf{R}}^\tau   + \frac{3}{{2\Delta t}}{\bf{R}}^t } \right)V^{n+1}- 
  \left( {\frac{\partial }{{\partial {\bf{Q}}}}{\bf{RHS}}^t} \right)^s } \right]
  \left( {\bf{Q}}^{n + 1,\,s + 1}  - {\bf{Q}}^{n + 1,\,s} \right) \! = \notag \\ \!=
  - {\bf{R}}^t \frac{{3({\bf{Q}}V)^{n + 1,\,s} - 4({\bf{Q}}V)^n + ({\bf{Q}}V)^{n - 1} }}
  {{2\Delta t}}V + ({\bf{RHS}}^t)^{n + 1,\,s}. 
  \label{Newton}
\end{gather}
При построении неявного оператора в левой части \eqref{Newton} предполагается, что на всех гранях 
невязкий поток аппроксимируется с первым порядком:
\begin{gather}
  \left( {{\bf{K}}_{\beta,\, inv}^t  \cdot {\bf{S}}} \right)_{m + \frac{1}{2}} = \notag \\ =
  \frac{1}{2}\left[ \left( {{\bf{K}}_{\beta,\, inv}^t \left( {{\bf{Q}}_m } \right) +
  {\bf{K}}_{\beta,\, inv}^t \left( {{\bf{Q}}_{m+1}} \right)} \right) \cdot {\bf{S}}_{m + 1/2} - 
  \left| {{\bf{A}}^t } \right|_{m+1/2} \Delta_{m+1/2} {\bf{Q}} \right],  
  \label{Kinv1} 
\end{gather}
а в вязком потоке оставляются производные только по той координате, поверхностью уровня которой является 
рассматриваемая грань. Таким образом вычисляя матрицу $\frac{\partial}{\partial {\bf{Q}}}{\bf{RHS}}^t$ и 
подставляя ее в \eqref{Newton}, получим следующую систему линейных уравнений:
\begin{gather}
  \left[ \left( {\frac{1}{{\Delta \tau }}{\bf{R}}^\tau + \frac{3}{{2\Delta t}}{\bf{R}}^t }\right)V^{n+1} + 
  {\bf{C}}_{i + 1/2}^- \Delta_{i + 1/2} + {\bf{C}}_{i - 1/2}^+ \Delta_{i - 1/2} +
  {\bf{C}}_{j + 1/2}^- \Delta_{j + 1/2} + \right. \notag \\ \left.
  + {\bf{C}}_{j - 1/2}^+ \Delta_{j - 1/2} +   {\bf{C}}_{k + 1/2}^- \Delta_{k + 1/2} + 
  {\bf{C}}_{k - 1/2}^+ \Delta_{k - 1/2} \right] \Delta^{s+1}{\bf{Q}} = \notag \\ =
  - {\bf{R}}^t \frac{3({\bf{Q}}V)^{n + 1,\,s} - 4({\bf{Q}}V)^n + 
  ({\bf{Q}}V)^{n - 1}}{2\Delta t} + ({\bf{RHS}}^t)^{n + 1,\,s}, 
  \label{discret} 
\end{gather}
где 
\begin{gather}
  \Delta^{s + 1}{\bf{Q}} = {\bf{Q}}^{n+1,\,s+1} - {\bf{Q}}^{n + 1,\,s}, \notag \\
  {\bf{C}}_{m + 1/2}^{\pm} = {\bf{A}}_{m + 1/2}^{\pm} \pm (\nu _{\text{eff}}{\bf{R}})_{m+1/2}.
\end{gather}
Матрицы ${\bf{R}}_{i+1/2},\ {\bf{R}}_{j+1/2},\ {\bf{R}}_{k+1/2}$ есть 
матрицы ${\bf{R}}^1,\ {\bf{R}}^2,\ {\bf{R}}^3$, вычисленные в приложении А и взятые на 
соответствующих гранях.

Система уравнений~\eqref{discret} переписывается в виде
\begin{gather}
  \left[{\bf{B}}+{\bf{C}}_{i + 1/2}^- T_i^+ - {\bf{C}}_{i - 1/2}^+ T_i^-
                +{\bf{C}}_{j + 1/2}^- T_j^+ - {\bf{C}}_{j - 1/2}^+ T_j^- \right. \notag \\
        \left.  +{\bf{C}}_{k + 1/2}^- T_k^+ - {\bf{C}}_{k - 1/2}^+ T_k^- \right] \Delta^{s+1}{\bf{Q}} 
        = \label{LU1}\\
        = - {\bf{R}}^t \frac{{3({\bf{Q}}V)^{n + 1,\,s} - 4({\bf{Q}}V)^n + ({\bf{Q}}V)^{n - 1} }}{{2\Delta t}}
        + ({\bf{RHS}}^t )^{n + 1,\,s}, \notag
\end{gather}
где $T_m^{\pm}$ --- оператор сдвига на один узел вперед ($+$) или назад ($-$) по индексу $m$,
\begin{equation}
  {\bf{B}}= \left({\frac{1}{{\Delta t}}{\bf{R}}^\tau + \frac{3}{{2\Delta t}}{\bf{R}}^t}\right)V^{n+1}
   + {\bf{C}}_{i -1/2}^+ - {\bf{C}}_{i +1/2}^- + {\bf{C}}_{j -1/2}^+ - {\bf{C}}_{j +1/2}^- 
   + {\bf{C}}_{k -1/2}^+ - {\bf{C}}_{k +1/2}^-. 
\end{equation}
После этого оператор в левой части \eqref{LU1} может быть приближенно представлен в виде произведения 
\begin{gather}
  \left[{\bf{B}}-{\bf{C}}_{i-1/2}^+T_i^- - {\bf{C}}_{j-1/2}^+T_j^- -{\bf{C}}_{k-1/2}^+T_k^- \right]
  {\bf{B}}^{-1} \times \notag \\ \times
  \left[{\bf{B}}+{\bf{C}}_{i+1/2}^-T_i^+ + {\bf{C}}_{j+1/2}^-T_j^+ +{\bf{C}}_{k+1/2}^-T_k^+ \right]
  \Delta ^{s + 1}{\bf{Q}} = \\ =
  - {\bf{R}}^t \frac{{3({\bf{Q}}V)^{n + 1,\,s} - 4({\bf{Q}}V)^n + ({\bf{Q}}V)^{n - 1} }}{{2\Delta t}}
        + ({\bf{RHS}}^t )^{n + 1,\,s}, \notag
\end{gather}
и последовательно обращен
\begin{gather}
  \Delta^{s+1/2} {\bf{Q}}_{ijk} = {\bf{B}}^{-1} \left[-{\bf{R}}^t 
  \frac{{3({\bf{Q}}V)^{n + 1,\,s} - 4({\bf{Q}}V)^n + ({\bf{Q}}V)^{n-1}}}{{2\Delta t}} + 
  ({\bf{RHS}}^t)^{n+1,\,s} + \right. \notag \\ \left.
  + {\bf{C}}_{i-1/2}^+ \Delta^{s+1/2}{\bf{Q}}_{i - 1jk} + {\bf{C}}_{j-1/2}^+ 
  \Delta^{s+1/2} {\bf{Q}}_{ij - 1k} + {\bf{C}}_{k -1/2}^+ \Delta^{s+1/2} {\bf{Q}}_{ijk-1} \right],
  \label{1step}
\end{gather}
\begin{gather}
  \Delta^{s+1} {\bf{Q}}_{ijk} = \Delta^{s+1/2} {\bf{Q}}_{ijk} - {\bf{B}}^{-1} 
  \left[ {\bf{C}}_{i + 1/2}^- \Delta^{s+1} {\bf{Q}}_{i+1jk} + \notag \right. \\ + \left.
  {\bf{C}}_{j+1/2}^- \Delta^{s+1} {\bf{Q}}_{ij+1k} + {\bf{C}}_{k+1/2}^- \Delta^{s+1} {\bf{Q}}_{ijk+1}\right]
  \label{2step}
\end{gather}
На первом шаге по формулам бегущего счета~\eqref{1step} совершается разовый обход области в направлении 
возрастания всех индексов и определяется вспомогательная величина $\Delta ^{s + 1/2} {\bf{Q}}_{ijk}$. 
На втором шаге по формулам бегущего счета~\eqref{2step} в направлении убывания индексов определяется 
величина $\Delta ^{s + 1} {\bf{Q}}_{ijk}$, по которой находится вектор неизвестных на $s+1$ итерации.
\begin{equation}
  ({\bf{Q}}_{ijk}^{n + 1} )^{s + 1}  = ({\bf{Q}}_{ijk}^{n + 1} )^s  + \Delta ^{s + 1} {\bf{Q}}_{ijk}. 
\end{equation}
Итерации по $s$ повторяются до достижения сходимости $({\bf{Q}}^{n + 1} )^s  \to {\bf{Q}}^{n + 1}$.

\section{Краевые условия на подвижной твердой границе}
\label{s:13}
Для скорости на подвижной границе $\Gamma$ задается условие прилипания
\begin{equation}
  \label{451}
  \left.{\bf{w}}\right|_\Gamma={\bf{v}}_\Gamma,
\end{equation}
где ${\bf{v}}_\Gamma$~--- скорость движения твердой границы.

Для выяснения вопроса о необходимости модификации условий для давления на твердой стенке в случае ее движения 
рассмотрим простейшую модельную ситуацию. Так как для расчета давления на поверхности тела используется 
уравнение количества движения в проекции на нормаль ${\bf{n}}$, то рассмотрим локально одномерную
задачу, в~которой твердой стенкой является плоскость $x=$~const. Нормальная составляющая уравнения количества 
движения к этой стенке есть
\begin{equation}
  \label{45} 
  u_t+uu_x+\frac{1}{\rho}p_x=\nu u_{xx}.
\end{equation}
Пусть стенка движется со скоростью $U$ в положительном направлении оси $Ox$. Введем систему координат
\begin{equation}
  x'=x-\int\limits_0^t Udt,
\end{equation}
движущуюся вместе с границей, и перепишем в ней уравнение \eqref{45}:
\begin{equation}
  \label{46} 
  u_t+(u-U)u_{x'}+\frac{1}{\rho}p_{x'}=\nu u_{x'x'}.
\end{equation}
Отнесем это уравнение к движущейся стенке. В силу условия прилипания \eqref{451} на стенке $u=U$ получим 
условие для давления в общем виде
\begin{equation}
  \label{47} 
  p_{x'}=\rho\left(-U_t+\nu u_{x'x'}\right).
\end{equation}
Производная $u_{x'x'}$ рассчитывается от скорости жидкости на стенке по нормали к ней. При $U=$~const и 
течении при больших числах Рейнольдса из \eqref{47} следует приближение пограничного слоя $p_{x'}=0,$
которое совпадает с условием для давления и в случае неподвижных сеток. Поэтому в настоящей работе для 
давления на подвижной твердой границе используется также приближение пограничного слоя.

\section{Стандартная $k-\boldmath{\varepsilon}$ модель турбулентности на подвижных сетках}
\label{s:14}
Выпишем модификацию метода решения уравнений $k-\varepsilon$ модели \cite{Cher} на подвижные сетки.

Каждое из уравнений $k-\varepsilon$ модели может быть записано в общем виде
\begin{equation}
  \label{50} 
  \frac{{\partial \varphi }}{{\partial t}} + \frac{\partial }{{\partial x_j }}\left( {\varphi w_j - 
  \left( {\nu  + \frac{{\nu _t }}{{\sigma _\varphi  }}} 
  \right)\frac{{\partial \varphi }}{{\partial x_j }}} \right) = H_\varphi.
\end{equation}
Интегрируя соотношение \eqref{50} по подвижному объему и применяя формулу
\begin{equation}
  \label{51} 
  \int\limits_{V(t)} {\frac{{\partial \varphi }}{{\partial t}}} dV = \frac{\partial }
  {{\partial t}}\int\limits_{V(t)} \varphi  dV - \displaystyle\oint\limits_{\partial V} 
  {\varphi {\bf{x}}_t  \cdot d{\bf{S}}},
\end{equation}
получим уравнение в виде интегрального закона сохранения
\begin{equation}
  \label{52} 
  \frac{\partial }{{\partial t}}\int\limits_{V(t)} \varphi dV = - 
  \displaystyle\oint\limits_{\partial V(t)} {\varphi \left( {{\bf{w}} - {\bf{x}}_t } 
  \right)\cdot d{\bf{S}}}  + \displaystyle\oint\limits_{\partial V(t)} {\left( {\nu + 
  \frac{{\nu _t }}{{\sigma _\varphi  }}} \right)\nabla \varphi d{\bf{S}}}  + 
  \int\limits_{V(t)} {H_\varphi dV}.
\end{equation}
Неявная аппроксимация интегрального уравнения~\eqref{52} приводит к системе нелинейных уравнений
\begin{equation}
  \label{53} 
  \frac{{3\left( {\varphi V} \right)_{ijk}^{n + 1} - 4\left( {\varphi V} \right)_{ijk}^n  + 
  \left( {\varphi V} \right)_{ijk}^{n - 1} }}{{2\Delta t}} = {\bf{RHS}}_{ijk}^{n + 1},
\end{equation}
где значения  $\varphi_{ijk}$ отнесены к центру ячейки и
\begin{gather}
  \label{54}
  {\bf{RHS}}_{ijk}^{n + 1} = - \sum\limits_{m = i,j,k} {\left[ {(\varphi {\bf{w}} 
  \cdot {\bf{S}})_{m + 1/2} - (\varphi {\bf{w}} \cdot {\bf{S}})_{m - 1/2} } \right]^{n + 1} }+ \notag\\
  + \sum\limits_{m = i,j,k} {\left[ {(\varphi {\bf{x}}_t  \cdot {\bf{S}})_{m + 1/2}  - (
  \varphi {\bf{x}}_t  \cdot {\bf{S}})_{m - 1/2} } \right]^{n + 1} } +\notag \\
  + \sum\limits_{m = i,j,k} {\left( {\left[ {\left( {\nu  + \frac{{\nu _t }}{{\sigma_\varphi  }}} \right)
  \left( {S_x \frac{{\partial \varphi }}{{\partial x}} + S_y \frac{{\partial \varphi }}{{\partial y}} + S_z 
  \frac{{\partial \varphi }}{{\partial z}}} \right)} \right]_{m + 1/2} } \right.}  - \notag \\
  \quad \quad \left. { - \left[ {\left( {\nu  + \frac{{\nu _t }}{{\sigma _\varphi  }}} \right)
  \left( {S_x \frac{{\partial \varphi }}{{\partial x}} + S_y \frac{{\partial \varphi }}{{\partial y}} + S_z 
  \frac{{\partial \varphi }}{{\partial z}}} \right)} \right]_{m - 1/2} } \right)^{n + 1}  + 
  \left( {H_\varphi V} \right)_{ijk}^{n + 1}.
\end{gather}
Схема для вычисления невязкого потока приобретает вид
\begin{gather}
  {\bf{F}}_{m + 1/2}^{inv}  = \frac{1}{2}\left[ \left(
  {\left( {\varphi {\bf{w}}} \right)_m  + \left( {\varphi {\bf{w}}}
  \right)_{m + 1} } \right) \cdot {\bf{S}}_{m + 1/2}-\right. \notag \\
  \left. - \left({\varphi _m  + \varphi _{m + 1} } \right) \cdot U_{g\, m + 1/2} -
  \left| {U - U_g } \right|_{m + 1/2} \left( {\varphi _{m+1} - \varphi _{m} } \right)  \right],
  \label{55}
\end{gather}
где $U = {\bf{w}} \cdot {\bf{S}},\ U_g = {\bf{x}}_t \cdot {\bf{S}}$. Скорость движения грани 
ячейки $U_{g\, m + 1/2}$ находится по формулам \eqref{31},~\eqref{Vni12}.

\section{Верификация и валидация предложенного метода}
\label{s:15}
\subsection{Расчет однородного потока на подвижной сетке}
\label{s:151}
\begin{figure}[b!]
  \label{fig1:2}
  \centering{
  \includegraphics[width=5cm]{Set_0.png}
  \includegraphics[width=5cm]{Set_2.png}
  \includegraphics[width=5cm]{Set_4.png}}
  \caption{Деформация сетки в кубической области течения однородного потока для трех шагов по $t$} 
\end{figure}

Целью численного эксперимента, описанного в настоящем разделе, является подтверждение точного выполнения 
дискретного УГК в предложенном численном алгоритме. Для этого рассматривается однородный поток несжимаемой 
жидкости с единичной скоростью через куб. Поскольку компоненты скорости 
потока и давление в нем постоянны, то погрешность аппроксимации разностной задачи будет определяться только 
погрешностью выполнения УГК~\eqref{eq1:17}. При этом не важен размер ячеек сетки. Необходимо 
продемонстрировать, что на существенно неравномерной сетке (рисунок~\ref{fig1:2}) погрешность аппроксимации 
остается равной нулю. В качестве краевых условий задавались параметры самого потока.
\begin{table}[t!]
  \centering
  \caption{Результаты расчета однородного потока на подвижной сетке}
  \vspace{2mm}
  \begin{tabular}{|c|c|c|c|}
  \hline
  &  &\multicolumn{2}{c|}{Предложенный метод} \\
  \cline{3-4}
  {Время $t$, c}&{Исходный метод  \cite{Cher}}  & с несогласованным & c согласованным\\
  &  & расчетом объемов& расчетом объемов\\
  \hline
  0.0 & 1.0000000 & 1.0000000 & 1.0000000  \\
  0.4 & 0.8785331 & 1.0025764 & 1.0000000  \\
  0.8 & 0.7646654 & 1.0056505 & 1.0000000  \\
  \hline
  \end{tabular}
  \label{tab:1}
\end{table}

Решение находилось тремя способами: исходным методом~\cite{Cher}, не учитывающим движение сетки, 
предложенным методом с несогласованным и согласованным расчетом 
объемов $V^n$ и $V^n_{m\pm 1/2}$ (см. раздел~\ref{s:124}). 
С помощью каждого способа было осуществлено пять шагов по времени с $\Delta t=0.2$~с. При 
этом на каждом новом шаге проводилась дополнительная деформация 
сетки (см. рисунок~\ref{fig1:2}). В таблице~\ref{tab:1} представлены рассчитанные скорости потока 
для каждого из примененных методов в разные моменты времени. Видно, что предложенный в диссертации 
метод с машинной точностью сохраняет однородный поток во времени.

\subsection{Движение кругового цилиндра в покоящейся несжимаемой вязкой жидкости}
\label{s:152}
В данном разделе исследуются свойства и возможности построенного метода расчета течений на подвижных сетках 
на задаче ламинарного обтекания кругового цилиндра, решение которой содержит большой набор достаточно сложных 
гидродинамических явлений. Адекватное предсказание последних является важным требованием, предъявляемым
к численным методам. Особенность проводимых исследований~---  нестационарная постановка задачи, в которой в
отличие от большинства используемых не жидкость обтекает неподвижный цилиндр, а цилиндр движется в 
покоящейся жидкости. Отметим, что важность этого теста состоит  в демонстрации эффективности и
надежности созданного программного инструментария.

\subsubsection{Постановка задачи}
\label{s:1521}
Движение цилиндра диаметра $d$ по области с покоящейся жидкостью задается посредством перемещения со временем 
части границы расчетной области, совпадающей с контуром цилиндра. Для построения сетки используется 
криволинейная система координат, нормально связанная с поверхностью цилиндра. Таким образом, вместе с
цилиндром движется и сетка (рисунок~\ref{fig1:3}). В качестве неподвижной внешней границы расчетной области 
принимается окружность радиуса $R$ с центром в начале декартовой системы координат $x,y$. Цилиндр
движется вдоль оси $Ox$ справа налево. В момент $t=0$ центр цилиндра находится в точке $(0.8R, 0)$, в 
конечный момент времени $t=T$~--- в точке $(-0.8R, 0)$. Движение начинается мгновенно с~постоянной 
скоростью $U$.

Пусть $0\leq \xi\leq 1$ есть продольная, а $0\leq \eta\leq 1$~--- поперечная координаты криволинейной системы 
координат, нормально связанной с цилиндром. Связь между декартовой и~криволинейной системами координат 
задается формулами
\begin{equation}
  \left\{
  \begin{array}{l}
  x\left( {\xi ,\eta ,t} \right) =  - r\left( {\xi ,\eta ,t} \right)\cos 2\pi\xi  + x_o \left( t \right), \\
  y\left( {\xi ,\eta ,t} \right) =    r\left( {\xi ,\eta ,t} \right)\sin 2\pi\xi , \\
  \end{array} \right.
\end{equation}
где $x_o \left( t \right) = 0.8\emph{R} - U  t$~--- координата двигающегося по оси $Ox$ центра цилиндра;
\begin{gather*}
  r\left( {\xi ,\eta ,t} \right) = \frac{\emph{d}}{2} + \left(
  {L\left( {\xi ,t} \right) - \frac{\emph{d}}{2}}
  \right)\frac{{\left( {1 + a} \right)^\eta - 1}}{a}, \\
  L\left( {\xi ,t} \right) = \sqrt {\emph{R}^2  - \left( {x_o
  \left( t \right)\sin 2\pi\xi } \right)^2 }  + x_o \left( t
  \right)\cos 2\pi\xi.
\end{gather*}
Кроме рассмотренных выше краевых условий на подвижной поверхности цилиндра, задаются все параметры 
невозмущенного потока на внешней границе расчетной области. На совпадающих границах $\Gamma^+$ и
$\Gamma^-$ (см. рисунок~\ref{fig1:3}) ставятся условия периодичности.
\begin{figure}[!t]
  \label{fig1:3}
  \centering\small \emph{а}\hspace*{81mm}\emph{б}\\
  \includegraphics[width=6.5cm]{set_r.png}\hspace*{20mm}\includegraphics[width=6.5cm]{set_l.png}
  \caption{Схема движения кругового цилиндра в области покоящейся несжимаемой вязкой жидкости: \emph{а}~--- 
           начальное, \emph{б}~--- конечное положения цилиндра}
\end{figure}

\subsubsection{Физические и схемные параметры задачи}
\label{s:1522}
Для расчета приняты следующие значения физических параметров задачи: $d=1$~м, $U=0.125$~м/с, коэффициент 
кинематической вязкости $\nu=0.003125$~м$^2$/с, давление в~покоящейся жидкости $p=0$. Этому режиму обтекания 
цилиндра соответствовало число Рейнольдса
\begin{equation*}
\mathrm{Re}=\frac{U  d}{\nu}=40.
\end{equation*}
К схемным параметрам относятся радиус внешней границы $R=200$~м, время движения цилиндра $T=2560$~c, параметр 
сгущения сетки по нормальному к цилиндру направлению $a=50$, шаг по физическому времени $\Delta t=0.1$~с, 
коэффициент искусственной сжимаемости $\beta=4$, шаг по псевдовремени $\Delta\tau=1$~с.

Сетка имеет 500 узлов в окружном и 500 узлов в нормальном к цилиндру направлениях. Заданное значение 
параметра сгущения $a$ обеспечивало отношение нормального размера ячейки у внешней границы к размеру ячейки у 
поверхности цилиндра, равное~50.

\subsubsection{Результаты расчета}
\label{s:1523}
\begin{figure}[b!]
  \label{fig1:4}
  \centering\includegraphics[width=10cm]{Phi_L_tec.png}  \\
  \caption{Картина установившегося обтекания цилиндра (линии тока): $\theta$~--- угол отрыва потока, 
           $L$~--- длина рециркуляционной зоны}
\end{figure}

Реализуемому в расчете числу Рейнольдса $\mathrm{Re}=40$ соответствует стационарный режим ламинарного 
обтекания цилиндра. В то же время расчет проводится в рамках нестационарной постановки задачи, в которой на 
каждом слое по времени $t$ решение устанавливается по псевдовремени $\tau$. Критерием сходимости
итераций $s$ по псевдовремени было условие
\begin{equation}
  \label{56} 
  \left\| {\bf{R}}^t\frac{3({\bf{Q}}V)^{n + 1, s} - 4\left( {{\bf{Q}}V} \right)^n  + \left( {{\bf{Q}}V} 
  \right)^{n - 1}}{2\Delta t}-\left({\bf{RHS}}^{\,t}\right)^{n+1, s}\right\|
  \leq10^{-6},
\end{equation}
где норма определяется как
\begin{equation*}
  \parallel {\mathbf F}\parallel=\max_{i,j,k,m}|F^m_{ijk}|.
\end{equation*}
Здесь $m=1,\ldots,4$ --- номер уравнения (соответствует номеру координаты вектора $\mathbf F$), $i,j,k$~--- 
номера ячеек по соответствующим координатным направлениям.

При выполнении условия \eqref{56} осуществляется переход на следующий слой по физическому времени и  
продвижение цилиндра по оси $Ox$ влево. В процессе движения цилиндра в его окрестности формируется и 
устанавливается поле течения, характерное для обтекания неподвижного цилиндра потоком жидкости с 
соответствующими параметрами. По достижении цилиндром центра расчетной области решение становится стационарным
и картина обтекания цилиндра приобретает вид, изображенный на рисунке~\ref{fig1:4}, где хорошо видны две 
сформировавшиеся симметричные рециркуляционные зоны на подветренной стороне цилиндра.

Полярный угол $\theta$ точки отрыва потока с цилиндра и длина рециркуляционной зоны~$L$ в~зависимости от 
положения цилиндра показаны на рисунке~\ref{fig1:5}.
\begin{figure}[t!]
  \label{fig1:5}
  \centering\small \emph{а}\hspace*{79mm}\emph{б}\\
  {\includegraphics[width=8.15cm]{Phi_sep_n.png}}\hfill
  {\includegraphics[width=8.15cm]{L_sep_n.png}}
  \caption{Зависимости $\theta$ (\emph{а}) и $L$ (\emph{б}) от положения центра цилиндра: серым 
           цветом выделены интервалы изменения значений параметров в сводных 
           данных~\cite{takami,dennis,belocirk,tuan,braza,coutan,tritt}, 
           пунктир~--- расчет в стационарной постановке \cite{Cher}}
\end{figure}
\begin{figure}[t!]\vspace*{2mm}
  \label{fig1:6}
  \centering\small \emph{а}\hspace*{79mm}\emph{б}\\
  {\includegraphics[width=8.15cm]{CD_n.png}}\hfill
  {\includegraphics[width=8.15cm]{Cp_n.png}}
  \caption{Зависимости $C_D$ (\emph{а}) и $C_p$ (\emph{б}) от положения центра цилиндра: серые 
           полосы~--- интервалы изменения значений параметров в сводных 
           данных~\cite{takami,dennis,belocirk,tuan,braza,coutan,tritt},
           штрих~--- расчет в стационарной постановке \cite{Cher}}
\end{figure}
Коэффициент сопротивления
\begin{equation*}
  C_{D} = \frac{2}{{\rho U^2}}\int\limits_S {\left( {p + \frac{2}{{{\mathop{\rm Re}\nolimits} }}
          \frac{{\partial u}}{{\partial n}}} \right)} n_x dS
\end{equation*}
и коэффициент давления  в передней критической точке
\begin{equation*}
  C_p=\frac{2(p-P)}{\rho U^2}
\end{equation*}
показаны в процессе движения цилиндра на рисунке~\ref{fig1:6}. Там же серым цветом отмечены интервалы 
изменения указанных параметров в сводных 
данных~\cite{takami,dennis,belocirk,tuan,braza,coutan,tritt}, а пунктиром~--- значения этих 
параметров, рассчитанные в постановке <<неподвижный цилиндр в потоке жидкости>> в стационарном приближении на 
неподвижной сетке \cite{Cher}. Видно хорошее совпадение параметров потока, полученных в нестационарной 
постановке на движущейся сетке, с экспериментальными данными и с результатами расчета стационарного обтекания 
цилиндра на неподвижной сетке.

Необходимо отметить, что рассмотренный подход  при решении данной задачи является более затратным по времени 
по сравнению с традиционным. Кроме того, как следует из рисунка~\ref{fig1:5}-\ref{fig1:6}, движение цилиндра 
сопровождается осцилляциями параметров рассчитываемого потока, природа которых связана с изменением на каждом 
шаге по времени аппроксимирующей поверхность цилиндра ломаной кривой~--- многогранника. Вершины этого 
многогранника меняют свое положение относительно сформировавшегося на предыдущем шаге по времени потока, что 
в свою очередь меняет значения наблюдаемых параметров. Возникающие осцилляции являются нефизическими~---  
обтекаемая поверхность колеблется как бы искусственно. В то же время в задачах с  подвижными границами 
подобное поведение параметров потока будет соответствовать физике явления и в задаче обтекания цилиндра в 
такой постановке продемонстрирована хорошая точность моделирования течения на подвижной сетке.


\subsection{Моделирование переходных процессов в гидротурбинах}
\label{s:153}
Для валидации разработанного численного метода было проведено сравнение результатов расчетов ряда 
переходных режимов с данными, полученными в натурных экспериментах. Результаты сравнения приводятся в 
разделе~\ref{s:24}. Анализ результатов решений показал, что метод корректно 
воспроизводит все рассмотренные характеристики и может быть использован для моделирования течений
в областях с подвижными границами в различных задачах вычислительной гидродинамики.

\section{Ускорение решения нестационарных задач динамики несжимаемой жидкости}
\label{s:16}
\subsection{Подходы к ускорению решения нестационарных задач}
\label{s:161}
Наиболее общей является постановка задачи гидродинамики турбин, в которой моделирование 
нестационарного течения проводится во всем проточном тракте. Нестационарная постановка позволяет моделировать 
весь диапазон режимов работы гидротурбин, в том числе и режимы неполной загрузки, учитывать взаимодействие 
ротора и статора турбины, описывать пульсации сил и моментов на лопатках, моделировать прецессирующий 
вихревой жгут за ротором и т.д. Для расчета потока строится многосвязная блочно-структурированная сетка, 
покрывающая весь проточный тракт турбины (рисунок~\ref{fig1:12}).  

\begin{figure}[ht!]
  \label{fig1:12}
  \centering                                                                                   
  \includegraphics[width=12cm]{SAKDO_domain.png} \\
  \caption{Блоки сеток, покрывающие проточный тракт гидротурбины}
\end{figure}

При этом нестационарная полная постановка требует значительных вычислительных 
ресурсов, а повышение точности расчетов за счет увеличения количества ячеек сетки приводит к нехватке 
оперативной памяти персонального компьютера. Например, расчет периодически нестационарного течения с 
прецессирующим вихревым жгутом на одном периоде (около трех оборотов рабочего колеса) с использованием сетки, 
содержащей  1 млн. узлов, требует 10 дней работы процессора Core2Duo 2.6 ГГц.

Подходы к ускорению нестационарных расчетов можно разделить на две группы. К первой отнесем все подходы, 
направленные на более быстрое решение системы линейных алгебраических уравнений (СЛАУ) 
\eqref{discret} на одном процессоре. Ко второй -- подходы, использующие 
параллельные алгоритмы расчета. В работе \cite{universe} 
проведено сравнение некоторых методов обращения матрицы СЛАУ, показавшее, что 
существенного уменьшения времени расчета относительно метода, использованного в диссертации, добиться не 
удается. Поэтому наиболее перспективным подходом к ускорению нестационарных расчетов является 
распараллеливание метода. 

Основной характеристикой, определяющей эффективность 
параллельной программы, является относительное 
сокращение времени счета, получаемое при использовании нескольких процессоров. 
Ускорение счета $S_N$ определяются следующим образом:
\begin{equation}
  S_N = \frac{T_1}{T_N},
\end{equation}
где $T_1$ -- время счета на одном процессоре, $T_N$ -- на $N$ процессорах. В идеальной ситуации ускорение счета равняется числу 
используемых процессоров.

В действительности же время, затрачиваемое на обмен данными между процессорами, неравномерная загрузка 
процессоров препятствуют идеальной ситуации и уменьшают получаемое ускорение счета.
При этом согласно закону Амдала \cite{amdal} ускорение работы программы на $N$ процессорах
\begin{equation}
  S_N \leq \frac{1}{f+\frac{1-f}{N}},
\end{equation}
где $f$ -- доля последовательного кода в программе. То есть, если например доля последовательного кода 
составляет 2~\%, то более чем 50-кратное ускорение в принципе получить невозможно. Таким образом, применение 
многопроцессорных ЭВМ является наиболее целесообразным подходом для создания модулей оперативного расчета 
динамических нагрузок, и преодоления ограничений, свойственных однопроцессорным машинам в терминах времени 
счета. Более того, лучше всего использовать геометрическое распараллеливание, заключающееся в декомпозиции
всей расчетной области на блоки, каждый из которых рассчитывается на отдельном ядре многопроцессорной 
вычислительной системы. В этом случае доля последовательного кода $f \sim 0$.

\subsection{Реализация геометрического распараллеливания}
\label{s:162}
В настоящей работе распараллеливание счета осуществляется распределением блоков расчетной сетки на процессоры 
кластера. Расчетная область естественным образом разбивается на 5 элементов -- спиральная камера, 
направляющий аппарат, рабочее колесо, диффузор и отсасывающая труба. Каждый элемент в свою очередь 
разбивается на односвязные блоки, топологически эквивалентные параллелепипедам. НА, РК, диффузор и бетонная 
спиральная камера представляют собой кольца из блоков, межпроцессорные обмены в них устроены одинаково. 
Для стальной спиральной камеры и ОТ были разработаны отдельные процедуры обмена полями течения, т.к. они 
имеют другую топологию, отличную от кольца. Вначале происходит параллельный расчет течения внутри блоков и 
обмены полями между блоками. Обмены между элементами расчетной области производятся после завершения расчета 
во всех блоках перед переходом на следующую итерацию по псевдовремени.

Коммуникации между процессорами осуществляются с использованием стандарта MPI~\cite{MPI}. 
Для посылки сообщений использовалась неблокирующая процедура MPI\_ISEND, для 
получения -- блокирующая MPI\_RECV. 
Если расчет блоков, имеющих общую границу, проводится на одном процессоре, то межпроцессорных обменов не 
проводится.

\subsection{Результаты численных экспериментов}
\label{s:163}
\subsubsection{Многопроцессорные системы}
\label{s:1631}
Решение задачи с использованием разработанного программного комплекса проводилось на многопроцессорных 
системах, представленных в таблице~\ref{tab:2}.
\begin{table}[h!]
  \caption{Рассмотренные многопроцессорные системы}
  \center
  \begin{tabular}{|c|c|c|c|c|}
  \hline
  Кластер & $N_{proc}$ (узлы$\times$ & Название проц.,    & RAM         &  Компилятор \\
          & $\times$проц.$\times$ядра) & размер кэш-памяти L3 & на узел &
  \\ \hhline{=|=|=|=|=} 
  ИВЦ НГУ & 1152  & Intel Xeon X5670  & 24 ГБ & Intel Fortran \\
          & (96 $\times$ 2 $\times$ 6)& 2.93 ГГц, 12 MБ & & 10.1.015
  \\ \hline
  ССКЦ  & 1152  & Intel Xeon X5670  & 24 ГБ & Intel Fortran \\
  НКС-30Т & (96 $\times$ 2 $\times$ 6)& 2.93 ГГц, 12 MБ & & 13.1.000
  \\ \hline
  Персональ- & 8  & Intel Core i7 950  & 12 ГБ & Intel Fortran \\
  ный комп.  & (1 $\times$ 2 $\times$ 4)& 3.06 ГГц, 8 MБ & & 11.1.051
  \\ \hline
  \end{tabular}
  \label{tab:2}
\end{table}
\begin{table}[h!]
  \caption{Число ячеек в расчетной сетке, покрывающей проточный тракт}
  \center
  \begin{tabular}{|c|c|c|c|c|c|c|}
  \hline
    & СК & НА & РК & ДФ & ОТ & Всего 
  \\ \hhline{=|=|=|=|=|=|=} 
   Число & 18  & 32 & 4  & 4 & 3 & 61 \\
   блоков&     &    &    &   &   & 
  \\ \hline
   Число  &         &        &         &       &         &   \\
   ячеек  &  29 232 & 44 573 &  64 757 & 8624  & 28 594  & - \\
   в блоке&         &        &         &       &         & 
  \\ \hline
   Общее  &          &           &         &        &         &           \\
   число  &  526 176 & 1 426 336 & 259 028 & 34 496 & 85 782  & 2 331 818 \\
   ячеек  &          &           &         &        &         & 
  \\ \hline
  \end{tabular}
  \label{tab:3}
\end{table}

\subsubsection{Расчетная область и ее размеры}
\label{s:1632}
Проточный тракт гидротурбины включает следующие элементы: спиральную камеру (СК), направляющий 
аппарат (НА), рабочее колесо (РК), конус отсасывающей трубы (диффузор, ДФ) и отсасывающую трубу (ОТ). 
Направляющий аппарат имеет 32 лопатки,
РК -- 4 лопасти. В таблице~\ref{tab:3} приведены количество блоков в каждом элементе гидротурбины и число ячеек 
в каждом блоке. В принятом подходе к распараллеливанию максимальное число процессоров, которое может 
использоваться  для расчета течения в элементе гидротурбины, не может превышать числа блоков, на которые 
разбит этот элемент.
\begin{figure}[h!]
  \label{fig1:13}
  \centering\small{\it а}\hspace*{90mm}{\it б}\\
  {\includegraphics[width=8.5cm]{speed_Up_HA.png}}\hfill
  {\includegraphics[width=8.5cm]{speed_Up_HA_eff.png}}
  \caption{Ускорения (\it{а}) и эффективности (\it{б}) распараллеливания НА, полученные на различных 
           многопроцессорных системах: \it{1} -- идеальное ускорение, \it{2} -- ССКЦ, 
           \it{3} -- ИВЦ НГУ, \it{4} -- ПК}
\end{figure}

\subsubsection{Расчет потока в направляющем аппарате}
\label{s:1633}
Проведены расчеты течения в направляющем аппарате на различном числе процессоров 
(1, 2, 3, 4, 6, 8, 12, 16, 24, 32). На рисунке~\ref{fig1:13} представлены полученные ускорения и 
эффективности счета $E_N = \dfrac{S_N}{N}$ на различных кластерах. 
Для сравнения там же приведено идеальное ускорение, равное числу используемых процессоров и идеальная
эффективность, тождественно равная 1. 
Видно, что на кластере НГУ наиболее оптимально задача решается при использовании 4 процессоров, 
при этом получено сверхлинейное ускорение счета. 
Такая особенность объясняется <<эффектом кэша>>, когда блоки расчетной сетки начинают 
полностью помещаться в кэш-память процессора. Дальнейшее увеличение числа используемых процессоров $N$ 
приводит к снижению эффективности $E_N$. Наибольшая эффективность счета достигается, когда число используемых 
процессоров является делителем числа блоков в элементе, так как в этом случае нагрузка на каждый процессор 
одинакова.

Двукратное отставание реального ускорения от идеального объясняется медленной скоростью обмена данными между
процессорами. Измерения времени в расчете показали, что время расчета одного блока НА на одном 
процессоре $t_\text{расчета}$ примерно равно времени обмена полями течения на границах между двумя 
процессорами $t_\text{обмена}$. Таким образом, 
\begin{equation}
  T_{32} = t_\text{расчета} + t_\text{обмена} \sim 2 t_\text{расчета},
\end{equation}
что и подтверждает рисунок~\ref{fig1:13}.

\subsubsection{Расчет потока во всем проточном тракте}
\label{s:1634}
Расчетная область включает спиральную камеру (18 блоков), направляющий аппарат (32 блока), рабочее колесо 
(4 блока), диффузор (4 блока) и отсасывающую трубу (3 блока). 
Число блоков и ячеек в расчетной сетке указаны в таблице~\ref{tab:3}.
Расчеты проводились на 1, 5, 6, 7, 9, 11, 13, 15, 17, 19, 21, 23, 25, 28, 36, 45 и 61 процессорах. 

Для получения наибольшего ускорения необходимо добиваться равномерности числа расчетных ячеек на каждый 
процессор. Внутри одного элемента гидротурбины, лучше брать число процессоров, являющееся делителем числа 
блоков этого элемента. Тогда для аппарата число процессоров может принимать значения степеней двойки (т.к. 
$32=2^5$), для рабочего колеса и диффузора -- 1, 2, 4, для трубы -- 1 или 3. 

\begin{figure}[ht]
  \label{fig1:14}
  \centering\small{\it а}\hspace*{90mm}{\it б}\\
  {\includegraphics[width=8.5cm]{speed_Up_all.png}}\hfill
  {\includegraphics[width=8.5cm]{speed_Up_all_eff.png}}
  \caption{Ускорения (\emph{а}) и эффективности (\emph{б}) распараллеливания всей гидротурбины, 
           полученные на различных многопроцессорных системах: \emph{1} -- идеальное ускорение, 
           \emph{2} -- ССКЦ, \emph{3} -- ИВЦ НГУ}
\end{figure}

Например, если взять максимально возможное число процессоров, равное 61, то вычислительная нагрузка для блока
рабочего колеса будет самой большой и составит 64 757 ячеек на процессор. На рисунке~\ref{fig1:14} 
приведены результаты распараллеливания расчета в полной постановке, полученные на кластерах ИВЦ НГУ и ССКЦ. 
Четырехкратное отставание от идеального значения обусловлено неравномерностью распределения ячеек расчетной 
сетки по процессорам и низкой скоростью обмена данными между процессорами. Из проведенных исследований 
следует, что при использовании 61 счетного процессора время решения задачи моделирования трехмерного 
турбулентного течения во всей гидротурбине на сетке с общим количеством ячеек около 2.5 млн сокращается более 
чем в 15 раз и составляет от 1 до 2 дней, что вполне приемлемо для практического применения.

  %!TEX root = dissertation.tex
% \pagestyle{plain}
\chapter*{Глава 2. Численная модель переходных процессов в гидротурбинах}
\label{s:2}
\setcounter{chapter}{2}
\addcontentsline{toc}{chapter}{Глава~\thechapter~ Численная модель переходных процессов в гидротурбинах}
\setcounter{section}{0}

\section{Проблемы, возникающие при моделировании переходных процессов в гидротурбинах}
\label{s:20}
Течения в проточном тракте (ПТ) гидротурбины (ГТ) подразделяются на установившиеся и переходные.
К установившимся относятся течения при неизменных значениях расхода воды $Q$ через ПТ, частоты
вращения рабочего колеса, нагрузки на вал РК. Эти течения являются стационарными или
периодически нестационарными и достаточно адекватно моделируются исследователями в полной
трехмерной постановке \cite{Cher,Rupr,Vunen1,Vunen2,Vu}.

Переходные режимы течений возникают при переводе работы ГТ из одного состояние в другое путем регулировки 
открытия направляющего аппарата или при увеличении (уменьшении) нагрузки на вал РК. Такие течения
являются существенно нестационарными и характеризуются резкими колебаниями расхода воды во времени. Это 
приводит к динамическому изменению давления в ПТ, проявляющемуся в форме гидравлического удара 
$\Delta H({\bf x}, t)$~\cite{jukovskii}. Гидравлический удар $\Delta H({\bf x}, t)$ может как повышать
общий напор на турбине
\begin{equation}
H({\bf x}, t)=H_0+\Delta H({\bf x}, t)
\end{equation}
в случае, если он положительный, так и понижать, когда он отрицательный. Здесь $H_0$~–-~разность полных 
энергий потоков между уровнями верхнего и нижнего бьефов гидроэлектростанции.
Следовательно, при расчете переходных режимов трехмерных течений возникают новые проблемы при создании для них 
численных моделей, не имевшиеся в случае моделирования установившихся течений. 

Первая проблема -- описание величины гидравлического удара $\Delta H({\bf x}, t)$. 
Явление гидравлического удара можно было бы
непосредственно моделировать, если учитывать упругие деформации аэрированной воды и стенок ПТ. При этом 
пришлось бы рассчитывать течение во всем ПТ ГТ, включая напорный водовод. Это привело бы к колоссальным 
вычислительным затратам и поэтому в настоящее время недостижимо. Если отказаться от учета деформации стенок 
ПТ, а моделирование течения воды проводить в приближении несжимаемой жидкости, то, рассчитав поток во всем ПТ, 
можно было бы получить картину <<жесткого>> гидравлического удара. Однако, найденное в такой постановке 
решение в случае длинного водовода и короткого времени изменения режима будет сильно отличаться от реального
переходного процесса в силу неограниченного возрастания величины гидравлического удара $\Delta H({\bf x}, t)$.

Вторая проблема -- использование на входной и выходной границах ПТ краевых условий, не фиксирующих расход воды 
$Q$ через них. Численное моделирование течений в ГТ осуществляется, как правило, не во всем ПТ, а в основных 
его элементах -- НА, РК и отсасывающей трубе (ОТ) \cite{Cher}. Одной из наиболее распространенных постановок 
условий во входном и выходном сечениях ПТ указанной цепочки элементов является задание распределения вектора 
скорости на входе и давления на выходе. Однако из-за изменения во времени расхода воды эта постановка при 
моделировании переходных процессов становится неприемлемой. Предложенная в \cite{bannikov} альтернативная 
постановка позволяет находить расход в процессе расчета течения. В этой постановке в обоих сечениях задаются 
полные энергии, кроме того, во входном сечении требуется информация о направлении потока, а в
выходном -- о профиле давления. Задаваемые в сечениях энергии могут меняться во времени. Такая постановка 
больше отвечает задачам о переходных процессах и будет использована при их решении.

При моделировании переходных течений в каналах НА с открывающимися или закрывающимися лопатками форма ячеек 
сетки не является фиксированной и подстраивается под положение поверхностей лопаток, являющихся границами 
расчетной области. Это приводит к необходимости использования подвижных, меняющихся со временем сеток. 
Распределение узлов сетки внутри расчетной области определяется распределением узлов на поверхностях лопаток. 
Движение поверхностей контролируется законом изменения регулирующего органа ГТ. Таким образом, третьей 
проблемой является обобщение разработанного ранее авторами численного метода решения трехмерных уравнений 
несжимаемой жидкости на фиксированных сетках на подвижные сетки. Особо важным требованием, которому
должен удовлетворять численный метод решения уравнений движения жидкости на подвижной сетке, является 
выполнение условия геометрической консервативности (УГК). Суть его состоит в том, что если решением 
рассматриваемой задачи является однородный поток, то метод с двигающейся сеткой должен выдавать этот же поток 
без какого-либо возмущения.

Наконец, четвертой проблемой, возникающей при обращении от моделирования установившихся течений к 
переходным, становится определение скорости вращения РК, которая в процессе переходного течения изменяется, 
подчиняясь закону вращения твердого тела под действием гидродинамического момента со стороны жидкости.

Изучение переходных процессов и правильный их учет при проектировании силового узла ГЭС представляют большой 
практический интерес. Задача анализа переходных процессов при изменении мощности состоит в том, чтобы найти 
оптимальные режимы регулирования, при которых изменение момента турбины будет происходить с наибольшей 
скоростью при соблюдении ограничений на возникающие динамические воздействия, в том числе от гидравлического 
удара.

\begin{figure}[h]
  \centering
  \includegraphics[width = 13cm]{VOD_GT_new.png}
  \caption{Схема проточной части ГЭС и геометрия турбины}
  \label{fig2:1}
\end{figure}

Предложенный гибридный подход включает в себя модель переходного процесса, 
состоящую из нестационарных усредненных по Рейнольдсу трехмерных уравнений Навье-Стокса, 
замкнутых $k-\varepsilon $ моделью турбулентности Кима-Чена, 
решаемых в областях с меняющимися во времени границами \cite{my1}; уравнения вращения РК как твердого целого и 
одномерных уравнений распространения упругого гидроудара в водоводе (рисунок~\ref{fig2:1}). 
Уравнения модели переходного процесса 
замыкаются новой постановкой краевых условий во входном и выходном сечениях ПТ ГТ, предложенной в 
\cite{bannikov}, и условиями сопряжения на границе водовод-гидротурбина.

Представлены результаты моделирования ряда переходных процессов в ГТ: пуска в турбинный режим, уменьшения 
мощности посредством закрытия лопаток НА, мгновенного сброса нагрузки. Проводится сравнение полученных 
результатов с экспериментальными данными.

\section{Основные уравнения гибридной модели}
\label{s:21}
\subsection{Уравнения Рейнольдса в форме интегральных законов сохранения для движущегося объема} Поскольку 
переходные процессы сопряжены с изменением границ расчетной области -- поворотом лопаток направляющего 
аппарата -- для моделирования пространственного течения несжимаемой жидкости используются
осредненные по Рейнольдсу уравнения Навье-Стокса, записанные в виде интегральных законов сохранения \eqref{6} 
для движущегося объема $V(t)$ \cite{my1}.

Величины $\nu_t$ и $k$ определяются по  $k-\varepsilon$ модели турбулентности Кима-Чена \cite{kimchen} с 
логарифмической пристеночной функцией вблизи твердых стенок.

\subsection{Уравнения $k-\varepsilon$ модели турбулентности Кима-Чена в интегральной форме для 
            движущегося объема} 
Каждое из уравнений $k-\varepsilon$ модели Кима-Чена \cite{kimchen} также может быть записано в виде
интегрального закона сохранения для движущегося объема $V(t)$
\begin{equation}
  \label{eq2:5} 
  \frac{\partial }{{\partial t}}\int\limits_{V\left( t \right)} {\varphi dV =  - } 
  \oint\limits_{\partial V\left( t \right)} {\varphi \left( {{\bf{w}} - {\bf{x}}_t } \right) 
  \cdot d{\bf{S}}} + \oint\limits_{\partial V\left( t \right)} {\left( {\nu  + \frac{{\nu _t }}
  {{\sigma _\varphi  }}} \right)\nabla \varphi \cdot d{\bf{S}}}  + \int\limits_{V\left( t \right)} 
  {H_\varphi dV},
\end{equation}
где $\varphi$ и $H_\varphi$ приведены в таблице~\ref{tab2:1}, 
${\bf{w}}=\left( {w_1 ,w_2 ,w_3 } \right)$, $\nu _t = C_\mu k^2/{\varepsilon }$.
\begin{table}[ht!]
  \label{tab2:1}
  \center
  \caption{Значения $\varphi$ и $H_\varphi$ в уравнении \eqref{eq2:5}}
  \begin{tabular}{|c|c|c|}
  \hline
   Уравнение & $\varphi$ & $H_\varphi$ \\
  \hline
  Для турбулентной  & $k$ & $G-\varepsilon$  \\
  кинетической энергии & &
  \\ \hline 
  Для скорости диссипации & $\varepsilon$ & 
  $C_{\varepsilon 1}\varepsilon G/k-C_{\varepsilon 2}\varepsilon^2/k + C_{\varepsilon 3} G^2 /{k}$ \\
  турб. кин. энергии & & \\ \hline 
  \end{tabular}
\end{table}

В таблице~\ref{tab2:1} $G = \tau _{ij} \dfrac{{\partial w_i }}{{\partial x_j }}$, значения констант
\begin{equation*}
C_\mu=0.09,\ C_{\varepsilon 1}=1.15,\ C_{\varepsilon 2}=1.9,\ C_{\varepsilon 3}=0.25,\ {{\sigma _k = 0.75}},
\ {{\sigma _\varepsilon = 1.15}}.
\end{equation*}
\subsection{Уравнение вращения рабочего колеса}
Большинство переходных процессов сопровождается изменением частоты вращения РК. Заранее эта зависимость не 
известна, поэтому одновременно с решением уравнений \eqref{6},\eqref{eq2:5} решается уравнение вращения РК 
как твердого целого
\begin{equation}
  \label{eq2:6} 
  I_z \frac{{d\omega }}{{dt}} = M_R (t) - M_{gen} (t) - sgn(\omega) M_{\text{тр}},
\end{equation}
где $I_z$ -- суммарный момент инерции РК и генератора, $M_R$ -- крутящий момент, обусловленный действием 
потока на РК, $M_{gen}$ -- момент полезной нагрузки, приложенный к валу электрогенератора,
$M_{\text{тр}}$ -- результирующий момент трения в электромеханической системе агрегата. Зависимость 
$M_{gen}(t)$ как правило, известна, в то время как $M_R(t)$ определяется гидродинамикой потока, т.е. находится 
из решения уравнений \eqref{6},\eqref{eq2:5}.

\subsection{Модель упругого гидравлического удара}
Для моделирования гидроакустических колебаний в напорном водоводе используется хорошо зарекомендовавшая себя 
одномерная модель упругого гидроудара \cite{jukovskii,krivch}. В случае, если трением жидкости о стенки
пренебрегается, эта модель может быть записана в виде системы дифференциальных уравнений
\begin{equation}
  \label{eq2:7}
  \left\{ \begin{aligned}
  \frac{{\partial m}}{{\partial t}} + \frac{{c^2 }}{{gS}}\frac{{\partial Q}}{{\partial \xi}} = 0 \\
  \frac{{\partial Q}}{{\partial t}} + gS\frac{{\partial m}}{{\partial \xi}} = 0 \\
  \end{aligned} \right.,\
  \xi \in [0,L],
\end{equation}
где $m(\xi,\ t) = \dfrac{p(\xi,\ t)}{{\rho g}} - z(\xi)$ -- потенциальный напор, $Q(\xi,\ t)$ -- расход 
жидкости, $S(\xi)$ -- площадь сечения водовода, $c(\xi)$ -- скорость распространения упругой волны удара, 
$L$ -- длина водовода. Скорость $c$ определяется концентрацией нерастворимого газа в воде и упругостью
стенок водовода \cite{krivch}. Для реальных водоводов $c= 1000\div 1450$ м/с.

\section{Краевые условия при совместном расчете течения в области водовод-гидротурбина}
\label{s:22}
\subsection{Входная и выходная границы} 
При расчете течений в гидротурбинах общеупотребительной постановкой краевых условий на входе и выходе области 
является постановка <<расход-давление>>, в которой во входном сечении задается расход жидкости и угол входа 
потока или распределение вектора скорости, а в выходном сечении -- распределение давления и касательные
составляющие скорости ${\bf w}\cdot\tau_i|_{\text{вых}},\ i=1,\,2$ ($\tau_1,\ \tau_2$ -- линейно 
независимые векторы, касательные к выходному сечению) \cite{AntKajMon}.

Однако в задачах моделирования переходных процессов расход жидкости меняется во времени и эта зависимость 
заранее не известна. В то же время полный напор, равный разности энергий на входе в водовод и на выходе из ОТ, 
остается неизменным даже в ходе переходного процесса. Поэтому в настоящей работе предлагается использовать 
краевые условия, которые позволяют по заданному напору $H_0$ определять расход жидкости одновременно с 
нахождением поля течения. Во входном сечении водовода $\xi=0$ (рисунок~\ref{fig2:1}) задается полная энергия потока
\begin{equation}
  \label{eq2:8}
  E_{\text{\it ВВ, вх}}\equiv m(0,t) + \frac{{Q^2 \left( {0,\,\,t} \right)}}{{2gS^2 }} = H_0.
\end{equation}
В выходном сечении ОТ задаются значение усредненной по расходу полной энергии
\begin{equation}
  \label{eq2:9}
  E_{\text{\it ОТ, вых}}\equiv \frac{1}{Q}\int\limits_{S_{\text{\it ОТ, вых}}}\left(\frac{p}
  {\rho g}-z+\frac{{\bf w}^2_{\text{\it ср}}}{2g}\right)\left({\bf w}d{\bf S}\right)=0,
\end{equation}
где ${\bf w}_{\text{\it ср}}=\frac{Q}{S_{\text{\it ОТ, вых}}} $, $S_{\text{\it ОТ, вых}}$ -- площадь выходного 
сечения ОТ, и условие на профиль статического давления
\begin{equation}
  p = p_0  + \rho g(z - z_0 ).
\end{equation}
Отметим, что значение $p_0$ не фиксируется априори, а определяется в процессе решения уравнений движения 
жидкости так, чтобы выполнялось равенство \eqref{eq2:9}.

Такая постановка входных и выходных условий отвечает физике переходного процесса. Фактически 
условия \eqref{eq2:8}-\eqref{eq2:9} означают, что уровни верхнего и нижнего бьефов не меняются.

\subsection{Граница обмена между водоводом и НА} 
При совместном расчете течения в области <<водовод-гидротурбина>> необходимо обеспечить корректную передачу 
параметров из одной области в другую. Трудность состоит в том, что в рассматриваемой экономичной циклической 
постановке спиральная камера (СК) и статор (СТ) не входят в расчетную область. Поэтому обмен параметрами 
течения нужно произвести между выходной границей водовода $\xi=L$ и входным сечением 
НА (рисунок~\ref{fig2:1}). Расход воды в 
этих сечениях одинаков, поэтому на входе в НА задаются значение расхода $Q_{\text{\it ВВ, вых}}$, полученное в 
ходе решения системы уравнений \eqref{eq2:7} в водоводе, и угол входа потока $\delta_{sp}=const$. Давление на 
входе в НА экстраполируется изнутри расчетной области.

Опишем более подробно передачу давления из области НА на выходную границу водовода. Выведем соотношение между 
давлением $p_{\text{\it ВВ, вых}}$ на выходе из водовода и давлением $p_{\text{\it НА, вх}}$ на входе в НА при 
условии, что суммарные потери энергии в СК и СТ оцениваются по формуле
\begin{equation}
  \Delta h_{\text{\it СК}}+\Delta h_{\text{\it СТ}} = \zeta_s H_0,
\end{equation}
где $\zeta_s$ -- коэффициент потерь в спиральной камере и статоре. Вообще, $\zeta_s\sim Q^2$~\cite{topaj}, 
но на практике для простоты можно принять, что $\zeta_s=const\sim 0.01$. Полная энергия на выходе из водовода
\begin{equation}
  E_{\text{\it ВВ, вых}}=(\frac{p}{\rho g}-z)_{\text{\it ВВ, вых}}+\frac{|{\bf w}|^2_{\text{\it ВВ, вых}}}{2g},
\end{equation}
где $|{\bf w}|_{\text{\it ВВ, вых}}=\frac{Q}{S_{\text{\it ВВ, вых}}}$, так как в модели предполагается, что 
скорость потока на выходе из водовода перпендикулярна сечению. На входе в НА
\begin{equation}
  E_{\text{\it НА, вх}}=(\frac{p}{\rho g}-z)_{\text{\it НА, вх}}+\frac{|{\bf w}|^2_{\text{\it НА, вх}}}{2g}.
\end{equation}
В цилиндрической системе координат вектор скорости на входе в НА ${\bf w}_{\text{\it НА, вх}}$ имеет 
компоненты $(c_r,c_z,c_u)$, при этом осевая компонента скорости $c_z=0$. Радиальная и окружная
связаны соотношением $\dfrac{c_r}{c_u}={\text tg }\,\delta_{sp}$. Отсюда
\begin{equation}
  |{\bf w}|_{\text{\it НА, вх}}=\sqrt{c_r^2+c_u^2}=\frac{c_r}{\sin\delta_{sp}}=\frac{Q}
  {S_{\text{\it НА, вх}}\sin \delta_{sp}}.
\end{equation}
Таким образом, учитывая
\begin{equation}
  E_{\text{\it ВВ, вых}}=E_{\text{\it НА, вх}}+\zeta_s H_0
\end{equation}
и равенство уровней $z_{\text{\it ВВ, вых}}=z_{\text{\it НА, вх}}$, получаем соотношение для статических 
давлений
\begin{equation}
  \label{eq2:16} 
  \frac{p_{\text{\it ВВ, вых}}}{\rho g}=\frac{p_{\text{\it НА, вх}}}{\rho g}+\frac{Q^2}{2g}
  \left(\frac{1}{S_{\text{\it НА, вх}}^2\, \sin^2\delta_{sp}}-\frac{1}{S_{\text{\it ВВ, вых}}^2 }\right)+
  \zeta_s H_0.
\end{equation}

\subsection{Остальные границы} 
Расчет трехмерного течения в НА и РК проводится в циклической постановке -- в одном межлопаточном канале НА, 
одном межлопастном канале РК и всей ОТ. На границах протекания жидкости из одного канала в другой ставится 
условие периодичности. На твердых стенках ставится условие прилипания потока. При передаче параметров на 
границах обмена между НА и РК, РК и ОТ производится осреднение потока в окружном направлении.

\section{Численный метод совместного решения уравнений модели переходного течения}
\label{s:23}
\subsection{Решение уравнений Рейнольдса}
Численный метод решения нестационарных уравнений \eqref{6} описан в п.~\ref{s:12}.

\subsection{Решение уравнений упругого гидроудара}
Система \eqref{eq2:7} -- одномерная гиперболическая система с постоянными коэффициентами. В векторном виде 
система записывается следующим образом
\begin{equation}
  \label{eq2:19} 
  \frac{{\partial {\bf{f}}}}{{\partial t}} +{\bf{A}}
  \frac{{\partial {\bf{f}}}}{{\partial \xi}} = 0,
\end{equation}
где ${\bf{f}} = \left( {\begin{array}{*{20}c} m \\ Q \\ \end{array}}\right),\ 
{\bf{A}} = \left( {\begin{array}{*{20}c} 0 & {c^2 /gS}  \\ {gS} & 0 \\ \end{array}} \right)$. Введем на отрезке
 $[0,L]$ равномерную сетку с шагом $\Delta \xi: \xi_1=0,\,\xi_2=\Delta \xi,\,\xi_3=2\Delta \xi,...,\,\xi_J=L.$
Уравнение \eqref{eq2:19} решается численно по неявной разностной схеме с использованием аппроксимаций против 
потока 1-го порядка по пространству и времени
\begin{equation}
  \label{eq2:20} 
  \frac{{{\bf{f}}_j^{s+1}  - {\bf{f}}_j^n }}{{\Delta t}} + {\bf{A}}^ +  \frac{{{\bf{f}}_j^{s+1} - 
  {\bf{f}}_{j - 1}^{s+1} }}{{\Delta \xi}} + {\bf{A}}^ -  
  \frac{{{\bf{f}}_{j + 1}^{s+1} - {\bf{f}}_j^{s+1} }}{{\Delta \xi}} = 0,\quad j=2,...,J-1,
\end{equation}
где ${\bf{A}}^+ +{\bf{A}}^-={\bf{A}}$, матрицы ${\bf{A}}^+=\left( {
\begin{array}{*{20}c}
   {c/2} & {c^2 /2gS}  \\
   {gS/2} & {c/2}  \\
\end{array}} \right)$ и ${\bf{A}}^-= \left( {\begin{array}{*{20}c}
   { - c/2} & {c^2 /2gS}  \\
   {gS/2} & { - c/2}  \\
\end{array}} \right)$ имеют только неотрицательные $(c,0)$ и неположительные $(-c,0)$ собственные значения, 
соответственно. Уравнение \eqref{eq2:20} итерируется по $s$, при этом ${\bf{f}}^{n+1}=\lim\limits_{s\to
\infty}{\bf{f}}^{s}$. Итерации по $s$ нужны из-за нелинейных краевых условий на входе в водовод \eqref{eq2:8}, 
а также для обмена параметрами с областью НА. Одна итерация по $s$ в области водовода соответствует одному 
шагу по псевдовремени в гидротурбине.

\subsection{Численная реализация краевых условий}
Граничное условие \eqref{eq2:8} во входном сечении водовода ($j=1$) реализуется следующим образом. Пусть
${\bf{f}}_2^{s}=(m_2^s,Q_2^s)$, тогда полагается ${\bf{f}}_1^{s+1}=(H_0-(Q_2^s)^2/{2gS^2},Q_2^s)$. Таким 
образом, во входном сечении значение расхода $Q$ экстраполируется изнутри области с предыдущей итерации, а 
величина потенциального напора $m$ вычисляется так, чтобы полная энергия, вычисленная в узле $j=1$, была равна 
$H_0$.

На границе обмена водовод-НА ($j=J$) полагается
\begin{equation}
  {\bf{f}}_J^{s+1} = \left( {
  \begin{array}{*{20}c}
   {m_{HA} }  \\
   Q^s_{J-1} \\
  \end{array}} \right),
\end{equation}
где $m_{HA}=p_{\text{\it ВВ, вых}}/{\rho g}-z_{HA}$, давление $p_{\text{\it ВВ, вых}}$ определяется 
соотношением \eqref{eq2:16}, в котором $p_{\text{\it НА, вх}}$~--~среднее по сечению статическое давление на 
входе в НА.

На выходе из ОТ гидростатическое распределение давления $p$ подбирается так, чтобы полная энергия потока на 
выходе из ОТ была равна нулю
\begin{equation}
  E_{\text{\it ОТ, вых}}=0,
\end{equation}
а значения скоростей экстраполируются изнутри расчетной области. Необходимо отметить, что при решении 
уравнений \eqref{6} давление определяется с точностью до константы, поэтому в проведенных расчетах энергия 
на выходе полагалась равной нулю. В реальности для рассматриваемой турбины энергия потока на выходе есть
\begin{equation}
  E_{\text{\it ОТ, вых}}=13.532 \text{м вод. ст}.
\end{equation}
Соответственно, для получения абсолютных значений давления в проточном тракте ГТ нужно увеличить приводимые рассчитанные величины на 13.532 м вод. ст.

\section{Результаты расчетов}
\label{s:24}
Разработанный метод применен для расчета основных переходных процессов в натурной радиально-осевой турбине 
с напором $ H_0=73.5$~м, диаметром РК $D_1=3.15$~м и номинальной частотой 
вращения $n_{\text{ном}}=200$ об/мин. 
Момент трения в \eqref{eq2:6} принят равным $M_{\text{тр}}=2$~т*м. Расчеты проведены в циклической
постановке в области, состоящей из водовода, одного межлопаточного канала НА, одного межлопастного канала РК и 
всей ОТ (рисунок~\ref{fig2:2}). Структурированная сетка в подвижной области НА строится автоматически на каждом 
шаге по времени. Расчетная сетка для всей области содержит суммарно около 100 тыс. ячеек.  Для области
водовода используется равномерная сетка с шагом по пространству $\Delta \xi=L/1000$. Во всех расчетах шаг по 
времени $\Delta t$ = 0.01~c, что соответствует повороту РК на $12^{\circ}$. На каждом шаге по времени $t$ 
проводилось 1500 итераций для установления по псевдовремени $\tau$. Характерное время проведения одного
нестационарного расчета с использованием 3 процессоров составляет 2 суток.

Все расчеты проведены в натурных параметрах, однако далее в тексте значения открытия НА $a_0$ (минимальное расстояние между соседними лопатками НА) указаны для соответствующей модели с диаметром РК $D_1=0.46$~м.
\begin{figure}[h]
  \centering
  \includegraphics[width = 9cm]{VV_OT.png}
  \caption{Расчетная область в циклической постановке: ВВ-НА-РК-ОТ}
  \label{fig2:2}
\end{figure}

\subsection{Моделирование переходного режима пуска турбины}
Пуск агрегата -- процесс, при котором РК из состояния покоя переводится в режим холостого хода с частотой 
$n_{\text{ном}}$ с последующей синхронизацией и включением генератора в сеть. При этом открытие НА меняется по 
закону, показанному на рисунке~\ref{fig2:3}, \emph{а}. Процесс пуска обычно имеет два пусковых открытия НА. 
Для рассмотренной турбины первое пусковое открытие НА $ a_{0, \text{по1}}=0.25a_{0,\max }$, где $a_{0,\max }$ 
-- максимальное открытие НА. При достижении РК частоты вращения, равной 90-95\% от номинальной 
$n_{\text{ном}}$, НА прикрывается на второе пусковое открытие $ a_{0, \text{по2}}=0.15a_{0,\max }$. Далее НА 
выходит на открытие холостого хода $a_{0,\text{хх}}=0.10a_{0,\max }$. Весь процесс пуска для рассматриваемой 
турбины в реальности занимает 30~с.

\begin{figure}[htb]
  \centering \small \rule{0mm}{0mm}\emph{a}\rule{90mm}{0mm}\emph{б}\\[1.5mm]
  {\includegraphics[width=8.5cm]{alfa0_a0_ot_article.png}}\hfill
  {\includegraphics[width=8.5cm]{n_ot_t.png}}
  \caption{Пуск в турбинный режим. \emph{а}) закон изменения открытия НА $a_0(t)$ в эксперименте ($\bullet$) 
  и принятый в расчете (\textcolor{blue}{$\genfrac{}{}{2pt}{2}{\quad}{  }$}); \emph{б}) 
  рассчитанная (\textcolor{blue}{$\genfrac{}{}{2pt}{2}{\quad}{  }$}) и 
  экспериментальная ($\bullet$) зависимости скорости вращения РК от времени}
  \label{fig2:3}
\end{figure}

Отметим, что требование невырожденности ячеек сетки не позволяет закрыть лопатки НА до 0~мм, поэтому в 
начальный момент времени $t=0$ задается поле течения, полученное в стационарной постановке при открытии 
$a_0=1$~мм и нулевой частоте вращения РК. Далее открытие лопаток НА за 2.6~с линейно растет до 
$a_{0,\text{по1}}$, при этом в первые 2.1~c частота вращения РК $n=0$~об/мин фиксирована (как в эскперименте). 
Начиная с момента времени $t=2.1$~c, частота $n$ находится в результате решения уравнения \eqref{eq2:6}.

На рисунке~\ref{fig2:3}, \emph{б} приведено сравнение рассчитанной и экспериментальной скоростей вращения РК. 
Получено хорошее качественное и количественное совпадение. При этом в первые 30~с наблюдается небольшое 
(меньше 7\%) отставание рассчитанной скорости вращения РК от экспериментальной. 
Возможно, это связано с неизбежным при 
использовании циклической постановки усреднением по окружности параметров течения при обмене на границе НА-РК.

\begin{figure}[ht!]
  \centering \small \rule{0mm}{0mm}\emph{a}\rule{95mm}{0mm}\emph{б}\\[1.5mm]
  {\includegraphics[width=8.5cm]{Qin_M_pusk_ppt.png}}\hfill
  {\includegraphics[width=8.5cm]{P_HA_exp_comp.png}}
  \caption{Рассчитанные зависимости расхода $Q$, момента $M_R$ (\emph{a}) и давления в НА (\emph{б}) от 
  времени, пуск в турбинный режим: \textcolor{blue}{$\genfrac{}{}{2pt}{2}{\quad}{  }$} 
  расчет, $\bullet$ эксперимент}
  \label{fig2:4}
\end{figure}

\begin{table}[!t]
  \label{tab2:2}
  \centering
  \caption{Параметры точек рассмотрения давления на лопасти РК в процессе пуска} 
  \begin{tabular}{|c|r|r|r|r|}
  \hline
  \No~п/п & время $t$, с & расход, м${^3}/$с & частота $n$, мин$^{-1}$ & момент сил $M_R$, тс$\cdot$м \\
  \hline
  1 & $0.5$ & $3.02$ & 0.00 & 13.34 \\
  2 & $3.0$ & $24.53$ & 12.23 & 142.41 \\
  3 & $15.0$ & $23.35$ & 162.78 & 73.19 \\
  4 & $30.0$ & $9.58$ & 192.08 & 8.49 \\
  \hline
  \end{tabular}
\end{table}

На рисунке~\ref{fig2:4}, \emph{а} изображены зависимости расхода воды $Q$ и момента $M_R$  турбины от времени. 
Зависимость расхода практически повторяет форму зависимости открытия 
НА от времени (рисунок~\ref{fig2:3}, \emph{а}). 
В первые 5~с гидравлический момент, действующий на лопасть РК быстро возрастает и достигает 55\% от
величины момента, действующего на лопасть в режиме оптимального КПД. После достижения максимума величина 
момента плавно снижается, при этом скорость вращения рабочего колеса продолжает возрастать.
При достижении режима холостого хода значение момента $M_R\sim M_{\text{тр}}$, что соответствует постоянному 
значению $n$. Параметры точек рассмотрения давления на лопасти РК приведены в таблице~\ref{tab2:2}. 
\begin{figure}[!h]
  \centering
  \begin{tabular}{cc}
    \includegraphics[natwidth=3702, natheight=2646,trim=  0 0 0 0,clip=true,scale=0.075]{k01_all_005_n.png}&
    \includegraphics[natwidth=3694, natheight=2634,trim=425 0 0 0,clip=true,scale=0.075]{k01_all_300_n.png} \\
    $t_1$=0.5 с&
    $t_2$=3.0 с \\
    \includegraphics[natwidth=3706, natheight=2638,trim=  0 0 0 0,clip=true,scale=0.075]{k01_all_1500_n.png}&
    \includegraphics[natwidth=3678, natheight=2658,trim=425 0 0 0,clip=true,scale=0.075]{k01_all_3000_n.png} \\
    $t_3$=15 с&
    $t_4$=30 с
  \end{tabular} \par
  \caption{Изменение давления на лопасти рабочего колеса при пуске в турбинный режим
  (для каждого момента времени рабочая сторона слева, тыльная -- справа)}
  \label{fig2:5}
\end{figure}

\begin{figure}[hb!]
  \includegraphics[width=8.5cm]{Cz_Cr_Cu.png} \hfill\includegraphics[width=8.5cm]{k01_vihr_article_2.png}\\
  \parbox{8.5cm} {\caption{Профили скорости на выходе из РК в режиме холостого хода}\label{fig2:6}}\hfill
  \parbox{8.5cm} {\caption{Срыв вихря с входной кромки лопасти РК в режиме холостого хода}\label{fig2:7}}
\end{figure}
\begin{figure}[hb!]
  \centering
  \includegraphics[width = 12cm]{z_vortexes.png}
  \caption{Линии тока и изолинии модуля скорости в двух сечениях РК}
  \label{fig2:8}
\end{figure}

На рисунке~\ref{fig2:4}, \emph{б} представлено сравнение рассчитанного давления во входном сечении НА с 
экспериментальным. Построенная модель правильно отражает величину гидроудара, возникающего 
при быстрых изменениях открытия НА.

Распределения давления на лопасти РК в выбранные моменты времени представлены на рисунке~\ref{fig2:5}. В первые 
3~с давление на рабочей стороне вблизи входной кромки лопасти резко возрастает, т.к. жидкость разгоняет РК. 
После этого распределение давления становится более равномерным -- давление на рабочей стороне лопасти 
становится меньше и компенсируется повышающимся давлением на тыльной стороне лопасти. Это приводит к 
уменьшению момента РК $M_R$ до величины момента трения $M_{\text{тр}}$.

На рисунке~\ref{fig2:6} изображены зависимости осевой $C_z$, радиальной $C_r$ и окружной $C_u$ компонент 
скорости от радиуса $R$ в сечении $z=2$~м отсасывающей трубы сразу за РК. При достижении режима холостого 
хода поток собирается и закручивается вблизи стенок ОТ, а в центре наблюдается обширная зона возвратного 
течения. Срыв вихря с входной кромки лопасти РК в режиме холостого хода представленный на 
рисунке~\ref{fig2:7} посредством линий тока, помогает понять структуру потока в канале РК. 
На рабочей стороне лопасти в ее нижней части 
формируется сильно закрученное течение, уходящее к стенкам ОТ. Вблизи обода РК наблюдается насосный вихрь 
(рисунок~\ref{fig2:8}), поток в котором идет вверх.

\subsection{Моделирование течения в гидротурбине при закрытии лопаток направляющего аппарата}
Ниже представлены результаты моделирования процесса уменьшения мощности турбины. 
В начальный момент времени $t=0$ задается стационарное поле течения, полученное при расчете режима 
максимального КПД. Далее открытие лопаток НА уменьшается линейно от $27.5$~мм (соответствует режиму 
максимального КПД) до $18$~мм (режим неполной загрузки) за $10$~c, начиная с момента времени $t=2.5$~c. 
Шаг по времени $\Delta t=0.01$~c соответствует повороту РК на $12^{\circ}$. Подвижная сетка в канале НА на 
каждом шаге по времени строилась алгебраически.

Известно, что в режиме максимального КПД поток за РК имеет слабую закрутку, вследствие чего течение в конусе 
отсасывающей трубы приобретает практически стационарный осесимметричный характер. В режиме неполной загрузки, 
напротив, окружная скорость потока за РК существенна, что приводит к формированию прецессирующего вихревого 
жгута в конусе ОТ.
\begin{figure}[b!]
  \label{fig1:9}
  \centering                                                                                   
  \includegraphics[width=12cm]{Q_M_new_i.png} \\
  \caption{Поведение момента сил $M_R$ и расхода $Q$ в процессе уменьшения мощности}
\end{figure}

\begin{figure}[ht]
  \label{fig1:10}
  \centering\small{\it а}\hspace*{78mm}{\it б}\\
  {\includegraphics[width=7.8cm]{pressureNA_1n.png}}\hfill
  {\includegraphics[width=7.8cm]{pressureOT_n1.png}}
  \caption{Эволюция давления перед НА (\emph{а}) и на стенке в конусе ОТ (\emph{б})}
\end{figure}

\begin{figure}[t!]
  \label{fig1:11}
  \centering\small
  \begin{tabular}{p{7.cm}p{7.cm}}
  \centering\small $t_1=6.02$~c
  \par \includegraphics[width=6.5cm]{vihr_0602.png}&
  \centering $t_2=8.06$~c \par
  \includegraphics[width=6.5cm]{vihr_0806.png}
  \end{tabular}\vspace*{5mm}
  \begin{tabular}{p{7.cm}p{7.cm}}
  \centering $t_3=10.02$~c \par
  \includegraphics[width=6.5cm]{vihr_1002.png}&
  \centering $t_4=12.80$~c \par
  \includegraphics[width=6.5cm]{vihr_1280.png}
  \end{tabular}\\[3mm]
  \caption{Рост интенсивности вихревого жгута (показаны изоповерхности давления в моменты 
           времени $t_1, t_2, t_3, t_4$, отмеченные на рисунке~\ref{fig1:10}, \emph{б})}
\end{figure}

На рисунке~\ref{fig1:9} представлены полученные в расчете изменения во времени момента сил $M_R$, 
действующего на РК, и расхода $Q$. На рисунке~\ref{fig1:10} показана эволюция давления перед НА 
и на стенке конуса ОТ. 
Снижение расхода при закрытии лопаток приводит к положительному гидроудару, который проявляется в повышении 
давления перед НА (рисунок~\ref{fig1:10},\,\emph{a}). Зародышевый вихревой жгут, наблюдающийся в начальный 
момент времени, по мере закрытия НА увеличивает свой размер и радиус винта (рисунок~\ref{fig1:11}), что 
вызвано ростом остаточной закрутки за рабочим колесом. 
После окончания движения лопаток~НА, т.\,е. при $t>12.5$~c, течение имеет периодически 
нестационарный характер. Вращение вихревого жгута вызывает пульсации расхода и момента сил на
валу гидротурбины~(рисунок~\ref{fig1:9}). Амплитуда пульсаций давления в конусе ОТ достигает 2.4\,\% от 
действующего напора, период прецессии жгута $T=1.41$~с, что согласуется со значениями, измеренным в 
эксперименте: 2.3\,\% и $T_{\text{exp}}=1.3$ с.

\subsection{Сброс нагрузки с последующим выходом на холостой ход}
Сброс нагрузки (отключение нагруженного генератора) -- процесс аварийный. Расчетный процесс сброса нагрузки с 
выходом на холостой ход определяется приведенным на рисунке~\ref{fig2:9}, \emph{а} законом закрытия лопаток.
В начальный момент времени $t=0$ задается стационарное поле течения, полученное при расчете режима 
максимальной мощности: $a_0=35$~мм, $n=200$~об/мин. Момент полезной нагрузки (см. уравнение \eqref{eq2:6}) 
становится нулевым при $t=2.5$~с, и тогда же начинают закрываться лопатки НА. В расчете процесса невозможно
закрыть НА до $a_0=0$ мм, поэтому в расчете на интервале $t\in(13.5;24.5)$ задавалось минимальное открытие 
$a_0=1$ мм. На рисунке~\ref{fig2:10} приведены сетки при максимальном и минимальном открытии НА, используемые 
в расчете процесса сброса нагрузки.

\begin{figure}[htb]
  \centering \small \rule{0mm}{0mm}\emph{a}\rule{95mm}{0mm}\emph{б}\\[1.5mm]
  {\includegraphics[width=8.5cm]{alfa0_a0_ot_t_sbros.png}}\hfill
  {\includegraphics[width=8.5cm]{n_lay_sbros.png}}
  \caption{Cброс нагрузки с выходом на холостой ход.
   \emph{а}) закон изменения открытия НА $a_0(t)$ в эксперименте ($\bullet$) и принятый в 
   расчете (\textcolor{blue}{$\genfrac{}{}{2pt}{2}{\quad}{  }$});
   \emph{б}) рассчитанная (\textcolor{blue}{$\genfrac{}{}{2pt}{2}{\quad}{  }$}) и 
   экспериментальная ($\bullet$) зависимости скорости вращения РК от времени}
  \label{fig2:9}
\end{figure}

\begin{figure}[t!]
  \centering
  \includegraphics[width = 16cm]{HA_set_35_1.png}
  \caption {Сетки в НА для двух положений лопаток процесса сброса нагрузки:
   $a_0(2.5 c)=35$~мм (слева) и $a_0(12.5 c)=1$~мм (справа)}
  \label{fig2:10}
\end{figure}
\begin{figure}[hb!]
  \centering \small \rule{0mm}{0mm}\emph{a}\rule{90mm}{0mm}\emph{б}\\[1.5mm]
  {\includegraphics[width=8.5cm]{Qin_sbros.png}}\hfill
  {\includegraphics[width=8.5cm]{M_H_sbros.png}}
  \caption{Рассчитанные зависимости расхода $Q$ (\emph{a}), момента $M_R$ и
   напора $H$ (\emph{б}) от времени, мгновенный сброс нагрузки}
  \label{fig2:11}
\end{figure}

На рисунках~\ref{fig2:9}, \emph{б} и \ref{fig2:11} приведены зависимости от времени 
в процессе сброса нагрузки частоты 
вращения РК $n$, расхода $Q$, гидродинамического момента РК $M_R$ и напора на турбине $H$. После отключения 
генератора частота вращения РК быстро увеличивается. Уменьшение открытия НА приводит к уменьшению расхода воды.
Момент РК падает до нуля и при $a_0<a_{0,\;\text{хх}}$ становится отрицательным. Частота вращения при этом 
проходит максимум и постепенно снижается, но пока она выше номинальной $n_{\text{ном}}$, направляющий аппарат 
продолжает закрываться и достигает своего минимального значения (1 мм в расчете, 0 мм в действительности), 
которое держится до тех пор, пока $n$ не приблизится к $n_{\text{ном}}$. После этого НА плавно открывается до 
$a_{0,\;\text{хх}}$ и ГТ переходит тем самым в режим холостого хода. Отметим, что рассчитанное увеличение
скорости вращения РК в первые 25~с вновь отстает от экспериментально полученного.
\begin{table}[h!]
  \label{tab2:3}
  \centering
  \caption{Параметры в процессе сброса нагрузки}
  \begin{tabular}{|c|r|r|r|r|}
  \hline
  \No~п/п & время $t$, с & расход, м${^3}/$с & частота $n$, мин$^{-1}$ & момент сил $M_R$, тс$\cdot$м \\
  \hline
  1 & $3.0$ & $84.42$ & 214.67 & 257.45 \\
  2 & $10.0$ & $5.35$ & 286.00 & -49.91 \\
  3 & $24.0$ & $2.68$ & 216.48 & -28.80 \\
  4 & $50.0$ & $10.38$ & 208.56 & 5.29 \\
  \hline
  \end{tabular}
\end{table}
\begin{figure}[h!]
  \centering
  \begin{tabular}{ccc} 
    \includegraphics[natwidth=3730, natheight=2674,trim=  0 0 0 0,clip=true,scale=0.08]{k01_0300_sbros_n.png}&
    \includegraphics[natwidth=3722, natheight=2638,trim=425 0 0 0,clip=true,scale=0.08]{k01_1000_sbros_n.png} 
     \vspace{-5mm} \\ 
    $t_1$=3.0 с&
    $t_2$=10.0 с \\ \vspace{-5mm}
    \includegraphics[natwidth=3734, natheight=2650,trim=  0 0 0 0,clip=true,scale=0.08]{k01_2400_sbros_n.png}&
    \includegraphics[natwidth=3702, natheight=2670,trim=425 0 0 0,clip=true,scale=0.08]{k01_5000_sbros_n.png} 
    \\ 
    $t_3$=24.0 с&
    $t_4$=50.0 с
  \end{tabular} \par \vspace{-5mm}
  \caption{Изменение давления на лопасти рабочего колеса при мгновенном сбросе нагрузки (для каждого момента 
  времени рабочая сторона слева, тыльная -- справа)}
  \label{fig2:12} \vspace{-5mm}
\end{figure}

Распределения давления на лопасти РК в выбранные моменты времени (таблица~\ref{tab2:3}) 
представлены на рисунке~\ref{fig2:12}. 
Вначале давление на рабочей стороне лопасти уменьшается, на тыльной возрастает. При $t=10$~с момент РК 
достигает своего минимального значения, после чего плавно возрастает до значения $M_{\text{тр}}$.
Так же, как и в процессе пуска, распределение давления жидкости по лопасти РК сильно зависит от времени. 
\vspace{-5mm}
\subsection{Траектории мгновенных режимов при различных переходных процессах}
Имея кривые изменения натурных параметров гидротурбины для различных переходных процессов, можно построить 
траекторию последовательного перемещения режимной точки в координатах приведенных к $H=1$~м, $D_1=1$~м расхода 
и частоты \vspace{-3mm}
$$
  Q_{11}=\frac{Q}{D_1^2\sqrt{H}},\ n_{11}=n\frac{D_1}{\sqrt{H}},
$$
т.е. изобразить переходной процесс в поле главной универсальной характеристики. Это позволяет наглядно 
представить диапазон изменения режимов и определить условия работы гидротурбины при различных переходных 
процессах. На рисунке~\ref{fig2:15}, \emph{а} приведена типичная универсальная характеристика РО турбины и линии 
мгновенных режимов, соответствующие различным типам переходных процессов, заимствованная из \cite{krivch}. На 
линии разгона $II$ гидродинамический момент РК $M_R=0$ и соответственно КПД 
$\eta = 0$. Эта линия делит универсальную характеристику на зону $I$ турбинных режимов ($M_R>0$) и зону $III$ 
тормозных режимов работы ($M_R<0$). Пунктирной линией ограничена область рабочих режимов при нормальной 
эксплуатации турбины. Рассчитанные в настоящей работе траектории представлены на 
рисунке~\ref{fig2:15}, \emph{б}. 
Получено хорошее качественное соответствие результатов расчетов экспериментальным данным. \vspace{-3mm}
\begin{figure}[!h]
  \begin{tabular}{p{8.0cm}p{8.0cm}}
  \centering
  \includegraphics[width=8.0cm]{UX_result_2ppt_1.png}\par \vspace{-4mm}
  \centering \emph{а}& \centering
  \includegraphics[width=8.0cm]{UX_count_2ppt_2.png}\par \vspace{-4mm}
  \centering \emph{б}
  \end{tabular} \vspace{-5mm}
  \caption{\emph{а}) типичная УХ РО турбины \cite{krivch}. I -- турбинные режимы, II -- разгонные режимы, 
  III -- тормозные режимы. \emph{б}) рассчитанные траектории движения режимных точек: 1~--~пуск в турбинный 
  режим, 2~--~снятие нагрузки, 3~--~сброс нагрузки} \vspace{-5mm}
  \label{fig2:15}
  % \rule{0mm}{12cm}
\end{figure}
%
 
  %!TEX root = dissertation.tex
% \pagestyle{plain}
\chapter*{Глава 3. Метод расчета осевых и радиальных нагрузок на рабочее колесо гидротурбины 
в нестационарном потоке}
\label{s:3}
\setcounter{chapter}{3}
\addcontentsline{toc}{chapter}{Глава~\thechapter~ Метод расчета осевых и радиальных нагрузок на рабочее 
колесо гидротурбины в нестационарном потоке}
\setcounter{section}{0}

% 
% \begin{abstract}
% 
% {\it Ключевые слова}: гидротурбина, нестационарные течения,
% лабиринтные уплотнения, радиальные и осевые нагрузки.
% \end{abstract}
%

\section{Постановка задачи}
\label{s:31}
Между вращающимися и неподвижными частями гидротурбины (ГТ) имеются зазоры протечек жидкости 
(рисунок~\ref{fig:01}). Для снижения объемных потерь в зазорах устанавливаются лабиринтные
уплотнения (ЛУ). Кольца лабиринтных уплотнений крепятся на верхнем и нижнем ободах или на камере рабочего 
колеса (РК). Ступица РК имеет разгрузочные отверстия, расположенные за выходными кромками лопастей, через 
которые полость над РК сообщается с основным проточным трактом (ПТ). На рисунке~\ref{fig3:2} приведена
схема течения жидкости в проточном тракте ГТ, лабиринтных уплотнениях, полости между ступицей и крышкой 
турбины, разгрузочных отверстиях, полости между ободом и камерой РК.

Кроме уменьшения потерь мощности в турбине за счет ограничения величин протечек воды между вращающимися и 
неподвижными частями, уплотнения обеспечивают также снижение осевых и радиальных нагрузок, действующих 
на РК~\cite{granovsky}.

Полная гидравлическая нагрузка $\textbf{F}$, действующая на РК, складывается из трех составляющих:
\begin{equation}
  \textbf{F} = \textbf{F}_1 + \textbf{F}_2 + \textbf{F}_3,
  \label{eq3:1}
\end{equation}
где $\textbf{F}_1$~--- нагрузка от основного потока на поверхность проточной части РК, $\textbf{F}_2$~--- 
нагрузка от течения через верхнюю область протечки, $\textbf{F}_3$~--- нагрузка от течения
через нижнюю область протечки (см.~рисунок~\ref{fig3:2}). В полной гидравлической 
нагрузке $\textbf{F}=(F_x,F_y,F_z)$ выделяются осевые $F_z$ и радиальные $\textbf{F}_R=(F_x,F_y)$ усилия. 
Индексами $x,\, y,\, z$ обозначены компоненты сил, действующие по соответствующим осям декартовой системы 
координат. Ось $Oz$ совпадает с осью РК.
\begin{figure}[t!]
  \centering
  \includegraphics[width=12cm]{fig2_geom.png}
  \caption{Основной проточный тракт и области протечек за ступицей и ободом рабочего колеса гидротурбины}
  \label{fig3:2}
\end{figure}

Основными источниками осевых и радиальных сил в нагрузке $\textbf{F}_1$ являются
неравномерность потока на выходе из спиральной камеры, ротор-статор взаимодействие между лопастями
РК и лопатками направляющего аппарата, прецессирующий вихревой жгут в конусе отсасывающей трубы,
гидроудар.

Существенный вклад в осевые $F_z$ и радиальные $\textbf{F}_R$ усилия
вносят также составляющие $\textbf{F}_2$ и $\textbf{F}_3$, вызванные течениями в ЛУ, полостях и
разгрузочных отверстиях.

Таким образом, для наиболее точного определения осевых и радиальных нагрузок (ОРН) следует проводить
совместные расчеты полей во всех указанных областях течений основного ПТ и протечек за ступицей и ободом РК и
определять по ним нагрузку $\textbf{F}$~\eqref{eq3:1}. 
Однако такой расчет потребует огромных затрат вычислительных ресурсов из-за необходимости 
его совместного проведения в областях, имеющих значительно отличающиеся характерные размеры. 
Так, диаметр РК натурной гидротурбины составляет
$D_1\sim 6$~м, а ширина зазоров $d$ при этом варьируется в пределах 2--3~мм. 

В настоящее время для предсказания ОРН используются приближенные методики, которые можно разделить на
инженерно-эмпирические~\cite{lomakin, mak_pilev, kuzminsk} и основанные на методах вычислительной 
гидродинамики~\cite{staubli, roy_vu, xi_rhode}.

\section{Обзор существующих методик}
\label{s:32}
\subsection{Инженерно-эмпирическая методика определения осевых нагрузок}
\label{s:321}
В этой методике расходы жидкости через верхнее и нижнее уплотнения находятся по формулам
\begin{gather*}
 q_B  = f_1 (\xi_B ,\Delta h_B ), \\
 q_H  = f_2 (\xi_H ,\Delta h_H ),
\end{gather*}
где  $\xi_B,\xi_H$~--- коэффициенты гидравлического сопротивления верхнего и нижнего уплотнений, приведенные 
к  входному сечению уплотнения, $\Delta h_B, \Delta h_H$~--- напоры, срабатываемые в уплотнениях. 
Конкретный вид зависимостей $f_1, f_2$ приведен в~\cite{mak_pilev}.

Коэффициент гидравлического сопротивления для уплотнения ступицы складывается из коэффициентов сопротивления 
лабиринта и разгрузочных устройств
\begin{equation*}
  \xi _B = \xi _{\text{вх}} \frac{{S_1^2}}{{S_{\text{вх}}^2}} + 
  \xi _{\text{ЛУ}} + \xi_p \frac{{S_1^2 }}{{\left( {N_p S_p } \right)^2 }}+
  \xi_{\text{щ}}\frac{{S_1^2}}{{S_{\text{щ}}^2 }},
\end{equation*}
где $S_1$~--- площадь сечения узкой части зазора ЛУ; $\xi_{\text{вх}}$~--- коэффициент гидравлического 
сопротивления щели входа в уплотнительный тракт ступицы в случае установки уплотнения не на 
периферии; $\xi_{\text{ЛУ}}$ ~--- коэффициент гидравлического сопротивления~ЛУ;
 $\xi_p$~--- коэффициент гидравлического сопротивления разгрузочных 
 отверстий; $N_p$~--- число разгрузочных отверстий; $S_p$~--- площадь сечения разгрузочного 
 отверстия, $\xi_{\text{щ}}$~--- коэффициент гидравлического сопротивления выходной щели.

Коэффициент гидравлического сопротивления $\xi_{\text{ЛУ}} = \xi_{\text{ЛУ}}(\lambda ,\xi_0),$ ЛУ вычисляется 
по эмпирической формуле, например из \cite{idelchik}. Здесь
где $\lambda = \lambda (\omega ,q)$, $\omega$~--- частота вращения РК; $\xi_0$~--- коэффициент 
гидравлического сопротивления ячейки расширения, зависящий от геометрии ячеек и числа Рейнольдса.

Коэффициент гидравлического сопротивления вращающихся разгрузочных отверстий определяется соотношением
\begin{equation*}
  \xi_p = \xi _p (W_0 ,V_p ,S'_p ),
\end{equation*}
где $W_0$~--- относительная скорость жидкости перед разгрузочным отверстием, $V_p$~--- среднерасходная 
скорость воды в разгрузочном отверстии, $S'_p$~--- относительная ширина полости над отверстием.

Коэффициент гидравлического сопротивления уплотнения обода определяется сопротивлением ЛУ, 
входа в ЛУ, если таковой имеется, и выхода из полости.

Напор  в уплотнении
\begin{equation*}
  \Delta h = h_{in} - h_{out} - h_\omega,
\end{equation*}
где $h_{in}$~--- напор на входе в уплотнение, $h_{out}$~--- напор на выходе из 
уплотнения, $h_\omega$~--- напор, создаваемый вращающимся объемом жидкости,
заключенным между ступицей и ободом и неподвижными крышкой турбины и камерой 
РК, $h_\omega$~--- функция  частоты вращения РК $\omega$ и геометрических размеров проточного тракта.

Так как коэффициенты гидравлического сопротивления лабиринтных уплотнений $\xi_{\text{ЛУ}}$ и 
разгрузки $\xi_p$ определяются расходом жидкости через уплотнения,
который в свою очередь зависит от коэффициентов $\xi_B$ и $\xi_H$, то расчет протечек ведется 
итерационным методом последовательных приближений.

Полная гидравлическая нагрузка на РК рассчитывается по формуле~\eqref{eq3:1}. Нагрузка воздействия потока на 
внутреннюю полость РК $F_{1,z}$ вычисляется по инженерно-эмпирическим формулам.
Нагрузка $F_{2,z}$, действующая на внешнюю часть ступицы РК, в общем случае складывается из трех составляющих
\begin{equation*}
  F_{2,z} = F'_{2\text{ст}} + F_{2\text{л}} + F''_{2\text{ст}},
\end{equation*}
где $F'_{2\text{ст}}$~--- сила давления жидкости на часть поверхности ступицы, расположенную между 
входом в полость <<ступица\,---\,крышка>> и ЛУ
(если ЛУ находится на периферии ступицы, то $F'_{2\text{ст}}=0$); $F_{2\text{л}}$~--- сила давления 
жидкости на ЛУ; $F''_{2\text{ст}}$~--- сила давления жидкости на часть поверхности ступицы, 
расположенную между ЛУ и уплотнением вала.
Для однощелевых уплотнений с односторонней и двусторонней ячейкой расширения $F_{2\text{л}}=0$,
а для уплотнений елочного и гребенчатого типа эта сила определяется по формуле~\cite{mak_pilev}
\begin{equation*}
  F_{2\text{л}} = f(\xi_B ,q_B ).
\end{equation*}
Силы $F'_{2\text{ст}}$ и $F''_{2\text{ст}}$ находятся интегрированием пьезометрического напора по 
поверхности ступицы. Приближенно можно положить
\begin{gather*}
 F'_{2\text{ст}}=\frac{1}{4}\pi\rho g\left( {D_{\text{вх}}^2 - D_{\text{л}}^2}\right)\left[{h_{\text{вх}} 
      -\xi_{\text{вх}}
       \left({\frac{{q_B}}{{S_{\text{вх}} }}} \right)^2  - \frac{\tilde{\omega}^2}{16g}
       \left( {D_{\text{вх}}^2  - D_{\text{л}}^2 } \right)} \right], \\
 F''_{2\text{ст}} = \frac{1}{4}\pi \rho g\left( {D_{\text{вых}}^2  - D_{\text{вал}}^2 } \right)
       \left[{h_{\text{вых}} - \frac{\tilde{\omega}^2}{16g}
       \left( {D_{\text{вых}}^2 - D_{\text{вал}}^2} \right)} \right],
\end{gather*}
где $D_{\text{вх}}$~--- диаметр входа в полость; $D_{\text{л}}$~--- диаметр ЛУ; $D_{\text{вал}}$~--- 
диаметр уплотнения вала;
$D_{\text{вых}}$~--- диаметр выхода из уплотнения;
$\tilde{\omega}=\tilde{\omega}(\omega)$~--- угловая скорость вращения 
жидкости; $h_{\text{вх}},~h_{\text{вых}}$~--- пьезометрический напор на входе и выходе.

Аналогично по инженерно-эмпирическим формулам определяется нагрузка на внешнюю часть обода РК $F_{3,\,z}$.

Отметим, что в силу предположения об осесимметричности потока в уплотнениях и~повторяемости
течения в каналах РК определение радиальных нагрузок по данной методике невозможно.

\subsection{Инженерно-эмпирическая методика определения радиальных нагрузок}
\label{s:322}
Для оценки радиальных сил, вызванных несоосностью ротора и статора для однощелевых уплотнений без ячеек 
расширения и без учета вращения смещенного ротора, в~\cite{lomakin} предложена инженерно-эмпирическая методика.
 В ней, как и в~\cite{idelchik}, предполагается, что падение давления на входе в уплотнение определяется 
выражением
\begin{equation}
  \Delta p_{in} = p_A -p_{in}  = \left( {1 + \xi _{in} } \right)\frac{{V_{in}^2 \left( \theta  \right)}}{{2g}},
  \label{eq3:pin}
\end{equation}
а изменение давления по всей длине уплотнения равно сумме потерь на входе и потерь 
от гидравлического трения по всей длине
\begin{gather}
  p_A  - p_{out}  = p_A  - p_{in}  + p_{in}  - p_{out}  =  \Delta p_{in}  + p_{in}  - p_{out}  = \notag \\
  = \left( 1 + \xi _{in}  + \frac{\lambda l}{2d^* \left(\theta \right)} \right)
  \frac{{V_{in}^2 \left( \theta  \right)}}{{2g}},
  \label{eq3:pall}
\end{gather}
где $\xi_{in}=0.5$~--- коэффициент гидравлического сопротивления входа; $V_{in}(\theta)$ ~--- скорость на 
входе в ЛУ, зависящая от окружной координаты $\theta$; $\lambda$~--- коэффициент гидравлического трения; 
$l$~--- длина уплотнения; $d^* \left(\theta \right)$~--- переменный зазор. Авторы~\cite{lomakin} не учитывали 
в \eqref{eq3:pall} зависимости~$\lambda$ от частоты вращения ротора $\omega$ и наличия ячеек расширения с их 
гидравлическим сопротивлением. Эти упрощения были сделаны с целью существенного сокращения времени расчета 
нагрузки. При известном перепаде давления скорость $V_{in}$ выражается из уравнения~\eqref{eq3:pall} в виде
\begin{equation*}
  V_{in}\left(\theta \right) = \sqrt{\frac{2g (p_A -p_{out})}{1+\xi_{in}+\cfrac{\lambda l}{2d^* 
        \left(\theta \right)}}}
\end{equation*}
и подставляется в уравнение~\eqref{eq3:pin}. Следовательно, перепад давлений на входе
\begin{equation*}
  \Delta p_{in}(\theta) = \frac{1+\xi_{in}}{1+\xi_{in}+\cfrac{\lambda l}{2d^* 
                \left(\theta \right)}}(p_A -p_{out})
\end{equation*}
меняется по окружности. Принимается, что давление $p$ по длине щели $z$ изменяется по линейному закону
\begin{equation*}
  p(z,\theta) = p_A - \Delta p_{in} - \left(p_A -p_{out}-\Delta p_{in} \right)\frac{z}{l}.
\end{equation*}
Наличие в уплотнении изменяющегося по окружности давления вызывает силу, действующую на 
ротор в радиальном направлении. Она в силу симметрии величины зазоров будет направлена по линии смещения 
центра ротора от оси уплотнительного кольца и равна
\begin{equation*}
  F = -\int\limits_0^l {\int\limits_0^{2\pi } {p\left( {z,\theta } \right)R_{\,r} \cos \theta } } dzd\theta,
\end{equation*}
где $R_r$~--- радиус ротора. При определении $F$ не учитывается возможность частичного 
выравнивания неравномерности распределения давления $p(z,\theta)$ по окружности из-за перетекания жидкости.

В настоящей работе данная методика усовершенствована путем учета вращения ротора и ячеек 
расширения лабиринтных уплотнений (см. ниже раздел~\ref{s:35}).

Существенный недостаток инженерно-эмпирических методик --- грубое представление в них потока жидкости в 
основном проточном тракте, не учитывающее нестационарность потока и форму лопасти, следствием чего становится 
неточное задание давлений на входе и выходе зазоров, приводящее к неправильному определению осевой силы, 
действующей на~РК.

В связи с этим в последнее время стали появляться работы, в которых моделирование течений в зазорах и полостях 
ГТ осуществляется на основе более совершенных методов вычислительной гидродинамики.

\subsection{Подходы, основанные на методах вычислительной гидродинамики}
\label{s:323}
В работе~\cite{staubli} в полной постановке рассчитывается трехмерное течение в 
области РК\,---\,ЛУ\,---\,по\-лос\-ть над ступицей\,---\,разгрузочные отверстия. Во входном сечении направляющего аппарата задается расход, в выходном сечении РК~--- давление. 
Области РК, полости за ступицей и разгрузочных отверстий рассчитываются во вращающейся системе координат, 
а область НА~--- в неподвижной. На стыке вращающихся и неподвижных областей осуществляется обмен всеми 
параметрами течения.
\begin{figure}[b!]
  \centering
  \includegraphics[width=12cm]{fig2_5_Vu_1_ru.png}
  \caption{Экспериментальные данные о зависимости типа течения в  прямом ЛУ от величины осевого числа 
  Рейнольдса $\mathrm{Re}$ и числа Тейлора $\mathrm{Ta}$~\cite{roy_vu}}
  \label{fig3:2.5}
\end{figure}

В рассматриваемых областях решаются нестационарные уравнения Рейнольдса с замыканием SST-моделью 
турбулентности. SST-модель турбулентности является синтезом моделей $k-\varepsilon$ и~$k-\omega$. 
Она основана на том, что модели типа $k-\varepsilon$ лучше описывают свойства свободных сдвиговых течений, 
а модель $k-\omega$ имеет преимущество при моделировании пристеночных течений. Плавный переход 
от $k-\omega$-модели в пристеночной области к $k-\varepsilon$-модели вдали от твердых стенок обеспечивается 
введением весовой эмпирической функции. Средняя величина нормализованного расстояния до стенки в
расчетах~\cite{staubli} равна $y_+=70$.
  
Приводятся распределения давления в ЛУ и по верхней части ступицы для трех режимов работы ГТ: неполной 
загрузки, оптимального КПД и максимальной мощности, и распределения давления по верхней части ступицы при 
наличии разгрузочного устройства для тех же режимов. Полученные зависимости качественно адекватно отражают  
моделируемые явления.

В~\cite{roy_vu} исследуется применимость различных вычислительных моделей и сеток для расчета течения в 
лабиринтных уплотнениях радиально-осевых турбин.
Осевое число Рейнольдса для рассмотренных в~\cite{roy_vu} конфигураций ЛУ и режимов течения варьируется 
в~диапазоне Re$_{axial}=\dfrac{U_{axial}d}{\nu}=2\cdot10^3\div6.4\cdot10^3$,
число Тейлора Ta$~=\dfrac{\omega\sqrt{R_1d^3}}{\nu}=300\div962$, в связи с чем на основании 
рисунка~\ref{fig3:2.5} делается вывод, что течение в зазоре является 
турбулентным. Утверждается также~\cite{roy_vu}, что при ширине зазора лабиринтного уплотнения $d$ 
много меньшей радиуса рабочего колеса $R_1$ вихри Тейлора~--- источник нестационарности~--- в нем не 
возникают.
Поэтому течение в зазоре ЛУ моделируется стационарными уравнениями Рейнольдса в двумерном осесимметричном 
приближении. На входе в ЛУ задается либо расход, либо полное давление, на выходе~--- статическое давление, 
которые берутся из эксперимента. Для замыкания уравнений используется Low-Re-SST-модель
турбулентности без пристеночных функций с использованием подробной сетки, разрешающей пограничный 
слой ($y_+\sim 1$). При  низких числах Рейнольдса толщина логарифмического пограничного слоя сопоставима 
с шириной зазора $d$. Поэтому метод пристеночных функций, требующий расположения
первого слоя сетки на расстоянии $y_+\sim 30\div 300$ от стенки, должен применяться с осторожностью. 
В~\cite{roy_vu} исследуется влияние сгущения сетки и режима течения, подбираются сетка, модель турбулентности 
и краевые условия для адекватного количественного описания течения в~ЛУ.
Проводится сравнение результатов расчетов с экспериментом и результатами, полученными по эмпирической модели.

Использование подходов работ~\cite{staubli, roy_vu} позволяет построить вычислительные модели для расчета 
объемных потерь через уплотнения, осевых и радиальных нагрузок на РК при условии совпадения оси вращения 
колеса и оси симметрии неподвижной камеры РК. Однако эти подходы не годятся для моделирования радиальных сил, 
связанных с несимметричностью расположения РК относительно статора.

\subsection{Моделирование течения в зазорах между РК и неподвижными 
            частями гидротурбины с учетом биения РК} \label{s:324}
При наличии эксцентриситета оси РК (рисунок~\ref{fig3:6}) область течения в зазорах на ступице и ободе 
становится областью с подвижными границами, что при расчете этих течений требует использования подвижных 
сеток, перестраивающихся на каждом шаге по времени. Это обстоятельство усложняет расчет осевых
и радиальных нагрузок. Однако в ряде случаев постановка задачи может быть упрощена.
\begin{figure}[b!]
  \centering
  \includegraphics[width=10cm]{fig_4_eks.png}
  \caption{Кольцевая щель лабиринтного уплотнения c осью вращения РК, смещенной относительно центра полости}
  \label{fig3:6}
\end{figure}

В работе~\cite{xi_rhode} численно моделируются течения в зазорах ЛУ паровой турбины. 
При этом рассчитываются радиальные силы, действующие на ротор со стороны потока при малых биениях ротора в 
радиальном направлении ($e \ll d$). Предполагается, что биение оси ротора периодическое по времени и вместе с
тем она перемещается в окружном направлении (прецессирует) с угловой скоростью $\Omega$ 
(см.~рисунок~\ref{fig3:6}). В этом случае течение приближенно представимо в виде
суперпозиции стационарного осесимметричного течения при фиксированном положении ротора в центре 
и малого периодически нестационарного несимметричного возмущения, 
порождаемого движением центра ротора относительно центра полости
\begin{equation*}
  \Phi {\rm{(}}x,r,\theta ,t{\rm{)}} = \Phi_0 {\rm{(}}x{\rm{,}}r{\rm{)}} +
  \frac{e}{d}\left( {\Phi _{1c} {\rm{(}}x{\rm{,}}r{\rm{)cos(}}\Omega t{\rm{ - }}
  \theta {\rm{)}}\,{\kern 1pt}  - \;\Phi _{1s} {\rm{(}}x{\rm{,}}r{\rm{)sin(}}\Omega t{\rm{ - }}
  \theta {\rm{)}}} \right).
\end{equation*}
Для нахождения первой, осесимметричной, составляющей 
течения $\Phi _0 {\rm{(}}x{\rm{,}}r{\rm{)}}$ используются уравнения Рейнольдса, замкнутые
стандартной $k-\varepsilon$ моделью турбулентности. Для нахождения второй, несимметричной, 
составляющей применяются уравнения Рейнольдса для возмущенных течений.
На основе полученной несимметричной составляющей поля давления вычисляется радиальная сила, действующая 
на ротор. Далее найденная сила используется для вычисления коэффициентов динамической устойчивости ротора, 
которые показывают, насколько лабиринтное уплотнение демпфирует колебания вала в радиальном
направлении. Расщепление на осесимметричную невозмущенную и несимметричную возмущенную части позволило 
в~\cite{xi_rhode} находить эти составляющие решением двумерных осесимметричных задач.

Подход~\cite{xi_rhode} может быть применен для вычисления радиальных гидродинамических сил, действующих 
на РК при наличии эксцентриситета и неподвижной поверхности уплотнений.
В более общем случае можно использовать трехмерное моделирование течения во всей кольцевой щели 
переменной ширины в неинерциальной системе отcчета, вращающейся со~скоростью прецессии 
вала $\Omega$ относительно центра камеры РК. При этом предполагается, что разгрузочные отверстия 
заменяются на кольцевую щель. В~данном случае не требуется перестроения сетки.

\section{Методика определения полной гидравлической нагрузки на РК}
\label{s:34}
В диссертации предложена методика определения осевых и радиальных нагрузок на
рабочее колесо гидротурбины, вызванных нестационарным течением рабочей жидкости в его межлопастных каналах, 
а также протечками в лабиринтных уплотнениях, полостях и разгрузочных отверстиях. Данная методика превосходит 
по точности инженерно-эмпирические подходы, так как основана на моделях трехмерных нестационарных
турбулентных течений и позволяет учитывать окружную неравномерность потока, прецессию вихревого жгута за 
рабочим колесом и другие особенности трехмерных нестационарных течений. Также проведено усовершенствование 
инженерно-эмпирического метода расчета радиальных нагрузок в лабиринтных уплотнениях, вызванных смещением
оси вращения рабочего колеса. При этом учтены влияние вращения ротора на коэффициент сопротивления узкой 
части лабиринта, сопротивления ячеек расширения, зависимости коэффициента сопротивления узкой части лабиринта 
и ячеек расширения от переменного зазора между статором и ротором.

\subsection{Определение нагрузки $\textbf{F}_1$ от основного потока}
\label{s:341}
Для определения гидравлической нагрузки $\textbf{F}_1$, действующей на РК, рассчитывается нестационарное 
течение в основной проточной части гидротурбины, включающей спиральную камеру, все каналы направляющего 
аппарата, все межлопастные каналы РК и отсасывающую трубу. В качестве граничного условия во входном сечении 
спиральной камеры задается равномерное распределение скорости, перпендикулярной сечению
\begin{equation*}
  \left| {\textbf{V}_{in}} \right| = \frac{Q}{{S_{in} }},
\end{equation*}
где $Q$~--- фиксированный расход жидкости, $S_{in}$~--- площадь входного сечения.

Для определения абсолютных значений давления в точках $A,\ B,\ C,\ D$ (см.~рисунок~\ref{fig3:2}) рабочего 
колеса, требуемых для постановки граничных условий при расчете течения жидкости через области протечки, 
нужно задать абсолютное давление в выходном сечении ОТ
\begin{equation}
  p(z) = p_{atm} + (z-z_\text{н.б}),
  \label{eq3:5}
\end{equation}
где $p_{atm}$~--- атмосферное давление, $z_\text{н.б}$~--- уровень нижнего бьефа. Полная 
нагрузка $\textbf{F}_1$, действующая на рабочее колесо, не зависит от константы, входящей в 
распределение давления (атмосферное давление). Но мы не можем посчитать вклад в $\textbf{F}_1$ атмосферного 
давления, действующего сверху на вал гидротурбины.
Поэтому для корректного расчета силы $\textbf{F}_1$ необходимо проводить вычисления при $p_{atm}=0$. Этот 
же подход использован в работе~\cite{mak_pilev}.

\subsection{Определение осевой составляющей нагрузки $\textbf{F}_2$}
\label{s:342}
Область протечки над ступицей включает (рисунок~\ref{fig3:4}) верхнее лабиринтное уплотнение,
полость между крышкой турбины и внешней поверхностью ступицы, разгрузочные отверстия (РО), кольцевую полость 
и щель, через которую жидкость попадает за лопасти~РК. Разгрузочные отверстия служат для снижения давления
на крышку турбины.

Из-за разгрузочных отверстий течение в полости не является осесимметричным. Поэтому все течение в верхней 
области протечки разбивается на два. Течение в самом ЛУ считается в осесимметричной постановке.
Область ЛУ содержит один слой сетки по окружному направлению. Трехмерное течение в остальной части 
предполагается циклически повторяющимся по числу разгрузочных отверстий и рассчитывается в секторе полости
между крышкой и ступицей с одним РО и в секторе кольцевой полости. При передаче параметров потока между этой 
областью и областью ЛУ проводится их усреднение по окружному направлению.

Расход воды через области протечки {\it априори} не известен и должен быть найден в~ходе решения задачи.
\begin{figure}[t!]
  \centering \small \emph{a}\rule{52mm}{0mm}\emph{б}\rule{55mm}{0mm}\emph{в}\\[1.5mm]
  {\includegraphics[width=10cm]{fig4a_1.png}}\hfill
  {\includegraphics[width=6cm]{fig4b_all_1.png}}
  \caption{Верхняя область протечки: \emph{а}~--- расчетная область периодического трехмерного течения в полости и осесимметричного течения в ЛУ;
  \emph{б}~--- сетка в одной ячейке ЛУ для расчета в ней осесимметричного течения;
  \emph{в}~--- $rz$-проекция верхней области протечки}
  \label{fig3:4}
\end{figure}

Поскольку в предлагаемом методе определения нагрузки на РК мы отказались от одновременного
совместного расчета течений в основном проточном тракте и в областях протечек, то возникает проблема обмена
между ними параметрами течений в точке $A$, расположенной у входного сечения $S_{in}$ в ЛУ, и в точке
$B$ у выходного сечения $S_{out}$ из кольцевой щели (см.~рисунок~\ref{fig3:4},\,{\it в}). Во-первых, 
в этом обмене исключено влияние течения в зазорах на течение в основном ПТ. В построенной методике течение 
в зазорах определяется течением в основном ПТ. Для сопряжения этих течений необходимо учесть входные потери 
энергии в точке~$A$ и выходные в точке~$B$. Согласно~\cite{idelchik}, потери полной энергии на входе в
лабиринтное уплотнение могут быть найдены по формуле
\begin{equation}
  \left({\Delta E}\right)_{in} = E_A  - E_{in} = \xi _{in} \frac{{\left| {{\bf{V}}_{in}} \right|^2 }}{{2g}},
  \label{eq3:6}
\end{equation}
где $\xi _{in}$~--- коэффициент гидравлического сопротивления входа в ЛУ полагается равным $\xi _{in}=0.5$. 
Поскольку полная энергия
\begin{equation}
  E = p - z + \frac{{\left| {\bf{V}} \right|^2 }}{{2g}},
  \label{eq3:7}
\end{equation}
то из~\eqref{eq3:6} следует
\begin{equation}
  p_A -z_A +\frac{{\left| {{\bf{V}}_A }\right|^2 }}{{2g}} -\left({p_{in} -z_{in} +
  \frac{{\left| {{\bf{V}}_{in}}\right|^2 }}{{2g}}}\right) =
  \xi _{in} \frac{{\left| {{\bf{V}}_{in}^{} } \right|^2 }}{{2g}}.
  \label{eq3:8}
\end{equation}
Считается~\cite{mak_pilev}, что течение в лабиринтном уплотнении не зависит от скорости жидкости в
основной части ГТ, а зависит только от давления. Поэтому в~\eqref{eq3:8} полагаем
${\bf{V}}_A  = 0$, а также считаем, что $z_A=z_{in}$. Тогда
\begin{equation}
  p_A  -p_{in} -\frac{{\left| {{\bf{V}}_{in}} \right|^2 }}{{2g}}=
  \xi _{in}\frac{{\left| {{\bf{V}}_{in}}\right|^2 }}{{2g}},
  \label{eq3:9}
\end{equation}
откуда получаем
\begin{equation}
  \left| {{\bf{V}}_{in}}\right| =\sqrt{2g\left({\frac{{p_A -p_{in}}}{{1+\xi_{in}}}}\right)}.
  \label{eq3:10}
\end{equation}
\begin{figure}[ht!]
  \centering
  \includegraphics[width=4.5cm]{fig5_Sout.png}
  \caption{К заданию давления в выходном сечении}
  \label{fig3:5}
\end{figure}

Таким образом, во входном сечении $S_{in}$ задается условие на модуль скорости, из которого находится 
расходная $z$-я составляющая, при этом тангенциальные составляющие скорости полагаются нулевыми. 
Давление $p_{in}$ в соотношении~\eqref{eq3:10} находится в процессе решения задачи.

Течение в выходной щели $K$ (рисунок~\ref{fig3:5}) не рассчитывается. Поэтому для задания
условия в сечении $S_{out}$ необходимо связать параметры потока в этом сечении с параметрами потока 
в точке $B$. Потери энергии на выходе из кольцевой полости (сечение $S_{out}$) складываются из
потерь на входе в кольцевую щель
\begin{equation}
  E_{out} -E_K =\xi _{in} \frac{{\left| {{\bf{V}}_{out}^{}} \right|^2 }}{{2g}}
  \label{eq3:11}
\end{equation}
и потерь на выходе из нее в область основного ПТ (см.~рисунок~\ref{fig3:5})
\begin{equation}
  E_K  -E_B  = \xi _{out} \frac{{\left| {{\bf{V}}_{out}^{} } \right|^2 }}{{2g}},
  \label{eq3:12}
\end{equation}
где, согласно~\cite{idelchik}, коэффициент гидравлического сопротивления входа $\xi_{in}=0.5$, коэффициент 
гидравлического сопротивления на выходе из кольцевой щели $\xi_{out}=1$. Тогда,
складывая соотношения~\eqref{eq3:11},~\eqref{eq3:12} и подставляя выражения для полной энергии, получим
\begin{equation}
  p_{out} -z_{out} +\frac{{\left| {{\bf{V}}_{out} }\right|^2 }}{{2g}} -\left( {p_B - z_B +\frac{{\left|{{\bf{V}}_B }\right|^2}}{{2g}}}\right) =
  \left( {\xi _{in} + \xi _{out}}\right)\frac{{\left| {{\bf{V}}_{out}^{}}\right|^2 }}{{2g}}.
  \label{eq3:13}
\end{equation}
Положив в~\eqref{eq3:13} ${\bf{V}}_B = 0$ и $z_B=z_{out}$, находим
\begin{equation}
  p_{out} = p_B + \xi _{in}\frac{{\left| {{\bf{V}}_{out}^{}}\right|^2 }}{{2g}}.
  \label{eq3:14}
\end{equation}
Таким образом, скорость $V_{out}$ находится из основных уравнений в процессе их решения, а~давление 
вычисляется по формуле~\eqref{eq3:14}, где $p_B$ предварительно определено из расчета потока 
в основной части ГТ.

\subsection{Определение радиальных составляющих нагрузок $\textbf{F}_2$ и $\textbf{F}_3$}
\label{s:343}
Силы $\textbf{F}_{2,R}$ и $\textbf{F}_{3,R}$ возникают главным образом в силу окружной
неравномерности потока в лабиринтных уплотнениях при смещении центра вращения РК относительно 
геометрического центра камеры рабочего колеса.

В общем случае полагаем, что имеются смещение оси РК на эксцентриситет $e$ и ее прецессия вокруг центра 
статорной части с угловой скоростью $\Omega$ (см. рисунок~\ref{fig3:6}). Течение в кольцевом зазоре
переменной ширины приводит к появлению радиальных сил, действующих со стороны протекающей в зазоре жидкости. 
Используются две методики расчета радиальных составляющих нагрузок $\textbf{F}_2$ и $\textbf{F}_3$ в модели с
эксцентриситетом. Первая основана на расчете трехмерного течения в несимметричной щели, 
вторая~--- усовершенствованный инженерно-эмпирический подход~\cite{lomakin}.

\bigskip
\noindent{\it Расчет трехмерного течения в ЛУ с эксцентриситетом}.
\label{s:3431}
\medskip

\noindent Если для описания течения в кольцевой щели лабиринтного уплотнения с осью вращения РК, 
смещенной относительно центра полости, перейти во вращающуюся вокруг центра полости систему 
координат $(x',y',z)$ (см.~рисунок~\ref{fig3:6}),
то относительно этой системы   течение будет стационарным. В самом деле, форма области зазора в 
координатах $(x',y')$ не меняется, сетку можно считать неподвижной, и необходимо  учесть только вращение 
ротора с угловой скоростью $\Omega - \omega$ и статора~--- со скоростью $\Omega$. Компоненты скорости 
движения точки $(x'_{\text{PK}},y'_{\text{PK}})$ на роторе будут следующими:
\begin{equation}
%  \begin{array}{l}
    u = y'_{\text{PK}}\left( {\Omega  - \omega } \right), \quad
    v =  - \left( {x'_{\text{PK}} - e} \right)\left( {\Omega  - \omega } \right), \quad
    w = 0,
%   \end{array}
   \label{eq3:15}
\end{equation}
а точки $(x'_{\text{ст}},y'_{\text{ст}})$ на статоре~---
\begin{equation}
  \begin{array}{l}
    u = y'_{\text{ст}}\Omega , \quad
    v =  - x'_{\text{ст}}\Omega , \quad
    w = 0. \\
  \end{array}
  \label{eq3:16}
\end{equation}
При этом в уравнениях количества движения~\eqref{1}--\eqref{2} в источниковых членах 
величину~$\omega$ следует заменить на $\Omega$.
\begin{figure}[t!]
  \centering \includegraphics[width=5.5cm]{fig7_in_out_1.png}
  \caption{К определению давления $p_{out}$}
  \label{fig3:7}
\end{figure}
Условия для входного и выходного сечений выводятся с
учетом входных и выходных потерь в полном соответствии с данными раздела~\ref{s:342}.
Связь скорости и давления на входе задает соотношение~\eqref{eq3:10}, из которого
 находится расходная $z$-компонента скорости $\textbf{V}_{in}$, остальные
компоненты скорости полагаются нулевыми.

Выражение для давления в выходном сечении (рисунок~\ref{fig3:7}) выводится из потерь энергии
\begin{equation}
  E_{out} - E_L = \xi _{out} \frac{{\left| {{\textbf{V}}_{out}} \right|^2}}{{2g}}.
  \label{eq3:17}
\end{equation}
С учетом того что $\textbf{V}_L=0$, $z_L=z_{out}$ и $\xi_{out}=1$, находим $p_{out}=p_L$.
Давление $p_L$ берется из расчета течения в верхней области протечки (см. раздел~\ref{s:342}).

\section{Улучшенная инженерно-эмпирическая методика определения радиальных 
составляющих нагрузок $\textbf{F}_2$ и $\textbf{F}_3$}
\label{s:35}
Методика~\cite{lomakin} позволяет вычислять действующие на ось РК радиальные нагрузки, вызванные
наличием эксцентриситета (см. рисунок~\ref{fig3:6}). Она применяется
только для щелевых уплотнений (без ячеек расширения) для
неподвижного РК и нулевой скорости прецессии вала $\Omega =0$. Ниже этот подход распространяется на случай 
лабиринтных уплотнений с ячейками расширения и вращающимся рабочим колесом.

\begin{figure}[h]
  \centering
  \includegraphics[width=5.5cm]{fig9_lu.png}
  \caption{Давления $ p_A,\,p_{in},\, p_{out}$  и коэффициенты сопротивления верхнего ЛУ}
  \label{fig3:9}
\end{figure}

Как и в~\cite{lomakin}, предполагается, что падение давления на входе в ЛУ (рисунок~\ref{fig3:9}) 
определяется выражением
\begin{equation}
  \Delta p_{in} = p_A -p_{in}  = \left({1 + \xi _{in}} \right)\frac{{V_{in}^2 \left( \theta  \right)}}{{2g}}.
  \label{eq3:18}
\end{equation}

Для изменения давления по всей длине ЛУ вместо~\eqref{eq3:pall} используем формулу из~\cite{idelchik},
в~которой $\lambda$ зависит от частоты вращения ротора $\omega$, учитываются ячейки расширения и~коэффициенты 
их гидравлического сопротивления. Тогда
\begin{gather}
  p_A  - p_{out}  = p_A  - p_{in}  + p_{in}  - p_{out}  = \notag \\
  = \Delta p_{in}  + p_{in}  - p_{out}   = \left( {1 + \xi _{in}  + 
  \frac{{\lambda \left( {\theta ,\omega } \right)l_{\text{уз}}}}
  {{2d^* \left( \theta  \right)}} + n\xi _0 \left( \theta  \right)} \right)
  \frac{{V_{in}^2 \left( \theta  \right)}}{{2g}},
  \label{eq3:19}
\end{gather}
где $\xi_{in}=0.5$~--- коэффициент гидравлического сопротивления входа; $V_{in}(\theta)$ ~--- 
скорость на входе в ЛУ, зависящая от окружной координаты $\theta$; 
$\lambda \left({\theta ,\omega}\right)$~--- коэффициент гидравлического трения;
$l_{\text{уз}}$~--- суммарная длина узкой части (щели без ячеек) уплотнения; 
$d^* \left(\theta \right)$~--- переменный зазор;
$n$~--- число ячеек расширения; $\xi_0\left(\theta\right)$~--- коэффициент гидравлического 
сопротивления одной ячейки.

Найдем зависимость величины зазора от окружного угла $d^* \left(\theta \right)$. Пусть $R_s$~--- радиус 
внутренней стенки статора, $R_r$~--- радиус ротора. Тогда (см.~рисунок~\ref{fig3:6})
\begin{equation*}
  R_s  = e\cos \left( {\theta  - \gamma } \right) + \left( {R_{\,r}  + d^* } \right)\cos \gamma.
\end{equation*}
При эксцентриситете $e<<R_r$ угол $\gamma <<1$, поэтому можно считать
\begin{equation*}
  R_s  = e\cos \theta  + R_{\,r}  + d^*.
\end{equation*}
Тогда
\begin{equation*}
  d^*  = R_s  - R_{\,r}  - e\cos \theta  = d - e\cos \theta.
\end{equation*}
Окончательно переменный зазор ЛУ
\begin{equation*}
  d^* \left( \theta  \right) = d\left( {1 - \varepsilon \cos \theta } \right),
\end{equation*}
где $d=R_s -R_{\,r}$~--- радиальный зазор, $\varepsilon=e/d$~--- относительный эксцентриситет,
отсчет угла $\theta$ идет от самой узкой части зазора.
Из соотношения~\eqref{eq3:19} находим
\begin{equation}
  V_{in}^2 \left( \theta  \right) = \frac{{2g\left( {p_A  - p_{out,\,l} } \right)}}
  {{1 + \xi _{in}  + \cfrac{{\lambda \left( {\theta ,\omega } \right)l_{\text{уз}} }}{{2d^* 
  \left( \theta  \right)}} + n\xi _0 \left( \theta  \right)}}.
  \label{eq3:20}
\end{equation}
В уравнении~\eqref{eq3:20} коэффициент гидравлического трения кольцевой 
щели $\lambda \left( {\theta ,\omega } \right)$ определяется по формуле~\cite{idelchik}
\begin{equation}
  \lambda \left({\theta ,\omega }\right) =\frac{{0.316}}{{{\mathop{\rm Re}\nolimits} ^{0.25} 
  \left( \theta  \right)}}
  \left[ {1 + \left({\frac{{{\mathop{\rm Re}\nolimits} _u \left( {\theta ,\omega } \right)}}
  {{{\mathop{\rm Re}\nolimits} \left( \theta  \right)}}} \right)^2 } \right]^{0.375},
  \label{eq3:21}
\end{equation}
где
\begin{equation}
  {\mathop{\rm Re}\nolimits} = V_{in} \left( \theta  \right)\frac{{2d^* \left( \theta  \right)}}{\nu },\quad
  {\mathop{\rm Re}\nolimits} _u = \frac{\omega }{2}  \frac{{2d^* \left( \theta  \right)}}{\nu },
  \label{eq3:22}
\end{equation}
$\nu$~--- коэффициент кинематической вязкости.
\begin{figure}[b!]
  \centering
  \includegraphics[width=9cm]{fig11_ab.png}
  \caption{Определение коэффициентов $a$ и $b$~\eqref{eq3:23} при $h_k>d_{\text{гр}}$}
  \label{fig3:11}
\end{figure}

Коэффициент гидравлического сопротивления ячеек расширения лабиринта $\xi_0 \left(\theta \right)$
определяется по формуле~\cite{idelchik}
\begin{equation}
  \xi _0  = a\left( \theta  \right) + 0.5b\left( \theta  \right),
  \label{eq3:23}
\end{equation}
где $a$~--- коэффициент потерь энергии постоянной массы в начале и в конце ячейки, 
$b$~--- коэффицент потерь энергии от внезапного сжатия ядра постоянной массы при втекании в щель.
Значения этих коэффициентов определяются в зависимости от соотношения величин
\begin{equation}
  d_{\text{гр}} = d^* \left(\theta \right)+0.24  S
  \label{eq3:24}
\end{equation}
и глубины камеры ячейки $h_k$ (см. рисунок~\ref{fig3:9}). Если $h_k>d_{\text{гр}}$, то $a$ и $b$ находятся 
из эмпирических зависимостей
(рисунок~\ref{fig3:11}), если $h_k<d_{\text{гр}}$, то по формулам
\begin{equation}
  a=1-\frac{d_{\text{гр}}}{h_k},\quad b=\left(1-\frac{d_{\text{гр}}}{h_k}\right)^2.
  \label{eq3:241}
\end{equation}

Далее, так как коэффициент гидравлического трения кольцевой 
щели $\lambda \left( {\theta ,\omega } \right)$ зависит от расходной составляющей
скорости в щели $V_{in} \left( \theta  \right)$~\eqref{eq3:21}--\eqref{eq3:22}, которая в свою 
очередь зависит от коэффициента гидравлического трения $\lambda \left( {\theta ,\omega } \right)$, то 
расчет этих двух величин ведется итерационным методом последовательных приближений.

Таким образом, в соответствии с~\eqref{eq3:19} распределение давления $p(z,\theta)$ 
по длине ЛУ $z\in[0,l]$ состоит из потерь давления на трение
$\left({\displaystyle\frac{{\lambda \left({\theta ,\omega}\right)l_{\text{уз}} }}
{{2d^* \left( \theta  \right)}}} \right)\displaystyle\frac{{V_{in}^2 \left(\theta \right)}}{{2g}}$
на каждом участке узкой части уплотнения и скачкообразных 
потерь $\xi _0 \left( \theta  \right)\displaystyle\frac{{V_{in}^2 \left( \theta  \right)}}{{2g}}$
на входе и выходе каждой ячейки расширения. На рисунке~\ref{fig3:12} приведена зависимость $p(z, 0)$ 
для значения относительного эксцентриситета
$\varepsilon=0.5$. Конфигурация ячеек расширения показана над осью абсцисс. При построении 
зависимости скачкообразные потери на каждой ячейке расширения разделены на входные и выходные
$$
a(\theta)\frac{V^2_{in}(\theta)}{2g}~~
\text{и}~~ 0.5b(\theta)\frac{V^2_{in}(\theta)}{2g}
$$
составляющие и отнесены к началу и концу ячейки соответственно. Поэтому на внутренней части 
ячейки давление не меняется.
\begin{figure}[t!]
  \centering
  \includegraphics[width=9cm]{fig12_p_1.png}
  \caption{Распределение давления вдоль длины ЛУ}
  \label{fig3:12}
\end{figure}

Радиальную составляющую нагрузки $\textbf{F}_2$ находим численным интегрированием найденного 
распределения давления $p(z,\theta)$ по поверхности ротора:
\begin{equation}
  |\textbf{F}_{2,\,R}| = -\int\limits_0^l {\int\limits_0^{2\pi } 
  {p\left( {z,\theta } \right)R_{\,r} \cos \theta } } dzd\theta.
  \label{eq3:242}
\end{equation}
Эта радиальная сила при $\omega= 0$ в силу симметрии зазоров будет направлена по линии смещения центра 
ротора к оси статора.

Радиальная составляющая нагрузки $\textbf{F}_3$ определяется аналогично, отличаются только
геометрические размеры и число ячеек расширения нижнего лабиринтного уплотнения.

В случае прецессии вала РК с ненулевой скоростью $\Omega$ этот подход определения радиальной силы не применим.

\section{Результаты расчетов осевых и радиальных нагрузок}
\label{s:36}
Разработанный метод использован для расчета основных радиальных нагрузок на РК в~двух режимах 
работы гидротурбины,
приведенных в таблице~\ref{tab3:1}. Для режима частичной нагрузки характерно наличие прецессирующего 
вихревого жгута в конусе
отсасывающей трубы, влияние которого на гидродинамические силы в проточном тракте представляет 
практический интерес. Расчеты проведены
в натурных параметрах, шаг по физическому времени $\Delta t=0.003126$ с соответствует повороту 
РК на угол $\theta = 3.75^\circ$.
Расчетная сетка содержит суммарно по рассмотренным спиральной камере, НА, РК и ОТ 2\,615\,925 внутренних 
ячеек.
\begin{table}[t!]
  \centering\small\caption{Режимы работы гидротурбины, в которых рассчитывались 
  основные радиальные нагрузки}\vspace*{2mm}

  \begin{tabular}{|c|c|c|c|c|} \hline
  Режим             & Открытие НА& Частота вращения РК & Напор   & Расход       \\
                    &  $a_0$, мм &   $n$, об/мин &  $H$, м & $Q$, м$^3$/с \\ \hline
  Максимальная мощность    &   219.13   &     200       &  73.5   &    87.980    \\ % \hline
  Частичная нагрузка&   123.26   &     200       &  73.5   &    50.615    \\ \hline
  \end{tabular}
\label{tab3:1}
\end{table}
\begin{figure}[!b]
  \centering
  {\includegraphics[width=8.2cm]{fig13_F1z_1.png}}\hfill
  {\includegraphics[width=8.2cm]{fig13_F1xy_1.png}}
  \caption{Пульсации компонент нагрузки $\textbf{F}_1$ в режиме максимальной мощности:
  {\sl 1}~--- $F_{1,\,z}$, {\sl 2}~--- $F_{1,\,x}$, {\sl 3}~--- $F_{1,\,y}$}
  \label{fig3:13}
\end{figure}

\vspace*{-1mm}

\subsection{Режим максимальной мощности}
\label{s:361}
\subsubsection{Сила $\textbf{F}_1$}
\label{s:3611}
На рисунке~\ref{fig3:13} представлены рассчитанные в полной постановке зависимости компонент 
силы $\textbf{F}_{1}$ от времени.
Период пульсаций $F_{1,\,z}$   $T = 0.01875$~с, их частота $f_b ={1}/{T} = 53.28$~Гц.
При этом частота вращения РК  $f_n  = 3.33$~Гц. Отношение  ${f_b}/{f_n}=16$  совпадает с количеством 
лопастей РК, т.\,е. наблюдаемые пульсации обусловлены лопастной системой РК и имеют в своей основе 
продольную (вдоль потока) природу. Та же самая частота наблюдается
для пульсаций радиальных компонент $F_{1,\,x}$ и $F_{1,\,y}$ нагрузки $\textbf{F}_1$. Среднее значение 
модуля радиальной силы
$|{\bf{F}}_{1,\,R}| =\sqrt {F_{1,\,x}^2 +F_{1,\,y}^2}= 3077$ кГс. Амплитуда колебаний в данном 
режиме пренебрежимо мала и составляет 1\,\%.

\subsubsection{Осевая составляющая $F_{2,\,z}$}
\label{s:3612}
\begin{table}[b!]
  \centering\small\caption{Давления $p_A$ и $p_B$ для расчета течения в верхней области протечки}\vspace*{2mm}

  \begin{tabular}{|c|c|c|} \hline
Давление  & Инженерно-эмпирическая & Расчет  \\
  & методика~\cite{mak_pilev} & в полной постановке  \\ \hline
    $p_A$   &                 54.570              &       52.465                \\ % \hline
    $p_B$   &                 -4.100              &       -0.535                \\ % \hline
  $p_A-p_B$ &                 58.670              &       53.000                \\ \hline
    \end{tabular}
  \label{tab3:2}
\end{table}
\begin{table}[b!]
  \centering\small  \caption{Характеристики течения в верхней области протечки}\vspace*{2mm}

  \begin{tabular}{|p{4cm}|c|c|c|} \hline
  &                            &                                          & Расчет с $p_A$ и $p_B$, \\
  Параметр & Данные~\cite{mak_pilev}    & Расчет с $p_A$ и $p_B$ из~\cite{mak_pilev} & полученными  \\
  &                            &                                         & в полной постановке      \\ \hline
Расход через уплотнение $Q$, м$^3$/с  &    0.295     &    0.258    &  0.239     \\ \hline
Осевая сила $F_{2,\,z}$, кГс          &    34546     &    33113    &  47694     \\ \hline
  \end{tabular}
  \label{tab3:3}
\end{table}
\begin{figure}[b!]\vspace*{4mm}
\centering \small \rule{0mm}{0mm}\emph{a}\rule{95mm}{0mm}\emph{б} \\[1.5mm]
  {\includegraphics[width=8.2cm]{fig14_a1_1.png}}\hfill
  {\includegraphics[width=8.2cm,trim = 7 7 7 7,clip=true]{fig14_b1_1.png}}
  \caption{Распределение давления в верхней области протечки ({\it а}) и линии тока вблизи разгрузочного 
  отверстия в относительном движении ({\it б})}
  \label{fig3:14}
\end{figure}

Расчет осевой составляющей   $F_{2,\,z}$ проводился в соответствии с постановкой, описанной в 
разделе~\ref{s:342}.
В частности, давления $p_A$ перед входной границей расчетной области и~$p_B$ за выходной границей 
(см. рисунок~\ref{fig3:4}) взяты из результатов расчета потока в полной постановке. С целью сравнения с 
результатами, полученными по инженерно-эмпирической методике~\cite{mak_pilev}, проведен расчет с 
давлениями $p_A$  и $p_B$, взятыми из этой работы (таблица~\ref{tab3:2}).

Расход и осевая сила $F_{2,\,z}$, найденные в расчетах, сравниваются с данными работы~\cite{mak_pilev} 
в таблице~\ref{tab3:3}. Видно, что в нашем расчете при одинаковых с [3] давлениях $p_A$ и $p_B$ расход 
получается на 12\,\% ниже, а сила $F_{2,\,z}$ хорошо совпадает с результатом~\cite{mak_pilev}. На 
рисунке~\ref{fig3:14},\,{\it а}  показано распределение давления во всей расчетной 
области, {\it б}~--- представлена картина течения в полости над ступицей и в разгрузочных отверстиях. 
На рисунке~\ref{fig3:15} приведены распределения давления~({\it а}) и отношения окружной
компоненты скорости $c_u$ к $\omega R_r$~({\it б}) вдоль средней линии полости. Результаты, полученные 
в расчете сравниваются с данными, полученными по~[3]. Видно, что предположение о квадратичной зависимости 
давления от расстояния до оси РК, принятое в инженерно-эмпирической методике~[3], не выполняется.
\begin{figure}[t!]%[htb]
\centering \small \emph{a}\rule{90mm}{0mm}\emph{б} \\[1.5mm]
  {\includegraphics[width=8.5cm]{fig15a_p_2.png}}\hfill
  {\includegraphics[width=8.5cm]{fig15b_w_1.png}}
  \caption{Давление ({\it а}) и распределение отношения $c_u/\omega R_r$ ({\it б}) в полости между 
  ступицей и крышкой турбины в режиме максимальной мощности:
  {\sl 1}~--- расчет при $p_A$ и $p_B$, взятых из~\cite{mak_pilev};
  {\sl 2}~--- расчет при $p_A$ и $p_B$ из полной постановки; $\Box$ и серый цвет~--- данные~\cite{mak_pilev}}
  \label{fig3:15}
\end{figure}

\subsubsection{Радиальные составляющие $\textbf{F}_{2,R}$ и $\textbf{F}_{3,R}$}
\label{s:3613}
Рассмотрена несоосность статора и ротора без прецессии ($\Omega=0$) и с 
прецессией (с~заданной частотой $\Omega$) вала ротора. В последнем случае исследован наиболее 
вероятный вариант~--- со скоростью вращения рабочего колеса  $\Omega=\omega$, соответствующий изгибу
вала ротора.

При расчете радиальной составляющей $\textbf{F}_{2,R}$ проведено сравнение
подхода (см. раздел~\ref{s:343}), основанного на трехмерном расчете всего ЛУ, и
подхода (см. раздел~\ref{s:35}), реализующего улучшенную инже\-нер\-но-эм\-пи\-рическую
методику. Результаты расчетов  $\textbf{F}_{2,R}$ в~случае
непрециссирующего ротора ($\Omega = 0$) представлены на рисунке~\ref{fig3:16}.
Видно, что модифицированная методика и  трехмерная модель
течения жидкости дают близкие между собой результаты, отличающиеся от данных
расчета по методике~\cite{lomakin}.
\begin{figure}[!ht]
  \centering\small \emph{a}\rule{90mm}{0mm}\emph{б} \\[1.5mm]
  {\includegraphics[width=8.2cm]{fig16_1.png}}\hfill
  {\includegraphics[width=8.2cm]{fig17_1.png}}
  \caption{Зависимости модуля радиальной силы $|\textbf{F}_{2,R}|$ от относительного 
  эксцентриситета $\varepsilon$ в случае непрециссирующего ротора с
  невращающимся РК ({\it а}) и вращающимся РК ({\it б}):
  $\circ$~--- ин\-же\-нер\-но-эмпирическая методика~\cite{lomakin}, $\triangle$~--- улучшенная 
  инженерно-эмпирическая методика,
  $\square$~--- трехмерный расчет}
  \label{fig3:16}
\end{figure}

Давления перед входом в верхнее ЛУ $p_A=54.57$~м, за выходом из ЛУ $p_{out}=12.16$~м взяты
из расчета в верхней области протечки. Дополнительные входные данные, необходимые
для определения радиальной нагрузки, следующие: радиус ротора  $R_r=1.38225$~м,
постоянный зазор щели ЛУ $d=0.002$~м, длина ЛУ $l=0.11$~м, число ячеек расширения $n=4$.

Трехмерный расчет требует 2--3 ч времени, а улучшенная методика --- 2--3 мин, что позволяет рекомендовать 
последнюю для практической оценки модуля радиальной силы.

Результаты трехмерных расчетов показали, что вращение вала РК и скорость его прецессии оказывают
существенное влияние на направление радиальной силы. Этот факт находится в соответствии с результатами 
работы\,\cite{marcinkovskii}.
Если РК не вращается ($\omega\,=\,0$), то ${\textbf{F}}_R =\left({F_x ,\,0}\right)$. Однако при вращении 
РК с угловой скоростью $\omega$, согласно~\cite{marcinkovskii}, в~уплотнении помимо компоненты $F_x$ 
радиальной силы, пропорциональной перепаду давления, возникает дополнительная
гидродинамическая сила, перпендикулярная оси $Ox$ и пропорциональная угловой скорости вращения 
РК (компонента $F_{2,\,y}$, рисунок~\ref{fig3:20}).
Кроме того, вращение РК приводит к появлению радиальной силы, направленной в сторону увеличения 
эксцентриситета, уменьшающей компоненту $F_{2,\,x}$ радиальной силы.

Наличие прецессии $\Omega$ существенно меняет направление радиальной силы $\textbf{F}_{2,\,R}$. 
За счет возникающего при $\Omega=\omega$ вязкого сопротивления поступательному перемещению центра вала 
компонента $F_{2,\,y}$ меняет знак (см. рисунок~\ref{fig3:20}).
\begin{figure}[!hb]
  \centering
  \includegraphics[width=8.2cm]{fig20_F2x_1.png}\hfill
  \includegraphics[width=8.2cm]{fig20_F2y_1.png}
  \caption{Зависимости $F_{2,\,x}$ и $F_{2,\,y}$ от $\varepsilon$ в расчете трехмерного течения в ЛУ:
  $\triangle$~--- неподвижное РК ($\omega=0,\ \Omega=0$), $\Box$~--- РК, вращающееся 
  со скоростью $\omega$ без прецессии вала ($\Omega=0$),
  $\circ$~--- РК, вращающееся со скоростью $\omega$ с прецессией вала ($\Omega=\omega$)}
  \label{fig3:20}
\end{figure}

\begin{figure}[!b]
  \includegraphics[width=8.2cm]{fig23_F1_reg2.png}\hfill \includegraphics[width=8.2cm]{pA_pB_1.png}\\[-5mm]
\parbox[t]{0.48\textwidth}{\caption{Пульсации осевой $F_{1,\,z}$ и 
радиальных $F_{1,\,x}$, $F_{1,\,y}$ составляющих нагрузки $\textbf{F}_1$ \label{fig3:23}}}\hfill
\parbox[t]{0.48\textwidth}{\caption{Пульсации давления в точках $A$ и~$B$}\label{fig3:25}}
\end{figure}

\subsection{Режим частичной нагрузки}
\label{s:362}
Для данного режима характерно наличие прецессирующего вихревого жгута в конусе отсасывающей трубы, 
оказывающего влияние на ГТ вверх по потоку. Частота прецессии вихря $f_\nu=0.74$ Гц явно выделяется
в пульсациях осевой и радиальной нагрузок, действующих на РК (рисунок~\ref{fig3:23}).
В режиме неполной нагрузки присутствует сильная динамическая составляющая 
сил $F_{1,\,z}$, $\textbf{F}_{1,\,R}$, вызванная прецессией вихревого жгута. Она имеет
для  $F_{1,\,z}$ величину порядка 10\,\%, для $\textbf{F}_{1,\,R}$~--- 100\,\% от среднего значения. 
Частота динамических составляющих равна $0.222 f_n$.
На рисунке~\ref{fig3:25} показаны пульсации давления в точках $A$ и $B$ в режиме частичной нагрузки.
При расчете нагрузок $\textbf{F}_2$ и $\textbf{F}_3$ давления $p_A$ и $p_B$ усредняются по времени.

\section{Исследование течения в полости над ступицей и в разгрузочных отверстиях}
\label{s:37}
Осевая сила $F_{2,\,z}$ определяется распределением давления в полости над ступицей. 
Распределение давления на внешней поверхности ступицы зависит в основном от давления $p_B$, 
сопротивления разгрузочного устройства и закона движения жидкости в полости над ступицей. Это давление 
практически не зависит от сопротивления в верхнем ЛУ, так как гидравлическое сопротивление разгрузочных 
устройств мало по сравнению с сопротивлением лабиринтного уплотнения. Таким образом, при заданном 
давлении $p_B$ точность определения осевой силы определяется точностью нахождения угловой скорости 
жидкости над ступицей $\tilde{\omega}$ и точностью определения сопротивления разгрузочных отверстий. 
Согласно экспериментальным данным, рассмотренным в \cite{makar}, коэффициент сопротивления разгрузочных 
отверстий зависит главным образом от 
соотношения $W_0 / V_p$, где $W_0 =\sqrt{c_r^2 + c_{u,rel}^2}$ -- относительная скорость потока 
перед входом в отверстие, $V_p$ -- скорость в самом отверстии. Кроме этого, при малом относительном осевом 
зазоре над  отверстием ($S_p / d_p < 1$) отношение $S_p / d_p$ также влияет на сопротивление (здесь 
$S_p$ -- высота верхней области протечки над разгрузочным отверстием, $d_p$ -- диаметр 
разгрузочного отверстия). Экспериментальные данные, проанализированные в \cite{makar}, обобщены на 
рисунке~\ref{fig3:26}. 
\begin{figure}[!t]
  \centering
  \includegraphics[width=12cm]{eksp_ot_Makar.jpg}\\[-5mm]
  \caption{Обобщенная экспериментальная зависимость из \cite{makar}}
  \label{fig3:26}
\end{figure}
\begin{figure}[!t]
  \centering
  \includegraphics[width=12cm]{geom_UpDomain.png}\\[-5mm]
  \caption{Рассмотренные варианты полости между ступицей и крышкой турбины. Серым цветом показана 
           полость над ступицей натурной турбины ГЭС Платановрисси ($S_p = 80$~мм). Темным выделена 
           область над ступицей, соответствующая варианту $S_p = 10$~мм}
  \label{fig3:27}
\end{figure}
С целью сравнения расчетных значений сопротивления разгрузочных отверстий с обобщенными экспериментальными 
данными, проведены расчеты течения в верхней области протечки гидротурбины ГЭС Платановрисси, в которых 
варьировалась высота $S_p$ полости между ступицей и крышкой турбины при неизменном диаметре $d_p$ и 
числе $z_p$ разгрузочных отверстий. Это позволило рассмотреть различные режимы натекания жидкости 
на разгрузочные отверстия, соответствующие различным отношениям $W_0/V_p$  и $S_p/d_p$. Рассмотрены 
варианты $S_p = 10,\ 20,\ 40,\ 60,\ 80,\ 120,\ 160$~мм (рисунок~\ref{fig3:27}). В расчетах анализировались 
также распределение закрутки потока $\tilde{\omega}$ по радиусу и картина течения над 
разгрузочными отверстиями.
\begin{figure}[!b]
  \includegraphics[width=8.2cm]{RO_p_distribution.png}\hfill 
  \includegraphics[width=8.2cm]{omega1_omegaR.png}\\[-5mm]
  \parbox[t]{0.48\textwidth}{\caption{Распределение давления вдоль линии $MM^\prime$, 
            проходящей через центр разгрузочного отверстия} \label{fig3:28}}\hfill
  \parbox[t]{0.48\textwidth}{\caption{Распределение угловой скорости потока над ступицей, 
            отнесенной к угловой скорости РК} \label{fig3:29}}
\end{figure}

Расчеты показали (рисунок~\ref{fig3:28}), что для всех значений $S_p$ основное падение давления происходит 
на входе в разгрузочное отверстие, что согласуется с заключением работы \cite{makar}. Резкое 
снижение давления в центре разгрузочного отверстия для варианта $S_p = 10$~мм связано с формированием 
вихревого жгута в разгрузочном отверстии (рисунок~\ref{fig3:31}, слева). 
На рисунке~\ref{fig3:29} показано распределение угловой скорости потока над ступицей $\tilde{\omega}$, к угловой 
скорости вращения РК $\omega_R$. Угловая скорость измерялась вдоль линии $m$ на рисунке~\ref{fig3:32}. 
На рисунке~\ref{fig3:29} серой полосой показан интервал для отношения $\tilde{\omega}/\omega_R$, 
который принимается для определения $W_0$ в инженерно-эмпирической методике.  
В таблице~\ref{tab3:4} представлены рассчитанные значения расхода 
и осевой силы $F_{2,\,z}$, действующей на внешнюю поверхность ступицы. 
Сравнение сопротивлений разгрузочных отверстий, рассчитанных по 
трехмерной модели с инженерно-эмпирической методикой затруднено тем, что течение в окрестности входа 
в разгрузочное отверстие и выхода из него имеет существенно трехмерный характер (рисунок~\ref{fig3:30}). 
Кроме этого, все параметры потока в полости существенно меняются в зависимости от радиуса. Поэтому имеется 
некий произвол в выборе характерных величин $c_r$, $\tilde{\omega}$ перед разгрузочным отверстием, 
а также перепада давления $\Delta p$, которые подставляются в инженерные формулы. 

В расчетах коэффициент сопротивления разгрузочных отверстий находился из соотношения 
\begin{equation}
  \Delta p = \xi _p \frac{{V_p^{\rm{2}} }}{{{\rm{2g}}}},
\end{equation}
где $\Delta p$ -- перепад давлений в точках 1 и 2 
(рисунок~\ref{fig3:32}), $ V_p = \frac{{4q_B}}{{z_p \pi d_p^2 }}$ -- средняя расходная скорость в 
разгрузочном отверстии. Для сравнения с экспериментальными данными (рисунок~\ref{fig3:26}) в таблице~\ref{tab3:4} 
представлено также рассчитанное значение коэффициента расхода $\mu _p  = 1/\sqrt {\xi _p } $. 
Для позиционирования значения $\mu _p$ на графике рисунка~\ref{fig3:26}, необходимо вычислить значение 
модуля относительной скорости $W_0  = \sqrt {c_r^2  + c_{u,rel}^2 }$ над разгрузочным отверстием. 
Для определения $c_r$ использовалась формула $c_r  = \frac{{q_B }}{{2\pi R_p S_p }}$, в то время как 
относительная окружная составляющая скорости $c_{u,rel}  = R_p (\omega _1  - \omega _R )$ полагалась равной 
нулю. На рисунке~\ref{fig3:33} показано сравнение полученной в результате зависимости $\mu _p (W_0 /V_p )$ с 
эмпирическими данными \cite{makar}. Наблюдается хорошее качественное и количественное совпадение с 
обобщенными экспериментальными данными. В настоящих расчетах коэффициент расхода выше примерно на 15~\%. 
\begin{table}[t!]
  \centering\small\caption{Интегральные параметры течения в верхней области протечки}\vspace*{2mm}
  \begin{tabular}{|c|c|c|c|c|c|c|c|c|c|} \hline
  $S_p$, м & $S_p / d_p$ & $q_B$, м$^3$/с & $F_{2,\,z}$, тс & $\Delta p$, м & $V_p$, м/с & 
  $\xi_{p,\,\text{расч}}$ & $\mu_{p,\,\text{расч}}$ & $C_r$, м/с & $W_0/V_p$ \\ \hline
    0.01  & 0.182 & 0.232 & 68.89 & 10.17 & 4.88 & 8.37 & 0.35 & 5.43 & 1.11 \\ % \hline
    0.02  & 0.364 & 0.257 & 38.83 & 3.79  & 5.41 & 2.54 & 0.63 & 3.01 & 0.56 \\ % \hline
    0.04  & 0.727 & 0.259 & 32.5  & 2.65  & 5.45 & 1.75 & 0.76 & 1.52 & 0.28 \\ % \hline
    0.06  & 1.091 & 0.259 & 32.02 & 2.58  & 5.45 & 1.70 & 0.77 & 1.01 & 0.19 \\ % \hline
    0.078 & 1.418 & 0.258 & 33.11 & 2.50  & 5.43 & 1.66 & 0.78 & 0.77 & 0.14 \\ % \hline
    0.12  & 2.182 & 0.257 & 33.06 & 2.43  & 5.41 & 1.63 & 0.78 & 0.50 & 0.09 \\ % \hline
    0.16  & 2.909 & 0.254 & 35.73 & 2.24  & 5.35 & 1.54 & 0.81 & 0.37 & 0.07 \\   \hline
  \end{tabular}
  \label{tab3:4}
\end{table}

\clearpage
\thispagestyle{empty}
\begin{figure}[!hb]
  \centering
  \text{внешняя пов. ступицы} \rule{15mm}{0mm} \text{средняя линия} 
  \rule{20mm}{0mm} \text{крышка турбины} \rule{5mm}{0mm} \\
  \includegraphics[width=5cm]{sp10l.png}\hfil\includegraphics[width=5cm]{sp10m.png}\hfil
  \includegraphics[width=5cm]{sp10r.png} \\
  $S_p = 10$~мм \\
  \includegraphics[width=5cm]{sp20l.png}\hfil\includegraphics[width=5cm]{sp20m.png}\hfil
  \includegraphics[width=5cm]{sp20r.png} \\
  $S_p = 20$~мм \\
  \includegraphics[width=5cm]{sp40l.png}\hfil\includegraphics[width=5cm]{sp40m.png}\hfil
  \includegraphics[width=5cm]{sp40r.png} \\
  $S_p = 40$~мм \\
  \includegraphics[width=5cm]{sp80l.png}\hfil\includegraphics[width=5cm]{sp80m.png}\hfil
  \includegraphics[width=5cm]{sp80r.png} \\
  $S_p = 80$~мм \\
  \includegraphics[width=5cm]{sp120l.png}\hfil\includegraphics[width=5cm]{sp120m.png}\hfil
  \includegraphics[width=5cm]{sp120r.png} \\
  $S_p = 120$~мм \\
  \caption{Картина относительного движения в полости над ступицей в окрестности входа в разгрузочное 
  отверстие. Показаны векторы скорости и линии тока вблизи внешней поверхности ступицы, в среднем по высоте сечении полости, вблизи крышки турбины}
  \label{fig3:30}
\end{figure}
 
\begin{figure}[!b]
  \centering
  \includegraphics[width=8.2cm]{sp10_flow.png}\hfill 
  \includegraphics[width=8.2cm]{sp80_flow.png}\\
  \text{$S_p = 10$ мм} \rule{55mm}{0mm} \text{$S_p = 80$ мм} \\
  \caption{Линии тока и изолинии давления изоповерхности давления на внешней поверхности 
  ступицы и в разгрузочном отверстии}
  \label{fig3:31}
\end{figure}

\begin{figure}[!b]
  \centering
  \includegraphics[width=12cm]{geom_p12.png}\\[-5mm]
  \caption{Точки 1 и 2 для расчета $\Delta p$}
  \label{fig3:32}
\end{figure}

\clearpage
\begin{figure}[!t]
  \centering
  \includegraphics[width=12cm]{count_eksp_UpDomain.png}\\[-5mm]
  \caption{Сравнение рассчитанного коэффициента расхода (\textcolor{red}{$\circ$}) с экспериментальными 
  данными ({\textbf{---}}). Цифрами показано отношение $S_p/d_p$}
  \label{fig3:33}
\end{figure}
\rule{0mm}{20mm}


  %!TEX root = dissertation.tex
% \pagestyle{plain}
\chapter*{Глава 4. Численная модель течения в гидротурбине с кольцевым затвором}
\label{s:4}
\setcounter{chapter}{4}
\addcontentsline{toc}{chapter}{Глава~\thechapter~ Численная модель течения в гидротурбине с кольцевым затвором}
\setcounter{section}{0}

\section{Общая постановка задачи}
\label{s:41}
В диссертации разработана численная модель и на ее основе осуществлено моделирование течений в гидротурбине 
либо при выдвинутом фиксированном положении затвора, либо в процессе его выдвижения. Проведен анализ 
осевых и радиальных сил, действующих на затвор.

Объектом исследований этой главы является проточная часть насос-турбины с кольцевым затвором и течение воды в 
ней. На рисунке~\ref{fig:1},~\emph{а} изображена расчетная область, состоящая из спиральной камеры с 22 
статорными колоннами и т.н. <<зубом>> (рисунок~\ref{fig:55}), направляющего аппарата с 24 лопатками и 
рабочего колеса, имеющего 8 лопастей. Там же приведены расчетные сетки в блоках СК, НА и РК, стрелкой 
показано направление потока воды. На рисунке~\ref{fig:1},~\emph{б} приведено меридиональное сечение расчетной 
области. Изображено положение затвора, выдвинутого в проточный тракт на 0.85 от высоты $b_0$ НА. 
Величину 
\begin{equation}
  s=\dfrac{b}{b_0},
\end{equation}
где $b$ -- высота выдвинутой части затвора, будем называть степенью закрытия.   
\begin{figure}[ht]
  \centering \small \rule{20mm}{0mm}\emph{a}\rule{90mm}{0mm}\emph{б}\\[1.5mm]
  {\includegraphics[width=10.5cm]{rasch_oblast_coarse_grid.png}}\hfill
  {\includegraphics[width=6.5cm]{mer_slide.png}}\\  
  \caption{Проточная часть насос-турбины с кольцевым затвором: \emph{а}) расчетная область с сетками 
           в трех блоках; \emph{б}) меридиональное сечение расчетной области}
  \label{fig:1}
\end{figure}

Наличие затвора, перекрывающего ПТ ГТ, можно учесть посредством задания условия твердой стенки на слое сетки, 
разделяющем статорные колонны и НА. При этом требуется минимальная модификация программных модулей, но не 
учитывается форма оголовка затвора и изменение геометрии проточной части. Вторым вариантом является добавление
нового элемента проточной части -- области под затвором, при этом геометрия проточного тракта моделируется 
точно. В работе предложены две основные модели: модель бесконечно тонкого затвора и 
модель затвора конечной реальной толщины. Каждая из этих моделей рассмотрена как в стационарной, так и в 
нестационарной постановке. При этом нестационарные расчеты проведены как при фиксированном, так и при 
движущемся затворе. 

Для верификации процедур построения сеток в добавленной области под оголовком затвора и процедур передачи 
данных из нее в каналы статора и направляющего аппарата проведены расчеты течений в области, включающей один 
канал статора, один канал направляющего аппарата и один канал рабочего колеса (циклическая постановка). 
В расчетах менялось открытие направляющего аппарата, а также изменялась степень закрытия затвора от 0 до 99~\%.

После этого была проведена адаптация программного комплекса CADRUN2 к задаче моделирования процесса 
закрытия затвора в полной постановке, описывающей течение в спиральной камере, во всех каналах статора, 
направляющего аппарата и рабочего колеса. В приближении уравнений Рейнольдса, замкнутых 
стандартной $k-\varepsilon$ моделью турбулентности, рассчитано течение для одного угла открытия направляющего 
аппарата и семи положений затвора. 
Исследовано влияние степени закрытия затвора на крутящие моменты лопаток направляющего аппарата и 
выталкивающую осевую силу $F_z$, действующую на оголовок затвора. Получены оценки 
радиальных сил $(F_x, F_y)$, действующих на боковую поверхность затвора, и других эффектов, связанных с 
неравномерностью потока в спиральной камере и ротор"=статор взаимодействием. 

\section{Модели затвора}
\label{s:42}
\subsection{Модель бесконечно тонкого затвора}
\label{s:421}
На основании  предоставленных филиалом <<Силовые машины>> -- ЛМЗ материалов, отражающих опыт 
фирмы <<Нейрпик>>, толщина кольцевого затвора составляет около 2\% от диаметра РК и около 10\% от высоты НА. 
Учитывая малую толщину затвора по сравнению с высотой НА, можно считать, что 
хорошим приближением для моделирования процесса закрытия является модель тонкого затвора. В этом случае 
кольцевой затвор считается бесконечно тонкой цилиндрической поверхностью постоянного радиуса, расположенной 
за выходной кромкой статорных колонн (рисунок~\ref{fig:2}). На ней ставится условие 
прилипания к твердой стенке. 
\begin{figure}[!h]
  \centering \includegraphics[width=12cm]{tonk_zatvor_cycle.png}\\  
  \caption{Расчетная область в циклической постановке для модели бесконечно тонкого затвора: $s=0.5$}
  \label{fig:2}
\end{figure}

\subsection{Модель затвора реальной толщины}
\label{s:422}
Также в работе рассмотрена модель затвора реальной толщины. В этом случае кольцевой затвор -- цилиндр с 
заданной формой оголовка (рисунок~\ref{fig:3}). 
Для реализации данной модели в программный комплекс введен новый 
элемент проточной части -- кольцевая область между оголовком затвора и нижней крышкой НА. 
Форма оголовка и сетка в ее окрестности приведены на рисунке~\ref{fig:31}. Также в программный комплекс 
добавлены новые процедуры построения сеток в области под затвором и передачи данных из нее 
в каналы статора и НА.
\begin{figure}[!hb]
  \centering \includegraphics[width=12cm]{tolst_zatvor_cycle.png}\\  
  \caption{Расчетная область в циклической постановке для модели затвора конечной толщины: $s=0.5$}
  \label{fig:3}
\end{figure}
\begin{figure}[!hb]
  \centering \includegraphics[width=12cm]{tolst_zatvor_mer_1.png}\\  
  \caption{Меридиональная проекция сетки в спиральной камере, области под затвором и направляющем 
           аппарате: $s=0.5$}
  \label{fig:31}
\end{figure}

\section{Входные и выходные условия}
\label{s:43}
На входе в область статора в циклической постановке задается полная энергия потока
\begin{equation*}
  E_{in} = H-\Delta h_{CK},
\end{equation*}
где $H$ -- напор на турбине, $\Delta h_{CK}$ -- потери энергии в спиральной камере. 
Предполагалось, что $\Delta h_{CK}=0.01H$, поэтому на входе в статор задается полная энергия 
потока $E_{in} = 0.99H$ и угол входа потока $\delta  = 30^\circ$. На выходе из РК держится полная 
энергия $E_{out} = \Delta h_{OT}$, 
где $\Delta h_{OT}$ -- потери энергии в отсасывающей трубе. Они оцениваются по инженерно-эмпирической формуле 
\cite{idelchik,etinberg}
\begin{equation*}
  \Delta h_{OT} = \xi_0 \frac{{\bar c_z^2}}{{2g}}+\frac{{\bar c_u^2 }}{{2g}},
\end{equation*}
где $\xi _0 \, = 0.15$ -- коэффициент потерь на трение в ОТ, $\bar c_u ,\,\bar c_z $ -- компоненты абсолютной 
скорости на выходе из рабочего колеса, усредненные по расходу (меняются в процессе расчета). 

Кроме этого, на выходе из РК реализуется условие радиального равновесия на профиль давления 
\begin{equation*}
  \frac{dp}{dr}=\frac{c_u^2}{r},
\end{equation*}
где $c_u$ -- окружная компонента скорости. Такие краевые условия позволяют 
определять расход $Q$ в процессе расчета течения.

В отличие от расчетов в циклической постановке, расчеты в полной постановке проведены с использованием 
традиционных граничных условий «расход на входе -- давление на выходе». На входе в спиральную камеру 
фиксировалось значение расхода $Q$, полученное для данного закрытия тонкого затвора в циклической постановке. 
На выходе из области РК задавалось постоянное давление $p = 0$.

\section{Силы, действующие на кольцевой затвор}
\label{s:44}
\subsection{Выталкивающая сила, действующая на оголовок}
\label{s:441}
Выталкивающая сила $F_z$, действующая со стороны жидкости на оголовок затвора, в модели тонкого затвора 
оценивалась по формуле
\begin{equation}
  F_z = R_0 d\int\limits_0^{2\pi}{p(\varphi)d\varphi}, 
  \label{eq:5}
\end{equation}
где $R_0$=0.71435~м -- внешний радиус реального затвора, $d$=0.03174~м -- толщина реального затвора, 
$p(\varphi)$ -- окружное распределение давления в потоке сразу под кончиком затвора. 

В случае затвора реальной толщины выталкивающая сила рассчитывается по формуле
\begin{equation}
  F_z = \int\limits_S p n_z dS,
  \label{eq:6}
\end{equation}
где $p$ -- давление воды, $S$ -- поверхность оголовка, $n_z$ -- $z$-я компонента вектора внутренней единичной 
нормали к поверхности оголовка затвора.
\begin{figure}[!hb]
  \centering \small \emph{a}\rule{90mm}{0mm}\emph{б}\\[1.5mm]
  {\includegraphics[width=8.5cm]{Q11_cycle.png}}\hfill
  {\includegraphics[width=8.5cm]{Fz_11_cycle.png}}\\  
  \caption{Зависимости приведенных расхода $Q_{11}$ (\emph{а}) и силы $F_{z,\, 11}$ (\emph{б}) от $s$ для
           различных открытий НА: $\blacktriangle$~~$\alpha_0=12^\circ$; $\bullet$~~$\alpha_0=26.5^\circ$; 
  $\blacksquare$~~$\alpha_0=40^\circ$. --~--~тонкий~затвор; $\genfrac{}{}{1pt}{2}{\quad}{  }$ затвор 
  реальной толщины}
  \label{fig:4}
\end{figure}

\subsection{Радиальная сила, действующая на боковую поверхность затвора}
\label{s:442}
Также представляет интерес радиальная сила $(F_x,F_y)$, действующая на боковую поверхность затвора. Она 
возникает вследствие окружной неравномерности потока в спиральной камере. Компоненты этой силы рассчитывались 
по формулам
\begin{gather}
  F_x = \int\limits_{S_1} p n_x dS + \int\limits_{S_2}p n_x dS + \int\limits_{S_3} p n_x dS,
  \label{eq:7} \\
  F_y = \int\limits_{S_1} p n_y dS + \int\limits_{S_2}p n_y dS + \int\limits_{S_3} p n_y dS,
  \label{eq:8}
\end{gather}
где $p$ -- давление жидкости, $S_1$ -- внешняя поверхность затвора (со стороны статора), $S_2$ -- внутренняя 
поверхность затвора (со стороны НА), $S_3$ -- поверхность оголовка, $(n_x,n_y)$ -- компоненты вектора 
внутренней единичной нормали к поверхности затвора.

\section{Результаты моделирования течений в гидротурбине с кольцевым затвором}
\label{s:45}
\subsection{Стационарная постановка}
\label{s:451}
Рассмотрены три угла открытия лопаток НА $\alpha_0=12^\circ,\ 26.5^\circ,\ 40^\circ$ и десять степеней 
закрытия кольцевого затвора $s$ от 0.00 до 0.95. На рисунке~\ref{fig:4} показано сравнение рассчитанных 
приведенных к $H=1$~м, $D_1=1$~м расхода $Q_{11}=\dfrac{Q}{D_1^2\sqrt{H}}$ и 
силы $ F_{z,\, 11}=\dfrac{F_z}{D^2_1 H}$ для двух рассмотренных моделей затвора. Видно заметное отличие 
расхода, полученного в двух постановках, уже при $s\geq 0.5$. В модели затвора реальной толщины расход выше. 
Отметим, что при $s\geq 0.9$ в этой модели выталкивающая сила становится отрицательной, т.е. затвор 
втягивается в проточную часть. 
\subsection{Нестационарная постановка}
\label{s:452}
В нестационарной постановке проведены расчеты с движущимся тонким затвором (циклическая постановка), также в 
модели затвора реальной толщины рассмотрены три степени закрытия кольцевого затвора $s=0.5;\ 0.7;\ 0.9$ 
(полная постановка, угол открытия НА $\alpha_0=26.5^\circ$). 
\subsubsection{Расход жидкости}
\label{s:4521}
На рисунке~\ref{fig:5} представлено сравнение 
зависимостей расхода $Q_{11}$ от степени закрытия, полученных в различных постановках.
\begin{figure}[htb]
  \includegraphics[width=9cm]{Q11_all.png} \hfill\includegraphics[width=8cm]{vane_numbers_3_ink.png}\\
  \parbox{9cm} {\caption{Зависимости приведенного расхода $Q_{11}$ от $s$ для открытия НА  
  $\alpha_0=26.5^\circ$, циклическая постановка: $\bullet$~фиксированные положения тонкого затвора; 
  \textcolor{blue}{$\genfrac{}{}{1pt}{2}{\quad}{  }$}~тонкий движущийся затвор; $\circ$~затвор реальной 
  толщины}
  \label{fig:5}}\hfill
  \parbox{8cm} {\caption{Взаимное положение лопаток направляющего аппарата и колонн статора. Приведена
                         нумерация лопаток НА}\label{fig:55}}
\end{figure}
\subsubsection{Моменты лопаток направляющего аппарата}
\label{s:4522}
На рисунке~\ref{fig:6} приведены моменты на лопатках НА, рассчитанные в различных постановках с фиксированными 
положениями затвора. Выбросы момента на кривой, полученной в полной стационарной постановке 
при $s=0.5$ ($\square$) находятся напротив лопастей РК. Максимальное значение крутящего момента получается на 
лопатке \No~18, находящейся вблизи <<зуба>> СК (рисунок~\ref{fig:55}). Необходимо отметить, что столь сильное 
влияние РК вызвано особенностями стационарной постановки: в процессе установления взаимное положение НА и РК 
остается неизменным. В действительности ротор-статор взаимодействие является более слабым, о чем 
свидетельствуют результаты расчетов в полной нестационарной 
постановке. Видно, 
что при больших закрытиях затвора ($s=0.9$) исчезает влияние «зуба» спирали на моменты лопаток НА. 
\begin{figure}[h!]
  \centering \small \emph{a}\rule{90mm}{0mm}\emph{б}\\[0.3mm]
  {\includegraphics[width=8.0cm]{M_HA_05.png}}\hfill
  {\includegraphics[width=8.0cm]{M_HA_09.png}}\\[0.3mm]  
  \caption{Гидравлический момент на лопатках направляющего аппарата: \emph{a}) $s=0.5$; \emph{б}) $s=0.9$.
  \textcolor{blue}{$\bullet$} -- циклическая постановка, тонкий затвор; 
  \textcolor{blue}{$\blacktriangle$} -- циклическая постановка, затвор реальной толщины; 
  $\square$ -- полная стационарная постановка, затвор реальной толщины; 
  (\textcolor{red}{|}) -- амплитуда колебаний, полная нестационарная постановка, затвор реальной толщины}
  \label{fig:6} \vspace{-5mm}
\end{figure}
\subsubsection{Пульсации сил, действующих на затвор}
\label{s:4523}
Скорость вращения РК $n'_1=83.4$~об/мин дает частоту вращения РК $f_R=\dfrac{n'_1}{60}$~Гц. Тогда частота 
прохождения одной лопасти $f_1=8f_R=11.12$~Гц. Также из"=за вращения РК возникает частота прохождения лопатки 
НА мимо фиксированной точки в РК, которая равна $f_3=24f_R=33.36$~Гц. В пульсациях сил для $s=0.5$ 
(рисунок~\ref{fig:62}) и $s=0.7$ (рисунок~\ref{fig:63}) явно выделяется частота прохождения 
лопасти РК $f_1$~Гц. Для закрытия $s=0.7$ наблюдаются низкочастотные пульсации с периодом 1.1~с, изображенные 
на рисунке~\ref{fig:63}. В колебаниях компонент силы для степени закрытия $s=0.9$ нет выделенных частот.

На рисунке~\ref{fig:7} показано сравнение осевых сил для всех проведенных расчетов. Приближение тонкого затвора 
дает завышенное значение силы $F_{z,\, 11}$, в особенности при больших значениях $s$. Видно, что в диапазоне 
степеней закрытия $0\leq s\leq 0.8$ все постановки с затвором реальной толщины дают очень близкие значения 
$F_{z,\, 11}$. Отличие составляет $s=0.9$, где полные стационарная и нестационарная постановки дают завышенное 
значение силы  $F_{z,\, 11}$. Из рисунка~\ref{fig:8} видно, что по мере закрытия затвора амплитуда пульсаций 
радиальных сил $F_x$ и $F_y$ возрастает. \vspace{-5mm}
\begin{figure}[h!]
  \centering \small \emph{a}\rule{90mm}{0mm}\emph{б}\\[0.5mm]
  {\includegraphics[width=8.5cm]{FxFy_s07_1_n.png}}\hfill
  {\includegraphics[width=8.5cm]{Fz_s07_1.png}}\\  
  \caption{Зависимости компонент силы, действующей на затвор, от времени при $s=0.7$}
  \label{fig:63}
\end{figure}

\begin{figure}[!p]
  \centering \small \emph{a}\rule{90mm}{0mm}\emph{б}\\[0.5mm]
  {\includegraphics[width=8cm]{Fx_s05.png}}\hfill
  {\includegraphics[width=8cm]{Fx_s05_fourie.png}}\\  
  {\includegraphics[width=8cm]{Fy_s05.png}}\hfill
  {\includegraphics[width=8cm]{Fy_s05_fourie.png}}\\  
  {\includegraphics[width=8cm]{Fz_s05.png}}\hfill
  {\includegraphics[width=8cm]{Fz_s05_fourie.png}}\\  
  \caption{Пульсации компонент силы, действующей на затвор, (слева) и их спектр (справа) при закрытии 
          затвора $s=0.5$: на спектрах цифрами отмечены частоты $f_1=11.075$~Гц, $f_2=22.144$~Гц, 
          $f_3=33.214$~Гц, $f_4=44.090$~Гц}
  \label{fig:62}
\end{figure}
\clearpage

\begin{figure}[p!]
  \centering \includegraphics[width=9cm]{Fz_11_all.png}\\  
  \caption{Осевая сила $F_{z,\, 11}$, действующая на оголовок затвора: $\bullet$ -- циклическая постановка, 
           тонкий затвор; сплошная линия -- циклическая постановка, тонкий движущийся затвор; 
           $\circ$ -- циклическая постановка, затвор реальной толщины; $\square$ -- полная стационарная 
           постановка; \textcolor{red}{$\blacktriangle$} -- полная нестационарная постановка; вертикальными 
           отрезками обозначена амплитуда колебаний в нестационарном расчете}
  \label{fig:7}
\end{figure}
\begin{figure}[p!]
  \centering \small \emph{a}\rule{90mm}{0mm}\emph{б}\\[1.5mm]
  {\includegraphics[width=8.5cm]{Fx_kg.png}}\hfill
  {\includegraphics[width=8.5cm]{Fy_kg.png}}\\  
  \caption{Компоненты   радиальной силы, действующей на боковую поверхность 
  затвора: $\square$ -- полная стационарная 
  постановка; вертикальные отрезки -- полная постановка, амплитуда колебаний в нестационарном расчете}
  \label{fig:8}
\end{figure}
\clearpage

\subsubsection{Структура течения вблизи затвора}
\label{s:4524}
На рисунке~\ref{fig:65} изображена структрура потока для разных степеней закрытия затвора. 
\begin{figure}[h!]
  \centering 
  \includegraphics[width=11cm]{Flow_zatvor_all.png}\\
  \caption{Линии тока и изолинии давления на боковых поверхностях статорных колонн и лопаток НА 
           при трех степенях закрытия затвора. Вид со стороны спиральной камеры}
  \label{fig:65}
\end{figure}

  %!TEX root = dissertation.tex
\chapter*{Заключение}
\addcontentsline{toc}{chapter}{Заключение}

Основные научные и практические результаты проведенных исследований заключаются в следующем.

\begin{enumerate}
  \item[1.] Обобщен на задачи с подвижными границами метод расчета течений несжимаемой жидкости в 
            неподвижных областях \cite{Cher}. Предложен подход, обеспечивающий точное выполнение 
            условия геометрической консервативности на дискретном уровне. Построенный метод применен 
            для решения тестовых и практических задач. Анализ результатов показал, что метод 
            корректно воспроизводит все рассмотренные характеристики и может быть использован для 
            моделирования течений в областях с~подвижными границами в различных задачах 
            вычислительной гидродинамики. 
  \item[2.] Разработана модель нестационарного трехмерного потока в 
            переходных режимах работы гидротурбин. Построенная модель учитывает явление гидроудара, 
            переменную частоту вращения РК и меняющийся расход воды, проходящей через водовод и гидротурбину. 
            Представлены результаты расчетов процессов пуска в турбинный режим, уменьшения мощности и 
            мгновенного сброса нагрузки. Исследованы структуры и особенности трехмерных течений при переходных
            режимах работы ГТ: развитие прецессирующего вихревого жгута за РК при уменьшении мощности; срыв 
            вихря с входной кромки лопасти РК и насосный вихрь, возникающие при пуске турбины. 
  \item[3.] Разработан комбинированный метод определения нестационарных осевых и радиальных 
            нагрузок (ОРН) на рабочее колесо, основанный на расчетах трехмерного течения 
            несжимаемой жидкости в основной проточной части и в 
            областях протечек гидротурбины, позволяющий рассчитывать осевые и радиальные нагрузки, 
            вызванные следующими факторами: неравномерностью потока в спиральной камере и статоре, 
            ротор-статор взаимодействием, влиянием вверх по потоку нестационарности 
            в конусе ОТ (вихревой жгут), несоосностью статора и ротора, изгибом вала ротора. Кроме 
            того, метод позволяет рассчитывать расход жидкости через зазоры между вращающимися и 
            неподвижными частями гидротурбины. Потери давления в лабиринтных уплотнениях, щелях и разгрузочных 
            отверстиях вычисляются непосредственно с использованием турбулентной модели, тем самым не 
            требуется эмпирическая информация о гидравлических сопротивлениях этих элементов. 
            С использованием нового метода обнаружено влияние высоты верхней области протечки на течение в 
            разгрузочном отверстии: при высоте менее 10 мм в разгрузочном отверстии формируется закрученное 
            вихревое течение, гидравлическое сопротивление при этом увеличивается в 4 раза. 
  \item[4.] Усовершенствована инженерно-эмпирическая методика~\cite{lomakin} для расчета радиальных нагрузок, 
            действующих на лабиринтное уплотнение вследствие смещения оси вращения ротора. В предложенной 
            методике учтены влияние вращения ротора на коэффициент сопротивления узкой части лабиринта, 
            сопротивление ячеек расширения, зависимость коэффициента сопротивления узкой части и ячеек 
            расширения от переменного зазора между статором и ротором. Усовершенствованная методика применима 
            для расчета радиальных нагрузок в случае несоосности статора и ротора и отсутствия 
            прецесии последнего. Результаты расчетов радиальной силы по предложенной методике хорошо совпадают 
            с данными трехмерных расчетов течения во всем ЛУ для диапазона рассмотренных значений 
            относительного эксцентриситета $\varepsilon$ от 0 до 1, что позволяет рекомендовать ее 
            для оценки возникающих радиальных сил, действующих на РК, при проектировании геометрии 
            областей протечек. Показано, что при относительном эксцентриситете $\varepsilon>0.6$ радиальные 
            силы, действующие на лабиринтные уплотнения, дают более 50\,\% итоговой радиальной нагрузки, 
            действующей на все рабочее колесо гидротурбины. 
  \item[5.] Создан метод расчета течения при наличии цилиндрического кольцевого затвора. С его помощью решены 
            практически важные задачи определения динамических нагрузок на элементы гидротурбины и давлений в 
            нестационарном потоке. Проведено сопоставление зависимостей сил, действующих на затвор, крутящих 
            моментов лопаток, полученных в полной и циклической постановках с затворами реальной толщины и 
            бесконечно тонкими. Установлены интервалы наиболее достоверных значений этих параметров при 
            различных положениях затвора. Показано, что амплитуды пульсаций резко возрастают, начиная от 
            степени закрытия 0.7 и большей. 
  \item[6.] Проведены верификация и валидация разработанных методов. Создан программный комплекс, реализующий 
            построенные численные алгоритмы на многопроцессорных вычислительных системах. Результаты 
            расчетов, выполненных с помощью разработанного программного комплекса, используются в проектных 
            исследованиях филиала ОАО <<Силовые машины>> <<ЛМЗ>> в г.~Санкт-Петербурге.
\end{enumerate}
%%%%%%%%%%%%%%%%%%%%%%%%%%%%%%%%%%%%%%%%%%%%%%%%%%%%%%%%%%%%%%%%%%%%%%%%%%%%%%%%%%%%%%%%%%%%%%%%%%%%

%
%

  %!TEX root = dissertation.tex
% \pagestyle{plain}
\begin{thebibliography}{999}
\addcontentsline{toc}{chapter}{Список литературы}
\baselineskip 21pt

%0000000000000000000000000000000000
%  introduction
%0000000000000000000000000000000000
\Rus
\bibitem{prt_RF}Об утверждении приоритетных направлений развития науки, технологий и техники в 
                Российской Федерации и перечня критических технологий Российской Федерации: Указ 
                Президента Рос. Федерации от 7~июля 2011~г. \No~899~// Собр. законодательства Рос. 
                Федерации.~--- 2011.~--- \No~28.~--- Ст.~4168.
% указ президента
\Eng
\bibitem{GSR_12}Renewables 2012 Global Status Report.~--- REN21, 2012.~--- Paris: 
                REN21 Secretariat.~--- 176~p. 
%
\Rus
\bibitem{Cher}  Черный,~С.\,Г. Численное моделирование течений в турбомашинах~/ С.\,Г.~Черный, Д.\,В.~Чирков, 
                В.\,Н.~Лапин, В.\,А.~Скороспелов, С.\,В.~Шаров.~--- Новосибирск:~Наука, 2006.~--- 202~с.
%
\Eng
\bibitem{Rupr}  Ruprecht,~A. Numerical prediction of vortex instabilities in turbomachinery~/ A.~Ruprecht // 
                Notes on Numerical Fluid Mechanics and Multidisciplinary Design.~--- Springer Verlag.~--- 
                2006.~--- Vol.~93.~--- P.~211--224.
%
\Eng
\bibitem{Vunen1} Ciocan,~G.\,D. Experimental Study and Numerical Simulation of the FLINDT Draft Tube 
                Rotating Vortex~/ G.\,D.~Ciocan, M.\,S.~Iliescu, T.\,C.~Vu, B.~Nennemann, F.~Avellan //
                J. of Fluids Engineering.~--- 2007.~--- Vol.~129, No.~2.~--- P.~146--158.
%
\Rus
\bibitem{Smir}  Смирнов,~Е.\,М. Опыт численного анализа пространственных турбулентных течений в 
                турбомашинах~/ Е.\,М.~Смирнов, А.\,И.~Кириллов, В.\,В.~Рис // Научно"=технические 
                ведомости.~--- СПб.: СПбГПУ.~--- 2004.~--- \No~2(36).~--- C.~55--70.
%
\Eng
\bibitem{Kuro}  Enomoto,~Y. Design optimization of Francis turbine runner using 
                multi"=objective genetic algorithm~/ Y.~Enomoto, S.~Kurosawa, T.~Suzuki // Proc. of 22nd 
                IAHR Symp. on Hydraulic Machinery and Systems.~--- 2000.~--- P.~1--10.
%
\Rus
\bibitem{granovsky} Грановский,~С.\,А. Конструкции и расчет гидротурбин~/ С.\,А.~Грановский, 
                    В.\,М.~Малышев, В.\,М.~Орго, Л.\,Г.~Смоляров.~--- Л.: Машиностроение, 1974.~--- 408~c.
%
\Rus
\bibitem{shevch} Шевченко,~Н.\,Г. Прогнозирование гидравлической силы воздействия потока воды на
                 кольцевой затвор гидротурбины~/ Н.\,Г.~Шевченко, Е.\,С.~Коваль // Сборник научных 
                 трудов SWorld. Матер. межд. научно"=практической конференции <<Современные 
                 направления теоретических и прикладных исследований'2013>>.~--- Выпуск~1. Том~6.~--- 
                 Одесса: КУПРИЕНКО, 2013.~--- ЦИТ: 113-0282.~--- С.~19-22.
%
\Rus
\bibitem{Jmud}  Жмудь,~А.\,Е. Гидравлический удар в гидротурбинных установках~/ А.\,Е.~Жмудь.~--- 
                Изд-во ГЭИ, 1953.~--- 235~с.
%
\Rus
\bibitem{krivch} Кривченко,~Г.\,И. Гидромеханические переходные процессы в гидроэнергетических 
                 установках~/ Г.\,И.~Кривченко, Н.\,Н.~Аршеневский, Е.\,Е.~Квятковский, В.\,М.~Клабуков.~--- 
                 Под ред. Г.\,И. Кривченко.~--- Москва:~Энергия, 1975.~--- 368~с.
%
\Eng 
\bibitem{Nicol} Nicolet,~C. Unstable Operation of Francis Pump-Turbine at runaway: Rigid and Elastic Water 
                Column Oscillation Modes~/ C.~Nicolet, S.~Alligne, B.~Kawkabani, J.-J.~Simond, F.~Avellan. //
                J. of Fluid Machinery and Systems.~--- 2009.~--- Vol.~2, No.~4.~--- P.~324--333.
%
\Rus
\bibitem{Popov} Попов,~Д.\,Н. Нестационарные гидромеханические процессы~/ Д.\,Н.~Попов.~--- 
                М.: Машиностроение, 1982.~--- 239~с.
%
\Rus
\bibitem{AleksMon} Алексеенко,~С.\,В. Введение в теорию концентрированных вихрей~/ С.\,В.~Алексеенко, 
                П.\,А.~Куйбин, В.\,Л.~Окулов.~--- Москва-Ижевск: Институт компьютерных исследований, 
                2005.~--- 504 С.
%
\Rus
\bibitem{KuiPilZah} Куйбин,~П.\,А. Развитие моделей для описания нестационарных явлений в гидротурбинах~/
                П.\,А.~Куйбин, И.\,М.~Пылев, А.\,В.~Захаров // Современная наука.~--- 2012.~--- 
                Т.~2, \No.~10.~--- С.~232-238.
%
\Rus
\bibitem{Samoil} Самойлович,~Г.\,С. Нестационарное обтекание и аэроупругие колебания решеток 
                турбомашин~/ Г.\,С.~Самойлович.~--- Москва: Изд-во Наука, 1969.~--- 444 с.
%
\Rus
\bibitem{GorKurBook} Горелов,~Д.\,Н. Аэродинамика решеток в нестационарном потоке~/ Д.\,Н.~Горелов, 
                В.\,Б.~Курзин, В.\,Э.~Сарен.~--- Новосибирск: Изд-во Наука, Сиб. Отд., 1971.~--- 272 с.
%
\Rus
\bibitem{GorKurAtlas} Горелов,~Д.\,Н. Атлас нестационарных аэродинамических характеристик решеток 
                профилей~/ Д.\,Н.~Горелов, В.\,Б.~Курзин, В.\,Э.~Сарен.~--- Новосибирск: Изд-во Наука, 
                Сиб. Отд., 1974.~--- 150 с.
%
\Rus
\bibitem{KurTurb} Курзин,~В.\,Б. Об акустическом резонансе в турбомашинах~/ В.\,Б.~Курзин // Проблемы 
                прочности.~--- 1974.~--- \No~8.~--- С.~94-98.
%
\Rus
\bibitem{Kur93} Курзин,~В.\,Б. Низкочастотные собственные акустические колебания в проточной части
                гидротурбин~/ В.\,Б.~Курзин // Прикл. механика и техн. физика.~--- 1993.~--- \No~2.~--- 
                С.~96-106.
%
\Rus
\bibitem{OkulPil} Окулов,~В.\,Л. Неустойчивость напорных систем~/ В.\,Л.~Окулов, И.\,М.~Пылев // Докл. РАН. 
                Энергетика.~--- 1995.~--- Т.~341, \No~4.~--- С.~470-473.
%
\Rus
\bibitem{ArkOkulPil} Арм,~В.\,Х. Неустойчивость напорных систем гидроэнергоблоков~/ В.\,Х.~Арм, 
                В.\,Л.~Окулов, И.\,М.~Пылев // Изв. РАН. Энергетика.~--- 1996.~--- \No~3.~--- С.~122-132.
%
\Rus
\bibitem{Kur2013} Курзин,~В.\,Б. О влиянии сжимаемости воды на нестационарные характеристики гидротурбин~/ 
                В.\,Б.~Курзин // Инженерно-физический журнал.~--- 2013.~--- Т.~86, \No~5.~--- С.~1127-1133.
%
\Eng 
\bibitem{Li}    Li,~J. 3D unsteady turbulent simulations of transients of the Francis 
                turbine~/ J.~Li, J.~Yu, Y.~Wu // Proc. of 25th IAHR Symposium on hydraulic machinery and 
                systems.~--- Timisoara, Romania, 20-24 September, 2010.
%
\Eng 
\bibitem{Liu1}  Liu,~S. Runaway transient simulation of a model Kaplan turbine~/ S.~Liu, D.~Zhou, D.~Liu, 
                Y.~Wu, M.~Nishi // Proc. of 25th IAHR Symposium on hydraulic machinery and systems.~--- 
                Timisoara, Romania, 20-24 September, 2010.
%
\Eng 
\bibitem{Liu2}  Liu,~J. Three-dimensional flow simulation of transient power interruption process 
                of a prototype pump-turbine at pump mode~/ J.~Liu, S.~Liu, Y.~Sun, L.~Jiao, Y.~Wu, L.~Wang // 
                J. of Mechanical Science and Technology.~--- 2013.~--- Vol.~27, No.~5.~--- P.~1305-1312.
%
\Eng 
\bibitem{Nicolle} Nicolle,~J. Transient CFD simulation of a Francis turbine startup~/ J.~Nicolle, 
                  J.\,F.~Morissette, A.\,M.~Giroux // Proc. of 26th IAHR Symposium on hydraulic machinery and 
                  systems.~--- Beijing, China, 19-23 August, 2012.
%
%**********************************
%  my
%**********************************
\Rus
\bibitem{my1} Авдюшенко,~А.\,Ю. Численный алгоритм моделирования пространственных течений несжимаемой 
              жидкости на подвижных сетках~/ А.\,Ю.~Авдюшенко, С.\,Г.~Черный, Д.\,В.~Чирков // 
              Вычислительные технологии.~--- 2012.~--- Т.~17, \No~6.~--- С.~3-25.
%~2.79 усл.-печ. л.
\Rus
\bibitem{my1_2} Бабаченко,~Ю.\,В. Расчетное исследование радиальных сил, действующих на 
              ротор радиально-осевой гидротурбины~/ Ю.\,В.~Бабаченко, А.\,Ю.~Авдюшенко // 
              Известия Самарского научного центра Российской академии наук.~--- 2013.~--- 
              Т.~15, \No~4(2).~--- С.~547-552.
             %~1.19 усл.-печ. л.  
\Rus
\bibitem{my2}  Авдюшенко,~А.\,Ю. Численное моделирование переходных процессов в гидротурбинах~/ 
               А.\,Ю.~Авдюшенко, С.\,Г.~Черный, Д.\,В.~Чирков, В.\,А.~Скороспелов, П.\,А.~Турук // 
               Теплофизика и аэромеханика.~--- 2013.~--- Т.~20, \No~5.~--- С.~587-604.
%~2.63 усл.-печ. л.
\Rus
\bibitem{my3} Авдюшенко,~А.\,Ю. Метод расчета осевых и радиальных нагрузок на 
              рабочее колесо гидротурбины в нестационарном потоке~/ А.\,Ю.~Авдюшенко, С.\,Г.~Черный, 
              Д.\,В.~Чирков // Вычислительные технологии.~--- 2013.~--- Т.~18., \No~4.~--- С.~3-25.
%~2.41 усл.-печ. л.
\Eng
\bibitem{my4} Avdyushenko,~A.\,Yu. High-perfomance computations in problems of simulation and optimization of 
              turbine hydrodynamics~/ A.\,Yu.~Avdyushenko, A.\,S.~Astrakova, A.\,E.~Lyutov, S.\,G.~Cherny, 
              D.\,V.~Chirkov // Proc. of Third International Conference 
              High Performance Computing (HPC-UA 2013).~--- Ukraine, Kiev, 2013. P.~28-35. --- 
              \href{http://hpc-ua.org/hpc-ua-13/files/proceedings/5.pdf}
              {http://hpc-ua.org/hpc-ua-13/files/proceedings/5.pdf}
%~1.21 усл.-печ. л. 
\Rus
\bibitem{my5} Черный,~С.\,Г. Численная модель течения в гидротурбине с затвором~/ С.\,Г.~Черный, 
              А.\,Ю.~Авдюшенко, Д.\,В.~Чирков // Совместный выпуск журналов <<Вестник 
              Восточно-Казахстанского государственного технического университета им. Д. Серикбаева>> 
              и <<Вычислительные технологии>>.~--- Усть-Каменогорск, Казахстан, 2013.~--- C.~266-274.
%~0.95 усл.-печ. л.
\Rus
\bibitem{my6} Авдюшенко,~А.\,Ю. Метод численного моделирования переходных 
              процессов в гидротурбинах~/ А.\,Ю.~Авдюшенко, С.\,Г.~Черный // 
              Матер. XIII всероссийской конф. молодых ученых по математическому моделированию и информационным 
              технологиям.~--- Новосибирск, 2012.~--- 
              \href{http://conf.nsc.ru/files/conferences/ym2012/fulltext/137843/139439/Large_thesis_Avd.pdf}
              {http://conf.nsc.ru/files/conferences/ym2012/fulltext/137843\linebreak[1]/139439/Large\_thesis\_Avd.pdf}
%~0.93 усл.-печ. л.
\Rus
\bibitem{my7} Авдюшенко,~А.\,Ю. Численное моделирование переходного процесса пуска 
              гидротурбины в турбинный режим~/ А.\,Ю.~Авдюшенко // 
              Матер. XI всероссийской конф. <<Краевые задачи и математическое моделирование>>.~--- 
              Новокузнецк, 2012.~--- С.~7-16.
%~0.52 усл.-печ. л. = 9/16*0.93
\Eng
\bibitem{my8} Chirkov,~D. CFD simulation of pressure and discharge surge in Francis turbine at off-design 
              conditions~/ D.~Chirkov, A.~Avdyushenko, L.~Panov, D.~Bannikov, S.~Cherny, V.~Skorospelov, 
              I.~Pylev // Proc. of 26th IAHR Symposium on hydraulic machinery and systems.~--- 
              Beijing, China, 2012.~---
              \href{http://iopscience.iop.org/1755-1315/15/3/032038/pdf/1755-1315_15_3_032038.pdf}
              {http://iopscience.iop.org/1755-1315/15/3/032038/pdf/1755-1315\_15\_3\_032038.pdf}
%~1.21 усл.-печ. л.
\Rus
\bibitem{my9} Черный,~С.\,Г. Численное решение сопряженных задач <<гидродинамика"=упругость>>~/ 
              С.\,Г.~Черный, Д.\,В.~Чирков, В.\,Н.~Лапин, Д.\,В.~Банников, Д.\,В.~Есипов, Д.\,С.~Куранаков, 
              А.\,Ю.~Авдюшенко, В.\,А.~Скороспелов, П.\,А.~Турук // 
              Матер. XVII междунар. конф. по вычислительной механике и современным прикладным 
              программным системам (ВМСППС’2011).~--- М.:~Изд-во МАИ-ПРИНТ, 
              2011. ~--- C.~169-171.
%~0.25 усл.-печ. л.
\Eng
\bibitem{my10} Cherny,~S. 3D Numerical simulation of transient processes in hydraulic 
               turbines~/ S.~Cherny, D.~Chirkov, D.~Bannikov, V.~Lapin, V.~Skorospelov, 
               I.~Eshkunova, A.~Avdushenko // Proc. of 25th IAHR Symposium on hydraulic machinery and 
               systems.~--- Timisoara, Romania, 2010.~--- 
               \href{http://iopscience.iop.org/1755-1315/12/1/012071/pdf/1755-1315_12_1_012071.pdf}
               {http://iopscience.iop.org/1755-1315/12/1/012071/pdf/1755-1315\_12\_1\_012071.pdf} 
%~1.19 усл.-печ. л.
\Rus
\bibitem{my11} Черный,~С.\,Г. Численное моделирование нестационарных пространственных течений в 
               каналах решеток гидротурбин~/ С.\,Г.~Черный, Д.\,В.~Чирков, В.\,Н.~Лапин, И.\,Ф.~Ешкунова, 
               Д.\,В.~Банников, А.\,Ю.~Авдюшенко, В.\,А.~Скороспелов, П.\,А.~Турук // Матер. VIII 
               междунар. конф. по неравновесным процессам в соплах и струях (NPNJ`10).~--- М: Изд-во МАИ-ПРИНТ, 2010.~--- C.~72-75.
%~0.47 усл.-печ. л.  
\Rus
\bibitem{my12} Черный,~С.\,Г. Программа расчета течений воды или кавитационной смеси воды с паром в проточных 
               частях различных типов гидротурбин в циклической постановке CADRUN/2013~/ С.\,Г.~Черный, 
               Д.\,В.~Чирков, В.\,Н.~Лапин, Л.\,В.~Панов, В.\,А.~Скороспелов, А.\,Ю.~Авдюшенко // 
               Свид. о гос. рег. прогр. для ЭВМ. Рег.~\No~2013611576. Роспатент.
% -.-- усл.-печ. л.
\Rus
\bibitem{my13} Черный,~С.\,Г. Программа расчета течений воды в проточных частях различных типов 
               гидротурбин в полной постановке CADRUN2/2013~/ С.\,Г.~Черный, Д.\,В.~Чирков, В.\,Н.~Лапин, 
               В.\,А.~Скороспелов, А.\,Ю.~Авдюшенко // Свид. о гос. рег. прогр. для ЭВМ. 
               Рег.~\No~2013611580. Роспатент.
% -.-- усл.-печ. л.
\Rus
\bibitem{my14} Авдюшенко,~А.\,Ю. Метод численного моделирования переходных 
               процессов в гидротурбинах~/ А.\,Ю.~Авдюшенко, С.\,Г.~Черный //
               Тез. докл. XIII всеросс. конф. молодых ученых по математическому моделированию и информационным 
               технологиям.~--- Новосибирск, 15-17 октября 2012.~--- С.~14.
%~0.04 усл.-печ. л.
\Rus
\bibitem{my15} Авдюшенко,~А.\,Ю. Численное моделирование переходного процесса пуска гидротурбины 
               в турбинный режим~/ А.\,Ю.~Авдюшенко // 
               Тез. докл. XI всеросс. конф. <<Краевые задачи и математическое моделирование>>.~--- 
               Новокузнецк, 10-11 октября 2012.~--- С.~3. 
%~0.04 усл.-печ. л.
\Rus
\bibitem{my16} Авдюшенко,~А.\,Ю. Некоторые задачи гидродинамики водяных турбин и численные алгоритмы 
               их решения~/ А.\,Ю.~Авдюшенко, С.\,Г.~Черный, Д.\,В.~Чирков //  Тез. докл. межд. 
               конф. <<Современные проблемы прикладной математики и 
               механики: теория, эксперимент и практика>>, посвященной 90-летию со дня 
               рождения Н.\,Н.~Яненко.~--- Новосибирск, 30 мая-4 июня 2011.~--- C.~79. 
% 0.02 усл.-печ. л.
\Rus
\bibitem{my17} Авдюшенко,~А.\,Ю. Модификация метода расчета нестационарных турбулентных течений 
               с отрывами потока~/ А.\,Ю.~Авдюшенко// Материалы XLVII междунар. научной студенческой 
               конф. <<Студент и научно-технический прогресс>>:~Математика. Новосиб. гос. ун-т.~--- 
               Новосибирск, 11-15 апреля 2009.~--- C.~240.           
% 0.02 усл.-печ. л.
\Rus
\bibitem{my18} Авдюшенко,~А.\,Ю. Численное моделирование нестационарного течения в 
               горизонтальной гидротурбине~/ А.\,Ю.~Авдюшенко // Материалы XLVI междунар. научной 
               студенческой конф. <<Студент и научно-технический прогресс>>:~Математика. 
               Новосиб. гос. ун-т.~--- Новосибирск, 26-30 апреля 2008.~--- C.~113.
%~0.02 усл.-печ. л.
\Rus
\bibitem{lomakin} Бедчер,~Ф.\,С. Определение критического числа оборотов ротора насоса с учетом
                  сил, возникающих в уплотнениях~/ Ф.\,С.~Бедчер, А.\,А.~Ломакин // М.-Л.: Паро- и 
                  газотурбостроение. Машгиз.~--- 1957.~--- c.~249--269. (Труды ЛМЗ, вып. 5)
%
%1111111111111111111111111111111111
%  chapter 1
%1111111111111111111111111111111111
% ниже ссылки из первой статьи по подвижным сеткам
\Eng
\bibitem{trulio}Trulio,~J.\,G. Numerical solution of the one-dimensional hydrodynamic equations
                in an arbitrary time-dependent coordinate system~/ J.\,G.~Trulio, K.\,R.~Trigger // 
                Techn. Rep. UCLR-6522. Univ. of California. Lawrence Radiation Laboratory.~--- 1961.
%
\Eng
\bibitem{thomas}Thomas,~P.\,D. Geometric conservation law and its application to flow 
                computations on moving grids~/ P.\,D.~Thomas, C.\,K.~Lombart // AIAA J.~--- 
                1979.~--- Vol.~17, No.~10.~--- P.~1030--1037.
%
\Eng
\bibitem{demper}Demirdzic,~I. Space conservation law in finite volume calculations of fluid 
                flow~/ I.~Demirdzic, M.~Peric // Intern. J. Numer. for Methods in 
                Fluids.~--- 1988.~--- Vol.~8.~--- P.~1037--1050.
%
\Eng
\bibitem{shyy} Shyy,~W. Computational fluid dynamics with moving boundaries~/ W.~Shyy, H.\,S.~Udaykumar, 
               M.\,M.~Rao, R.\,W.~Smith.~--- Washington, DC: Taylor and Francis, 1996.~--- 304~p.
%
\Eng
\bibitem{lesoin}Lesoinne,~M. Geometric conservation laws for flow problems with moving boundaries and
                deformable meshes, and their impact on aerolastic computations~/ M.~Lesoinne, C.~Farhat // 
                Comput. Methods in Appl. Mech. and Eng.~--- 1996.~--- Vol.~134.~--- P.~71--90.
%
\Eng
\bibitem{koobus}Koobus,~B. Second-order time-accurate and geometrically conservative implicit 
                schemes for flow computations on unstructured dynamic meshes~/ B.~Koobus, C.~Farhat // 
                Comput. Methods in Appl. Mech. and Eng.~--- 1999.~--- Vol.~170.~--- P.~103--129.
%
\Eng
\bibitem{forst} Forster,~C. On the geometric conservation law in transient flow calculations on deforming 
                domains~/ C.~Forster, W.\,A.~Wall, E.~Ramm // Intern. J. for Numer. Methods in Fluids.~--- 
                2006.~--- Vol.~50.~--- P.~1369--1379.
%
\Eng
\bibitem{engel} Engel,~M. Flow simulation on moving boundary-fitted grids and application to 
                fluid-structure interaction problems~/ M.~Engel, M.~Griebel // Intern. J. for Numer. 
                Methods in Fluids.~--- 2006.~--- Vol.~50.~--- P.~437--468.
%
\Rus
\bibitem{zaits} Зайцев,~Н.\,О. Применение деформируемых сеток для численного моделирования течений 
                в областях с подвижными границами~/ Н.\,О.~Зайцев, Н.\,А.~Щур // Научно"=техн. ведомости 
                Санкт"=Петербургского гос. ун-та.~--- 2006.~--- Т.~47.~--- С.~15--22.
%
\Rus
\bibitem{volkov}Волков,~К.\,Н. Дискретизация уравнений Навье"=Стокса на подвижных неструктурированных 
                сетках~/ К.\,Н.~Волков // Вычисл. методы и программирование.~--- 2008.~--- Т.~9.~--- 
                С.~256--273.
%
\Eng
\bibitem{sharma} Launder,~B.\,E. Application of energy dissipation model of turbulence to
                 the calculation of flow near a spinning disc~/ B.\,E.~Launder, B.\,I.~Sharma // 
                 Letters in Heat and Mass Transfer.~--- 1974.~--- Vol.~1, No.~2.~--- P.~131--138.
%
\Rus
\bibitem{fiht}  Фихтенгольц,~Г.\,М. Курс дифференциального и интегрального 
                исчисления~/ Г.\,М.~Фихтенгольц.~--- М.: Физматлит, 2001.~--- Т.~2.~--- 810~с.
%
\Eng
\bibitem{RizziErikson} Rizzi,~A. Computation of flow around wings based on the Euler
                       equations~/ A.~Rizzi, L."=E.~Eriksson // J. Fluid. Mech.~--- 1984.~--- 
                       Vol.~148.~--- P.~45--71.
%
\Eng
\bibitem{anders}Anderson,~W.\,K. Comparison of finite volume flux vector splittings 
                for the euler equations~/ W.\,K.~Anderson, J.\,L.~Thomas, B.~van Leer // AIAA J.~--- 1986.~--- Vol.~24, No.~9.~--- P.~1453--1460.
%
\Eng
\bibitem{takami}Takami,~H. Steady two-dimensional viscous flow of an incompressible fluid past a 
                circular cylinder~/ H.~Takami, H.\,B.~Keller // Phys. Fluids Suppl.~--- 1969.~--- 
                Vol.~12.~--- P.~11--51.
%
\Eng
\bibitem{dennis}Dennis,~S. Numerical solutions for steady flow past a circular cylinder at 
                Reynolds numbers up to 100~/ S.~Dennis, G.Z.~Chang // J. of Fluid Mech.~--- 
                1970.~--- Vol.~42.~--- P.~471--493.
%
\Rus
\bibitem{belocirk} Белоцерсковский,~О.М. Численное моделирование в механике сплошных сред: 2-е изд., 
                   перераб. и доп~/ О.М.~Белоцерсковский.~--- М.: Физматлит, 1994.~--- 448~с.
%
\Eng
\bibitem{tuan} Tuann,~S.\,Y. Numerical studies of the flow around a circular cylinder by a 
               finite element method~/ S.\,Y.~Tuann, M.\,D.~Olson // 
               Computers and Fluids.~--- 1978.~--- Vol.~6.~--- P.~219--234.
%
\Eng
\bibitem{braza} Braza,~M. Numerical study and physical analysis of the pressure
                and velocity fields in the near wake of a circular cylinder~/ M.~Braza, P.~Chassaing, 
                H.~Minh // J. of Fluid Mech.~--- 1986.~--- Vol.~8.~--- P.~79--98.
%
\Eng
\bibitem{coutan}Coutanceau,~M. Experimental determination of the main features of the viscous flow 
                in the wake of a circular cylinder in uniform translation. Part 1. Steady flow~/ 
                M.~Coutanceau, R.~Bernard // J. of Fluid Mech.~--- 1977.~--- Vol.~79.~--- P.~231--254.
%
\Eng
\bibitem{tritt} Tritton,~D.\,J. Experiments on the flow past a circular cylinder at low Reynolds numbers~/ 
                D.\,J.~Tritton // J. of Fluid Mech.~--- 1959.~--- Vol.~6.~--- P.~547--567.
%
\Rus
\bibitem{universe}Ешкунова,~И.\,Ф. Ускорение сходимости решения нестационарных 
                  задач динамики несжимаемой жидкости~/ И.\,Ф.~Ешкунова, С.\,Г.~Черный, Д.\,В.~Чирков // 
                  Вычислительные технологии.~--- 2011.~--- Т.~16, \No~5.~--- C.~30--49.
%
\Eng
\bibitem{amdal} Amdahl,~G. Validity of the single-processor approach to achieving 
                large-scale computing capabilities~/ G.~Amdahl // 
                Proc. of AFIPS Conf., AFIPS Press.~--- 1967.~--- V.~30.~--- P.~483--485.                
%
\Eng
\bibitem{MPI} MPI: A Message-Passing Interface Standart. Version 3.0.~--- University 
              of Tennessee, Knoxville, Tennessee.~--- 2012.~--- 852~p.                
%
%2222222222222222222222222222222222
%  chapter 2
%2222222222222222222222222222222222
% ниже ссылки из второй статьи по переходным процессам
\Eng
\bibitem{Vunen2}Vu,~T.\,C. Experimental study and unsteady simulation of the FLINDT draft tube rotating 
                vortex rope~/ T.\,C.~Vu, B.~Nennemann, G.\,D.~Ciocan, M.\,S.~Iliescu, O.~Braun, F.~Avellan // 
                Proceedings of the Hydro 2004 Conference.~--- Porto, Portugal, FE-05-1175.
%
\Eng
\bibitem{Vu}    Vu,~T.\,C. Accuracy assessment of current CFD tools to predict hydraulic turbine 
                efficiency hill chart~/ T.\,C.~Vu, S.~Retieb // Proceedings of the XXI IAHR Symposium on 
                Hydraulic Machinery and Systems.~--- 2002.~--- Lausanne, Switzerland.
%
\Rus
\bibitem{jukovskii} Жуковский~Н.\,Е. О гидравлическом ударе в водопроводных трубах~/ Н.\,Е.~Жуковский.~--- 
                    Л.: Гостехиздат, 1949. ~--- 103~с.
%
\Rus
\bibitem{bannikov} Банников,~Д.\,В. Оптимизационное проектирование ротора гидротурбины по критериям 
                   эффективность"=прочность~/ Д.\,В.~Банников, Д.\,В.~Есипов, С.\,Г.~Черный, Д.\,В.~Чирков // 
                   Теплофизика и аэромеханика.~--- 2010.~--- Т.~17, \No~4.~--- C.~651--658.
%
\Eng
\bibitem{kimchen} Chen,~Y.\,S. Computation of turbulent flows using an extended k-$\varepsilon$
                  turbulence closure model~/ Y.\,S.~Chen, S.\,W.~Kim.~--- NASA CR-179204.~--- 1987.
%
\Eng
\bibitem{nicolet} Nicolet,~C. Hydroacoustic modelling and numerical simulation of unsteady operation of 
                  hydroelectric systems: Ph. D. Thesis~/ Nicolet Christophe.~--- 2007.~--- EPFL No~3751.~--- 
                  314~p.
%
\Rus
\bibitem{AntKajMon} Антонцев,~С.\,Н. Краевые задачи механики неоднородных жидкостей~/ С.\,Н.~Антонцев, 
                    А.\,В.~Кажихов, В.\,Н.~Монахов.~--- Новосибирск: Наука.~--- 1983.~--- 320~с.
%
\Rus
\bibitem{topaj} Топаж,~Г.\,И. Расчет интегральных гидравлических показателей гидромашин~/ Г.\,И.~Топаж.~--- 
                Л: Издательство Ленинградского университета.~--- 1989.~--- 208~с.
%
%3333333333333333333333333333333333
%  chapter 3
%3333333333333333333333333333333333
% ниже ссылки из третьей статьи по лабиринтам
\Rus
\bibitem{mak_pilev} Макаров,~В.\,В. Объемные и дисковые потери в радиально-осевых 
                    гидротурбинах~/ В.\,В.~Макаров, И.\,М.~Пылев, В.\,И.~Пьянов // Л.: 
                    Энергомашиностроение.~--- 1982.~--- \No~1.~--- С.~11--15.
%
\Rus
\bibitem{kuzminsk} Кузьминский,~С.\,С. Методические вопросы определения давлений и осевой силы в 
                   радиально"=осевых гидротурбинах~/ С.\,С.~Кузьминский, И.\,И.~Федорова, И.\,М.~Пылев 
                   // Л.: Энергомашиностроение.~--- 1979.~--- \No~3.~--- С.~6--9.
%
\Eng
\bibitem{staubli}  Staubli,~T. Parameters to adjust axial thrust~/ T.~Staubli, M.~Sallaberger, F.~Senn // 
                   Proceeding of internatioanl conference HYDRO2006. Porto Carras, Greece, 2006.
%
\Eng
\bibitem{roy_vu} Le Roy,~V. Validation of CFD model for hydraulic seals~/ V.~Le Roy, F.~Guibault, T.~Vu // 
                 International Journal of Fluid Machinery and Systems.~--- 2009.~--- Vol.~2, No.~4.~--- 
                 P.~400--408.
%
\Eng
\bibitem{xi_rhode} Xi,~J. Rotordynamics of Turbine Labyrinth Seals with Rotor Axial 
                   Shifting~/ J.~Xi, D.\,L.~Rhode // International Journal of Rotating 
                   Machinery.~--- 2006.~--- Vol.~2006.~--- P.~1--11.
%
\Rus
\bibitem{idelchik} Идельчик,~И.\,Е. Справочник по гидравлическим сопротивлениям/ И.\,Е.~Идельчик.~--- 
                   М.: Машиностроение, 1992.~--- 672~с.
%
\Rus
\bibitem{marcinkovskii} Марцинковский,~В.\,А. Гидродинамика и прочность центробежных насосов~/ 
                        В.\,А.~Марцинковский.~--- М.: Машиностроение, 1970.~--- 296~c.
%
\Rus
\bibitem{makar} Макаров,~В.\,В. Исследование гидравлических сопротивлений вращающихся разгрузочных отверстий 
                и разработка рекомендаций по их определению/ В.\,В.~Макаров.~--- Тех. информация по этапу 13 
                хоз. договора \No~8123/3061.~--- Л., 1984.~--- 29~с. 
%
%4444444444444444444444444444444444
%  chapter 4
%4444444444444444444444444444444444
% ниже ссылки на литературу по кольцевым затворам
\Rus
\bibitem{etinberg} Этинберг,~И.\,Э. Гидродинамика гидравлических турбин~/ И.\,Э.~Этинберг, Б.\,С.~Раухман.~--- 
                   Л.: Машиностроение (Ленингр. отд-ние).~--- 1978.~--- 280~с.
%

\end{thebibliography}

  %!TEX root = dissertation.tex
\newpage
\appendix
% \chapter*{Приложение А Вспомогательные матрицы}

\begin{flushleft}
  \LARGE
  \textbf{Приложение A \\ Вспомогательные матрицы}
\end{flushleft}
\label{s:A}
\setcounter{chapter}{1}
\addcontentsline{toc}{chapter}{Приложение~\thechapter~ Вспомогательные матрицы}

\section{Матрица Якоби невязкого потока и её разложение}
Матрица Якоби невязкого потока имеет вид
\begin{gather}
  {\bf{A}}^t ({\bf{Q}}) = \frac{{\partial ({\bf{K}}_{\beta,\, inv}^t ({\bf{Q}}) \cdot {\bf{S}})}}{{\partial
  {\bf{Q}}}} \!= \notag \\ =\! \left[ {\begin{array}{*{20}c}
   0 \!&\! {\beta S_x } \!&\! {\beta S_y } \!&\! {\beta S_z }  \\
   {S_x } \!&\! {w_1 S_x  + U_{}  - U_g } \!&\! {w_1 S_y } \!&\! {w_1 S_z }  \\
   {S_y } \!&\! {w_2 S_x } \!&\! {w_2 S_y  + U_{}  - U_g } \!&\! {w_2 S_z }  \\
   {S_z } \!&\! {w_3 S_x } \!&\! {w_3 S_y } \!&\! {w_3 S_z  + U_{}  - U_g }  \\
  \end{array}} \right],
  \label{A1} 
\end{gather}
где $U = {\bf{w}} \cdot {\bf{S}},\ U_g  = {\bf{x}}_t  \cdot {\bf{S}}$. 

Собственные значения матрицы ${\bf{A}}^t$:
\begin{gather}
  \label{A2}
  \lambda _{1,2} = U_{} - U_g , \quad \notag \\
  \lambda _{3} = U_{} - \frac{1}{2}U_g + c, \\
  \lambda _{4} = U_{} - \frac{1}{2}U_g - c. \notag \\
  {\bf{D}} = Diag(\lambda_{1},\lambda_{2},\lambda_{3},\lambda_{4}),
\end{gather}
где $c = \sqrt {\left({U_{} -\cfrac{1}{2}U_g} \right)^2 + \beta \left( {S_x^2  + S_y^2  + S_z^2 } \right)}$.
\begin{equation}
  {\bf{A}}^t = {\bf{R}}\cdot{\bf{D}}\cdot{\bf{L}}.
\end{equation}

Матрица правых собственных векторов ${\bf{R}}$ имеет вид:
\begin{equation}
  {\bf{R}} = \left( {\begin{array}{*{20}c}
   0 & 0 & {\beta \left( {c + U_g /2} \right)} & { - \beta \left( {c - U_g /2} \right)}  \\
   { - S_z } & { - S_y } & {u\lambda _3  + \beta S_x } & {u\lambda _4  + \beta S_x }  \\
   0 & {S_x } & {v\lambda _3  + \beta S_y } & {v\lambda _4  + \beta S_y }  \\
   {S_x } & 0 & {w\lambda _3  + \beta S_z } & {w\lambda _4  + \beta S_z }  \\
  \end{array}} \right).
\end{equation}

Матрица левых собственных векторов ${\bf{L}}={\bf{R}}^{-1}$, т.е. 
\begin{equation}
  {\bf{R}}\cdot{\bf{L}}={\bf{I}}.
\end{equation}

\section{Матрица Якоби вязкого потока}
Вектор вязкого потока ${\bf{K}}_{\beta,\, vis} \cdot {\bf{S}}$ запишем в виде суммы
\begin{equation}
  {\bf{K}}_{\beta,\, vis} \cdot {\bf{S}} = 
  -\nu _{\text{eff}}{\bf{G}}({\bf{Q}}_{\xi},{\bf{Q}}_{\eta},{\bf{Q}}_{\zeta}) = 
  -\nu _{\text{eff}}\left[{\bf{G}}^1({\bf{Q}}_{\xi})+{\bf{G}}^2({\bf{Q}}_{\eta})+
  {\bf{G}}^3({\bf{Q}}_{\zeta})\right], 
\end{equation}
где
\begin{gather*}
  {\bf{Q}}_{\xi}=
  \dfrac{\partial {\bf{Q}}}{\partial \xi},\ {\bf{Q}}_{\eta} =
  \dfrac{\partial {\bf{Q}}}{\partial \eta},\ {\bf{Q}}_{\zeta} =
  \dfrac{\partial {\bf{Q}}}{\partial \zeta}, \notag \\ 
  {\bf{G}}^1({\bf{Q}}_{\xi})\!=\!\left( \begin{array}{c}
   0  \\
   \left(\omega_1 +\dfrac{\partial \xi}{\partial x}S_x \right)\dfrac{\partial u}{\partial \xi} +
   \dfrac{\partial \xi}{\partial x}S_y\dfrac{\partial v}{\partial \xi} + 
   \dfrac{\partial \xi}{\partial x}S_z\dfrac{\partial w}{\partial \xi} \\
   \dfrac{\partial \xi}{\partial y}S_x\dfrac{\partial u}{\partial \xi} + 
   \left(\omega_1 +\dfrac{\partial \xi}{\partial y}S_y \right)\dfrac{\partial v}{\partial \xi} +
   \dfrac{\partial \xi}{\partial y}S_z\dfrac{\partial w}{\partial \xi} \\
   \dfrac{\partial \xi}{\partial z}S_x\dfrac{\partial u}{\partial \xi} + 
   \dfrac{\partial \xi}{\partial z}S_y\dfrac{\partial v}{\partial \xi} +
   \left(\omega_1 +\dfrac{\partial \xi}{\partial z}S_z \right)\dfrac{\partial w}{\partial \xi} \\
  \end{array} \right), \notag \\  
  {\bf{G}}^2({\bf{Q}}_{\eta})\!=\!\left( \begin{array}{c}
   0  \\
   \left(\omega_2 +\dfrac{\partial \eta}{\partial x}S_x \right)\dfrac{\partial u}{\partial \eta} +
   \dfrac{\partial \eta}{\partial x}S_y\dfrac{\partial v}{\partial \eta} + 
   \dfrac{\partial \eta}{\partial x}S_z\dfrac{\partial w}{\partial \eta} \\
   \dfrac{\partial \eta}{\partial y}S_x\dfrac{\partial u}{\partial \eta} + 
   \left(\omega_2 +\dfrac{\partial \eta}{\partial y}S_y \right)\dfrac{\partial v}{\partial \eta} +
   \dfrac{\partial \eta}{\partial y}S_z\dfrac{\partial w}{\partial \eta} \\
   \dfrac{\partial \eta}{\partial z}S_x\dfrac{\partial u}{\partial \eta} + 
   \dfrac{\partial \eta}{\partial z}S_y\dfrac{\partial v}{\partial \eta} +
   \left(\omega_2 +\dfrac{\partial \eta}{\partial z}S_z \right)\dfrac{\partial w}{\partial \eta} \\
  \end{array} \right), \\
  {\bf{G}}^3({\bf{Q}}_{\zeta})\!=\!\left( \begin{array}{c}
   0  \\
   \left(\omega_3 +\dfrac{\partial \zeta}{\partial x}S_x \right)\dfrac{\partial u}{\partial \zeta} +
   \dfrac{\partial \zeta}{\partial x}S_y\dfrac{\partial v}{\partial \zeta} + 
   \dfrac{\partial \zeta}{\partial x}S_z\dfrac{\partial w}{\partial \zeta} \\
   \dfrac{\partial \zeta}{\partial y}S_x\dfrac{\partial u}{\partial \zeta} + 
   \left(\omega_3 +\dfrac{\partial \zeta}{\partial y}S_y \right)\dfrac{\partial v}{\partial \zeta} +
   \dfrac{\partial \zeta}{\partial y}S_z\dfrac{\partial w}{\partial \zeta} \\
   \dfrac{\partial \zeta}{\partial z}S_x\dfrac{\partial u}{\partial \zeta} + 
   \dfrac{\partial \zeta}{\partial z}S_y\dfrac{\partial v}{\partial \zeta} +
   \left(\omega_3 +\dfrac{\partial \zeta}{\partial z}S_z \right)\dfrac{\partial w}{\partial \zeta} \\
  \end{array} \right). 
\end{gather*}
Далее находим матрицы Якоби выписанных составляющих вектора вязкого потока
\begin{equation}
  {\bf{R}}^k =\dfrac{\partial {\bf{G}}^k}{\partial {\bf{Q}}_\gamma} = \left( {\begin{array}{*{20}c}
  0 & 0 & 0 & 0 \\
  0 & \omega_k+\dfrac{\partial \gamma}{\partial x}S_x & \dfrac{\partial \gamma}{\partial x}S_y & 
      \dfrac{\partial \gamma}{\partial x}S_z \\
  0 & \dfrac{\partial \gamma}{\partial y}S_x & \omega_k+\dfrac{\partial \gamma}{\partial y}S_y & 
      \dfrac{\partial \gamma}{\partial y}S_z\\
  0 & \dfrac{\partial \gamma}{\partial z}S_x & \dfrac{\partial \gamma}{\partial z}S_y &  
      \omega_k+\dfrac{\partial \gamma}{\partial z}S_z 
  \end{array}} \right),
\end{equation}
где $\gamma = \xi$ для $k=1$, $\gamma = \eta$ для $k=2$ и $\gamma = \zeta$ для $k=3$.

  %!TEX root = dissertation.tex
 % \chapter*{Приложение B}
\vspace{20mm}

\begin{flushleft}
  \LARGE
  \textbf{Приложение В \\ Свидетельства о государственной регистрации программ и акт об использовании}
\end{flushleft}

На следующих страницах приведены два свидетельства о государственной регистрации программ для ЭВМ и акт об 
использовании в филиале ОАО «Силовые машины» «ЛМЗ» в г.~Санкт-Петербурге научных результатов диссертационной 
работы.
\setcounter{chapter}{2}
\addcontentsline{toc}{chapter}{Приложение~\thechapter~ Свидетельства о государственной регистрации программ}

%\newpage
%\centering
%\includegraphics[width=18.0cm]{Cadrun1.png}
%\pagestyle{empty}
%
%\includegraphics[width=18.0cm]{Cadrun2.png}
%
%\includegraphics[width=18.0cm]{Akt.png}
%

\end{document}
