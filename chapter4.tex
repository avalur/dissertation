%!TEX root = dissertation.tex
% \pagestyle{plain}
\chapter*{Глава 4. Численная модель течения в гидротурбине с кольцевым затвором}
\label{s:4}
\setcounter{chapter}{4}
\addcontentsline{toc}{chapter}{Глава~\thechapter~ Численная модель течения в гидротурбине с кольцевым затвором}
\setcounter{section}{0}

\section{Общая постановка задачи}
\label{s:41}
В диссертации разработана численная модель и на ее основе осуществлено моделирование течений в гидротурбине 
либо при выдвинутом фиксированном положении затвора, либо в процессе его выдвижения. Проведен анализ 
осевых и радиальных сил, действующих на затвор.

Объектом исследований этой главы является проточная часть насос-турбины с кольцевым затвором и течение воды в 
ней. На рисунке~\ref{fig:1},~\emph{а} изображена расчетная область, состоящая из спиральной камеры с 22 
статорными колоннами и т.н. <<зубом>> (рисунок~\ref{fig:55}), направляющего аппарата с 24 лопатками и 
рабочего колеса, имеющего 8 лопастей. Там же приведены расчетные сетки в блоках СК, НА и РК, стрелкой 
показано направление потока воды. На рисунке~\ref{fig:1},~\emph{б} приведено меридиональное сечение расчетной 
области. Изображено положение затвора, выдвинутого в проточный тракт на 0.85 от высоты $b_0$ НА. 
Величину 
\begin{equation}
  s=\dfrac{b}{b_0},
\end{equation}
где $b$ -- высота выдвинутой части затвора, будем называть степенью закрытия.   
\begin{figure}[ht]
  \centering \small \rule{20mm}{0mm}\emph{a}\rule{90mm}{0mm}\emph{б}\\[1.5mm]
  {\includegraphics[width=10.5cm]{rasch_oblast_coarse_grid.png}}\hfill
  {\includegraphics[width=6.5cm]{mer_slide.png}}\\  
  \caption{Проточная часть насос-турбины с кольцевым затвором: \emph{а}) расчетная область с сетками 
           в трех блоках; \emph{б}) меридиональное сечение расчетной области}
  \label{fig:1}
\end{figure}

Наличие затвора, перекрывающего ПТ ГТ, можно учесть посредством задания условия твердой стенки на слое сетки, 
разделяющем статорные колонны и НА. При этом требуется минимальная модификация программных модулей, но не 
учитывается форма оголовка затвора и изменение геометрии проточной части. Вторым вариантом является добавление
нового элемента проточной части -- области под затвором, при этом геометрия проточного тракта моделируется 
точно. В работе предложены две основные модели: модель бесконечно тонкого затвора и 
модель затвора конечной реальной толщины. Каждая из этих моделей рассмотрена как в стационарной, так и в 
нестационарной постановке. При этом нестационарные расчеты проведены как при фиксированном, так и при 
движущемся затворе. 

Для верификации процедур построения сеток в добавленной области под оголовком затвора и процедур передачи 
данных из нее в каналы статора и направляющего аппарата проведены расчеты течений в области, включающей один 
канал статора, один канал направляющего аппарата и один канал рабочего колеса (циклическая постановка). 
В расчетах менялось открытие направляющего аппарата, а также изменялась степень закрытия затвора от 0 до 99~\%.

После этого была проведена адаптация программного комплекса CADRUN2 к задаче моделирования процесса 
закрытия затвора в полной постановке, описывающей течение в спиральной камере, во всех каналах статора, 
направляющего аппарата и рабочего колеса. В приближении уравнений Рейнольдса, замкнутых 
стандартной $k-\varepsilon$ моделью турбулентности, рассчитано течение для одного угла открытия направляющего 
аппарата и семи положений затвора. 
Исследовано влияние степени закрытия затвора на крутящие моменты лопаток направляющего аппарата и 
выталкивающую осевую силу $F_z$, действующую на оголовок затвора. Получены оценки 
радиальных сил $(F_x, F_y)$, действующих на боковую поверхность затвора, и других эффектов, связанных с 
неравномерностью потока в спиральной камере и ротор"=статор взаимодействием. 

\section{Модели затвора}
\label{s:42}
\subsection{Модель бесконечно тонкого затвора}
\label{s:421}
На основании  предоставленных филиалом <<Силовые машины>> -- ЛМЗ материалов, отражающих опыт 
фирмы <<Нейрпик>>, толщина кольцевого затвора составляет около 2\% от диаметра РК и около 10\% от высоты НА. 
Учитывая малую толщину затвора по сравнению с высотой НА, можно считать, что 
хорошим приближением для моделирования процесса закрытия является модель тонкого затвора. В этом случае 
кольцевой затвор считается бесконечно тонкой цилиндрической поверхностью постоянного радиуса, расположенной 
за выходной кромкой статорных колонн (рисунок~\ref{fig:2}). На ней ставится условие 
прилипания к твердой стенке. 
\begin{figure}[!h]
  \centering \includegraphics[width=12cm]{tonk_zatvor_cycle.png}\\  
  \caption{Расчетная область в циклической постановке для модели бесконечно тонкого затвора: $s=0.5$}
  \label{fig:2}
\end{figure}

\subsection{Модель затвора реальной толщины}
\label{s:422}
Также в работе рассмотрена модель затвора реальной толщины. В этом случае кольцевой затвор -- цилиндр с 
заданной формой оголовка (рисунок~\ref{fig:3}). 
Для реализации данной модели в программный комплекс введен новый 
элемент проточной части -- кольцевая область между оголовком затвора и нижней крышкой НА. 
Форма оголовка и сетка в ее окрестности приведены на рисунке~\ref{fig:31}. Также в программный комплекс 
добавлены новые процедуры построения сеток в области под затвором и передачи данных из нее 
в каналы статора и НА.
\begin{figure}[!hb]
  \centering \includegraphics[width=12cm]{tolst_zatvor_cycle.png}\\  
  \caption{Расчетная область в циклической постановке для модели затвора конечной толщины: $s=0.5$}
  \label{fig:3}
\end{figure}
\begin{figure}[!hb]
  \centering \includegraphics[width=12cm]{tolst_zatvor_mer_1.png}\\  
  \caption{Меридиональная проекция сетки в спиральной камере, области под затвором и направляющем 
           аппарате: $s=0.5$}
  \label{fig:31}
\end{figure}

\section{Входные и выходные условия}
\label{s:43}
На входе в область статора в циклической постановке задается полная энергия потока
\begin{equation*}
  E_{in} = H-\Delta h_{CK},
\end{equation*}
где $H$ -- напор на турбине, $\Delta h_{CK}$ -- потери энергии в спиральной камере. 
Предполагалось, что $\Delta h_{CK}=0.01H$, поэтому на входе в статор задается полная энергия 
потока $E_{in} = 0.99H$ и угол входа потока $\delta  = 30^\circ$. На выходе из РК держится полная 
энергия $E_{out} = \Delta h_{OT}$, 
где $\Delta h_{OT}$ -- потери энергии в отсасывающей трубе. Они оцениваются по инженерно-эмпирической формуле 
\cite{idelchik,etinberg}
\begin{equation*}
  \Delta h_{OT} = \xi_0 \frac{{\bar c_z^2}}{{2g}}+\frac{{\bar c_u^2 }}{{2g}},
\end{equation*}
где $\xi _0 \, = 0.15$ -- коэффициент потерь на трение в ОТ, $\bar c_u ,\,\bar c_z $ -- компоненты абсолютной 
скорости на выходе из рабочего колеса, усредненные по расходу (меняются в процессе расчета). 

Кроме этого, на выходе из РК реализуется условие радиального равновесия на профиль давления 
\begin{equation*}
  \frac{dp}{dr}=\frac{c_u^2}{r},
\end{equation*}
где $c_u$ -- окружная компонента скорости. Такие краевые условия позволяют 
определять расход $Q$ в процессе расчета течения.

В отличие от расчетов в циклической постановке, расчеты в полной постановке проведены с использованием 
традиционных граничных условий «расход на входе -- давление на выходе». На входе в спиральную камеру 
фиксировалось значение расхода $Q$, полученное для данного закрытия тонкого затвора в циклической постановке. 
На выходе из области РК задавалось постоянное давление $p = 0$.

\section{Силы, действующие на кольцевой затвор}
\label{s:44}
\subsection{Выталкивающая сила, действующая на оголовок}
\label{s:441}
Выталкивающая сила $F_z$, действующая со стороны жидкости на оголовок затвора, в модели тонкого затвора 
оценивалась по формуле
\begin{equation}
  F_z = R_0 d\int\limits_0^{2\pi}{p(\varphi)d\varphi}, 
  \label{eq:5}
\end{equation}
где $R_0$=0.71435~м -- внешний радиус реального затвора, $d$=0.03174~м -- толщина реального затвора, 
$p(\varphi)$ -- окружное распределение давления в потоке сразу под кончиком затвора. 

В случае затвора реальной толщины выталкивающая сила рассчитывается по формуле
\begin{equation}
  F_z = \int\limits_S p n_z dS,
  \label{eq:6}
\end{equation}
где $p$ -- давление воды, $S$ -- поверхность оголовка, $n_z$ -- $z$-я компонента вектора внутренней единичной 
нормали к поверхности оголовка затвора.
\begin{figure}[!hb]
  \centering \small \emph{a}\rule{90mm}{0mm}\emph{б}\\[1.5mm]
  {\includegraphics[width=8.5cm]{Q11_cycle.png}}\hfill
  {\includegraphics[width=8.5cm]{Fz_11_cycle.png}}\\  
  \caption{Зависимости приведенных расхода $Q_{11}$ (\emph{а}) и силы $F_{z,\, 11}$ (\emph{б}) от $s$ для
           различных открытий НА: $\blacktriangle$~~$\alpha_0=12^\circ$; $\bullet$~~$\alpha_0=26.5^\circ$; 
  $\blacksquare$~~$\alpha_0=40^\circ$. --~--~тонкий~затвор; $\genfrac{}{}{1pt}{2}{\quad}{  }$ затвор 
  реальной толщины}
  \label{fig:4}
\end{figure}

\subsection{Радиальная сила, действующая на боковую поверхность затвора}
\label{s:442}
Также представляет интерес радиальная сила $(F_x,F_y)$, действующая на боковую поверхность затвора. Она 
возникает вследствие окружной неравномерности потока в спиральной камере. Компоненты этой силы рассчитывались 
по формулам
\begin{gather}
  F_x = \int\limits_{S_1} p n_x dS + \int\limits_{S_2}p n_x dS + \int\limits_{S_3} p n_x dS,
  \label{eq:7} \\
  F_y = \int\limits_{S_1} p n_y dS + \int\limits_{S_2}p n_y dS + \int\limits_{S_3} p n_y dS,
  \label{eq:8}
\end{gather}
где $p$ -- давление жидкости, $S_1$ -- внешняя поверхность затвора (со стороны статора), $S_2$ -- внутренняя 
поверхность затвора (со стороны НА), $S_3$ -- поверхность оголовка, $(n_x,n_y)$ -- компоненты вектора 
внутренней единичной нормали к поверхности затвора.

\section{Результаты моделирования течений в гидротурбине с кольцевым затвором}
\label{s:45}
\subsection{Стационарная постановка}
\label{s:451}
Рассмотрены три угла открытия лопаток НА $\alpha_0=12^\circ,\ 26.5^\circ,\ 40^\circ$ и десять степеней 
закрытия кольцевого затвора $s$ от 0.00 до 0.95. На рисунке~\ref{fig:4} показано сравнение рассчитанных 
приведенных к $H=1$~м, $D_1=1$~м расхода $Q_{11}=\dfrac{Q}{D_1^2\sqrt{H}}$ и 
силы $ F_{z,\, 11}=\dfrac{F_z}{D^2_1 H}$ для двух рассмотренных моделей затвора. Видно заметное отличие 
расхода, полученного в двух постановках, уже при $s\geq 0.5$. В модели затвора реальной толщины расход выше. 
Отметим, что при $s\geq 0.9$ в этой модели выталкивающая сила становится отрицательной, т.е. затвор 
втягивается в проточную часть. 
\subsection{Нестационарная постановка}
\label{s:452}
В нестационарной постановке проведены расчеты с движущимся тонким затвором (циклическая постановка), также в 
модели затвора реальной толщины рассмотрены три степени закрытия кольцевого затвора $s=0.5;\ 0.7;\ 0.9$ 
(полная постановка, угол открытия НА $\alpha_0=26.5^\circ$). 
\subsubsection{Расход жидкости}
\label{s:4521}
На рисунке~\ref{fig:5} представлено сравнение 
зависимостей расхода $Q_{11}$ от степени закрытия, полученных в различных постановках.
\begin{figure}[htb]
  \includegraphics[width=9cm]{Q11_all.png} \hfill\includegraphics[width=8cm]{vane_numbers_3_ink.png}\\
  \parbox{9cm} {\caption{Зависимости приведенного расхода $Q_{11}$ от $s$ для открытия НА  
  $\alpha_0=26.5^\circ$, циклическая постановка: $\bullet$~фиксированные положения тонкого затвора; 
  \textcolor{blue}{$\genfrac{}{}{1pt}{2}{\quad}{  }$}~тонкий движущийся затвор; $\circ$~затвор реальной 
  толщины}
  \label{fig:5}}\hfill
  \parbox{8cm} {\caption{Взаимное положение лопаток направляющего аппарата и колонн статора. Приведена
                         нумерация лопаток НА}\label{fig:55}}
\end{figure}
\subsubsection{Моменты лопаток направляющего аппарата}
\label{s:4522}
На рисунке~\ref{fig:6} приведены моменты на лопатках НА, рассчитанные в различных постановках с фиксированными 
положениями затвора. Выбросы момента на кривой, полученной в полной стационарной постановке 
при $s=0.5$ ($\square$) находятся напротив лопастей РК. Максимальное значение крутящего момента получается на 
лопатке \No~18, находящейся вблизи <<зуба>> СК (рисунок~\ref{fig:55}). Необходимо отметить, что столь сильное 
влияние РК вызвано особенностями стационарной постановки: в процессе установления взаимное положение НА и РК 
остается неизменным. В действительности ротор-статор взаимодействие является более слабым, о чем 
свидетельствуют результаты расчетов в полной нестационарной 
постановке. Видно, 
что при больших закрытиях затвора ($s=0.9$) исчезает влияние «зуба» спирали на моменты лопаток НА. 
\begin{figure}[h!]
  \centering \small \emph{a}\rule{90mm}{0mm}\emph{б}\\[0.3mm]
  {\includegraphics[width=8.0cm]{M_HA_05.png}}\hfill
  {\includegraphics[width=8.0cm]{M_HA_09.png}}\\[0.3mm]  
  \caption{Гидравлический момент на лопатках направляющего аппарата: \emph{a}) $s=0.5$; \emph{б}) $s=0.9$.
  \textcolor{blue}{$\bullet$} -- циклическая постановка, тонкий затвор; 
  \textcolor{blue}{$\blacktriangle$} -- циклическая постановка, затвор реальной толщины; 
  $\square$ -- полная стационарная постановка, затвор реальной толщины; 
  (\textcolor{red}{|}) -- амплитуда колебаний, полная нестационарная постановка, затвор реальной толщины}
  \label{fig:6} \vspace{-5mm}
\end{figure}
\subsubsection{Пульсации сил, действующих на затвор}
\label{s:4523}
Скорость вращения РК $n'_1=83.4$~об/мин дает частоту вращения РК $f_R=\dfrac{n'_1}{60}$~Гц. Тогда частота 
прохождения одной лопасти $f_1=8f_R=11.12$~Гц. Также из"=за вращения РК возникает частота прохождения лопатки 
НА мимо фиксированной точки в РК, которая равна $f_3=24f_R=33.36$~Гц. В пульсациях сил для $s=0.5$ 
(рисунок~\ref{fig:62}) и $s=0.7$ (рисунок~\ref{fig:63}) явно выделяется частота прохождения 
лопасти РК $f_1$~Гц. Для закрытия $s=0.7$ наблюдаются низкочастотные пульсации с периодом 1.1~с, изображенные 
на рисунке~\ref{fig:63}. В колебаниях компонент силы для степени закрытия $s=0.9$ нет выделенных частот.

На рисунке~\ref{fig:7} показано сравнение осевых сил для всех проведенных расчетов. Приближение тонкого затвора 
дает завышенное значение силы $F_{z,\, 11}$, в особенности при больших значениях $s$. Видно, что в диапазоне 
степеней закрытия $0\leq s\leq 0.8$ все постановки с затвором реальной толщины дают очень близкие значения 
$F_{z,\, 11}$. Отличие составляет $s=0.9$, где полные стационарная и нестационарная постановки дают завышенное 
значение силы  $F_{z,\, 11}$. Из рисунка~\ref{fig:8} видно, что по мере закрытия затвора амплитуда пульсаций 
радиальных сил $F_x$ и $F_y$ возрастает. \vspace{-5mm}
\begin{figure}[h!]
  \centering \small \emph{a}\rule{90mm}{0mm}\emph{б}\\[0.5mm]
  {\includegraphics[width=8.5cm]{FxFy_s07_1_n.png}}\hfill
  {\includegraphics[width=8.5cm]{Fz_s07_1.png}}\\  
  \caption{Зависимости компонент силы, действующей на затвор, от времени при $s=0.7$}
  \label{fig:63}
\end{figure}

\begin{figure}[!p]
  \centering \small \emph{a}\rule{90mm}{0mm}\emph{б}\\[0.5mm]
  {\includegraphics[width=8cm]{Fx_s05.png}}\hfill
  {\includegraphics[width=8cm]{Fx_s05_fourie.png}}\\  
  {\includegraphics[width=8cm]{Fy_s05.png}}\hfill
  {\includegraphics[width=8cm]{Fy_s05_fourie.png}}\\  
  {\includegraphics[width=8cm]{Fz_s05.png}}\hfill
  {\includegraphics[width=8cm]{Fz_s05_fourie.png}}\\  
  \caption{Пульсации компонент силы, действующей на затвор, (слева) и их спектр (справа) при закрытии 
          затвора $s=0.5$: на спектрах цифрами отмечены частоты $f_1=11.075$~Гц, $f_2=22.144$~Гц, 
          $f_3=33.214$~Гц, $f_4=44.090$~Гц}
  \label{fig:62}
\end{figure}
\clearpage

\begin{figure}[p!]
  \centering \includegraphics[width=9cm]{Fz_11_all.png}\\  
  \caption{Осевая сила $F_{z,\, 11}$, действующая на оголовок затвора: $\bullet$ -- циклическая постановка, 
           тонкий затвор; сплошная линия -- циклическая постановка, тонкий движущийся затвор; 
           $\circ$ -- циклическая постановка, затвор реальной толщины; $\square$ -- полная стационарная 
           постановка; \textcolor{red}{$\blacktriangle$} -- полная нестационарная постановка; вертикальными 
           отрезками обозначена амплитуда колебаний в нестационарном расчете}
  \label{fig:7}
\end{figure}
\begin{figure}[p!]
  \centering \small \emph{a}\rule{90mm}{0mm}\emph{б}\\[1.5mm]
  {\includegraphics[width=8.5cm]{Fx_kg.png}}\hfill
  {\includegraphics[width=8.5cm]{Fy_kg.png}}\\  
  \caption{Компоненты   радиальной силы, действующей на боковую поверхность 
  затвора: $\square$ -- полная стационарная 
  постановка; вертикальные отрезки -- полная постановка, амплитуда колебаний в нестационарном расчете}
  \label{fig:8}
\end{figure}
\clearpage

\subsubsection{Структура течения вблизи затвора}
\label{s:4524}
На рисунке~\ref{fig:65} изображена структрура потока для разных степеней закрытия затвора. 
\begin{figure}[h!]
  \centering 
  \includegraphics[width=11cm]{Flow_zatvor_all.png}\\
  \caption{Линии тока и изолинии давления на боковых поверхностях статорных колонн и лопаток НА 
           при трех степенях закрытия затвора. Вид со стороны спиральной камеры}
  \label{fig:65}
\end{figure}
